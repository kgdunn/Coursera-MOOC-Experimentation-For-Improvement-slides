\usepackage[latin1]{inputenc}  % allows direct input of Unicode characters
% %\usepackage[utf8x]{inputenc}   % allows direct input of Unicode characters
\inputencoding{utf8} %<--- try this next time
%\usepackage{textcomp}
\usepackage{amsmath,amssymb,amsfonts,euscript,mathrsfs,wasysym,textcomp}
\usepackage{array}
\usepackage{transparent} 		% to get alpha channel in "\color"
\usepackage{multirow}
\usepackage{multimedia}
\usepackage{fancybox}
\usepackage{mathtools}
\usepackage{psfrag}
\usepackage{listings}
\usepackage{hyperref}
\usepackage[normalem]{ulem}  	% For strikeout text: \sout{text goes here}
\usepackage{datetime} 			% For better date/time display
\usepackage{comment}  			% To block out parts of the notes: \iffalse ....  \fi
\usepackage[multidot]{grffile} 	% Handles file names with spaces and dots in them
\usepackage{cancel}				% To cancel terms in an equation
\usepackage{etoolbox}			% For selective compiling: figure sources might be on different machines
\usepackage{tabulary}			% Better package for tables?
\usepackage{url}		

\providetoggle{unixmac}
\settoggle{unixmac}{true}
\iftoggle{unixmac}
{
	\newcommand{\imagedir}{/Users/kevindunn/Dropbox/Personal/repos/figures}
}{
	\newcommand{\imagedir}{C:/Figures}
}
\iftoggle{unixmac}
{
	\newcommand{\courseradir}{/Users/kevindunn/Dropbox/Career/Coursera/Media}
}{
	\newcommand{\imagedir}{C:/Figures}
}

%\newcommand{\angstrom}{\mbox{\normalfont\AA}}
% \tiny\scriptsize\footnotesize\small\normalsize\large\Large\LARGE\huge\Huge

\usetheme{default} 
\setbeamertemplate{navigation symbols}{}          % suppress all navigation symbols
\setbeamertemplate{blocks}[rounded][shadow=true]  % use rounded blocks (boxes), with shadows

% http://www.latex-community.org/forum/viewtopic.php?f=4&t=2251
\newenvironment<>{varblock}[2][\textwidth]{%
	\begin{center}
		\begin{minipage}{#1}
			\setlength{\textwidth}{#1}
			\begin{actionenv}#3%
				\def\insertblocktitle{#2}%
				\par%
				\usebeamertemplate{block begin}}
				{\par%
				\usebeamertemplate{block end}%
			\end{actionenv}
		\end{minipage}
	\end{center}
}


\definecolor{brickred}{rgb}{0.71, 0.2, 0.11}
\definecolor{todoGreen}{rgb}{0.0, 0.9, 0.0}
\definecolor{myGreen}{rgb}{0.,0.4,0.}
\definecolor{myOrange}{rgb}{1.,0.5,0.}
\definecolor{myBlue}{rgb}{0.0,0.1,0.9}
\definecolor{myRed}{rgb}{1.0,0.0,0.0}
\definecolor{myLightGrey}{rgb}{0.8,0.8,0.8}
\definecolor{Brown}{cmyk}{0,0.81,1,0.60}
\definecolor{OliveGreen}{cmyk}{0.64,0,0.95,0.40}
\definecolor{CadetBlue}{cmyk}{0.62,0.57,0.23,0}
\definecolor{purple}{rgb}{0.70,0.22,0.92}

%https://tex.stackexchange.com/questions/31323/cross-out-with-arrow-as-in-goes-to-zero
\renewcommand{\CancelColor}{\color{red}} %change cancel color to red
\makeatletter
\let\my@cancelto\cancelto %copy over the original cancelto command
\newcommand<>{\cancelto}[2]{\alt#3{\my@cancelto{#1}{#2}}{\mathrlap{#2}\phantom{\my@cancelto{#1}{#2}}}}
% redefine the cancelto command, using \phantom to assure that the
% result doesn't wiggle up and down with and without the arrow
\makeatother


% Show page numbers and dates on slides
% KGD: removed footer completely: 01 Oct 2012: Presentation Zen: no redundant information
%\setbeamertemplate{footline}{\begin{beamercolorbox}[right]{section in head/foot}{\color{black}{\tiny \insertframenumber}} ~~~ \vskip5pt \end{beamercolorbox}}

\setbeamercovered{transparent=1} 

\hypersetup{colorlinks        = true,    
 			linkcolor         = blue,    
%}
% 			linkbordercolor   = {1 0 0},    
% 			urlcolor          = cyan,    
% 			bookmarks         = {true,},    
% 			bookmarksopen     = {true,},    
% 			bookmarksnumbered = {false,},    
 			pdftitle          = {Kevin Dunn},
 			pdfsubject        = {Kevin Dunn},
 			pdfauthor         = {Kevin Dunn},
 			pdfproducer       = {LaTeX, wiki2beamer, beamer, BeamerPDF},    
 			pdfkeywords       = https://learnche.org,
 		}
\usepackage{pgfpages}

\makeatletter
\def\hlinewd#1{%
\noalign{\ifnum0=`}\fi\hrule \@height #1 %
\futurelet\reserved@a\@xhline}
\makeatother

% Source code listings
\lstdefinestyle{python}{
    captionpos=t,%
    basicstyle=\footnotesize\ttfamily,%
    numberstyle=\tiny,%
    numbers=none,%
    stepnumber=1,%
    frame=single,%
    showspaces=false,%
    showstringspaces=false,%
    showtabs=false,%
    keywordstyle=\color{blue},%
    identifierstyle=,%
    commentstyle=\color{gray},%
    stringstyle=\color{blue}%
}
% Override the Python settings with these
\lstset{language=python,
    basicstyle=\normalsize,
    frame=tb,
    framesep=5pt,
    keywordstyle=\ttfamily\color{OliveGreen},
    identifierstyle=\ttfamily\color{CadetBlue}\bfseries, 
    commentstyle=\color{Brown},
    stringstyle=\ttfamily\color{blue},
    xleftmargin=2pt,xrightmargin=2pt,
    showstringspaces=false
}

\lstset{language=R,
    basicstyle=\small,
    frame=tb,
    framesep=5pt,
    keywordstyle=\ttfamily\color{OliveGreen},
    identifierstyle=\ttfamily\color{CadetBlue}\bfseries, 
    commentstyle=\color{Brown},
    stringstyle=\ttfamily\color{blue},
    xleftmargin=2pt,xrightmargin=2pt,
    showstringspaces=false
}

% Some definitions
\newcommand{\todo}[1]{{\center{\color{todoGreen} #1}}}
\newcommand{\q}{{\textbf{Q}}}
\newcommand{\adv}{{\small {\color{Brown} (advanced)}}}
\newcommand{\extra}{{\small {\color{Brown} (extra)}}}
\newcommand{\liturl}[2]{\href{#1}{#2 {\tiny ~(#1)}}}
\newcommand{\micron}{$\mu $m}
\newcommand{\see}[1]{{\tiny [{\color{myBlue}{#1}}]}}
\newcommand{\seefull}[1]{{\color{myBlue}{#1}}}
\newcommand{\degC}{$^\circ\text{C}$}
\newcommand{\myhrule}{\vspace{4pt}\hrule\vspace{4pt}}