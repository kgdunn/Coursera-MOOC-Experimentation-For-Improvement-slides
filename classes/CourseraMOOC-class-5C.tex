\begin{frame}\frametitle{}
	\centerline{\includegraphics[height=.8\textheight]{../5C/Supporting material/flickr-quinnanya-2186957732_12f67e557b_o-grocery-shelf.jpg}}
	\see{\href{https://secure.flickr.com/photos/quinnanya/2186957732}{Flickr: quinnanya}}
\end{frame}
\begin{frame}\frametitle{}
	\centerline{\includegraphics[height=\textheight]{../5C/Supporting material/COST-contours-shopping-01a.png}}
\end{frame}
\begin{frame}\frametitle{}
	\centerline{\includegraphics[height=\textheight]{../5C/Supporting material/COST-contours-shopping-01b.png}}
\end{frame}
\begin{frame}\frametitle{}
	\centerline{\includegraphics[height=\textheight]{../5C/Supporting material/COST-contours-shopping-02a.png}}
\end{frame}
\begin{frame}\frametitle{}
	\centerline{\includegraphics[height=\textheight]{../5C/Supporting material/COST-contours-shopping-02b.png}}
\end{frame}
\begin{frame}\frametitle{}
	\centerline{\includegraphics[height=\textheight]{../5C/Supporting material/COST-contours-shopping-03.png}}
\end{frame}
\begin{frame}\frametitle{}
	\centerline{\includegraphics[height=\textheight]{../5C/Supporting material/COST-contours-shopping-04.png}}
\end{frame}
\begin{frame}\frametitle{}
	\centerline{\includegraphics[height=\textheight]{../5C/Supporting material/COST-contours-shopping-05.png}}
\end{frame}
\begin{frame}\frametitle{}
	\centerline{\includegraphics[height=\textheight]{../5C/Supporting material/COST-contours-shopping-06a.png}}
\end{frame}
\begin{frame}\frametitle{}
	\centerline{\includegraphics[height=\textheight]{../5C/Supporting material/COST-contours-shopping-06b.png}}
\end{frame}
\begin{frame}\frametitle{}
	\centerline{\includegraphics[height=\textheight]{../5C/Supporting material/COST-contours-shopping-07.png}}
\end{frame}
\begin{frame}\frametitle{}
	\centerline{\includegraphics[height=\textheight]{../5C/Supporting material/COST-contours-shopping-08.png}}
\end{frame}
\begin{frame}\frametitle{}
	\centerline{\includegraphics[height=\textheight]{../5C/Supporting material/COST-contours-shopping-09.png}}
\end{frame}
\begin{frame}\frametitle{}
	\centerline{\includegraphics[height=\textheight]{../5C/Supporting material/COST-contours-shopping-10.png}}
\end{frame}
\begin{frame}\frametitle{}
	\centerline{\includegraphics[height=\textheight]{../5C/Supporting material/COST-contours-shopping-11.png}}
\end{frame}
\begin{frame}\frametitle{}
	\centerline{\includegraphics[height=\textheight]{../5C/Supporting material/COST-contours-shopping-12.png}}
\end{frame}


\begin{frame}\frametitle{The COST approach: {\color{myGreen}``Change One Single Thing'' at a time}}
	\begin{columns}[b]
		\column{0.6\textwidth}
			\begin{exampleblock}{}
				\begin{itemize}
					\item	leads you into a false belief that you have reached the optimum \pause
					\item	COST can possibly work in 2-dimensions, and maybe with 3 variables
					\item	but the chance you hit the optimum gets lower \pause
					\item	interactions and other unusual surface shapes makes COST inefficient \pause
					
					\item	OFAT = one factor at a time \\(another name for COST)
				\end{itemize}
			\end{exampleblock}
		
		\column{0.40\textwidth}
			\centerline{\includegraphics[width=1.1\textwidth]{../5C/Supporting material/COST-contours-shopping-12.png}}
	\end{columns}
	
	\pause
	\vspace{1cm}
	{\color{myOrange}COST works in a lab: to prove a conclusive cause-effect relationship}
	\\
	\pause
	\vspace{0.5cm}
	{\color{blue} Recall, RSM {\small (response surface methods)} is used \textbf{after} screening for known causal factors}

\end{frame}

\begin{frame}\frametitle{Interpreting what contour plots are}
	\begin{columns}[b]
		\column{0.5\textwidth}
			\centerline{\includegraphics[width=1.1\textwidth]{../5C/Supporting material/wikipedia-20121125213159!FujiSunriseKawaguchiko2025WP.jpg}}
			
			\see{Wikipedia: \href{https://commons.wikimedia.org/wiki/File:FujiSunriseKawaguchiko2025WP.jpg}{Wikipedia}}
		
		\column{0.5\textwidth}
			\centerline{\includegraphics[width=1.102\textwidth]{../5C/Supporting material/Google-Map-Screenshot-wide.png}}

			\see{\href{https://www.google.com/maps/preview?f=q&hl=en&geocode=&q=Mt.+Fuji&ie=UTF8&t=p&ll=35.366656,138.733292&spn=0.099668,0.207367&z=13&iwloc=addr;}{Google Maps link}}
	\end{columns}
	
\end{frame}

\begin{frame}\frametitle{Why we need response surface methods}
	\centerline{\includegraphics[width=.7\textwidth]{../5C/Supporting material/flickr-slgc-5812112251_f0b0e253c5_o-basement-construction.jpg}}
	\see{\href{https://secure.flickr.com/photos/slgc/5812112251}{Flickr: slgc}}	
\end{frame}

\begin{frame}\frametitle{Some questions on how we will use response surface methods }
	\begin{columns}[c]
		\column{0.5\textwidth}
			\centerline{\includegraphics[width=1.102\textwidth]{../5C/Supporting material/COST-contours-shopping-11.png}}

		\column{0.5\textwidth}
			\begin{itemize}
				\item	Observing the surface is costly
						
						\hspace{1ex} {\scriptsize each touch on the surface = experiment }\\
						\hspace{1ex} {\scriptsize we want to do as few experiments as possible }
				\pause
				\item	How do we get to the top quickly and efficiently?
				
				\pause
				\item	How will we know we are actually at the top?
			\end{itemize}
	\end{columns}
	
\end{frame}

\begin{frame}\frametitle{What does the response surface consist of?}
	
	The response surface is a plot of the outcome variable:
	\begin{itemize}
		\item	total sales (to maximize)
		\item	total number of unburned popcorn (to maximize)
		\item	height of plants (to maximize)
	\end{itemize}
	\vspace{1cm}
	If you are stuck, use ``profit''
	\begin{exampleblock}{}
		\centerline{\color{myGreen}Profit = (total income) $-$ (total expenses)}
	\end{exampleblock}
	
\end{frame}

\begin{frame}\frametitle{What do we do if we want to \emph{\textbf{minimize}}?}
	
	We still maximize, but just turn our response surface upside-down
	
	\vspace{.5cm}
	\begin{exampleblock}{}
		\centerline{\color{myGreen}maximization = $-$ (minimization)}
	\end{exampleblock}
	\centerline{\includegraphics[width=.8\textwidth]{../5C/Supporting material/waste-water-screenshot-example.png}}
	
\end{frame}

\begin{frame}\frametitle{Some questions to leave you with for the next video }
	\begin{columns}[c]
		\column{0.5\textwidth}
			\centerline{\includegraphics[width=1.102\textwidth]{../5C/Supporting material/COST-contours-shopping-11.png}}

		\column{0.5\textwidth}
			\begin{itemize}
				\item	Which direction should we climb up that mountain? 
				\item	How do we get to the top quickly and efficiently? \pause
				\item	What size of steps should we take? \pause
				\item	What if that surface is nonlinear? \pause
				\item	When do we stop? ``How do we confirm we are at the top?''
			\end{itemize}
	\end{columns}
	 
	\pause
	\vspace{1cm}
	{\color{myOrange}\emph{Hint}: some of the answers were given in the prior video (popcorn optimization)}
\end{frame}
