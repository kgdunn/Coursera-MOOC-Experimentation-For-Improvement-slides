
\begin{frame}\frametitle{Back to our example}
	\begin{columns}[T]
		\column{0.45\textwidth}
		
			\centerline{\includegraphics[width=\textwidth]{\imagedir/examples/distillation/flickr-mhl20-6095987122_ac6e275e1c_o.jpg}}
			
			 \see{\href{https://secure.flickr.com/photos/mhl20/6095987122}{Flickr: mhl20}}
		
		\column{0.48\textwidth}
			
			\begin{enumerate}
				\item	What is my {\color{myOrange}Objective}?
					\begin{itemize}
						\onslide+<2->{\item	usually has a ``direction''}
					\end{itemize}
				\item	What is my {\color{myOrange} Outcome}?
					\begin{itemize}
						\onslide+<3->{\item	needs to be measurable}
					\end{itemize}
				\item	What are my {\color{myOrange}Factors}?
					\begin{itemize}
						\onslide+<4->{\item	two \emph{types} of factors}
					\end{itemize}
			\end{enumerate}
	\end{columns}
\end{frame}

\begin{frame}\frametitle{{\Large Draw a distillation column and determine factors to change}}

\end{frame}

\begin{frame}\frametitle{Now write down a few experiments you will run}
	\begin{itemize}
		\item	only consider two factors
		\item	you don't need to use all rows in the table
		\item	write the specific values you will use for factors \textbf{A} and \textbf{B}
	\end{itemize}
	
	\vspace{1cm}
	\begin{tabulary}{\linewidth}{c|c|c|c}
		\textbf{\relax Experiment} & \textbf{\relax A = \_\_\_\_\_\_\_\_} & \textbf{\relax B = \_\_\_\_\_\_\_\_} & \textbf{\relax Outcome variable } \\ \hline &  &  & \\
		\textbf{1} & \_\_\_\_\_\_\_\_  & \_\_\_\_\_\_\_\_ &   \\ &  &  & \\
		\textbf{2} & \_\_\_\_\_\_\_\_  & \_\_\_\_\_\_\_\_ &   \\ &  &  & \\
		\textbf{3} & \_\_\_\_\_\_\_\_  & \_\_\_\_\_\_\_\_ &   \\ &  &  & \\
		\textbf{4} & \_\_\_\_\_\_\_\_  & \_\_\_\_\_\_\_\_ &   \\ &  &  & \\
		\textbf{5} & \_\_\_\_\_\_\_\_  & \_\_\_\_\_\_\_\_ &   \\ &  &  & \\
	\end{tabulary}	
\end{frame}

\begin{frame}\frametitle{How many experiments should I run?}
	For this course, a good start is to run $2^k$ experiments
	\begin{itemize}
		\item	Every factor ($k$) is at 2 levels
		\item	For example: $k$ = 3 factors, requires 8 experiments
	\end{itemize}
	
	
	
	\begin{columns}[c]
		\column{0.65\textwidth}
			\centerline{\includegraphics[width=\textwidth]{\imagedir/doe/half-fraction-in-3-factors-mooc-all-8-no-labels.png}}
		\column{0.35\textwidth}
			\onslide+<2->{{\color{red}BUT, always run the experiments in \textbf{random} order}}

	\end{columns}
	
\end{frame}

\begin{frame}\frametitle{Back to our example}
	\begin{columns}[T]
		\column{0.45\textwidth}
		
			\centerline{\includegraphics[width=\textwidth]{\imagedir/examples/distillation/flickr-mhl20-6095987122_ac6e275e1c_o.jpg}}
			
			\see{\href{https://secure.flickr.com/photos/mhl20/6095987122}{Flickr: mhl20}}
		
		\column{0.48\textwidth}
			
			\begin{enumerate}
				\item	What is my {\color{myOrange}Objective}?
					\begin{itemize}
						\onslide+<1->{\item	usually has a ``direction''}
					\end{itemize}
				\item	What is my {\color{myOrange} Outcome}?
					\begin{itemize}
						\onslide+<1->{\item	needs to be measurable}
					\end{itemize}
				\item	What are my {\color{myOrange}Factors}?
					\begin{itemize}
						\onslide+<1->{\item	two \emph{types} of factors}
					\end{itemize}
					
				\item	What {\color{myOrange}levels} do I pick for the factors?
					\begin{itemize}
						\onslide+<2->{\item	not at the maximum and minimum (extremes)}
					\end{itemize}
					
				\item	\onslide+<3->{Next, we {\color{myOrange}analyze the data}}
					\begin{itemize}
						\item	\onslide+<3->{We will use a visual analysis for now}
					\end{itemize}
			\end{enumerate}
	\end{columns}
\end{frame}

\begin{frame}\frametitle{How do I analyze the data afterwards? By example}
	\begin{tabulary}{\linewidth}{c|c|c|c}
		\textbf{\relax Experiment} & \textbf{\relax A = reflux ratio} & \textbf{\relax B = feed flow} & \textbf{\relax Outcome} \\ \hline &  &  & \\
		\textbf{1} & 1.1 \qquad {\color{blue}[$-1$]} & 20 \qquad {\color{blue}[$-1$]}& {\color{myOrange}50}  \\ &  &  & \\
		\textbf{2} & 1.5 \qquad {\color{blue}[$+1$]} & 20 \qquad {\color{blue}[$-1$]}& {\color{myOrange}70}  \\ &  &  & \\
		\textbf{3} & 1.1 \qquad {\color{blue}[$-1$]} & 28 \qquad {\color{blue}[$+1$]}& {\color{myOrange}60}  \\ &  &  & \\
		\textbf{4} & 1.5 \qquad {\color{blue}[$+1$]} & 28 \qquad {\color{blue}[$+1$]}& {\color{myOrange}80}  \\ &  &  & \\
		\textbf{5} & 1.3 \qquad\,\, {\color{blue}[$0$]} & 24 \qquad\,\, {\color{blue}[$0$]}& {\color{myOrange}65}  \\ &  &  & \\
	\end{tabulary}
\end{frame}

\begin{frame}\frametitle{Always start by visualizing the system}
	\begin{columns}[T]
		\column{0.7\textwidth}
			\centerline{\includegraphics[width=\textwidth]{\imagedir/doe/cube-plot-2-factors.png}}
		\column{0.4\textwidth}
			\begin{tabulary}{\linewidth}{c|c|c|c}
				Expt & \textbf{\relax A} & \textbf{\relax B} & \textbf{\relax $y$} \\ \hline &  &  & \\
				1 &{\color{blue}$-1$} & {\color{blue}$-1$}& {\color{myOrange}\textbf{50}}  \\ &  &  & \\
				2 &{\color{blue}$+1$} & {\color{blue}$-1$}& {\color{myOrange}\textbf{70}}  \\ &  &  & \\
				3 &{\color{blue}$-1$} & {\color{blue}$+1$}& {\color{myOrange}\textbf{60}}  \\ &  &  & \\
				4 &{\color{blue}$+1$} & {\color{blue}$+1$}& {\color{myOrange}\textbf{80}}  \\ &  &  & \\
				5 &{\color{blue}$0$ } & {\color{blue}$0$ }& {\color{myOrange}\textbf{65}}  \\ &  &  & \\
			\end{tabulary}
	\end{columns}
\end{frame}

\begin{frame}\frametitle{Let's try a different system now}
	% \begin{columns}[T]
% 		\column{0.7\textwidth}
% 			\centerline{\includegraphics[width=\textwidth]{\imagedir/doe/cube-plot-2-factors-interactions.png}}
% 		\column{0.4\textwidth}
% 			\begin{tabulary}{\linewidth}{c|c|c|c}
% 				Expt & \textbf{\relax A} & \textbf{\relax B} & \textbf{\relax $y$} \\ \hline &  &  & \\
% 				1 &{\color{blue}$-1$} & {\color{blue}$-1$}& {\color{myOrange}\textbf{50}}  \\ &  &  & \\
% 				2 &{\color{blue}$+1$} & {\color{blue}$-1$}& {\color{myOrange}\textbf{60}}  \\ &  &  & \\
% 				3 &{\color{blue}$-1$} & {\color{blue}$+1$}& {\color{myOrange}\textbf{60}}  \\ &  &  & \\
% 				4 &{\color{blue}$+1$} & {\color{blue}$+1$}& {\color{myOrange}\textbf{85}}  \\ &  &  & \\
% 				5 &{\color{blue}$0$ } & {\color{blue}$0$ }& {\color{myOrange}\textbf{70}}  \\ &  &  & \\
% 			\end{tabulary}
% 	\end{columns}
\end{frame}

\begin{frame}\frametitle{What if I have 3 or more factors? Can I solve this mathematically?}
	
	We don't have time in today's class for some of the details. We will cover this in \texttt{CHE 4C3}.
	
	\vspace{12pt}
	But you don't have to wait for \texttt{CHE 4C3}; watch videos at \href{http://yint.org/experiments}{http://yint.org/experiments}
	\begin{itemize}
		\item	video 2A
		\item	video 2B
		\item	video 2C
		\item	video 2D
		\item	video 3A
		\item	video 3B
	\end{itemize}
	
	\begin{exampleblock}{}
		There are less that 90 minutes of self-directed learning videos that extend what we learned today.
	\end{exampleblock}
	
\end{frame}

\begin{frame}\frametitle{How \textbf{NOT} to run the experiments}
	(Re)watch  video ``5C'' from the Coursera course at \href{http://yint.org/media/5C.mp4}{http://yint.org/media/5C.mp4}
\end{frame}


\begin{frame}\frametitle{R code used to solve the problems in these slides}
	Enter the code at \href{http://yint.org/Rweb}{http://yint.org/Rweb}
	
	\vspace{12pt}
	
	\texttt{A <- c(-1,  +1,  -1,  +1,   0)}\\
	\texttt{B <- c(-1,  -1,  +1,  +1,   0)}\\
	\texttt{y <- c(50, 70, 60, 80, 65)}\\
	\texttt{model.1 <- lm(y $\sim$ A*B)} \\
	\texttt{summary(model.1)}\\
	
	\vspace{24pt}
	\texttt{A <- c(-1,  +1,  -1,  +1,   0)}\\
	\texttt{B <- c(-1,  -1,  +1,  +1,   0)}\\
	\texttt{y <- c(50, 60, 60, 90, 70)}\\
	\texttt{model.2 <- lm(y $\sim$ A*B)}\\
	\texttt{summary(model.2)}
	
\end{frame}

