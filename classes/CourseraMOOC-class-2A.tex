\begin{frame}\frametitle{Two-factor experiments: a recap from the last module}
    \includegraphics[width=\textwidth]{\imagedir/doe/two-types-of-variables.png}
\end{frame}

\begin{frame}\frametitle{Two-factor experiments: a recap from the last module}
    \includegraphics[width=\textwidth]{\imagedir/doe/two-types-of-variables-expanded.png}
\end{frame}

{\usebackgroundtemplate{\vbox to \paperheight{\vfil\hbox to \paperwidth{\hfil  
    \includegraphics[width=\paperwidth]
	{../2A/Supporting files/flickr-4137738962_064e1f4604_o.jpg}  \hfil}\vfil}}
\begin{frame}\frametitle{}
	\vspace{-5cm}
	\mode<handout>{
		\see{Flickr: ed\_welker \href{https://www.flickr.com/photos/ed_welker/4137738962/}{4137738962}}
	}
\end{frame}}

\begin{frame}\frametitle{A systematic approach}
	 \begin{enumerate}
	 	\item	What's my outcome? \pause
	 	\item	What's my objective? \pause
	 	\item	Which factors? \pause
		\item	At what levels?\pause
	 	\item	Plan the experiment \pause
	 	\item	Implement the experiment\pause
	 	\item	Analyze the results\pause
		\item	Repeat (if required)
	 \end{enumerate}
	 \vspace{5cm}
\end{frame}

{\usebackgroundtemplate{\vbox to \paperheight{\vfil\hbox to \paperwidth{\hfil  
    \includegraphics[width=\paperwidth]
	{../2A/Supporting files/flickr-13193465325_b35360c4b8_o.jpg}  \hfil}\vfil}}
\begin{frame}\frametitle{}
	
	{\color{white}
		\textbf{Outcome}
		
		\qquad Number of popped corn
		
		\vspace{18pt}
		
		\textbf{Objective}
		
		\qquad Maximizing number of popped corn
	

		\vspace{12pt}
		\onslide+<2->{
			This is equivalent to:

			``\emph{minimize the number of unpopped corn}''
		}

	}
	\vspace{4cm}
	\mode<handout>{
		\see{Flickr: booleansplit \href{https://www.flickr.com/photos/booleansplit/13193465325/}{13193465325}}
	}
\end{frame}}

{\usebackgroundtemplate{\vbox to \paperheight{\vfil\hbox to \paperwidth{\hfil  
    \includegraphics[width=0.95\paperwidth,trim=0 1.63cm 0 1.63cm, clip]
	{../2A/Slides/04Screen Shot 2014-07-15 at 10.14.11 .png}  \hfil}\vfil}}
\begin{frame}\frametitle{}
\end{frame}}

{\usebackgroundtemplate{\vbox to \paperheight{\vfil\hbox to \paperwidth{\hfil  
    \includegraphics[width=0.95\paperwidth,trim=0 1.63cm 0 1.63cm, clip]
	{../2A/Slides/05Screen Shot 2014-07-15 at 10.14.20 .png}  \hfil}\vfil}}
\begin{frame}\frametitle{}
\end{frame}}

{\usebackgroundtemplate{\vbox to \paperheight{\vfil\hbox to \paperwidth{\hfil  
    \includegraphics[width=0.95\paperwidth,trim=0 1.63cm 0 1.63cm, clip]
	{../2A/Slides/06Screen Shot 2014-07-15 at 10.14.48 .png}  \hfil}\vfil}}
\begin{frame}\frametitle{}
\end{frame}}

\begin{frame}\frametitle{}
	
     \includegraphics[width=0.5\textwidth]{\imagedir/doe/examples/advice-logo.png}
	 \pause
	 \vspace{2cm}
	 \begin{center}
		\color{purple}
		 \Large Always run experiments
		 in random order
	 \end{center}
     \vspace{2cm}
\end{frame}

% {\usebackgroundtemplate{\vbox to \paperheight{\vfil\hbox to \paperwidth{\hfil
%     \includegraphics[width=0.95\paperwidth,trim=0 1.63cm 0 1.63cm, clip]
% 	{../2A/Slides/07Screen Shot 2014-07-15 at 10.14.52 .png}  \hfil}\vfil}}
% \begin{frame}\frametitle{}
% \end{frame}}

\begin{frame}\frametitle{Various options for selecting the random order of experiments}
	Let's assume you need 8 random numbers:
	\begin{enumerate}
		\item	write numbers 1, 2, ... 8 on pieces of paper / cards
		\item	spreadsheets: use the following code (ignore any duplicates)
			\begin{varblock}[5cm]{}
				\texttt{=1 + INT( 8 * RAND() )}
			\end{varblock}

		\item	some spreadsheets have a special function
			\begin{varblock}[5cm]{}
				\texttt{=RANDBETWEEN(1, 8)}
				\begin{center}
					% FIGURE GOT DELETED
					% \includegraphics[width=0.4\textwidth]{\imagedir/doe/screenshots/randbetween.png}
				\end{center}
			\end{varblock}

		\item	In R (statistical software)
			\begin{varblock}[5cm]{}
				\texttt{sample(8)}
			\end{varblock}
	\end{enumerate}
\end{frame}

{\usebackgroundtemplate{\vbox to \paperheight{\vfil\hbox to \paperwidth{\hfil  
    \includegraphics[width=0.95\paperwidth,trim=0 1.63cm 0 1.63cm, clip]
	{../2A/Slides/08Screen Shot 2014-07-15 at 10.15.11.png}  \hfil}\vfil}}
\begin{frame}\frametitle{}
\end{frame}}

{\usebackgroundtemplate{\vbox to \paperheight{\vfil\hbox to \paperwidth{\hfil  
    \includegraphics[width=0.95\paperwidth,trim=0 1.63cm 0 1.63cm, clip]
	{../2A/Slides/09Screen Shot 2014-07-15 at 10.15.29 .png}  \hfil}\vfil}}
\begin{frame}\frametitle{}
\end{frame}}

{\usebackgroundtemplate{\vbox to \paperheight{\vfil\hbox to \paperwidth{\hfil  
    \includegraphics[width=0.95\paperwidth,trim=0 1.63cm 0 1.63cm, clip]
	{../2A/Slides/10Screen Shot 2014-07-15 at 10.15.35 .png}  \hfil}\vfil}}
\begin{frame}\frametitle{}
\end{frame}}

{\usebackgroundtemplate{\vbox to \paperheight{\vfil\hbox to \paperwidth{\hfil  
    \includegraphics[width=0.95\paperwidth,trim=0 1.63cm 0 1.63cm, clip]
	{../2A/Slides/11Screen Shot 2014-07-15 at 10.15.48 .png}  \hfil}\vfil}}
\begin{frame}\frametitle{}
\end{frame}}

{\usebackgroundtemplate{\vbox to \paperheight{\vfil\hbox to \paperwidth{\hfil  
    \includegraphics[width=0.95\paperwidth,trim=0 1.63cm 0 1.63cm, clip]
	{../2A/Slides/12Screen Shot 2014-07-15 at 10.16.00 .png}  \hfil}\vfil}}
\begin{frame}\frametitle{}
\end{frame}}

{\usebackgroundtemplate{\vbox to \paperheight{\vfil\hbox to \paperwidth{\hfil  
    \includegraphics[width=0.95\paperwidth,trim=0 1.63cm 0 1.63cm, clip]
	{../2A/Slides/13Screen Shot 2014-07-15 at 10.16.06 .png}  \hfil}\vfil}}
\begin{frame}\frametitle{}
\end{frame}}

{\usebackgroundtemplate{\vbox to \paperheight{\vfil\hbox to \paperwidth{\hfil  
    \includegraphics[width=0.95\paperwidth,trim=0 1.63cm 0 1.63cm, clip]
	{../2A/Slides/14Screen Shot 2014-07-15 at 10.16.20 .png}  \hfil}\vfil}}
\begin{frame}\frametitle{}
\end{frame}}

{\usebackgroundtemplate{\vbox to \paperheight{\vfil\hbox to \paperwidth{\hfil  
    \includegraphics[width=0.95\paperwidth,trim=0 1.63cm 0 1.63cm, clip]
	{../2A/Slides/15Screen Shot 2014-07-15 at 10.16.28 .png}  \hfil}\vfil}}
\begin{frame}\frametitle{}
\end{frame}}

% {\usebackgroundtemplate{\vbox to \paperheight{\vfil\hbox to \paperwidth{\hfil
%     \includegraphics[width=0.95\paperwidth,trim=0 1.63cm 0 1.63cm, clip]
% 	{../2A/Slides/16Screen Shot 2014-07-15 at 10.16.31 .png}  \hfil}\vfil}}
% \begin{frame}\frametitle{}
% \end{frame}}

{\usebackgroundtemplate{\vbox to \paperheight{\vfil\hbox to \paperwidth{\hfil  
    \includegraphics[width=0.95\paperwidth,trim=0 1.63cm 0 1.63cm, clip]
	{../2A/Slides/17Screen Shot 2014-07-15 at 10.16.42 .png}  \hfil}\vfil}}
\begin{frame}\frametitle{}
\end{frame}}

\begin{frame}\frametitle{Simple visualization of these factorial designs are powerful!}
	\begin{columns}[T]
		\column{0.5\textwidth}
			1. Tables show numeric trends
			
			\vspace{2pt}
		    \includegraphics[width=0.6\textwidth]{../2A/Slides/popcorn-example-table.png}
			
			
			2.	Cube plots indicate important factors
			
			\includegraphics[width=0.55\textwidth]{../2A/Slides/popcorn-example-cube.png}
			
		\column{0.01\textwidth}
			\rule[3mm]{0.01cm}{85mm}%
			
		\column{0.5\textwidth}
			3.	Contour plots show where to move next
			
			\includegraphics[width=0.55\textwidth]{../2A/Slides/popcorn-example-contour.png}
			
			4.	Interaction plots show synergies [next...]
			
			\includegraphics[width=0.55\textwidth]{../2A/Slides/popcorn-example-interaction.png}
			
	\end{columns}
	
	
\end{frame}

