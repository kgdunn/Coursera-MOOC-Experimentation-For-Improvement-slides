
%$\hat{y} = b_0 + b_\text{A}x_\text{A} + {\color{blue}b_\text{AA}x^2_\text{AA}}$
%$\hat{y} = 91.8 + 14.9 x_\text{A}  -2 x^2_\text{AA}$
%$\color{red}\hat{y} = 91.8 + 14.9 (+1)  -2 (+1)^2 \approx 105 $
% \color[rgb]{0.538744,0.117572,0.032380}

\begin{frame}\frametitle{The {\color{red} critical concepts} covered in this video using the {\color{myOrange}popcorn case study}}
	\begin{enumerate}
		\item	Real-world units and coded units
		\item	\textbf{Linear vs nonlinear systems}
		\item	Prediction models are wrong, but still useful
		\item	\textbf{Noise and error} and the need for replicated experiments
		\item	How to systematically reach an optimum
		\item	Justifying the choice of every experiment
	\end{enumerate}
\end{frame}

\begin{frame}\frametitle{The connection between our model's coded units, and real-world units}
	\begin{columns}[b]
		
		
		\column{0.65\textwidth}
			\textbf{The general formula for continuous variables}
			\begin{flalign*}
				\onslide+<2->{
					\text{coded value} &= \dfrac{(\text{real value}) - (\text{center value})}{\tfrac{1}{2}\left(\text{range}\right)}\\ \\
				}
				\onslide+<3->{
					\text{center value} &= \dfrac{(\text{low value}) + (\text{high value})}{2}\\ \\
				}
				\onslide+<4->{
					\text{range} &= (\text{high value}) - (\text{low value}) &\\
				}
			\end{flalign*}
			
		\column{0.02\textwidth}
			
			\rule[3mm]{0.01cm}{55mm}%
			
		
		\column{0.40\textwidth}
			\textbf{Example:} cooking time, factor \textbf{A}
			\begin{flalign*}
				\onslide+<5->{	
					x_\text{A} &= \dfrac{\,\text{A} - 135}{\tfrac{1}{2}\left(30\right)\,} = \dfrac{\,\text{A} - 135}{15}\\ \\
				}
				\onslide+<3->{
					\text{center}_\text{A} &= \dfrac{120 + 150}{2} = 135\\ \\
				}
				\onslide+<4->{
					\text{range}_\text{A} &= 150 - 120 = 30 & \\
				}
			\end{flalign*}		
			
	\end{columns}
	\vspace{-.2cm}
	\centerline{\includegraphics[width=.4\textwidth]{../5B/Supporting materials/popcorn-coded-real.png}}
\end{frame}

\begin{frame}\frametitle{Let's try it out}
	
	\begin{flalign*}
		\text{coded value} &= \dfrac{(\text{real value}) - (\text{center value})}{\tfrac{1}{2}\left(\text{range}\right)} &
		\qquad x_\text{A} &= \dfrac{\,\text{A} - 135}{\tfrac{1}{2}\left(30\right)\,} = \dfrac{\,\text{A} - 135}{15}\\ 
		\\
		\onslide+<2->{
			&\text{If cooking time is 135 seconds, what is $x_\text{A}$?} &
		}
		\onslide+<3->{
			x_\text{A} &= \dfrac{135 - 135}{\tfrac{1}{2}\left(30\right)\,} = \dfrac{0}{15} = 0
		}\\ \\
		\onslide+<4->{
			&\text{If cooking time is 150 seconds, what is $x_\text{A}$?} &
		}
		\onslide+<5->{
			x_\text{A} &= \dfrac{150 - 135}{\tfrac{1}{2}\left(30\right)\,} = \dfrac{15}{15} = +1
		}
	\end{flalign*}
			
	
\end{frame}

\begin{frame}\frametitle{The connection between real-world units and  coded units (in reverse\emph{!})}
	\begin{columns}[b]
				
		\column{0.65\textwidth}
			\begin{flalign*}
				\intertext{\textbf{Going forwards}:}
				\text{coded value} &= \dfrac{(\text{real value}) - (\text{center value})}{\tfrac{1}{2}\left(\text{range}\right)}\\ \\
				\intertext{\textbf{Going backwards}:}
				\text{real value} &= (\text{coded value}) \times \tfrac{1}{2}\left(\text{range}\right) + (\text{center value}) \\ \\ \\
			\end{flalign*}
			
		\column{0.02\textwidth}
			\rule[3mm]{0.01cm}{55mm}
		
		\column{0.40\textwidth}
			
			\begin{flalign*}
				\intertext{\textbf{Example:} cooking time, factor \textbf{A}}
				x_\text{A} &= \dfrac{\,\text{A} - 135}{\tfrac{1}{2}\left(30\right)\,} = \dfrac{\,\text{A} - 135}{15}\\ \\
				\intertext{\color{myOrange} What is $x_\text{A} = +2$?:} 
				\onslide+<2->{
				\text{real value} &= (+2) \times \tfrac{1}{2}\left(30\right) + (135)\\
					\onslide+<3->{
					 			  &= (+2) \times 15 + (135)}\\
					\onslide+<4->{
								  &= 30 + 135 = 165\\}
				}
			\end{flalign*}		
			
	\end{columns}
	
	\vspace{2cm}
	\vfill
\end{frame}

\begin{frame}\frametitle{}
	\centerline{\includegraphics[width=.98\textwidth]{../5B/Supporting materials/popcorn-experiments-01.png}}
\end{frame}
\begin{frame}\frametitle{}
	\centerline{\includegraphics[width=.98\textwidth]{../5B/Supporting materials/popcorn-experiments-02.png}}
\end{frame}
\begin{frame}\frametitle{}
	\centerline{\includegraphics[width=.98\textwidth]{../5B/Supporting materials/popcorn-experiments-03.png}}
\end{frame}
\begin{frame}\frametitle{}
	\centerline{\includegraphics[width=.98\textwidth]{../5B/Supporting materials/popcorn-experiments-04.png}}
\end{frame}
\begin{frame}\frametitle{}
	\centerline{\includegraphics[width=.98\textwidth]{../5B/Supporting materials/popcorn-experiments-05.png}}
\end{frame}
\begin{frame}\frametitle{}
	\centerline{\includegraphics[width=.98\textwidth]{../5B/Supporting materials/popcorn-experiments-06.png}}
\end{frame}
\begin{frame}\frametitle{The famous quote by George Box}
	\begin{quote}
		``... all models are wrong, but some are useful.''
	\end{quote}
	
	\begin{quote}
		``...the practical question is how wrong do they have to be [before they are] not useful?''
	\end{quote}
	
	\vspace{1cm}
	G. E. P. Box and  N. R. Draper (1987), ``\emph{Empirical Model Building and Response Surfaces}'', John Wiley \& Sons, New York, NY.
\end{frame}
\begin{frame}\frametitle{Answering George Box's question for the popcorn example}
	
	What is ``\emph{not useful}\,'' in this case?
	
	\vspace{1cm}
	In an earlier video we said: ``always have an objective in mind''.
	
	\vspace{1cm}
	\begin{exampleblock}{The answer to the question}
		Our model is not useful when the predictions are not accurate; we need accurate predictions to optimize.
	\end{exampleblock}
\end{frame}
\begin{frame}\frametitle{}
	\centerline{\includegraphics[width=.98\textwidth]{../5B/Supporting materials/popcorn-experiments-07.png}}
\end{frame}
\begin{frame}\frametitle{}
	\centerline{\includegraphics[width=.98\textwidth]{../5B/Supporting materials/popcorn-experiments-08.png}}
\end{frame}
\begin{frame}\frametitle{}
	\centerline{\includegraphics[width=.98\textwidth]{../5B/Supporting materials/popcorn-experiments-09.png}}
\end{frame}
\begin{frame}\frametitle{}
	\centerline{\includegraphics[width=.98\textwidth]{../5B/Supporting materials/popcorn-experiments-10.png}}
\end{frame}
\begin{frame}\frametitle{}
	\centerline{\includegraphics[width=.98\textwidth]{../5B/Supporting materials/popcorn-experiments-11.png}}
\end{frame}
\begin{frame}\frametitle{}
	\centerline{\includegraphics[width=.98\textwidth]{../5B/Supporting materials/popcorn-experiments-12.png}}
\end{frame}
\begin{frame}\frametitle{}
	\centerline{\includegraphics[width=.98\textwidth]{../5B/Supporting materials/popcorn-experiments-13.png}}
\end{frame}
\begin{frame}\frametitle{}
	\centerline{\includegraphics[width=.98\textwidth]{../5B/Supporting materials/popcorn-experiments-14.png}}
\end{frame}
\begin{frame}\frametitle{}
	\centerline{\includegraphics[width=.98\textwidth]{../5B/Supporting materials/popcorn-experiments-15.png}}
\end{frame}
\begin{frame}\frametitle{}
	\centerline{\includegraphics[width=.98\textwidth]{../5B/Supporting materials/popcorn-experiments-16.png}}
\end{frame}
\begin{frame}\frametitle{}
	\centerline{\includegraphics[width=.98\textwidth]{../5B/Supporting materials/popcorn-experiments-17.png}}
\end{frame}
\begin{frame}\frametitle{}
	\centerline{\includegraphics[width=.98\textwidth]{../5B/Supporting materials/popcorn-experiments-18.png}}
\end{frame}
\begin{frame}\frametitle{}
	\centerline{\includegraphics[width=.98\textwidth]{../5B/Supporting materials/popcorn-experiments-19.png}}
\end{frame}
\begin{frame}\frametitle{}
	\centerline{\includegraphics[width=.98\textwidth]{../5B/Supporting materials/popcorn-experiments-20.png}}
\end{frame}
\begin{frame}\frametitle{}
	\centerline{\includegraphics[width=.98\textwidth]{../5B/Supporting materials/popcorn-experiments-21.png}}
\end{frame}
\begin{frame}\frametitle{}
	\centerline{\includegraphics[width=.98\textwidth]{../5B/Supporting materials/popcorn-experiments-22.png}}
\end{frame}
\begin{frame}\frametitle{}
	\centerline{\includegraphics[width=.98\textwidth]{../5B/Supporting materials/popcorn-experiments-23.png}}
\end{frame}
\begin{frame}\frametitle{}
	\centerline{\includegraphics[width=.98\textwidth]{../5B/Supporting materials/popcorn-experiments-24.png}}
\end{frame}
\begin{frame}\frametitle{}
	\centerline{\includegraphics[width=.98\textwidth]{../5B/Supporting materials/popcorn-experiments-25.png}}
\end{frame}
\begin{frame}\frametitle{}
	\centerline{\includegraphics[width=.98\textwidth]{../5B/Supporting materials/popcorn-experiments-26.png}}
\end{frame}
\begin{frame}\frametitle{}
	\centerline{\includegraphics[width=.98\textwidth]{../5B/Supporting materials/popcorn-experiments-27.png}}
\end{frame}
\begin{frame}\frametitle{}
	\centerline{\includegraphics[width=.98\textwidth]{../5B/Supporting materials/popcorn-experiments-28.png}}
\end{frame}
\begin{frame}\frametitle{}
	\centerline{\includegraphics[width=.98\textwidth]{../5B/Supporting materials/popcorn-experiments-29.png}}
\end{frame}
\begin{frame}\frametitle{}
	\centerline{\includegraphics[width=.98\textwidth]{../5B/Supporting materials/popcorn-experiments-30.png}}
\end{frame}
\begin{frame}\frametitle{}
	\centerline{\includegraphics[width=.98\textwidth]{../5B/Supporting materials/popcorn-experiments-31.png}}
\end{frame}
\begin{frame}\frametitle{}
	\centerline{\includegraphics[width=.98\textwidth]{../5B/Supporting materials/popcorn-experiments-32.png}}
\end{frame}
\begin{frame}\frametitle{}
	\centerline{\includegraphics[width=.98\textwidth]{../5B/Supporting materials/popcorn-experiments-33.png}}
\end{frame}
\begin{frame}\frametitle{}
	\centerline{\includegraphics[width=.98\textwidth]{../5B/Supporting materials/popcorn-experiments-34.png}}
\end{frame}
\begin{frame}\frametitle{}
	\centerline{\includegraphics[width=.98\textwidth]{../5B/Supporting materials/popcorn-experiments-35.png}}
\end{frame}
\begin{frame}\frametitle{}
	\centerline{\includegraphics[width=.98\textwidth]{../5B/Supporting materials/popcorn-experiments-36.png}}
\end{frame}
\begin{frame}\frametitle{Recap of each experiment used so far}
	\begin{itemize}
		\item	\textbf{1, 2, 3, 4}
		
			\hspace{1ex} the used to rule out factor B (oil type)\\
			\hspace{1ex} used  to provide an initial model
			
		\pause
		
		\item	\textbf{5}
		
			\hspace{1ex} center point: confirmed our model was linear\\
			\hspace{1ex} used to check prediction quality
		
		\pause
			
		\item	\textbf{6}
		
			\hspace{1ex} exploratory first step outside the initial factorial\\
			\hspace{1ex} to test the model's prediction quality\\
			\hspace{1ex} it suggested we rebuild the model (add quadratic term)
		
		\pause
		
		\item	\textbf{7}
		
			\hspace{1ex} seems like we are near an optimum

	\end{itemize}
	\onslide+<5->{
		\hspace{1cm} \includegraphics[width=0.3\textwidth]{\imagedir/doe/examples/advice-logo.png}
				\,\,{\color{blue}are you able to justify the need for each experiment?}
	}
\end{frame}
\begin{frame}\frametitle{The {\color{red} critical concepts} covered in this video using the {\color{myOrange}popcorn case study}}
	\begin{enumerate}
		\item	Real-world units and coded units
		\item	Linear vs nonlinear systems
		\item	Prediction models are wrong, but still useful
		\item	Noise and error and the need for replicated experiments
		\item	How to systematically reach an optimum
		\item	Justifying the choice of every experiment
	\end{enumerate}
\end{frame}
