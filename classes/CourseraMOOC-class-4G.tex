% \begin{comment}
% 	\begin{frame}\frametitle{\includegraphics[width=0.3\textwidth]{\imagedir/doe/examples/advice-logo.png}\,\, plan your experiments carefully ahead of time}
% 	\end{frame}
%
% 	\begin{columns}[T]
% 		\column{0.45\textwidth}
% 			\includegraphics[width=0.7\textwidth]{\imagedir/statistics/flicfcb_o.jpg}
%
% 			{\scriptsize (p. 230 in Box, Hunter and Hunter, 2$^\text{nd}$ ed)}
%
% 		\column{0.48\textwidth}
% 			\includegraphics[width=\textwidth]{\imagedir/doe/examples/solar-panel-mendelu-cz-website.png}
%
%
% 			\see{\href{http://yint.org/solar-panel-study}{http://yint.org/solar-panel-study}}
% 	\end{columns}
%
% 	\begin{center}\rule[8mm]{4cm}{0.01cm}\end{center}
% 	\rule[3mm]{0.01cm}{25mm}%
% \end{comment}

\begin{frame}\frametitle{Cell-culture example: creating the fractional factorial design}
	
	\vspace{0.5cm}
	\begin{columns}[T]
		\column{0.45\textwidth}
			{\scriptsize
				\begin{tabulary}{\linewidth}{ccccccc}
					\textbf{\relax Experiment} & \textbf{\relax A } & \textbf{\relax B} & \textbf{\relax C } & \textbf{\relax D = AB} & \textbf{\relax E = AC}\\ \cline{1-6}
					\textbf{1} & \(-\) & \(-\) & \(-\) & \(+\) & \(+\) \\
					\textbf{2} & \(+\) & \(-\) & \(-\) & \(-\) & \(-\) \\
					\textbf{3} & \(-\) & \(+\) & \(-\) & \(-\) & \(+\) \\
					\textbf{4} & \(+\) & \(+\) & \(-\) & \(+\) & \(-\) \\
					\textbf{5} & \(-\) & \(-\) & \(+\) & \(+\) & \(-\) \\
					\textbf{6} & \(+\) & \(-\) & \(+\) & \(-\) & \(+\) \\
					\textbf{7} & \(-\) & \(+\) & \(+\) & \(-\) & \(-\) \\
					\textbf{8} & \(+\) & \(+\) & \(+\) & \(+\) & \(+\) \\
				\end{tabulary}
			}
		
		\column{0.48\textwidth}
			\vspace{-0.2cm}
			\centerline{\includegraphics[height=.5\textwidth]{../4F/Supporting material/flickr-493740898_b98113ae44_o-cell-culture-mod.png} \see{\href{https://secure.flickr.com/photos/londonmatt/493740898}{Flickr: londonmatt}}}
			
	\end{columns}
	
	\vspace{0.2cm}
	\hrule
	\begin{enumerate}
		\item	Read the generators from the trade off table 
			\begin{itemize}
				\item		$\textbf{D = AB}$  and $\textbf{E = AC}$ 
			\end{itemize}
		
		\item	Rearrange the generators as  $\textbf{I = \ldots}$
			\begin{itemize}
				\item	$\textbf{I = ABD}$ and $\textbf{I = ACE}$ 
			\end{itemize}
			
		\item	Form the {\color{purple}\textbf{defining relationship}} taking all combinations of the words: $\textbf{I = \ldots}$
			\begin{itemize}
				\item	$\textbf{I = ABD = ACE = BCDE}$
			\end{itemize}
			
		\item	Ensure the defining relationship has $2^p$ words
			\begin{itemize}
				\item	$p=2$, and we have 4 words.
			\end{itemize}
		\item	{\color{myOrange}\textbf{In this example:}} we will use the defining relationship to compute the aliasing pattern
		
	\end{enumerate}

		
	
\end{frame}

\begin{frame}\frametitle{The purpose of the defining relationship: to calculate all possible aliases}
	\newcommand{\mo}{\color{myOrange}}
	
	\begin{exampleblock}{}
		\color{myGreen}By example: what are the aliases (confounding) with factor \textbf{\mo B}?
	\end{exampleblock}
	\begin{align*}
		\textbf{I}\onslide+<2->{\mo\,\textbf{B}} &&=&& \textbf{ABD}\onslide+<2->{\mo\,\textbf{B}} &&=&& \textbf{ACE}\onslide+<2->{\mo\,\textbf{B}} &&=&& \textbf{BCDE}\onslide+<2->{\mo\,\textbf{B}}\\		
		\onslide+<3->{\textbf{\mo B} &&=&& \textbf{A(B{\mo B})D} &&=&& \textbf{A{\mo B}CE} &&=&& \textbf{(B{\mo B})CDE}\\}
		\onslide+<4->{\textbf{B} &&=&& \textbf{AID} &&=&& \textbf{ABCE} &&=&& \textbf{ICDE}\\}
		\onslide+<5->{\textbf{B} &&=&& \textbf{AD} &&=&& \textbf{ABCE} &&=&& \textbf{CDE}}
	\end{align*}
	
	\begin{itemize}
		\item	Take the defining relationship
		\onslide+<2->{
			\item	Multiply every word by a \textbf{\mo B} term
		}
		\onslide+<3->{
			\item	Rearrange the order
		}
		\onslide+<4->{
			\item	Simplify by using the rules: \textbf{AA=I}; \textbf{BB=I}; ... \textbf{DD=I}, \emph{etc}
		}
		\onslide+<5->{
			\item	Drop out the unnecessary identity terms
		}
	\end{itemize}
\end{frame}

\begin{frame}\frametitle{How do we use this aliasing information we obtain?}
	Now we know the aliases of \textbf{B}:
	\begin{align*}
		\textbf{B} &&=&& \textbf{AD} &&=&& \textbf{ABCE} &&=&& \textbf{CDE}
	\end{align*}
	
	\vspace{1cm}
	\begin{itemize}
		\item	These are the terms that B is going to be confounded with
		\onslide+<2->{\item	We cannot tell B apart from the AD interaction.}
	\end{itemize}
\end{frame}

\begin{frame}\frametitle{How do we use this aliasing information we obtain?}
	Now we know the aliases of \textbf{B}:
	\begin{align*}
		\textbf{B} &&=&& \textbf{AD} &&\color{lightgray}=&& {\color{lightgray}\textbf{ABCE}} &&\color{lightgray}=&& \color{lightgray} \textbf{CDE}
	\end{align*}
	
	\vspace{1cm}
	\begin{itemize}
		\item	These are the terms that B is going to be confounded with
		\onslide+<2->{\item	We cannot tell B apart from the AD interaction.}
	\end{itemize}
\end{frame}

\begin{frame}\frametitle{How do we use this aliasing information we obtain?}
	Try this yourself: what are the aliases of \textbf{C}?
	\pause
	\begin{align*}
		\textbf{C} &&=&& \color{lightgray}\textbf{ABCD} &&=&& \textbf{AE} &&=&&\color{lightgray} \textbf{BDE}
	\end{align*}
	
	
	\vspace{1cm}
	\begin{itemize}
		\item	These are the terms that C is going to be confounded with
		\onslide+<2->{\item	We cannot tell C apart from the AE interaction.}
	\end{itemize}
\end{frame}

\begin{frame}\frametitle{How do we use this aliasing information we obtain?}
	Try this yourself: what are the aliases of \textbf{A}?
	\pause
	\begin{align*}
		\textbf{A} &&=&& \textbf{BD} &&=&& \textbf{CE} &&=&&\color{lightgray} \textbf{ABCDE}
	\end{align*}
		
	\vspace{1cm}
	\begin{itemize}
		\item	These are the terms that A is going to be confounded with
		\onslide+<2->{\item	We cannot tell A apart from the BD interaction and the CE interaction.}
	\end{itemize}
\end{frame}

\begin{frame}\frametitle{How do we use this aliasing information we obtain?}
	
	\vspace{1cm}
	Write out all the aliases for the 5 main effects:
	
	\vspace{0.5cm}
	\begin{align*}
		\textbf{A} &&=&& \textbf{BD} &&=&& \textbf{CE} &&=&&\color{lightgray} \textbf{ABCDE}\\
		\textbf{B} &&=&& \textbf{AD} &&=&& {\color{lightgray}\textbf{ABCE}} &&=&& \color{lightgray} \textbf{CDE}\\
		\textbf{C} &&=&& \color{lightgray}\textbf{ABCD} &&=&& \textbf{AE} &&=&&\color{lightgray} \textbf{BDE}\\
		\textbf{D} &&=&& \textbf{AB} &&=&& \color{lightgray} \textbf{ACDE} &&=&& \color{lightgray}\textbf{BCE} \\
		\textbf{E} &&=&& \color{lightgray} \textbf{ABDE} &&=&& \textbf{AC} &&=&&\color{lightgray} \textbf{BCD}\\
	\end{align*}
	
\end{frame}

\begin{frame}\frametitle{How do we use this aliasing information we obtain?}
	
	\vspace{1cm}
	Aliases for the 5 main effects (dropping out 3rd order and higher interactions):
	
	\vspace{0.5cm}
	\begin{align*}
		\textbf{A} &&=&& \textbf{BD} &&=&& \textbf{CE} \\
		\textbf{B} &&=&& \textbf{AD} &&&& &&&& \\
		\textbf{C} &&=&& \textbf{AE} &&&& &&&& \\
		\textbf{D} &&=&& \textbf{AB} &&&& &&&& \\
		\textbf{E} &&=&& \textbf{AC} &&&& &&&& \\
		\color{white}  \textbf{E} &&\color{white}=&& \color{white} \textbf{ABCD}	 &&\color{white}=&& \color{white} \textbf{ABDE}	&&\color{white}=&& \color{white} \textbf{ABCDE}
	\end{align*}
		
	
\end{frame}

\begin{frame}\frametitle{Back to some more baking experiments}
	\begin{columns}[T]
		\column{0.45\textwidth}
		
			\vspace{1cm}
			{\small Recipe: \href{http://yint.org/honeycomb-cake}{http://yint.org/honeycomb-cake}}
		
			\vspace{1cm}
		
			\centerline{\includegraphics[height=.5\textwidth]{../4G/Supporting material/flickr-5268081618_97f2626f2a_b-baking-experiment.jpg}}
			
			 \see{\href{https://secure.flickr.com/photos/andrea_nguyen/5268081724}{Flickr: andrea\_nguyen}}
		
		\column{0.48\textwidth}
			Potential factors to consider are
			\begin{itemize}
				\item	stirring speed
				\item	type of coconut milk
				\item	baking soda
				\item	amount of wheat starch
				\item	baking temperature
			\end{itemize}
			
			\vspace{0.4cm}
			\hrule
			
			\vspace{0.1cm}
			Aliasing pattern for a $2^{5-2}_{\textrm{III}}$ design:
			\begin{tabulary}{\linewidth}{ccccccc}				
				\textbf{A} & = & \textbf{BD} & = & \textbf{CE}  \\
				\textbf{B} & = & \textbf{AD} & & \\
				\textbf{C} & = & \textbf{AE} & & \\
				\textbf{D} & = & \textbf{AB} & & \\
				\textbf{E} & = & \textbf{AC} & & 
			\end{tabulary}
			
	\end{columns}
	
	\vspace{1cm}

	
\end{frame}

\begin{frame}\frametitle{Back to some more baking experiments}
	\begin{columns}[T]
		\column{0.45\textwidth}
		
			\vspace{1cm}
			{\small Recipe: \href{http://yint.org/honeycomb-cake}{http://yint.org/honeycomb-cake}}
		
			\vspace{1cm}
		
			\centerline{\includegraphics[height=.5\textwidth]{../4G/Supporting material/flickr-5268081618_97f2626f2a_b-baking-experiment.jpg}}
			
			 \see{\href{https://secure.flickr.com/photos/andrea_nguyen/5268081724}{Flickr: andrea\_nguyen}}
		
		\column{0.48\textwidth}
			{\color{myOrange}So my factor assignment so far is:}
			\begin{itemize}
				\item	stirring speed
				\item	type of coconut milk
				\item	baking soda
				\item	amount of wheat starch
				\item	\textbf{A} = baking temperature
			\end{itemize}
			
			\vspace{0.4cm}
			\hrule
			
			\vspace{0.1cm}
			Aliasing pattern for a $2^{5-2}_{\textrm{III}}$ design:
			\begin{tabulary}{\linewidth}{ccccccc}				
				\textbf{A} & = & \textbf{BD} & = & \textbf{CE}  \\
				\textbf{B} & = & \textbf{AD} & & \\
				\textbf{C} & = & \textbf{AE} & & \\
				\textbf{D} & = & \textbf{AB} & & \\
				\textbf{E} & = & \textbf{AC} & & 
			\end{tabulary}
			
	\end{columns}
	
	\vspace{1cm}

	
\end{frame}

\begin{frame}\frametitle{Back to some more baking experiments}
	\begin{columns}[T]
		\column{0.45\textwidth}
		
			\vspace{1cm}
			{\small Recipe: \href{http://yint.org/honeycomb-cake}{http://yint.org/honeycomb-cake}}
		
			\vspace{1cm}
		
			\centerline{\includegraphics[height=.5\textwidth]{../4G/Supporting material/flickr-5268081618_97f2626f2a_b-baking-experiment.jpg}}
			
			 \see{\href{https://secure.flickr.com/photos/andrea_nguyen/5268081724}{Flickr: andrea\_nguyen}}
		
		\column{0.48\textwidth}
			{\color{myOrange}So my factor assignment so far is:}
			\begin{itemize}
				\item	\textbf{B} = stirring speed {\tiny (to avoid an AB interaction)}
				\item	type of coconut milk
				\item	baking soda
				\item	amount of wheat starch
				\item	\textbf{A} = baking temperature
			\end{itemize}
			
			\vspace{0.4cm}
			\hrule
			
			\vspace{0.1cm}
			Aliasing pattern for a $2^{5-2}_{\textrm{III}}$ design:
			\begin{tabulary}{\linewidth}{ccccccc}				
				\textbf{A} & = & \textbf{BD} & = & \textbf{CE}  \\
				\textbf{B} & = & \textbf{AD} &  & \\
				\textbf{C} & = & \textbf{AE} &  & \\
				\textbf{D} & = & \textbf{AB} &  & \\
				\textbf{E} & = & \textbf{AC} &  & 
			\end{tabulary}
			
	\end{columns}
	
	\vspace{1cm}

	
\end{frame}

\begin{frame}\frametitle{Back to some more baking experiments}
	\begin{columns}[T]
		\column{0.45\textwidth}
		
			\vspace{1cm}
			{\small Recipe: \href{http://yint.org/honeycomb-cake}{http://yint.org/honeycomb-cake}}
		
			\vspace{1cm}
		
			\centerline{\includegraphics[height=.5\textwidth]{../4G/Supporting material/flickr-5268081618_97f2626f2a_b-baking-experiment.jpg}}
			
			 \see{\href{https://secure.flickr.com/photos/andrea_nguyen/5268081724}{Flickr: andrea\_nguyen}}
		
		\column{0.48\textwidth}
			{\color{myOrange}So my factor assignment so far is:}
			\begin{itemize}
				\item	\textbf{B} = stirring speed {\tiny (to avoid an AB interaction)}
				\item	type of coconut milk
				\item	\textbf{D} = baking soda {\tiny (we expect this to be sensitive)}
				\item	amount of wheat starch
				\item	\textbf{A} = baking temperature
			\end{itemize}
			
			\vspace{0.4cm}
			\hrule
			
			\vspace{0.1cm}
			Aliasing pattern for a $2^{5-2}_{\textrm{III}}$ design:
			\begin{tabulary}{\linewidth}{ccccccc}				
				\textbf{A} & = & \textbf{BD} & = & \textbf{CE}  \\
				\textbf{B} & = & \textbf{AD} &  \\
				\textbf{C} & = & \textbf{AE} &  \\
				\textbf{D} & = & \multicolumn{5}{l}{$\cancelto{_0}{\textbf{AB}}$ {\tiny (to get an unbiased estimate)}}   \\
				\textbf{E} & = & \textbf{AC} & & 
			\end{tabulary}
			
	\end{columns}
	
	\vspace{1cm}
		
\end{frame}

\begin{frame}\frametitle{How do we use this aliasing information we obtain?}
	
	\vspace{1cm}
	The aliases for the 5 main effects in this a $2^{5-2}_{\textrm{III}}$ design:
	
	\vspace{0.5cm}
	\begin{align*}
		\textbf{A} &&=&& \textbf{BD} &&=&& \textbf{CE} &&=&&\color{lightgray} \textbf{ABCDE}\\
		\textbf{B} &&=&& \textbf{AD} &&=&& \color{lightgray} \textbf{CDE} &&=&& {\color{lightgray}\textbf{ABCE}}\\
		\textbf{C} &&=&& \textbf{AE} &&=&& \color{lightgray} \textbf{BDE} &&=&& \color{lightgray}\textbf{ABCD}\\
		\textbf{D} &&=&& \textbf{AB} &&=&& \color{lightgray}\textbf{BCE}  &&=&& \color{lightgray} \textbf{ACDE}\\
		\textbf{E} &&=&& \textbf{AC} &&=&& \color{lightgray} \textbf{BCD} &&=&& \color{lightgray} \textbf{ABDE}\\
	\end{align*}
	
\end{frame}

\begin{frame}\frametitle{Is is possible to get a more favourable fractional factorial? }
	
	\begin{exampleblock}{}
		In other words, can we get better confounding?
	\end{exampleblock}
	
	
	
	\begin{itemize}
		\item	In the 5-factor example:\\
			\qquad\qquad \emph{main effects are confounded with two factor interactions}
		
	\end{itemize}
	\vspace{1cm}
	
			Is it possible to achieve a design where main effects are confounded with 3-factor interactions?
			
	\vspace{1cm}
	
		\pause
		These are called resolution \textrm{IV} designs.
\end{frame}

\begin{frame}\frametitle{Example to try yourself: find the defining relationship}
	
	\vspace{0.5cm}
	You'd like to investigate 6 factors, and have a budget for 15 to 20 experiments.
	
	\vspace{0.5cm}
	\begin{columns}[T]
		\column{0.7\textwidth}
			\begin{enumerate}
				\item	Read the generators from the table 
				\item	Rearrange the generators as  $\textbf{I = \ldots}$
			 	\item	Form the {\color{purple}defining relationship} taking all combinations of the words, so that $\textbf{I = \ldots}$
			 	\item	Ensure the defining relationship has $2^p$ words
				\item	Use this defining relationship to compute the aliasing pattern
			\end{enumerate}
			
		\column{0.3\textwidth}

	\end{columns}

	
\end{frame}

\begin{frame}\frametitle{Example for practice: {\color{myOrange}solution} for finding the defining relationship}
	
	\vspace{0.5cm}
	You'd like to investigate 6 factors, and have a budget for 15 to 20 experiments.
	
	\vspace{0.5cm}
	\begin{columns}[T]
		\column{0.85\textwidth}
			\begin{enumerate}
				\item	Read the generators from the table for $k=6$, and 16 experiments
					\begin{itemize}
						\item	\textbf{E = ABC}	and  \textbf{F = BCD}
					\end{itemize}
				\item	Rearrange the generators as  $\textbf{I = \ldots}$
					\begin{itemize}
						\item	\textbf{I = ABCE}	and  \textbf{I = BCDF}
					\end{itemize}
			 	\item	Form the {\color{purple}defining relationship} taking all combinations of the words, so that $\textbf{I = \ldots}$
					\begin{itemize}
						\item	\textbf{I = ABCE = BCDF = A(BB)(CC)DEF = ADEF}
					\end{itemize}
			 	\item	Ensure the defining relationship has $2^p$ words
					\begin{itemize}
						\item	It does; since $p=2$ in this case
					\end{itemize} 
				\item	We will use this defining relationship now
			\end{enumerate}
			
		\column{0.3\textwidth}
			\centerline{\includegraphics[height=.7\textwidth]{../4F/Supporting material/ivq-final-solution.png}}

	\end{columns}

	
\end{frame}

\begin{frame}\frametitle{The 6-factor example in 16 experiments: a resolution \textrm{IV} design}

	
	\begin{exampleblock}{}
		\color{myGreen}By example: what are the aliases (confounding) with factor \textbf{\color{myOrange} A}?
	\end{exampleblock}
	
	
	\begin{align*}
		\intertext{The defining relationship is}
		\textbf{I} &&=&& \textbf{ABCE} &&=&& \textbf{BCDF} &&=&& \textbf{ADEF}\\
		\onslide+<2->{
			\intertext{Now multiply all words by \textbf{\color{myOrange} A}:}
			\textbf{\color{myOrange} A} &&=&& \textbf{BCE} &&=&& \textbf{ABCDF} &&=&& \textbf{DEF}
		}
	\end{align*}
	
	\onslide+<3->{
		\vspace{0.5cm}
		There are only 3-factor and higher level interactions\emph{!}
	}
	
\end{frame}

\begin{frame}\frametitle{The 6-factor example in 16 experiments: a resolution \textrm{IV} design}
	\begin{exampleblock}{}
		\color{myGreen}What are the aliases (confounding) with a two-factor interaction term: \textbf{\color{myOrange} CD}?
	\end{exampleblock}
	
	\begin{align*}
		\intertext{The defining relationship is}
		\textbf{I} &&=&& \textbf{ABCE} &&=&& \textbf{BCDF} &&=&& \textbf{ADEF}\\
		\onslide+<2->{
			\intertext{Now multiply all words by \textbf{\color{myOrange} CD}:}
			\textbf{\color{myOrange} CD} &&=&& \textbf{ABDE} &&=&& \textbf{BF} &&=&& \textbf{ACEF}
		}
	\end{align*}
	
	\onslide+<3->{
		\vspace{0.5cm}
		Two factor interactions are aliased with other two factor interactions.
	}

\end{frame}

\begin{frame}\frametitle{}
	%\vspace{-10pt}
	\centerline{\includegraphics[width=1\textwidth]{\imagedir/doe/DOE-trade-off-table.png}}
\end{frame}

\begin{frame}\frametitle{What the design's resolution tells us}

	\textbf{Resolution \textrm{III} designs}
	\begin{itemize}
		\item	Excellent for an initial screening
			\begin{itemize}
				\item	e.g. developing a new product 
				\item	e.g. troubleshooting a process, such as moving production from one location to another, but struggling to get a similar product on the two different machines
			\end{itemize}
	\end{itemize}

	\pause
	\vspace{0.5cm}
	\textbf{Resolution \textrm{IV} designs}
	\begin{itemize}
		\item	Used for learning about, and understanding, a system (characterization)
	\end{itemize}
	
	\pause
	\vspace{0.5cm}
	\textbf{Resolution \textrm{V} designs and higher, and full factorial designs}
	\begin{itemize}
		\item	Used for optimizing a process, understanding complex effects
		\item	To develop high-accuracy models
		\item	You will need to justify the budget for this carefully: expensive\emph{!}
	\end{itemize}
\end{frame}

\begin{frame}\frametitle{}
	\vspace{0.5cm}
	\centerline{\includegraphics[height=1.1\textheight]{../4G/Supporting material/course-textbook-fractional-factorial-example.png}}
		
\end{frame}

\begin{frame}\frametitle{}
	\vspace{0.5cm}
	\centerline{\includegraphics[height=1.1\textheight]{../4G/Supporting material/course-textbook-fractional-factorial-example-continued.png}}
		
\end{frame}

\begin{frame}\frametitle{General approach for figuring out the aliasing \emph{before starting} the experiments}
	
	\vspace{0.5cm}
	Define the number of factors to investigate; and determine what your budget is.
	
	\vspace{0.5cm}
	\begin{columns}[T]
		\column{0.9\textwidth}
			\begin{enumerate}
				\item	Read the generators from the trade off table 
				\item	Rearrange the generators as  $\textbf{I = \ldots}$
			 	\item	Form the {\color{purple}defining relationship} taking all combinations of the words, so that $\textbf{I = \ldots}$
			 	\item	Ensure the defining relationship has $2^p$ words
				\item	Compute the aliasing pattern
				\item	{\color{myGreen}Is the aliasing problematic?}
				
					\hbox{\hspace{3.5em}If \textbf{yes}: reassign factor letters; or pick another design (start over)}
					
					\hbox{\hspace{3.5em}If \textbf{no}: you are ready to start your experiments}
			\end{enumerate}
			
		\column{0.1\textwidth}
			\onslide+<2->{
				\centerline{\includegraphics[width=2\textwidth]{../4G/Supporting material/flickr-9568156463_1809c97b21_o-checklist}}
			
				\see{\href{https://secure.flickr.com/photos/ajc1/9568156463}{Flickr: ajc1}}
			}
	\end{columns}

	
\end{frame}

\begin{frame}\frametitle{}
	%\vspace{-10pt}
	\centerline{\includegraphics[width=1\textwidth]{\imagedir/doe/DOE-trade-off-table.png}}
\end{frame}

\begin{frame}\frametitle{\includegraphics[width=0.3\textwidth]{\imagedir/doe/examples/advice-logo.png}\,\, pick a design that meets the objective}
	
	\begin{columns}[T]
		\column{0.7\textwidth}
		
			\begin{itemize}
				\item	If you are just starting out, avoid eliminating factors to simply get a full factorial.
				\item	Use the experimental evidence to eliminate factors.
			\end{itemize}
			
			
			\vspace{1cm}
			\onslide+<2->{
				\begin{itemize}
					\item	Remember: these are experimental building blocks. The experiments you run first can be extended on later.
					\vspace{0.5cm}
					\onslide+<3->{
						\item	In the next example, we show how factors are eliminated, {\color{myOrange}\emph{based on evidence}}.
					}
				\end{itemize}
			}
	
			\vspace{1cm}
			
		\column{0.3\textwidth}
			\onslide+<2->{
				\centerline{\includegraphics[width=\textwidth]{../4G/Supporting material/flickr-6479064129_25ce3bb07f_o-building-block.png}}
		
				\see{\href{https://secure.flickr.com/photos/rahego/6479064129}{Flickr: rahego}}
			}
	\end{columns}
\end{frame}
