\begin{frame}\frametitle{Some important terminology we will use all the time}
	
	\begin{columns}[T]
		\column{0.75\textwidth}
			\textbf{{\color{purple} Outcome}}
				\begin{itemize}
					\item	What we measure after the experiment is finished  \pause
					\item	It is the aspect you are interested in improving.\pause
			 
				\end{itemize}
		\column{0.3\textwidth}
		
			\centerline{\includegraphics[width=\textwidth]{\imagedir/doe/measure-4904403417_93baa750a6-flickr.jpg}}

	\end{columns}
	
			
	\vspace{24pt}
	\textbf{{\color{purple} Factors}}
		\begin{itemize}
			\item	Things which you actively change to influence the outcome.
			\item	We typically change 2, 3, 4, or many more factors. \pause
			\item	Don't fixate on changing 1 factor at a time.
		\end{itemize}
\end{frame}

\begin{frame}\frametitle{}
	
	\begin{columns}[c]
		\column{0.5\textwidth}
			\centerline{\includegraphics[width=0.8\textwidth]{../1B/Supporting-material/plant.png}}
		
		\column{0.5\textwidth}
			Various outcomes are possible in your experiment:
			
			\begin{itemize}
				\item	height of the plant
				\item	average length of leaves
				\item	the number of flowers
			\end{itemize}
			
			\vspace{12pt}
			These are examples of numeric measurements (quantitative).

	\end{columns}
\end{frame}

\begin{frame}\frametitle{}
	
	\begin{columns}[c]
		\column{0.5\textwidth}
			\centerline{\includegraphics[width=0.8\textwidth]{../1B/Supporting-material/plant.png}}
		
		\column{0.5\textwidth}
			%Different outcomes are possible:
			
			\begin{itemize}
				\item	colour of the flower
			\end{itemize}
			
			\vspace{12pt}
			This is a qualitative measurement.
			
			(We use qualitative outcomes infrequently)

	\end{columns}
\end{frame}

\begin{frame}\frametitle{}
	
	\textbf{{\color{purple} Outcome}} \onslide+<5->{ = \textbf{{\color{purple} Response}}}
		\begin{itemize}
			\item	What we measure after the experiment is finished  
			\item	It is the aspect you are interested in improving.
		\end{itemize}
	
	\vspace{12pt}
	\pause
	
	\textbf{{\color{purple} Objective}}
	
	
		\qquad combine the {\color{purple} outcome} with ``a desire to \emph{adjust} the outcome''
	\vspace{24pt}
	\pause
	
	{\textbf{{Various examples of ``objectives''}}}
	
		\begin{itemize}
			\item	maximize $(\uparrow)$ the profit
			\item	maximize $(\uparrow)$ the height of the plant
			\item	minimize $(\downarrow)$ pollution
			\item	minimize $(\downarrow)$ energy used to produce a product
		\end{itemize} 	
		
	\pause
		\vspace{12pt}
		But sometimes the objective is ``the same as before'' $(=)$
\end{frame}

\begin{frame}\frametitle{}
	
	\textbf{{\color{purple} Factors}} \onslide+<6->{ = \textbf{{\color{purple} Variables}}}
	\pause
		
		\begin{columns}[T]
			\column{0.33\textwidth}
				\onslide+<2->{\centerline{\includegraphics[width=\textwidth]{../1B/Supporting-material/water.png}}}
				
				
			\column{0.33\textwidth}
				\onslide+<3->{\centerline{\includegraphics[width=\textwidth]{../1B/Supporting-material/fertilizer.png}}}
				
				
			\column{0.33\textwidth}
				\onslide+<4->{\centerline{\includegraphics[width=\textwidth]{../1B/Supporting-material/soil.png}}}
			
		\end{columns}
		
		
	\vspace{24pt}
	
	\onslide+<5->{
		\textbf{{\color{purple} Types of factors}}
	
	
		\qquad \emph{numeric} factors (quantitative) can be measured and adjusted to different levels
		
			\qquad \qquad 
		\vspace{12pt}
	
		\qquad \emph{categorical} factors (qualitative) are simply selected for their characteristic
	}
	
	
\end{frame}

{\usebackgroundtemplate{\vbox to \paperheight{\vfil\hbox to \paperwidth{\hfil  
    \includegraphics[width=0.95\paperwidth, clip]
	{../1B/Slides/01Screen Shot 2015-08-06 at 22.40.09 .png}  \hfil}\vfil}}
\begin{frame}\frametitle{}
\end{frame}}

{\usebackgroundtemplate{\vbox to \paperheight{\vfil\hbox to \paperwidth{\hfil  
    \includegraphics[width=0.95\paperwidth, clip]
	{../1B/Slides/02Screen Shot 2015-08-06 at 22.40.32 .png}  \hfil}\vfil}}
\begin{frame}\frametitle{}
\end{frame}}

{\usebackgroundtemplate{\vbox to \paperheight{\vfil\hbox to \paperwidth{\hfil  
    \includegraphics[width=0.95\paperwidth, clip]
	{../1B/Slides/03Screen Shot 2015-08-06 at 22.40.40 .png}  \hfil}\vfil}}
\begin{frame}\frametitle{}
\end{frame}}


