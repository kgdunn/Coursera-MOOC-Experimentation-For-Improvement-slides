\begin{frame}\frametitle{\includegraphics[width=0.3\textwidth]{\imagedir/doe/examples/advice-logo.png}\,\, pick a design that meets the objective}
	
	\begin{columns}[T]
		\column{0.7\textwidth}
		
			\begin{itemize}
				\item	If you are just starting out, avoid eliminating factors to simply get a full factorial.
				\item	Use the experimental evidence to eliminate factors.
			\end{itemize}
			
			
			\vspace{1cm}
			\onslide+<2->{
				\begin{itemize}
					\item	Remember: these are experimental building blocks. The experiments you run first can be extended on later.
					\vspace{0.5cm}
					\onslide+<3->{
						\item	In the next example, we show how factors are eliminated, {\color{myOrange}\emph{based on evidence}}.
					}
				\end{itemize}
			}
	
			\vspace{1cm}
			
		\column{0.3\textwidth}
			\onslide+<2->{
				\centerline{\includegraphics[width=\textwidth]{../4G/Supporting material/flickr-6479064129_25ce3bb07f_o-building-block.png}}
		
				\see{\href{https://secure.flickr.com/photos/rahego/6479064129}{Flickr: rahego}}
			}
	\end{columns}
\end{frame}

\begin{frame}\frametitle{An example to demonstrate a saturated fractional factorial analyis}
	We have 7 factors: \textbf{A}, \textbf{B}, \textbf{C}, \textbf{D}, \textbf{E}, \textbf{F}, and \textbf{G}.
	
	\vspace{1cm}
	The fewest number of experiments is: 8 runs
	\vfill
\end{frame}

\begin{frame}\frametitle{Creating the standard order table for the fractional factorial}
	\vspace{0.4cm}
	\begin{tabulary}{\linewidth}{c|ccccccc}
		\textbf{\relax Experiment} & \textbf{\relax A } & \textbf{\relax B} & \textbf{\relax C } & \onslide+<3->{\textbf{\relax D = AB}} & \onslide+<3->{\textbf{\relax E = AC}} & \onslide+<3->{\textbf{\relax F = BC}} & \onslide+<3->{\textbf{\relax G = ABC}} \\ \cline{1-8}
		\textbf{1} & \(-\) & \(-\) & \(-\) & \onslide+<3->{\(+\)} & \onslide+<3->{\(+\)} & \onslide+<3->{\(+\)} & \onslide+<3->{\(-\)} \\
		\textbf{2} & \(+\) & \(-\) & \(-\) & \onslide+<3->{\(-\)} & \onslide+<3->{\(-\)} & \onslide+<3->{\(+\)} & \onslide+<3->{\(+\)} \\
		\textbf{3} & \(-\) & \(+\) & \(-\) & \onslide+<3->{\(-\)} & \onslide+<3->{\(+\)} & \onslide+<3->{\(-\)} & \onslide+<3->{\(+\)} \\
		\textbf{4} & \(+\) & \(+\) & \(-\) & \onslide+<3->{\(+\)} & \onslide+<3->{\(-\)} & \onslide+<3->{\(-\)} & \onslide+<3->{\(-\)} \\
		\textbf{5} & \(-\) & \(-\) & \(+\) & \onslide+<3->{\(+\)} & \onslide+<3->{\(-\)} & \onslide+<3->{\(-\)} & \onslide+<3->{\(+\)} \\
		\textbf{6} & \(+\) & \(-\) & \(+\) & \onslide+<3->{\(-\)} & \onslide+<3->{\(+\)} & \onslide+<3->{\(-\)} & \onslide+<3->{\(-\)} \\
		\textbf{7} & \(-\) & \(+\) & \(+\) & \onslide+<3->{\(-\)} & \onslide+<3->{\(-\)} & \onslide+<3->{\(+\)} & \onslide+<3->{\(-\)} \\
		\textbf{8} & \(+\) & \(+\) & \(+\) & \onslide+<3->{\(+\)} & \onslide+<3->{\(+\)} & \onslide+<3->{\(+\)} & \onslide+<3->{\(+\)} \\  
	\end{tabulary}
	
	\vspace{0.4cm}
	
	
	\begin{columns}[T]
		
		\column{0.45\textwidth}
			\centerline{\includegraphics[width=.5\textwidth]{../4G/Supporting material/trade-off-design-example-saturated.png}}
		
		\column{0.48\textwidth}
			\onslide+<2->{
				{\huge $2^{k-p}$\\
				{\normalsize with} $p=4$}
			}
	\end{columns}	
\end{frame}

\begin{frame}\frametitle{Creating the standard order table for the fractional factorial}
	\vspace{0.4cm}
	\begin{tabulary}{\linewidth}{c|ccccccc|c}
		\textbf{\relax Experiment} & \textbf{\relax A } & \textbf{\relax B} & \textbf{\relax C } & \onslide+<1->{\textbf{\relax D = AB}} & \onslide+<1->{\textbf{\relax E = AC}} & \onslide+<1->{\textbf{\relax F = BC}} & \onslide+<1->{\textbf{\relax G = ABC}} & \onslide+<1->{$y$}\\ \cline{1-9}
		\textbf{1} & \(-\) & \(-\) & \(-\) & \onslide+<1->{\(+\)} & \onslide+<1->{\(+\)} & \onslide+<1->{\(+\)} & \onslide+<1->{\(-\)} \\
		\textbf{2} & \(+\) & \(-\) & \(-\) & \onslide+<1->{\(-\)} & \onslide+<1->{\(-\)} & \onslide+<1->{\(+\)} & \onslide+<1->{\(+\)} \\
		\textbf{3} & \(-\) & \(+\) & \(-\) & \onslide+<1->{\(-\)} & \onslide+<1->{\(+\)} & \onslide+<1->{\(-\)} & \onslide+<1->{\(+\)} \\
		\textbf{4} & \(+\) & \(+\) & \(-\) & \onslide+<1->{\(+\)} & \onslide+<1->{\(-\)} & \onslide+<1->{\(-\)} & \onslide+<1->{\(-\)} \\
		\textbf{5} & \(-\) & \(-\) & \(+\) & \onslide+<1->{\(+\)} & \onslide+<1->{\(-\)} & \onslide+<1->{\(-\)} & \onslide+<1->{\(+\)} \\
		\textbf{6} & \(+\) & \(-\) & \(+\) & \onslide+<1->{\(-\)} & \onslide+<1->{\(+\)} & \onslide+<1->{\(-\)} & \onslide+<1->{\(-\)} \\
		\textbf{7} & \(-\) & \(+\) & \(+\) & \onslide+<1->{\(-\)} & \onslide+<1->{\(-\)} & \onslide+<1->{\(+\)} & \onslide+<1->{\(-\)} \\
		\textbf{8} & \(+\) & \(+\) & \(+\) & \onslide+<1->{\(+\)} & \onslide+<1->{\(+\)} & \onslide+<1->{\(+\)} & \onslide+<1->{\(+\)} \\ 
	\end{tabulary}
	
	\vspace{0.4cm}
	
	
	\begin{columns}[T]
		
		\column{0.45\textwidth}
			\centerline{\includegraphics[width=.5\textwidth]{../4G/Supporting material/trade-off-design-example-saturated.png}}
		
		\column{0.48\textwidth}
			\onslide+<1->{
				{\huge $2^{k-p}$\\
				{\normalsize with} $p=4$}
			}
	\end{columns}
\end{frame}

\begin{frame}\frametitle{\includegraphics[width=.07\textwidth]{../4G/Supporting material/flickr-9568156463_1809c97b21_o-checklist.jpg} Let's follow the recommended approach shown earlier in the video}
	
	\begin{enumerate}
		\item	Read the generators from the trade off table 
			\begin{itemize}
				\item		$\textbf{D = AB}$  and $\textbf{E = AC}$ and $\textbf{F = BC}$  and $\textbf{G = ABC}$ 
			\end{itemize}
\onslide+<2->{
		\item	Rearrange the generators as  $\textbf{I = \ldots}$
			\begin{itemize}
				\item	$\textbf{I = ABD}$ and $\textbf{I = ACE}$ and $\textbf{I = BCF}$ and $\textbf{I = ABCG}$
			\end{itemize}
}
\onslide+<3->{
		\item	Form the {\color{purple}\textbf{defining relationship}} taking all combinations of the words: $\textbf{I = \ldots}$
}
\onslide+<4->{
	   	 \fbox{\parbox[b][4.5em][t]{0.92\textwidth}{
			\textbf{I = ABD = ACE = BCF = ABCG = $\ldots$} {\small \color{myOrange} \hfill $\longleftarrow$ that's 5 so far} \\
\onslide+<5->{
			\,\,\textbf{BDCE}}\onslide+<6->{\textbf{ = ACDF}}\onslide+<7->{\textbf{ = CDG = ABEF = BEG = AFG = $\ldots$} {\small \color{myOrange} \hfill $\longleftarrow$ that's 11} \\
}
\onslide+<8->{
			\textbf{DEF}} \onslide+<9->{\textbf{= ADEG = CEFG = BDFG = $\ldots$}{\small \color{myOrange} \hfill $\longleftarrow$ we are up to 15} \\
}
\onslide+<10->{
			\textbf{ABCDEFG}{\small \color{myOrange} \hfill $\longleftarrow$ we have all 16 here} \\
}
		 	
	   	 } }
}
\onslide+<5-5>{\textbf{(ABD)(ACE) = BCDE}}\,\onslide+<6-6>{\textbf{(ABD)(BCF) = ACDF}}\, \onslide+<8-8>{\textbf{(ABD)(ACE)(BCF) = DEF}}
\onslide+<3->{
		\item	Ensure the defining relationship has $2^p$ words
			\begin{itemize}
				\item	$p=4$, so we have 16 words.
			\end{itemize}
}
\onslide+<11->{
		\item	Use the defining relationship to compute the aliasing pattern
		\item	Ensure the aliasing is acceptable
}		
	\end{enumerate}
\end{frame}

\begin{frame}\frametitle{Check the alias patterns by using the defining relationship}
	
	
	\begin{align*}
		\textbf{A} &= \textbf{BD = CE = FG \color{lightgray} = BCG = CDF = BEF = DEG =  ABCF = ABEG = $\ldots$} \\		
		&\qquad\qquad\qquad\color{lightgray} \textbf{ = ACDG = ADEF = ABDCE = ABDFG = ACEFG = BCDEFG}\\
\onslide+<2->{
		\textbf{B} &= \textbf{AD = CF = EG} \color{lightgray} \,+\, \text{other higher order interactions} \\
		\textbf{C} &= \textbf{AE = BF = DG}\\
		\textbf{D} &= \textbf{AB = CG = EF}\\
		\textbf{E} &= \textbf{AC = BG = DF}\\
		\textbf{F} &= \textbf{BC = AG = DE}\\
		\textbf{G} &= \textbf{CD = BE = AF}
}
	\end{align*}
\end{frame}

\begin{frame}\frametitle{Finally! You get to do the experiments and record the outcome value}
	\vspace{0.4cm}
	\begin{tabulary}{\linewidth}{c|ccccccc|c}\hline
		\textbf{\relax Experiment} & \textbf{\relax A } & \textbf{\relax B} & \textbf{\relax C } & \textbf{\relax D = AB} & \textbf{\relax E = AC} & \textbf{\relax F = BC} & \textbf{\relax G = ABC} & $y$\\ \hline
		\textbf{1} & \(-\) & \(-\) & \(-\) & \(+\) & \(+\) & \(+\) & \(-\) & 320\\
		\textbf{2} & \(+\) & \(-\) & \(-\) & \(-\) & \(-\) & \(+\) & \(+\) & 276\\
		\textbf{3} & \(-\) & \(+\) & \(-\) & \(-\) & \(+\) & \(-\) & \(+\) & 306\\
		\textbf{4} & \(+\) & \(+\) & \(-\) & \(+\) & \(-\) & \(-\) & \(-\) & 290\\
		\textbf{5} & \(-\) & \(-\) & \(+\) & \(+\) & \(-\) & \(-\) & \(+\) & 272\\
		\textbf{6} & \(+\) & \(-\) & \(+\) & \(-\) & \(+\) & \(-\) & \(-\) & 274\\
		\textbf{7} & \(-\) & \(+\) & \(+\) & \(-\) & \(-\) & \(+\) & \(-\) & 290\\
		\textbf{8} & \(+\) & \(+\) & \(+\) & \(+\) & \(+\) & \(+\) & \(+\) & 255\\ \hline
	\end{tabulary}
\end{frame}

\begin{frame}\frametitle{The reduced model: only has four main effect factors}
	\renewcommand{\lg}{\color{lightgray}}
	
	\vspace{0.4cm}
	\begin{tabulary}{\linewidth}{c|ccccccc|c}\hline
		\textbf{\relax Experiment} & \textbf{\relax A } & \textbf{\relax C} & \textbf{\relax E } & \textbf{\relax G} & \lg \textbf{\relax B} & \lg \textbf{\relax D} & \lg \textbf{\relax F} & $y$\\ \hline
		\textbf{1} & \(-\) & \(-\) & \(+\) & \(-\) & \lg \(-\) & \lg  \(+\) & \lg  \(+\) & 320\\
		\textbf{2} & \(+\) & \(-\) & \(-\) & \(+\) & \lg \(-\) & \lg  \(-\) & \lg  \(+\) & 276\\
		\textbf{3} & \(-\) & \(-\) & \(+\) & \(+\) & \lg \(+\) & \lg  \(-\) & \lg  \(-\) & 306\\
		\textbf{4} & \(+\) & \(-\) & \(-\) & \(-\) & \lg \(+\) & \lg  \(+\) & \lg  \(-\) & 290\\
		\textbf{5} & \(-\) & \(+\) & \(-\) & \(+\) & \lg \(-\) & \lg  \(+\) & \lg  \(-\) & 272\\
		\textbf{6} & \(+\) & \(+\) & \(+\) & \(-\) & \lg \(-\) & \lg  \(-\) & \lg  \(-\) & 274\\
		\textbf{7} & \(-\) & \(+\) & \(-\) & \(-\) & \lg \(+\) & \lg  \(-\) & \lg  \(+\) & 290\\
		\textbf{8} & \(+\) & \(+\) & \(+\) & \(+\) & \lg \(+\) & \lg  \(+\) & \lg  \(+\) & 255\\ \hline
	\end{tabulary}
	
	\vspace{1cm}
	Those that are more math oriented: please verify that each column is uncorrelated with the others. So rebuilding the model implies the factor estimates are the same.
\end{frame}

\begin{frame}\frametitle{Other fractional factorial designs: Plackett-Burman designs}
 	\begin{columns}[T]
 		
 		\column{0.55\textwidth}
			
			Plackett-Burman designs exists in \\
			multiples of 4:
			
				\begin{itemize}
					\item	4, 8, 12, 16, 20, 24, 28, 32, \\
							36, 40, $\ldots$
					
					\item	Main effects are confounded \\
							with two-factor interactions,\\
							but in a complicated way.
							
					\item	Such designs are most usefully \\
							created by software, unlike the\\
							fractional factorial designs \\
							shown in this section.
							
					\item	e.g. a Placket-Burman design\\
							in 20 runs, can screen for\\
							19 factors\emph{!}
							
				\end{itemize}
 		\column{0.5\textwidth}
			\hbox{\hspace{-5.5em}
				\includegraphics[height=.9\textheight]{\imagedir/doe/DOE-trade-off-table-MOOC-resolution-plain-cropped.png}
			}
					
 	\end{columns}
	 
	 
	
\end{frame}

\begin{frame}\frametitle{Definitive Screening Designs: a type of optimal design}
	\begin{columns}[T]
		\column{0.6\textwidth}
			\textbf{Optimal designs}
			
			\begin{itemize}
				\item	There are several desirable mathematical criteria that can be optimal.
				\item	We won't go into the details, but factorial designs often meet these optimal criteria.
				
				\item	Interested in the details? Search for:
					\begin{itemize}
						\item	\texttt{D-optimal designs}\\
							it is the most common optimal design
					\end{itemize}
			\end{itemize}
			
			\onslide+<2->{
			\textbf{Definitive screening designs}
				\begin{itemize}
					\item	Factors can be at 3 levels (not 2!)
					\item	Small number of runs
					\item	Main effects and 2-factor interactions \textbf{are not} aliased - a great advantage.
				\end{itemize}
			}
		\column{0.02\textwidth}
			\rule[3mm]{0.01cm}{90mm}
		\column{0.48\textwidth}
			
			\centerline{\includegraphics[width=.8\textwidth]{../4G/Supporting material/D-optimal-paper.png}}
			\see{\href{http://www.jstor.org/discover/10.2307/1267635}{http://www.jstor.org/discover/10.2307/1267635}}
		
			\onslide+<2->{
				\vspace{1cm}
				\centerline{\includegraphics[width=.8\textwidth]{../4H/Supporting material/definitive-screening-design-paper.png}}
				\see{\href{http://yint.org/dsdesign}{http://yint.org/dsdesign}}
			}
			
	\end{columns}	
\end{frame}

\begin{frame}\frametitle{\includegraphics[width=0.3\textwidth]{\imagedir/doe/examples/advice-logo.png}\,\, Practice, fail, start over, and persist}

	\begin{itemize}
		\item	There are case studies in the course textbook
		\item	There are other textbooks, listed on the course website
		\item	Create your own datasets
			\begin{itemize}
				\item	biscuits
				\item	coffee
				\item	growing plants, or
				\item	many of the experiments suggested in the course forums
				
			\end{itemize}
	\end{itemize}
\end{frame}



