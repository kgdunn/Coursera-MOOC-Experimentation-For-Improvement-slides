\begin{comment}
		
	\begin{frame}\frametitle{\includegraphics[width=0.3\textwidth]{\imagedir/doe/examples/advice-logo.png}\,\, plan your experiments carefully ahead of time}
	\end{frame}
	
	\begin{columns}[T]
		\column{0.45\textwidth}
			\includegraphics[width=0.7\textwidth]{\imagedir/statistics/flicfcb_o.jpg}
		
			{\scriptsize (p. 230 in Box, Hunter and Hunter, 2$^\text{nd}$ ed)}
		
		\column{0.48\textwidth}
			\includegraphics[width=\textwidth]{\imagedir/doe/examples/solar-panel-mendelu-cz-website.png}
		
		
			\see{\href{http://yint.org/solar-panel-study}{http://yint.org/solar-panel-study}}
	\end{columns}
	
	\begin{center}\rule[8mm]{4cm}{0.01cm}\end{center}
	\rule[3mm]{0.01cm}{25mm}%

\begin{frame}\frametitle{The variables affecting our experimental system}
	We may classify them in several ways:
	
	\begin{itemize}
		\item	those we know about, and those that are unknown
			\pause
		\item	variables we can control, and uncontrolled variables: \emph{we'll consider these today}
			\pause
		\item	some we can measure, and others we cannot measure: \emph{we'll consider these today}
		
	\end{itemize}	
\end{frame}

\begin{frame}\frametitle{Controllable and measurable variables and the need for randomization}
	Use a familiar example: {\color{myOrange} ginger biscuits}
	\vspace{0.25cm}
	\begin{columns}[T]
		\column{0.4\textwidth}
			We have 3 factors:
			\begin{enumerate}
				\item	\textbf{A}: baking temperature
				\item	\textbf{B}: amount of baking soda
				\item	\textbf{C}: baking time
			\end{enumerate}
			
			
			\vspace{1cm}
			
			\includegraphics[width=0.8\textwidth]{../4C/Supporting materials/4C-2-ginger-biscuits-3242839562_10c30d1aa3_o.jpg}
		
		
			\see{\href{https://secure.flickr.com/photos/babbagecabbage/3242839562}{Flickr: babbagecabbage}}
			
		
		\column{0.58\textwidth}
		
			\onslide+<2->{
			
				Order of experiments$^\ast$:
			
				\begin{tabulary}{\linewidth}{ccccc}\hline 
					\textbf{\relax Experiment } & \textbf{\relax A } & \textbf{\relax B } & \textbf{\relax C } & \textbf{\relax Time finished} \\
					\hline 
					\textbf{5} & \(-\) & \(-\) & \(+\) & 08:32\\
					\textbf{1} & \(-\) & \(-\) & \(-\) & 09:46\\
					\textbf{8} & \(+\) & \(+\) & \(+\) & 10:50\\
					\textbf{3} & \(-\) & \(+\) & \(-\) & 12:05\\
					\textbf{6} & \(+\) & \(-\) & \(+\) & 13:16\\
					\textbf{2} & \(+\) & \(-\) & \(-\) & 14:30\\
					\textbf{4} & \(+\) & \(+\) & \(-\) & 15:57\\
					\textbf{7} & \(-\) & \(+\) & \(+\) & 17:09\\
					
					 \hline
				\end{tabulary}
				
				\vspace{0.2cm}
				{\scriptsize $^\ast$ note the randomized order}
			}
	\end{columns}	
\end{frame}

\begin{frame}\frametitle{Disturbances are uncontrolled and unmeasured variables}
	
	\begin{columns}[T]
		\column{0.45\textwidth}
			Any potential impact on our system which
			\begin{itemize}
				\item	is not controlled, and
		
				\item	is not measured
			\end{itemize}
			
			\vspace{1cm}
			In the ginger biscuit example, it might include:
			\begin{itemize}
				\item	ambient humidity
				\item	ambient temperature
				\item	impurities in the ingredients
				\item	other examples ...
			\end{itemize}
		
		\column{0.48\textwidth}
			\includegraphics[width=\textwidth]{../4C/Supporting materials/4C-1-tiredness.jpg}
	\end{columns}
\end{frame}

\begin{frame}\frametitle{``Control'' and ``Measure'': let's clarify what we mean by those terms}
	
	{\color{myOrange}Given enough resources (usually money), we can control and measure most things.}
	
	\begin{columns}[T]
		\column{0.49\textwidth}
			\begin{center}\textbf{Control}\end{center}
				
			We can not actively control these:
			\begin{itemize}
				\item	ambient humidity \onslide+<2->{(unless indoors)}
				\item	ambient temperature \onslide+<2->{(unless indoors)}
			\end{itemize}
			
			\vspace{1cm}
			Practically, it might be expensive to control certain variables. For example: 
			
			impurities in the ingredients. I could buy enough
			ingredients, mix them all up, and split them into 8 pieces, so all
			8 experiments have the same ``impurity''.
		
		\column{0.48\textwidth}
			\onslide+<3->{
				\begin{center}\textbf{Measure}\end{center}
			
				Some variables are too expensive or unreliable to measure.
			
				\onslide+<4->{
					\begin{center}
						\includegraphics[width=.6\textwidth]{../4C/Supporting materials/5937291945_4e961c8baa_o-flickr-humidity-modified.png}
						\see{\href{https://secure.flickr.com/photos/sidelong/5937291945}{Flickr: sidelong}}
					\end{center}
				}
			}
			
	\end{columns}
	
\end{frame}

\begin{frame}\frametitle{Uncontrolled and unmeasurable variables can negate all your hard work}
	

	\begin{columns}[T]
		\column{0.6\textwidth}
			%Order of experiments$^\ast$:
			
			\vspace{1cm}
			\begin{tabulary}{\linewidth}{ccccc}\hline 
				\textbf{\relax Experiment } & \textbf{\relax A } & \textbf{\relax B } & \textbf{\relax C } & \textbf{\relax Time finished} \\
				\hline 
				\textbf{1} & \(-\) & \(-\) & \(-\) & 08:32\\
				\textbf{2} & \(+\) & \(-\) & \(-\) & 09:46\\
				\textbf{3} & \(-\) & \(+\) & \(-\) & 10:50\\
				\textbf{4} & \(+\) & \(+\) & \(-\) & 12:05\\
				\textbf{5} & \(-\) & \(-\) & \(+\) & 13:16\\
				\textbf{6} & \(+\) & \(-\) & \(+\) & 14:30\\
				\textbf{7} & \(-\) & \(+\) & \(+\) & 15:57\\
				\textbf{8} & \(+\) & \(+\) & \(+\) & 17:09\\
				 \hline
			\end{tabulary}
			
			\vspace{0.2cm}
			{\scriptsize $^\ast$ note the experiments are in \emph{standard order this time}}

		\column{0.38\textwidth}
	\end{columns}	
\end{frame}

\begin{frame}\frametitle{Take a moment to think about this ...}
	\begin{exampleblock}{}
		\emph{Pause the video}: what problems do you think might be caused by this confounding?
		
		\vspace{1cm}
		\begin{itemize}
			\item	Recall what the term ``confounding'' means (from the previous video)
			\item	Guess what will happen when you analyze the outcome variable, $y$
			\item	Now resume the video ...
			
		\end{itemize}
	\end{exampleblock}
\end{frame}

\begin{frame}\frametitle{\includegraphics[width=0.3\textwidth]{\imagedir/doe/examples/advice-logo.png} when experimenting with computer simulations}
	\begin{columns}[T]
		\column{0.48\textwidth}
		
			\textbf{The same as regular experiments}:
			
			\vspace{12pt}
			\begin{itemize}
				\item	you must follow a systematic method
				\item	don't ``play around'' with the software: trial-and-error
			\end{itemize}
			
			\begin{center}\rule[8mm]{4cm}{0.01cm}\end{center}
		\column{0.01\textwidth}
			\rule[3mm]{0.01cm}{25mm}%
			
		\column{0.48\textwidth}
			\textbf{Different to regular experiments:}
			\vspace{12pt}
			\begin{enumerate}
				\item	We can often run computer simulations in parallel
				\item	Computer experiments (mostly$^\ast$) are deterministic
					\begin{itemize}
						\item	i.e. if you repeat the experiments, you get the identical results
						\item	this indicates there are no disturbances that affect the outcome
						\item	this implies you do not need to randomize the order
						\item	or even repeat experiments!
						
					\end{itemize}
			\end{enumerate}
			{\scriptsize $^\ast$ {\emph{except those that use random number generators}}}
	\end{columns}
	
	\vspace{-90pt}

	 \fbox{\parbox[b][8em][t]{0.45\textwidth}{
	 	{\footnotesize
		 	Many experiments are run this \\way these days:
			\begin{itemize}
				\item	bridge/building design
				\item	chemical factory design
				\item	traffic light timing and queuing
	 			\item	stock market buy/sell strategy\\ evaluations \tiny{(see \href{http://yint.org/trading-expt}{ http://yint.org/trading-expt})}
	 		\end{itemize}
		}
	 } }
\end{frame}

\begin{frame}\frametitle{Experiments that could be problematic without randomization}
	\begin{columns}[T]
		\column{0.33\textwidth}
			\includegraphics[width=\textwidth]{../4C/Supporting materials/4C-5-Italian.png}
		
		\column{0.33\textwidth}
			\includegraphics[width=\textwidth]{../4C/Supporting materials/4C-4-memorize.png}
		
		\column{0.33\textwidth}
			\includegraphics[width=\textwidth]{../4C/Supporting materials/4C-3-gym.png}
			
	\end{columns}
\end{frame}

\begin{frame}\frametitle{Other systems experience deterioration over time: we must randomize to counteract this}
	\begin{columns}[T]
		\column{0.35\textwidth}
			Gas mileage in a vehicle: 
			
			\includegraphics[width=\textwidth]{../4C/Supporting materials/4C-6-gas-mileage.png}
			
			\vspace{2.5cm}
			{\scriptsize (Question: how would you describe the variable: ``\textbf{R} = \emph{rain on the road}\,''?)}
		
		\column{0.33\textwidth}
			
			\onslide+<2->{
				Ginger biscuit baking experiment: 
			
				\includegraphics[width=\textwidth]{../4C/Supporting materials/4C-1-tiredness.jpg}
			}
		
		\column{0.33\textwidth}
		
			\onslide+<3->{
				\includegraphics[width=.8\textwidth]{../4C/Supporting materials/7629039140_c2e2cde0c8_o-flickr-rusting.jpg}
				
				\vspace{-0.52cm}
				{\tiny \href{https://secure.flickr.com/photos/mikecogh/7629039140}{Flickr: mikecogh}}
				
				
				{\scriptsize 
					This is a slow moving disturbance: we must account for it though if we run experiments that span several months, or years.
				}
			}
			
	\end{columns}
\end{frame}

\begin{frame}\frametitle{\includegraphics[width=0.3\textwidth]{\imagedir/doe/examples/advice-logo.png}\,\, Always randomize!}
	
	\begin{exampleblock}{Why do we randomize?}
		So when we analyze the effect of the factors -- we can be almost certain they are not confound by unmeasured, and uncontrolled disturbances.
	\end{exampleblock}
	
\end{frame}

\begin{frame}\frametitle{\includegraphics[width=0.25\textwidth]{\imagedir/doe/examples/advice-logo.png}\,\,disturbances which can be measured, should be recorded}
	
	\begin{tabulary}{\linewidth}{c|ccc|cc|c}\hline 
		& \multicolumn{3}{c|}{\textbf{\relax Factors}} & \multicolumn{2}{c|}{\textbf{\relax Measured disturbances}}  & \textbf{\relax Outcome variable}                      \\
		& \multicolumn{3}{c|}{$\overbrace{ \hspace{2.2cm}}{}$}  & \multicolumn{2}{c|}{$\overbrace{ \hspace{4cm}}{}$} & $\overbrace{\hspace{3cm}}{}$ \\
		\textbf{\relax Experiment$^\ast$} & \textbf{\relax A } & \textbf{\relax B } & \textbf{\relax C } & \textbf{\relax T}emperature & \textbf{\relax H}umidity & \textbf{\relax $y$ = breakability}\\
		\hline 
		\textbf{1} & \(-\) & \(-\) & \(-\) & 23 & 32 \\
		\textbf{2} & \(+\) & \(-\) & \(-\) & 21 & 56 \\
		\textbf{3} & \(-\) & \(+\) & \(-\) & 24 & 24 \\
		\textbf{4} & \(+\) & \(+\) & \(-\) & 22 & 24 \\
		\textbf{5} & \(-\) & \(-\) & \(+\) & 23 & 30 \\
		\textbf{6} & \(+\) & \(-\) & \(+\) & 22 & 54 \\
		\textbf{7} & \(-\) & \(+\) & \(+\) & 23 & 36 \\
		\textbf{8} & \(+\) & \(+\) & \(+\) & 24 & 24 \\
		\textbf{9} &  &  &  &  & \\
		$\vdots$   &  &  &  &  & \\ \hline
	\end{tabulary}
	
	\vspace{0.05cm}
	{\scriptsize $^\ast$ Experiments were run in random order; but are reported in standard order}
	
	\onslide+<2->{
		\vspace{0.25cm}
		{\small ``{\color{purple} measured disturbances}'' = ``variables which are measured, but not controlled'' = ``{\color{purple}  covariates}''}
	}	
\end{frame}

\begin{frame}\frametitle{Using the covariate information}
	
	
	\begin{tabulary}{\linewidth}{c|ccc|cc|c}\hline 
		& \multicolumn{3}{c|}{\textbf{\relax Factors}} & \multicolumn{2}{c|}{\textbf{\relax Measured disturbances}}  & \textbf{\relax Outcome variable}                      \\
		& \multicolumn{3}{c|}{$\overbrace{ \hspace{2.2cm}}{}$}  & \multicolumn{2}{c|}{$\overbrace{ \hspace{4cm}}{}$} & $\overbrace{\hspace{3cm}}{}$ \\
		\textbf{\relax Experiment$^\ast$} & \textbf{\relax A } & \textbf{\relax B } & \textbf{\relax C } & \textbf{\relax T}emperature & \textbf{\relax H}umidity & \textbf{\relax $y$ = breakability}\\
		\hline 
		\textbf{1} & \(-\) & \(-\) & \(-\) & 23 & 32 & 4 \\
		\textbf{2} & \(+\) & \(-\) & \(-\) & 21 & 56 & 5 \\
		\textbf{3} & \(-\) & \(+\) & \(-\) & 24 & 24 & 5 \\
		\textbf{4} & \(+\) & \(+\) & \(-\) & 22 & 24 & 6 \\
		\textbf{5} & \(-\) & \(-\) & \(+\) & 23 & 77 & 3 \\
		\textbf{6} & \(+\) & \(-\) & \(+\) & 22 & 54 & 8 \\
		\textbf{7} & \(-\) & \(+\) & \(+\) & 23 & 36 & 6 \\
		\textbf{8} & \(+\) & \(+\) & \(+\) & 24 & 24 & 9 \\
		\textbf{9} & $0$   & $0$   & $0$   & 23 & 39 & 5 \\
		$\vdots$   &  &  &  &  & \\ \hline
	\end{tabulary}
	
	\onslide+<2->{
		\vspace{0.25cm}
		$\hat{y} = b_0 + b_\text{A}x_\text{A} + b_\text{B}x_\text{B} + b_\text{C}x_\text{C} +  b_\text{AB}x_\text{A}x_\text{B} + \dots +\underbrace{ \color{myOrange} b_\text{T}\,x_\text{T} \,\, + \,\, b_\text{H}\,x_\text{H}}_{\mathclap{\text{covariate terms are added to the model$^\ast$}}}$
		
		
		
		%\begin{flushright}
		{\tiny $^\ast$ but you will require more than 8 experiments to build this model}
		%\end{flushright}	
	}
\end{frame}

\begin{frame}\frametitle{Cellphone app example: ``CalApp''}	
	\begin{columns}[T]
		\column{0.25\textwidth}
			\includegraphics[width=\textwidth]{../4C/Supporting materials/4C-7-cell-phone-2830319467_1faaecc974_o-flickr.jpg}
			\\
			\tiny{\href{https://secure.flickr.com/photos/williamhook/2830319467/}{Flickr: williamhook}}
			
		\column{0.80\textwidth}
		
			The app has various upgradable features, called ``in-app purchases''
			
			\begin{itemize}
				\item	sync-to-other-devices
				\item	text message reminders
				\item	integrate with desktop calendar
			\end{itemize}
			
			\vspace{1cm}
			
			
		\onslide+<2->{	
		   	 \fbox{\parbox[b][5em][t]{0.65\textwidth}{
			 	Your marketing idea for experimenting: 
				\begin{itemize}
					\item	each test group has 2000 people
					\item	calculate the percentage using the app after 60 days; that's your outcome, $y$
				\end{itemize}
		   	 } }
		}
		 
	\end{columns}	
\end{frame}

\begin{frame}\frametitle{Cellphone app example: ``CalApp''}	
	
			{\color{myOrange} The factors you are actively testing} {\small (confirm whether they are controllable)}
			\vspace{12pt}
			
			\begin{tabulary}{\linewidth}{l|ll}\hline
				& \textbf{\relax Low level $-$} & \textbf{\relax High level $+$}\\ \hline  \\
				\textbf{A}: ``Promotion'' & 1 free in-app upgrade & 30-day trial of all features\\ \\
				\textbf{B}: ``Message'' & \parbox[t]{5.cm}{``CalApp has your schedule available at your fingertips, on any device.''} & \parbox[t]{5.5cm}{``CalApp features are configurable; only pay for the features you want.''} \\ \\
				\textbf{C}: ``Price'' & in-app purchase price is 89c &  in-app purchase price is 99c  \\& \\\hline
			\end{tabulary}	
\end{frame}

\begin{frame}\frametitle{Cellphone app example: test your understanding}
	
	
	Are these ``disturbances'' (\emph{not measured, not controlled}), or \\
	\qquad\quad\,\,\,\,\,\, ``covariates'' \,\,\,\, (\emph{measured, but not controlled}), or \\
	\qquad\quad\,\,\,\,\,\, ``neither of these'':
	
	\vspace{0.5cm}
	
	\begin{itemize}
		\item	\textbf{E}: smartphone user's age
		\item	\textbf{N}: smartphone user's gender
		\item	\textbf{S}: smartphone user's connection speed (e.g. cell, or wifi)
		\item	\textbf{R}: amount of free memory (RAM) on the device
		\item	\textbf{F}: whether the advert/message is delivered via ad network G, or ad network H
		\item	\textbf{D}: if the user's phone is Android or Apple		
	\end{itemize}
	
	\vspace{0.5cm}
	
	Participate in the forums and share your opinion at \href{http://yint.org/cal-app}{http://yint.org/cal-app}
\end{frame}

\end{comment}

\begin{frame}\frametitle{The variables affecting our experimental system}
	We may classify them in several ways:
	
	\begin{itemize}
		\item	those we know about, and those that are unknown
			\pause
		\item	variables we can control, and uncontrolled variables: \emph{we'll consider these today}
			\pause
		\item	some we can measure, and others we cannot measure: \emph{we'll consider these today}
		
	\end{itemize}	
\end{frame}

\begin{frame}\frametitle{Controllable and measurable variables and the need for randomization}
	Use a familiar example: {\color{myOrange} ginger biscuits}
	\vspace{0.25cm}
	\begin{columns}[T]
		\column{0.4\textwidth}
			We have 3 factors:
			\begin{enumerate}
				\item	\textbf{A}: baking temperature
				\item	\textbf{B}: amount of baking soda
				\item	\textbf{C}: baking time
			\end{enumerate}
			
			
			\vspace{1cm}
			
			\includegraphics[width=0.8\textwidth]{../4C/Supporting materials/4C-2-ginger-biscuits-3242839562_10c30d1aa3_o.jpg}
		
		
			\see{\href{https://secure.flickr.com/photos/babbagecabbage/3242839562}{Flickr: babbagecabbage}}
			
		
		\column{0.58\textwidth}
		
			\onslide+<2->{
			
				Order of experiments$^\ast$:
			
				\begin{tabulary}{\linewidth}{ccccc}\hline 
					\textbf{\relax Experiment } & \textbf{\relax A } & \textbf{\relax B } & \textbf{\relax C } & \textbf{\relax Time finished} \\
					\hline 
					\textbf{5} & \(-\) & \(-\) & \(+\) & 08:32\\
					\textbf{1} & \(-\) & \(-\) & \(-\) & 09:46\\
					\textbf{8} & \(+\) & \(+\) & \(+\) & 10:50\\
					\textbf{3} & \(-\) & \(+\) & \(-\) & 12:05\\
					\textbf{6} & \(+\) & \(-\) & \(+\) & 13:16\\
					\textbf{2} & \(+\) & \(-\) & \(-\) & 14:30\\
					\textbf{4} & \(+\) & \(+\) & \(-\) & 15:57\\
					\textbf{7} & \(-\) & \(+\) & \(+\) & 17:09\\
					
					 \hline
				\end{tabulary}
				
				\vspace{0.2cm}
				{\scriptsize $^\ast$ note the randomized order}
			}
			
			\vspace{0.75cm}
			%{\scriptsize See forum postings at \href{http://yint.org/forum-randomization}{http://yint.org/forum-randomization}}
	\end{columns}	
\end{frame}

\begin{frame}\frametitle{Disturbances are uncontrolled and unmeasured variables}
	
	\begin{columns}[T]
		\column{0.45\textwidth}
			Any potential impact on our system which
			\begin{itemize}
				\item	is not controlled, and
		
				\item	is not measured
			\end{itemize}
			
			\vspace{1cm}
			In the ginger biscuit example, it might include:
			\begin{itemize}
				\item	ambient humidity
				\item	ambient temperature
				\item	impurities in the ingredients
				\item	other examples ...
			\end{itemize}
		
		\column{0.48\textwidth}
			\includegraphics[width=\textwidth]{../4C/Supporting materials/4C-1-tiredness.jpg}
	\end{columns}
\end{frame}

\begin{frame}\frametitle{``Control'' and ``Measure'': let's clarify what we mean by those terms}
	
	{\color{myOrange}Given enough resources (usually money), we can control and measure most things.}
	
	\begin{columns}[T]
		\column{0.49\textwidth}
			\begin{center}\textbf{Control}\end{center}
				
			We can not actively control these:
			\begin{itemize}
				\item	ambient humidity \onslide+<2->{(unless indoors)}
				\item	ambient temperature \onslide+<2->{(unless indoors)}
			\end{itemize}
			
			\vspace{1cm}
			Practically, it might be expensive to control certain variables. For example: 
			
			impurities in the ingredients. I could buy enough
			ingredients, mix them all up, and split them into 8 pieces, so all
			8 experiments have the same ``impurity''.
		
		\column{0.48\textwidth}
			\onslide+<3->{
				\begin{center}\textbf{Measure}\end{center}
			
				Some variables are too expensive or unreliable to measure.
			
				\onslide+<4->{
					\begin{center}
						\includegraphics[width=.6\textwidth]{../4C/Supporting materials/5937291945_4e961c8baa_o-flickr-humidity-modified.png}
						\see{\href{https://secure.flickr.com/photos/sidelong/5937291945}{Flickr: sidelong}}
					\end{center}
				}
			}
			
	\end{columns}
	
\end{frame}

\begin{frame}\frametitle{Uncontrolled and unmeasurable variables can negate all your hard work}
	

	\begin{columns}[T]
		\column{0.6\textwidth}
			%Order of experiments$^\ast$:
			
			\vspace{1cm}
			\begin{tabulary}{\linewidth}{ccccc}\hline 
				\textbf{\relax Experiment } & \textbf{\relax A } & \textbf{\relax B } & \textbf{\relax C } & \textbf{\relax Time finished} \\
				\hline 
				\textbf{1} & \(-\) & \(-\) & \(-\) & 08:32\\
				\textbf{2} & \(+\) & \(-\) & \(-\) & 09:46\\
				\textbf{3} & \(-\) & \(+\) & \(-\) & 10:50\\
				\textbf{4} & \(+\) & \(+\) & \(-\) & 12:05\\
				\textbf{5} & \(-\) & \(-\) & \(+\) & 13:16\\
				\textbf{6} & \(+\) & \(-\) & \(+\) & 14:30\\
				\textbf{7} & \(-\) & \(+\) & \(+\) & 15:57\\
				\textbf{8} & \(+\) & \(+\) & \(+\) & 17:09\\
				 \hline
			\end{tabulary}
			
			\vspace{0.2cm}
			{\scriptsize $^\ast$ note the experiments are in \emph{standard order this time}}

		\column{0.38\textwidth}
	\end{columns}	
\end{frame}

\begin{frame}\frametitle{Take a moment to think about this ...}
	\begin{exampleblock}{}
		\emph{Pause the video}: what problems do you think might be caused by this confounding?
		
		\vspace{1cm}
		\begin{itemize}
			\item	Recall what the term ``confounding'' means (from the previous video)
			\item	Guess what will happen when you analyze the outcome variable, $y$
			\item	Now resume the video ...
			
		\end{itemize}
	\end{exampleblock}
\end{frame}

\begin{frame}\frametitle{\includegraphics[width=0.3\textwidth]{\imagedir/doe/examples/advice-logo.png} when experimenting with computer simulations}
	\begin{columns}[T]
		\column{0.48\textwidth}
		
			\textbf{The same as regular experiments}:
			
			\vspace{12pt}
			\begin{itemize}
				\item	you must follow a systematic method
				\item	don't ``play around'' with the software: trial-and-error
			\end{itemize}
			
			\begin{center}\rule[8mm]{4cm}{0.01cm}\end{center}
		\column{0.01\textwidth}
			\rule[3mm]{0.01cm}{25mm}%
			
		\column{0.48\textwidth}
			\textbf{Different to regular experiments:}
			\vspace{12pt}
			\begin{enumerate}
				\item	We can often run computer simulations in parallel
				\item	Computer experiments (mostly$^\ast$) are deterministic
					\begin{itemize}
						\item	i.e. if you repeat the experiments, you get the identical results
						\item	this indicates there are no disturbances that affect the outcome
						\item	this implies you do not need to randomize the order
						\item	or even repeat experiments!
						
					\end{itemize}
			\end{enumerate}
			{\scriptsize $^\ast$ {\emph{except those that use random number generators}}}
	\end{columns}
	
	\vspace{-90pt}

	 \fbox{\parbox[b][8em][t]{0.45\textwidth}{
	 	{\footnotesize
		 	Many experiments are run this \\way these days:
			\begin{itemize}
				\item	bridge/building design
				\item	chemical factory design
				\item	traffic light timing and queuing
	 			%\item	stock market buy/sell strategy\\ evaluations \tiny{(see \href{http://yint.org/trading-expt}{ http://yint.org/trading-expt})}
	 		\end{itemize}
		}
	 } }
\end{frame}

\begin{frame}\frametitle{Experiments that could be problematic without randomization}
	\begin{columns}[T]
		\column{0.33\textwidth}
			\includegraphics[width=\textwidth]{../4C/Supporting materials/4C-5-Italian.png}
		
		\column{0.33\textwidth}
			\includegraphics[width=\textwidth]{../4C/Supporting materials/4C-4-memorize.png}
		
		\column{0.33\textwidth}
			\includegraphics[width=\textwidth]{../4C/Supporting materials/4C-3-gym.png}
			
	\end{columns}
\end{frame}

\begin{frame}\frametitle{Other systems experience deterioration over time: we must randomize to counteract this}
	\begin{columns}[T]
		\column{0.35\textwidth}
			Gas mileage in a vehicle: 
			
			\includegraphics[width=\textwidth]{../4C/Supporting materials/4C-6-gas-mileage.png}
			
			\vspace{2.5cm}
			{\scriptsize (Question: how would you describe the variable: ``\textbf{R} = \emph{rain on the road}\,''?)}
		
		\column{0.33\textwidth}
			
			\onslide+<2->{
				Ginger biscuit baking experiment: 
			
				\includegraphics[width=\textwidth]{../4C/Supporting materials/4C-1-tiredness.jpg}
			}
		
		\column{0.33\textwidth}
		
			\onslide+<3->{
				\includegraphics[width=.8\textwidth]{../4C/Supporting materials/7629039140_c2e2cde0c8_o-flickr-rusting.jpg}
				
				\vspace{-0.52cm}
				{\tiny \href{https://secure.flickr.com/photos/mikecogh/7629039140}{Flickr: mikecogh}}
				
				
				{\scriptsize 
					This is a slow moving disturbance: we must account for it though if we run experiments that span several months, or years.
				}
			}
			
	\end{columns}
\end{frame}

\begin{frame}\frametitle{\includegraphics[width=0.3\textwidth]{\imagedir/doe/examples/advice-logo.png}\,\, Always randomize!}
	
	\begin{exampleblock}{Why do we randomize?}
		So when we analyze the effect of the factors -- we can be almost certain they are not confound by unmeasured, and uncontrolled disturbances.
	\end{exampleblock}
	
\end{frame}

\begin{frame}\frametitle{\includegraphics[width=0.25\textwidth]{\imagedir/doe/examples/advice-logo.png}\,\,disturbances which can be measured, should be recorded}
	
	\begin{tabulary}{\linewidth}{c|ccc|cc|c}\hline 
		& \multicolumn{3}{c|}{\textbf{\relax Factors}} & \multicolumn{2}{c|}{\textbf{\relax Measured disturbances}}  & \textbf{\relax Outcome variable}                      \\
		& \multicolumn{3}{c|}{$\overbrace{ \hspace{2.2cm}}{}$}  & \multicolumn{2}{c|}{$\overbrace{ \hspace{4cm}}{}$} & $\overbrace{\hspace{3cm}}{}$ \\
		\textbf{\relax Experiment$^\ast$} & \textbf{\relax A } & \textbf{\relax B } & \textbf{\relax C } & \textbf{\relax T}emperature & \textbf{\relax H}umidity & \textbf{\relax $y$ = breakability}\\
		\hline 
		\textbf{1} & \(-\) & \(-\) & \(-\) & 23 & 32 \\
		\textbf{2} & \(+\) & \(-\) & \(-\) & 21 & 56 \\
		\textbf{3} & \(-\) & \(+\) & \(-\) & 24 & 24 \\
		\textbf{4} & \(+\) & \(+\) & \(-\) & 22 & 24 \\
		\textbf{5} & \(-\) & \(-\) & \(+\) & 23 & 30 \\
		\textbf{6} & \(+\) & \(-\) & \(+\) & 22 & 54 \\
		\textbf{7} & \(-\) & \(+\) & \(+\) & 23 & 36 \\
		\textbf{8} & \(+\) & \(+\) & \(+\) & 24 & 24 \\
		\textbf{9} &  &  &  &  & \\
		$\vdots$   &  &  &  &  & \\ \hline
	\end{tabulary}
	
	\vspace{0.05cm}
	{\scriptsize $^\ast$ Experiments were run in random order; but are reported in standard order}
	
	\onslide+<2->{
		\vspace{0.25cm}
		{\small ``{\color{purple} measured disturbances}'' = ``variables which are measured, but not controlled'' = ``{\color{purple}  covariates}''}
	}	
\end{frame}

\begin{frame}\frametitle{Using the covariate information}
	
	
	\begin{tabulary}{\linewidth}{c|ccc|cc|c}\hline 
		& \multicolumn{3}{c|}{\textbf{\relax Factors}} & \multicolumn{2}{c|}{\textbf{\relax Measured disturbances}}  & \textbf{\relax Outcome variable}                      \\
		& \multicolumn{3}{c|}{$\overbrace{ \hspace{2.2cm}}{}$}  & \multicolumn{2}{c|}{$\overbrace{ \hspace{4cm}}{}$} & $\overbrace{\hspace{3cm}}{}$ \\
		\textbf{\relax Experiment$^\ast$} & \textbf{\relax A } & \textbf{\relax B } & \textbf{\relax C } & \textbf{\relax T}emperature & \textbf{\relax H}umidity & \textbf{\relax $y$ = breakability}\\
		\hline 
		\textbf{1} & \(-\) & \(-\) & \(-\) & 23 & 32 & 4 \\
		\textbf{2} & \(+\) & \(-\) & \(-\) & 21 & 56 & 5 \\
		\textbf{3} & \(-\) & \(+\) & \(-\) & 24 & 24 & 5 \\
		\textbf{4} & \(+\) & \(+\) & \(-\) & 22 & 24 & 6 \\
		\textbf{5} & \(-\) & \(-\) & \(+\) & 23 & 77 & 3 \\
		\textbf{6} & \(+\) & \(-\) & \(+\) & 22 & 54 & 8 \\
		\textbf{7} & \(-\) & \(+\) & \(+\) & 23 & 36 & 6 \\
		\textbf{8} & \(+\) & \(+\) & \(+\) & 24 & 24 & 9 \\
		\textbf{9} & $0$   & $0$   & $0$   & 23 & 39 & 5 \\
		$\vdots$   &  &  &  &  & \\ \hline
	\end{tabulary}
	
	\onslide+<2->{
		\vspace{0.25cm}
		$\hat{y} = b_0 + b_\text{A}x_\text{A} + b_\text{B}x_\text{B} + b_\text{C}x_\text{C} +  b_\text{AB}x_\text{A}x_\text{B} + \dots +\underbrace{ \color{myOrange} b_\text{T}\,x_\text{T} \,\, + \,\, b_\text{H}\,x_\text{H}}_{\mathclap{\text{covariate terms are added to the model$^\ast$}}}$
		
		
		
		%\begin{flushright}
		{\tiny $^\ast$ but you will require more than 8 experiments to build this model}
		%\end{flushright}	
	}
\end{frame}

\begin{frame}\frametitle{Cellphone app example: ``CalApp''}	
	\begin{columns}[T]
		\column{0.25\textwidth}
			\includegraphics[width=\textwidth]{../4C/Supporting materials/4C-7-cell-phone-2830319467_1faaecc974_o-flickr.jpg}
			\\
			\tiny{\href{https://secure.flickr.com/photos/williamhook/2830319467/}{Flickr: williamhook}}
			
		\column{0.80\textwidth}
		
			The app has various upgradable features, called ``in-app purchases''
			
			\begin{itemize}
				\item	sync-to-other-devices
				\item	text message reminders
				\item	integrate with desktop calendar
			\end{itemize}
			
			\vspace{1cm}
			
			
		\onslide+<2->{	
		   	 \fbox{\parbox[b][5em][t]{0.65\textwidth}{
			 	Your marketing idea for experimenting: 
				\begin{itemize}
					\item	each test group has 2000 people
					\item	calculate the percentage using the app after 60 days; that's your outcome, $y$
				\end{itemize}
		   	 } }
		}
		
		\vspace{0.5cm}
		{\scriptsize Read the Harvard Business Review article at \href{http://yint.org/hbr-article}{http://yint.org/hbr-article}}
		 
	\end{columns}	
\end{frame}

\begin{frame}\frametitle{Cellphone app example: ``CalApp''}	
	
			{\color{myOrange} The factors you are actively testing} {\small (confirm whether they are controllable)}
			\vspace{12pt}
			
			\begin{tabulary}{\linewidth}{l|ll}\hline
				& \textbf{\relax Low level $-$} & \textbf{\relax High level $+$}\\ \hline  \\
				\textbf{A}: ``Promotion'' & 1 free in-app upgrade & 30-day trial of all features\\ \\
				\textbf{B}: ``Message'' & \parbox[t]{5.cm}{``CalApp has your schedule available at your fingertips, on any device.''} & \parbox[t]{5.5cm}{``CalApp features are configurable; only pay for the features you want.''} \\ \\
				\textbf{C}: ``Price'' & in-app purchase price is 89c &  in-app purchase price is 99c  \\& \\\hline
			\end{tabulary}	
\end{frame}

\begin{frame}\frametitle{Cellphone app example: test your understanding}
	
	
	Are these ``disturbances'' (\emph{not measured, not controlled}), or \\
	\qquad\quad\,\,\,\,\,\, ``covariates'' \,\,\,\, (\emph{measured, but not controlled}), or \\
	\qquad\quad\,\,\,\,\,\, ``neither of these'':
	
	\vspace{0.5cm}
	
	\begin{itemize}
		\item	\textbf{E}: smartphone user's age
		\item	\textbf{N}: smartphone user's gender
		\item	\textbf{S}: smartphone user's connection speed (e.g. cell, or wifi)
		\item	\textbf{R}: amount of free memory (RAM) on the device
		\item	\textbf{F}: whether the advert/message is delivered via ad network G, or ad network H
		\item	\textbf{D}: if the user's phone is Android or Apple		
	\end{itemize}
	
	\vspace{0.5cm}
	
	Participate in the forums and share your opinion at \href{http://yint.org/cal-app}{http://yint.org/cal-app}
\end{frame}