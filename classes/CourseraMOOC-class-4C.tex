\begin{frame}\frametitle{Now that we understand aliasing; how can we work with it in our system?}
	
	We'd like to take advantage of doing half the work, but still get the most benefit. But, 
	recall we showed that this will lead to a loss of some accuracy:
	
	\vspace{0.2cm}
	
	
	\vspace{-0.2cm}
	\begin{columns}[T]
		\column{0.35\textwidth}
			\begin{center}\textbf{Full factorial model}\end{center}
				\begin{align*}
					\hat{y} &= {\color{myOrange}11.25}\\
							&+ {\color{myOrange}6.25\,}x_\text{A}\\
							&+ {\color{blue}0.75\,}x_\text{B}\\
							&- {\color{myOrange}7.25\,}x_\text{C}\\
							&+ {\color{myOrange}0.25\,}x_\text{A}x_\text{B}\\
							&- {\color{blue}6.75\,}x_\text{A}x_\text{C}\\
							&- {\color{myOrange}0.25\,}x_\text{B}x_\text{C}\\
							&- {\color{myOrange}0.25\,}x_\text{A}x_\text{B}x_\text{C}		
				\end{align*}
		\column{0.02\textwidth}
			\vspace{1cm}
			\rule[3mm]{0.03cm}{65mm}
		\column{0.35\textwidth}
			\begin{center}\textbf{Fractional factorial model}\end{center}				
				\begin{align*}
					\hat{y} &= {\color{myOrange}11.0}\\
							&+ {\color{myOrange}6.0\,}x_\text{A}\\
							&- {\color{blue}6.0\,}x_\text{B}\\
							&- {\color{myOrange}7.0\,}x_\text{C}\\
							& \color{lightgray}+ \cancel{b_\text{AB}\,x_\text{A}x_\text{B}}\\
							& \color{lightgray}+ \cancel{b_\text{AC}\,x_\text{A}x_\text{C}}\\
							& \color{lightgray}+ \cancel{b_\text{BC}\,x_\text{B}x_\text{C}}\\
							& \color{lightgray}+ \cancel{b_\text{ABC}\,x_\text{A}x_\text{B}x_\text{C}}\\
				\end{align*}

	\end{columns}
\end{frame}

\begin{frame}\frametitle{Example: treating water (again)}
	
	\begin{columns}[T]
		\column{0.65\textwidth}
			We have identified 3 factors to investigate
			\begin{enumerate}
				\item	Water treatment chemical used
				\item	Temperature of the treatment
				\item	Stirring speed
			\end{enumerate}
		
			\vspace{1cm}
			\emph{Budget}: for 4, maybe 5, experiments
			
			\onslide+<2->{
				\vspace{1cm}
				Other criteria:
				\begin{itemize}
					\item	We don't expect interactions between temperature, and stirring speed
					\onslide+<3->{
						\item	We want a good (i.e. unbiased) estimate of the chemical effect
					}
				\end{itemize}
			}
			
		
		\column{0.38\textwidth}
			\includegraphics[width=\textwidth]{../4B/Supporting files/177506595-thinkstock-reduced.jpg}
		
		
			
	\end{columns}
\end{frame}

\begin{frame}\frametitle{Example: treating water (again)}
	
	\begin{columns}[T]
		\column{0.70\textwidth}
			Now we choose to assign the factor letters: \textbf{A}, \textbf{B}, and \textbf{C}
			
			\vspace{0.25cm}
			\begin{itemize} 
				\item	\textbf{A}: Temperature of the treatment
				\item	\textbf{B}: Stirring speed				
				\item	\textbf{C}: Water treatment chemical used

			\end{itemize}
			
			\vspace{0.5cm}
			Reasons for this choice of letters (assignment):
			\begin{itemize}
				\item	We don't expect interaction between temperature (\textbf{A}), and stirring speed (\textbf{B}). That implies $b_\text{AB} \approx 0$
				\item	We want a clear, unbiased estimate of the chemical effect (\textbf{C}). So an unbiased $b_\text{C}$ is desired.
				\item	We know that $\hat{b}_\text{C}$ will be the estimated chemical effect in a half-fraction. It will
					 	be confounded:
				\begin{itemize}
					\item	$\hat{b}_\text{C}  = b_\text{C} + b_\text{AB}  = b_\text{C} + \cancelto{\approx 0}{b_\text{AB}} \qquad$ so then $\hat{b}_C \approx b_\text{C}$
				\end{itemize}
			\end{itemize}
		
		\column{0.35\textwidth}
			\includegraphics[width=\textwidth]{../4B/Supporting files/177506595-thinkstock-reduced.jpg}
	\end{columns}
\end{frame}

\begin{frame}\frametitle{\includegraphics[width=0.3\textwidth]{\imagedir/doe/examples/advice-logo.png}\,\, plan your experiments carefully ahead of time}
	
	\vspace{24pt}
	{\color{myOrange} 	\emph{Experiments are expensive to run!}}
	

	\begin{exampleblock}{}
		\vspace{12pt}
		\begin{itemize}
			\item	so plan not only the experiments, but also \textbf{how you will analyze} the results
			\item	you don't need outcome values ($y$-values) to do this
			
			\vspace{1cm}

			\item	Re-allocate your letters if you don't like the confounding
		\end{itemize} 
	\end{exampleblock}
	
\end{frame}

\begin{frame}\frametitle{}
	%\vspace{-0.5cm}
	\includegraphics[height=.9\textheight]{\imagedir/doe/DOE-trade-off-table-half-fractions.png}
	
	\vspace{-10pt}
	Notice where all the half fractions lie in the table. \\
	Do you observe the recurring pattern?
\end{frame}

\begin{frame}\frametitle{\includegraphics[width=0.3\textwidth]{\imagedir/doe/examples/advice-logo.png}\,\, the utility of half-fractions}
	
	\vspace{24pt}
	Four factors: requires 16 runs for a full factorial. Why not start with a half fraction?
	
	\vspace{24pt}
	\begin{itemize}
		\item	Start with 8 runs
		\item	Then come back and do the other 8 later on
	\end{itemize}
	
	\pause
	\vspace{24pt}
	{\color{myOrange} Coming up later in this module: we will see quarter, or even $\tfrac{1}{8}^\text{th}$ fractions, applied in the same way.}
	
\end{frame}

\begin{frame}\frametitle{Demonstration on an example we can visualize: a 3-factor system}
	\begin{columns}
		\column{0.6\textwidth}
			\begin{center}
				\includegraphics[width=.9\textwidth]{\imagedir/doe/half-fraction-in-3-factors-MOOC-no-labels.png}
			\end{center}
			
		\column{0.55\textwidth}
		
			{\color{blue}Start with a half fraction $^\ast$}
			\begin{tabulary}{\linewidth}{cccl}\hline 
				\textsf{\relax Experiment } & \textbf{\relax A } & \textbf{\relax B } & \textbf{\relax C = AB} \\
				\hline 
				\textbf{1} & \(-\) & \(-\) & \(+\) \\
				\textbf{2} & \(+\) & \(-\) & \(-\) \\
				\textbf{3} & \(-\) & \(+\) & \(-\) \\
				\textbf{4} & \(+\) & \(+\) & \(+\) \\  \hline
			\end{tabulary}
			
			{\color{myOrange}These are the 4 runs with the open circles.}

			\small
			\vspace{0.5cm}
			
			
			\onslide+<2->{
			{\color{blue}\emph{Later on} ... come back and complete the complementary half-fraction.$^\ast$}
			
			\begin{tabulary}{\linewidth}{cccl}\hline 
				\textsf{\relax Experiment } & \textbf{\relax A } & \textbf{\relax B } & \textbf{\relax C = $-$AB } \\
				\hline 
				\textbf{5} & \(-\) & \(-\) & \(-\) \\
				\textbf{6} & \(+\) & \(-\) & \(+\) \\
				\textbf{7} & \(-\) & \(+\) & \(+\) \\
				\textbf{8} & \(+\) & \(+\) & \(-\) \\ \hline
			\end{tabulary}
			
			{\color{myOrange}These are the 4 runs with the closed circles.}
			}
			
			\vspace{6pt}
			{\color{blue}$^\ast$\scriptsize {Always randomize within the group of 4!}}
	\end{columns}
\end{frame}

\begin{frame}\frametitle{}
	%\vspace{-0.5cm}
	\includegraphics[height=.9\textheight]{\imagedir/doe/DOE-trade-off-table-half-fractions.png}
	
	\vspace{-10pt}
	We will see how entries in this table are good building blocks.\\
	Consider how you might use it for your experiments.
\end{frame}

\begin{frame}\frametitle{Half fractions are  one of the experimental building blocks available to you}
	
	\begin{columns}[C]
		\column{0.3\textwidth}
		
	
			Initial groups of experiments can be build on, and extended later.
			
		\column{0.7\textwidth}
			
				\centerline{\includegraphics[width=0.5\textwidth]{../4G/Supporting material/flickr-6479064129_25ce3bb07f_o-building-block.png} \quad
				\see{\href{https://secure.flickr.com/photos/rahego/6479064129}{Flickr: rahego}}
				}
		
				
		
	\end{columns}
\end{frame}