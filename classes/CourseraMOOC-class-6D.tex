\begin{frame}\frametitle{The course outline over the past several weeks}
	\begin{columns}[T]
		\column{0.75\textwidth}
			\centerline{\includegraphics[width=\textwidth]{../6D/Supporting material/Week-1.pdf}}
		\column{0.3\textwidth}
			\centerline{\includegraphics[width=\textwidth]{../6D/Supporting material/weekly-plan-for-Coursera-MOOC.png}}
	\end{columns}
\end{frame}

\begin{frame}\frametitle{The course outline over the past several weeks}
	\begin{columns}[T]
		\column{0.75\textwidth}
			\centerline{\includegraphics[width=\textwidth]{../6D/Supporting material/Week-2-mod.pdf}}
		\column{0.3\textwidth}
			\centerline{\includegraphics[width=\textwidth]{../6D/Supporting material/weekly-plan-for-Coursera-MOOC.png}}
	\end{columns}
\end{frame}

\begin{frame}\frametitle{The course outline over the past several weeks}
	\begin{columns}[T]
		\column{0.75\textwidth}
			\centerline{\includegraphics[width=\textwidth]{../6D/Supporting material/Week-3.pdf}}
		\column{0.3\textwidth}
			\centerline{\includegraphics[width=\textwidth]{../6D/Supporting material/weekly-plan-for-Coursera-MOOC.png}}
	\end{columns}
\end{frame}

\begin{frame}\frametitle{The course outline over the past several weeks}
	\begin{columns}[T]
		\column{0.75\textwidth}
			\centerline{\includegraphics[width=\textwidth]{../6D/Supporting material/Week-4.pdf}}
		\column{0.3\textwidth}
			\centerline{\includegraphics[width=\textwidth]{../6D/Supporting material/weekly-plan-for-Coursera-MOOC.png}}
	\end{columns}
\end{frame}

\begin{frame}\frametitle{The course outline over the past several weeks}
	\begin{columns}[T]
		\column{0.75\textwidth}
			\centerline{\includegraphics[width=\textwidth]{../6D/Supporting material/Week-5.pdf}}
		\column{0.3\textwidth}
			\centerline{\includegraphics[width=\textwidth]{../6D/Supporting material/weekly-plan-for-Coursera-MOOC.png}}
	\end{columns}
\end{frame}

\begin{frame}\frametitle{The course outline over the past several weeks}
	\begin{columns}[T]
		\column{0.75\textwidth}
			\centerline{\includegraphics[width=\textwidth]{../6D/Supporting material/Week-6.pdf}}
		\column{0.3\textwidth}
			\centerline{\includegraphics[width=\textwidth]{../6D/Supporting material/weekly-plan-for-Coursera-MOOC.png}}
	\end{columns}
\end{frame}

\begin{frame}\frametitle{}
			\centerline{\includegraphics[height=\textheight]{../6A/Supporting materials/RSM-51.png}}
\end{frame}

\begin{frame}\frametitle{Dealing with multiple criteria in experiments}
	\begin{columns}[c]
		\column{0.75\textwidth}
			\begin{itemize}
				\item	The weighted sum = $\varphi$
				\begin{align*}
					\varphi &&=&& {\color{purple}w_1} {\color[rgb]{0,0.56,0.93}(\text{colour})} &&+&& {\color{purple}w_2} {\color[rgb]{0.93,0,0}(\text{breakability})} \\
				\onslide+<2->{	
				\text{For example}\qquad	\varphi &&=&& {\color{purple}0.3} {\color[rgb]{0,0.56,0.93}(\text{colour})} &&+&& {\color{purple}0.7} {\color[rgb]{0.93,0,0}(\text{breakability})}
				}
				\end{align*}
			\end{itemize}
			
			\centerline{\includegraphics[width=.7\textwidth]{\imagedir/doe/examples/snackfood-tradeoffs-01.png}}

		\column{0.3\textwidth}
			Optimizing the properties of a snack food product that is fried in oil
			\centerline{\includegraphics[width=\textwidth]{../6D/Supporting material/snack-food.jpg}}
	\end{columns}
\end{frame}

\begin{frame}\frametitle{Dealing with multiple criteria in experiments}
	\begin{columns}[c]
		\column{0.75\textwidth}
			\begin{itemize}
				\item	The weighted sum = $\varphi$
				\begin{align*}
					\varphi &&=&& {\color{purple}w_1} {\color[rgb]{0,0.56,0.93}(\text{colour})} &&+&& {\color{purple}w_2} {\color[rgb]{0.93,0,0}(\text{breakability})} \\
				\onslide+<1->{	
				\text{For example}\qquad	\varphi &&=&& {\color{purple}0.3} {\color[rgb]{0,0.56,0.93}(\text{colour})} &&+&& {\color{purple}0.7} {\color[rgb]{0.93,0,0}(\text{breakability})}
				}
				\end{align*}
			\end{itemize}
			
			\centerline{\includegraphics[width=.7\textwidth]{\imagedir/doe/examples/snackfood-tradeoffs-02.png}}

		\column{0.3\textwidth}
			Optimizing the properties of a snack food product that is fried in oil
			\centerline{\includegraphics[width=\textwidth]{../6D/Supporting material/snack-food.jpg}}
	\end{columns}
\end{frame}

\begin{frame}\frametitle{Dealing with multiple criteria in experiments}
	\begin{columns}[c]
		\column{0.75\textwidth}
			\begin{itemize}
				\item	The weighted sum = $\varphi$
				\begin{align*}
					\varphi &&=&& {\color{purple}w_1} {\color[rgb]{0,0.56,0.93}(\text{colour})} &&+&& {\color{purple}w_2} {\color[rgb]{0.93,0,0}(\text{breakability})} \\
				\onslide+<1->{	
				\text{For example}\qquad	\varphi &&=&& {\color{purple}0.3} {\color[rgb]{0,0.56,0.93}(\text{colour})} &&+&& {\color{purple}0.7} {\color[rgb]{0.93,0,0}(\text{breakability})}
				}
				\end{align*}
			\end{itemize}
			
			\centerline{\includegraphics[width=.7\textwidth]{\imagedir/doe/examples/snackfood-tradeoffs-03.png}}

		\column{0.3\textwidth}
			Optimizing the properties of a snack food product that is fried in oil
			\centerline{\includegraphics[width=\textwidth]{../6D/Supporting material/snack-food.jpg}}
	\end{columns}
\end{frame}

\begin{frame}\frametitle{The book by Goos and Jones: \emph{Optimal Design of Experiments}}
	\begin{itemize}
		\item	A case-study based approach \pause
		\item	RSM with categorical factors \pause
		\item	Screening designs \pause
		\item	Mixture designs \pause
		\item	Blocking and covariates \pause
		\item	Split-plot designs (an important, practical topic) 
	\end{itemize}
\end{frame}

\begin{frame}\frametitle{}
	\centerline{\includegraphics[width=.96\textwidth]{../5B/Supporting materials/popcorn-experiments-32.png}}
\end{frame}

\begin{frame}\frametitle{Some additional advice on response surface methods}
	
	\begin{columns}[T]
		\column{0.95\textwidth}
			\begin{itemize}
				\item	RSM with multiple factors, e.g. \textbf{A}, \textbf{B}, \textbf{C}, and \textbf{D}. Pick a
				value for $\Delta x_\text{A}$, then:
				\begin{itemize}
					\item	$\Delta x_\text{B} = \dfrac{b_\text{B}}{b_\text{A}} \Delta x_\text{A}$
					\item	$\Delta x_\text{C} = \dfrac{b_\text{C}}{b_\text{A}} \Delta x_\text{A}$
					\item	$\Delta x_\text{D} = \dfrac{b_\text{D}}{b_\text{A}} \Delta x_\text{A}$
				\end{itemize}
				
				\item	Now with 4 factors, it does not make sense to run $2^4=16$ experiments in every factorial. Use fractional factorials.
				\item	Unless, you are doing the experiments on a small scale, and they are cheap(er) than full-scale experiments.
				\item	But watch the aliases when you approach the optimum. You will likely need to run a full CCD near the optimum to estimate curvature correctly.
						\\
				\item	Check out the {\color{myOrange}``rsm''} package in R
						
			\end{itemize}
		\column{0.05\textwidth}
			
	\end{columns}
% categorical -> continuous
\end{frame}

\begin{frame}\frametitle{\color{white}This design is already D-optimal}
	\color{white} in the case when you want to run 8 experiments in 3 factors
	\centerline{\includegraphics[height=.8\textheight]{\imagedir/doe/half-fraction-in-3-factors-MOOC-all-8-no-labels.png}}	
\end{frame}

\begin{frame}\frametitle{This design is already D-optimal}
	\color{myOrange} in the case when you want to run 8 experiments in 3 factors
	\centerline{\includegraphics[height=.8\textheight]{\imagedir/doe/half-fraction-in-3-factors-MOOC-all-8-no-labels.png}}	
\end{frame}

\begin{frame}\frametitle{This design is already D-optimal}
	\color{myOrange} in the case when you want to run 4 experiments in 3 factors
	\centerline{\includegraphics[height=.8\textheight]{\imagedir/doe/half-fraction-in-3-factors-MOOC-optimum-4-coloured.png}}	
\end{frame}



