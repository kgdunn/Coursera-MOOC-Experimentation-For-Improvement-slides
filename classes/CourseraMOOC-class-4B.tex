% \begin{comment}
%
%
% 	\begin{frame}\frametitle{\includegraphics[width=0.3\textwidth]{\imagedir/doe/examples/advice-logo.png}\,\, plan your experiments carefully ahead of time}
% 	\end{frame}
%
% 	\begin{columns}[T]
% 		\column{0.45\textwidth}
% 			\includegraphics[width=0.7\textwidth]{\imagedir/statistics/flicfcb_o.jpg}
%
% 			{\scriptsize (p. 230 in Box, Hunter and Hunter, 2$^\text{nd}$ ed)}
%
% 		\column{0.48\textwidth}
% 			\includegraphics[width=\textwidth]{\imagedir/doe/examples/solar-panel-mendelu-cz-website.png}
%
%
% 			\see{\href{http://yint.org/solar-panel-study}{http://yint.org/solar-panel-study}}
% 	\end{columns}
%
% 	\begin{center}\rule[8mm]{4cm}{0.01cm}\end{center}
% 	\rule[3mm]{0.01cm}{25mm}%
%
% \end{comment}

\begin{frame}\frametitle{}
	\begin{center}
	\includegraphics[height=.9\textheight]{\imagedir/doe/DOE-trade-off-table-no-annotation.png}
	\end{center}
\end{frame}

\begin{frame}\frametitle{}
	\begin{center}
	\includegraphics[height=.9\textheight]{\imagedir/doe/DOE-trade-off-table.png}
	\end{center}
\end{frame}

\begin{frame}\frametitle{}
	\begin{center}
	\includegraphics[height=.9\textheight]{\imagedir/doe/DOE-trade-off-table-cell1.png}
	\end{center}
\end{frame}

\begin{frame}\frametitle{For your own case: be prepared to remap your letters ... temporarily}
	\begin{columns}[b]
		\column{0.45\textwidth}
			\emph{Original factor names}
			
			\begin{itemize}
				\item	\textbf{C}: chemical variable
				\item	\textbf{T}: temperature variable
				\item	\textbf{S}: stirring speed variable
			\end{itemize}
			
			\vspace{1cm}
			\textbf{$+$S = CT}
			
		\onslide+<3->{
			\vspace{1cm}
			\textbf{$-$S = CT}
		}
			
		\column{0.45\textwidth}
		\onslide+<2->{
			\emph{Factor names in the table}
			
			\begin{itemize}
				\item	\textbf{A}
				\item	\textbf{B}
				\item	\textbf{C}
			\end{itemize}
			
			\vspace{1cm}
			\textbf{$+$C = AB}
		}
		
		\onslide+<3->{	
			\vspace{1cm}
			\textbf{$-$C = AB}
		}
	\end{columns}	
\end{frame}

\begin{frame}\frametitle{Setting up the half-fraction in 3 factors: using the complementary half}
	\begin{columns}
		\column{0.6\textwidth}
			\begin{center}
				\includegraphics[width=.9\textwidth]{\imagedir/doe/half-fraction-in-3-factors-MOOC-no-labels.png}
			\end{center}
			
		\column{0.55\textwidth}
			\begin{tabulary}{\linewidth}{|c||c|c|c|}\hline 
				\textsf{\relax Experiment } & \textbf{\relax A } & \textbf{\relax B } & \textbf{\relax C = $-$AB } \\
				\hline \textbf{1} & \(-\) & \(-\) & \(-(-)(-) = -\) \\
				\hline \textbf{2} & \(+\) & \(-\) & \(-(+)(-) = +\) \\
				\hline \textbf{3} & \(-\) & \(+\) & \(-(-)(+) = +\) \\
				\hline \textbf{4} & \(+\) & \(+\) & \(-(+)(+) = -\) \\
				\hline
			\end{tabulary}
			
			\small
			\vspace{1cm}
			\textbf{\relax $-$C = AB} can be rewritten as \textbf{\relax C = $-$AB}
			
			\vspace{1cm}
			Notice how the 4 runs generated with \textbf{\relax C = $-$AB} correspond to the closed circles.

			\pause
			\vspace{1cm}
			The 4 runs generated with \textbf{\relax C = $+$AB} correspond\\
			to the open circles.
			
	\end{columns}
\end{frame}

\begin{frame}\frametitle{{\large Creating and understanding fractional factorials from a half-fraction example}}
	\begin{columns}
		\column{0.65\textwidth}
			\begin{center}
				\includegraphics[width=.9\textwidth]{\imagedir/doe/half-fraction-in-3-factors-MOOC-optimum-4.png}
			\end{center}
			
		\onslide+<2->	{
		\column{0.55\textwidth}
			\begin{tabulary}{\linewidth}{cccl}\hline 
				\textsf{\relax Experiment } & \textbf{\relax A } & \textbf{\relax B } & \textbf{\relax C } \\
				\hline 
				1 & \(-\) & \(-\) & \(-\) \\
				\color{myOrange} \textbf{2} & \(+\) & \(-\) & \(-\) \\
				\color{myOrange} \textbf{3} & \(-\) & \(+\) & \(-\) \\
				4 & \(+\) & \(+\) & \(-\) \\
				\color{myOrange} \textbf{5} & \(-\) & \(-\) & \(+\) \\
				6 & \(+\) & \(-\) & \(+\) \\
				7 & \(-\) & \(+\) & \(+\) \\
				\color{myOrange} \textbf{8} & \(+\) & \(+\) & \(+\) \\  \hline
			\end{tabulary}
		}
			
	\end{columns}	
\end{frame}

\begin{frame}\frametitle{{\large Creating and understanding fractional factorials from a half-fraction example}}
	\begin{columns}
		\column{0.65\textwidth}
			\begin{center}
				\includegraphics[width=.9\textwidth]{\imagedir/doe/half-fraction-in-3-factors-MOOC-optimum-4.png}
			\end{center}

		\column{0.55\textwidth}
			\begin{tabulary}{\linewidth}{cccl}\hline 
				\textsf{\relax Experiment } & \textbf{\relax A } & \textbf{\relax B } & \textbf{\relax C } \\
				\hline 
				 & & & \\
				\color{myOrange} \textbf{2} & \(+\) & \(-\) & \(-\) \\
				\color{myOrange} \textbf{3} & \(-\) & \(+\) & \(-\) \\
				 & & & \\
				\color{myOrange} \textbf{5} & \(-\) & \(-\) & \(+\) \\
				 & & & \\
				 & & & \\
				\color{myOrange} \textbf{8} & \(+\) & \(+\) & \(+\) \\ \hline
			\end{tabulary}
	\end{columns}	
\end{frame}

\begin{frame}\frametitle{{\large Creating and understanding fractional factorials from a half-fraction example}}
	\begin{columns}
		\column{0.65\textwidth}
			\begin{center}
				\includegraphics[width=.9\textwidth]{\imagedir/doe/half-fraction-in-3-factors-MOOC-optimum-4.png}
			\end{center}

		\column{0.55\textwidth}
			\begin{tabulary}{\linewidth}{cccl}\hline 
				\textsf{\relax Experiment } & \textbf{\relax A } & \textbf{\relax B } & \textbf{\relax C } \\
				\hline 
				\color{myOrange} \textbf{5} & \(-\) & \(-\) & \(+\) \\
				\color{myOrange} \textbf{2} & \(+\) & \(-\) & \(-\) \\
				\color{myOrange} \textbf{3} & \(-\) & \(+\) & \(-\) \\
				 & & & \\
				 & & & \\
				 & & & \\
				 & & & \\
				\color{myOrange} \textbf{8} & \(+\) & \(+\) & \(+\) \\  \hline
			\end{tabulary}
	\end{columns}	
\end{frame}

\begin{frame}\frametitle{{\large Creating and understanding fractional factorials from a half-fraction example}}
	\begin{columns}
		\column{0.65\textwidth}
			\begin{center}
				\includegraphics[width=.9\textwidth]{\imagedir/doe/half-fraction-in-3-factors-MOOC-optimum-4.png}
			\end{center}

		\column{0.55\textwidth}
			\begin{tabulary}{\linewidth}{cccl}\hline 
				\textsf{\relax Experiment } & \textbf{\relax A } & \textbf{\relax B } & \textbf{\relax C } \\
				\hline 
				 \color{myOrange} \textbf{5} & \(-\) & \(-\) & \(+\) \\
				 \color{myOrange} \textbf{2} & \(+\) & \(-\) & \(-\) \\
				 \color{myOrange} \textbf{3} & \(-\) & \(+\) & \(-\) \\
				 \color{myOrange} \textbf{8} & \(+\) & \(+\) & \(+\) \\ \hline
				  & & & \\
				  & & & \\
				  & & & \\
				  & & & 
			\end{tabulary}
	\end{columns}	
\end{frame}

\begin{frame}\frametitle{{\large Creating and understanding fractional factorials from a half-fraction example}}
	\begin{columns}
		\column{0.65\textwidth}
			\begin{center}
				\includegraphics[width=.9\textwidth]{\imagedir/doe/half-fraction-in-3-factors-MOOC-optimum-4.png}
			\end{center}

		\column{0.55\textwidth}
			\begin{tabulary}{\linewidth}{cccl}\hline 
				\textsf{\relax Experiment } & \textbf{\relax A } & \textbf{\relax B } & \textbf{\relax C = AB } \\
				\hline 
				 \color{myOrange} \textbf{5} & \(-\) & \(-\) & \(+\) \\
				 \color{myOrange} \textbf{2} & \(+\) & \(-\) & \(-\) \\
				 \color{myOrange} \textbf{3} & \(-\) & \(+\) & \(-\) \\
				 \color{myOrange} \textbf{8} & \(+\) & \(+\) & \(+\) \\ \hline
				  & & & \\
				  & & & \\
				  & & & \\
				  & & & 
			\end{tabulary}
	\end{columns}	
\end{frame}

\begin{frame}\frametitle{Our predictive model \color[rgb]{0,0,0}{\scriptsize  $ \hat{y} = $ {\color[rgb]{0.54,0.12,0.03}$b_\text{0}$}  
	+ {\color[rgb]{0.54,0.12,0.03}$b_\text{C}$}$x_\text{C}$ 
	+ {\color[rgb]{0.54,0.12,0.03}$b_\text{T}$}$x_\text{T}$ 
	+ {\color[rgb]{0.54,0.12,0.03}$b_\text{S}$}$x_\text{S}$ 
	+ {\color[rgb]{0.54,0.12,0.03}$b_\text{CT}$}$x_\text{C}x_\text{T}$ 
	+ {\color[rgb]{0.54,0.12,0.03}$b_\text{CS}$}$x_\text{C}x_\text{S}$ 
	+ {\color[rgb]{0.54,0.12,0.03}$b_\text{TS}$}$x_\text{T}x_\text{S}$ 
	+ {\color[rgb]{0.54,0.12,0.03}$b_\text{CTS}$}$x_\text{C}x_\text{T}x_\text{S}$}}  
	
	% The CTS one	
	
	\newcommand{\mw}{\color[rgb]{1,1,1}}
	\newcommand{\mm}{\color{lightgray}}
		
	\vspace{-0.8cm}
	 {\LARGE
	\begin{flalign*}
				&{\mw =}\normalsize  \qquad\,\,\begin{matrix} \mm \,b_\text{0} & \mm \quad b_\text{C} & \mm \quad b_\text{T} & \mm \quad b_\text{S}
				& \mm \,\,\, b_\text{CT} & \mm \,\,\, b_\text{CS} & \mm \,\,\,b_\text{TS} & \mm \,\,b_\text{CTS} 
			\end{matrix}
				\\
		\begin{pmatrix}y_1\\y_2\\y_3\\y_4\\y_5\\y_6\\y_7\\y_8\end{pmatrix} &= 
		\left(\begin{matrix}
			+1 & -1 & -1 & -1 & +1 & +1 & +1 & -1 \\ 
			+1 & +1 & -1 & -1 & -1 & -1 & +1 & +1 \\ 
			+1 & -1 & +1 & -1 & -1 & +1 & -1 & +1 \\
			+1 & +1 & +1 & -1 & +1 & -1 & -1 & -1 \\
			+1 & -1 & -1 & +1 & +1 & -1 & -1 & +1 \\ 
			+1 & +1 & -1 & +1 & -1 & +1 & -1 & -1 \\ 
			+1 & -1 & +1 & +1 & -1 & -1 & +1 & -1 \\
			+1 & +1 & +1 & +1 & +1 & +1 & +1 & +1 \\
		\end{matrix}\right)	
		\begin{pmatrix}{\color[rgb]{0.54,0.12,0.03}b_0}\\
		{\color[rgb]{0.54,0.12,0.03}b_\text{C} }\\
		{\color[rgb]{0.54,0.12,0.03}b_\text{T}} \\
		{\color[rgb]{0.54,0.12,0.03}b_\text{S}} \\
		{\color[rgb]{0.54,0.12,0.03}b_\text{CT}} \\
		{\color[rgb]{0.54,0.12,0.03}b_\text{CS}} \\
		{\color[rgb]{0.54,0.12,0.03}b_\text{TS}} \\
		{\color[rgb]{0.54,0.12,0.03}b_\text{CTS}}\\
		 \end{pmatrix}	 \\
		 \mathbf{y} &= \mathbf{X}{\color[rgb]{0.54,0.12,0.03}\mathbf{b}}	 		
	\end{flalign*}
	}
\end{frame}

\begin{frame}\frametitle{Our predictive model \color[rgb]{0,0,0}{\scriptsize  $ \hat{y} = $ {\color[rgb]{0.54,0.12,0.03}$b_\text{0}$}  
	+ {\color[rgb]{0.54,0.12,0.03}$b_\text{A}$}$x_\text{A}$ 
	+ {\color[rgb]{0.54,0.12,0.03}$b_\text{B}$}$x_\text{B}$ 
	+ {\color[rgb]{0.54,0.12,0.03}$b_\text{C}$}$x_\text{C}$ 
	+ {\color[rgb]{0.54,0.12,0.03}$b_\text{AB}$}$x_\text{A}x_\text{B}$ 
	+ {\color[rgb]{0.54,0.12,0.03}$b_\text{AC}$}$x_\text{A}x_\text{C}$ 
	+ {\color[rgb]{0.54,0.12,0.03}$b_\text{BC}$}$x_\text{B}x_\text{C}$ 
	+ {\color[rgb]{0.54,0.12,0.03}$b_\text{ABC}$}$x_\text{A}x_\text{B}x_\text{C}$}}   % The ABC, with all the "-1" and "+1"
	
	
	\newcommand{\mw}{\color[rgb]{1,1,1}}
	\newcommand{\mm}{\color{lightgray}}
		
	\vspace{-0.8cm}
	{\LARGE
	\begin{flalign*}
			&{\mw =}\normalsize  \qquad\,\,\begin{matrix} \mm \,b_\text{0} & \mm \quad b_\text{A} & \mm \quad b_\text{B} & \mm \quad b_\text{C}
			& \mm \,\,\, b_\text{AB} & \mm \,\,\, b_\text{AC} & \mm \,\,\,b_\text{BC} & \mm \,\,b_\text{ABC} 
		\end{matrix}
			\\ 
		\begin{pmatrix}y_1\\y_2\\y_3\\y_4\\y_5\\y_6\\y_7\\y_8\end{pmatrix} &= 
		\left(\begin{matrix}
			+1 & -1 & -1 & -1 & +1 & +1 & +1 & -1 \\ 
			+1 & +1 & -1 & -1 & -1 & -1 & +1 & +1 \\ 
			+1 & -1 & +1 & -1 & -1 & +1 & -1 & +1 \\
			+1 & +1 & +1 & -1 & +1 & -1 & -1 & -1 \\
			+1 & -1 & -1 & +1 & +1 & -1 & -1 & +1 \\ 
			+1 & +1 & -1 & +1 & -1 & +1 & -1 & -1 \\ 
			+1 & -1 & +1 & +1 & -1 & -1 & +1 & -1 \\
			+1 & +1 & +1 & +1 & +1 & +1 & +1 & +1 \\
		 \end{matrix}\right)	
		\begin{pmatrix}{\color[rgb]{0.54,0.12,0.03}b_0}\\
		{\color[rgb]{0.54,0.12,0.03}b_\text{A} }\\
		{\color[rgb]{0.54,0.12,0.03}b_\text{B}} \\
		{\color[rgb]{0.54,0.12,0.03}b_\text{C}} \\
		{\color[rgb]{0.54,0.12,0.03}b_\text{AB}} \\
		{\color[rgb]{0.54,0.12,0.03}b_\text{AC}} \\
		{\color[rgb]{0.54,0.12,0.03}b_\text{BC}} \\
		{\color[rgb]{0.54,0.12,0.03}b_\text{ABC}}\\
		 \end{pmatrix}\\
			 \mathbf{y} &= \mathbf{X}{\color[rgb]{0.54,0.12,0.03}\mathbf{b}}	 		
	\end{flalign*}
	}	
\end{frame}

\begin{frame}\frametitle{Our predictive model \color[rgb]{0,0,0}{\scriptsize  $ \hat{y} = $ {\color[rgb]{0.54,0.12,0.03}$b_\text{0}$}  
	+ {\color[rgb]{0.54,0.12,0.03}$b_\text{A}$}$x_\text{A}$ 
	+ {\color[rgb]{0.54,0.12,0.03}$b_\text{B}$}$x_\text{B}$ 
	+ {\color[rgb]{0.54,0.12,0.03}$b_\text{C}$}$x_\text{C}$ 
	+ {\color[rgb]{0.54,0.12,0.03}$b_\text{AB}$}$x_\text{A}x_\text{B}$ 
	+ {\color[rgb]{0.54,0.12,0.03}$b_\text{AC}$}$x_\text{A}x_\text{C}$ 
	+ {\color[rgb]{0.54,0.12,0.03}$b_\text{BC}$}$x_\text{B}x_\text{C}$ 
	+ {\color[rgb]{0.54,0.12,0.03}$b_\text{ABC}$}$x_\text{A}x_\text{B}x_\text{C}$}}  % The ABC, now just with signs
	
	\newcommand{\mw}{\color[rgb]{1,1,1}}
	\newcommand{\mm}{\color{lightgray}}
	
	\vspace{-0.8cm}
	{\LARGE
	\begin{flalign*}
			&{\mw =}\normalsize  \qquad\,\,\begin{matrix} \mm b_\text{0} & \mm b_\text{A} & \mm \,b_\text{B} & \mm \,b_\text{C}
			& \mm b_\text{AB} & \mm b_\text{AC} & \mm \hspace{-0.03cm}b_\text{BC} & \mm \hspace{-0.13cm}b_\text{ABC} 
		\end{matrix}
			\\
		\begin{pmatrix}y_1\\y_2\\y_3\\y_4\\y_5\\y_6\\y_7\\y_8\end{pmatrix} &= 
		\left(\begin{matrix}
			+ & - & - & - & + & + & + & - \\ 
			+ & + & - & - & - & - & + & + \\ 
			+ & - & + & - & - & + & - & + \\
			+ & + & + & - & + & - & - & - \\
			+ & - & - & + & + & - & - & + \\ 
			+ & + & - & + & - & + & - & - \\ 
			+ & - & + & + & - & - & + & - \\
			+ & + & + & + & + & + & + & + \\
		 \end{matrix}\right)	
		\begin{pmatrix}{\color[rgb]{0.54,0.12,0.03}b_0}\\
		{\color[rgb]{0.54,0.12,0.03}b_\text{A} }\\
		{\color[rgb]{0.54,0.12,0.03}b_\text{B}} \\
		{\color[rgb]{0.54,0.12,0.03}b_\text{C}} \\
		{\color[rgb]{0.54,0.12,0.03}b_\text{AB}} \\
		{\color[rgb]{0.54,0.12,0.03}b_\text{AC}} \\
		{\color[rgb]{0.54,0.12,0.03}b_\text{BC}} \\
		{\color[rgb]{0.54,0.12,0.03}b_\text{ABC}}\\
		 \end{pmatrix}\\
			 \mathbf{y} &= \mathbf{X}{\color[rgb]{0.54,0.12,0.03}\mathbf{b}} & % trailing "&" is required for flalign
	\end{flalign*}
	}	
\end{frame}

\begin{frame}\frametitle{Our predictive model \color[rgb]{0,0,0}{\scriptsize  $ \hat{y} = $ {\color[rgb]{0.54,0.12,0.03}$b_\text{0}$}  
	+ {\color[rgb]{0.54,0.12,0.03}$b_\text{A}$}$x_\text{A}$ 
	+ {\color[rgb]{0.54,0.12,0.03}$b_\text{B}$}$x_\text{B}$ 
	+ {\color[rgb]{0.54,0.12,0.03}$b_\text{C}$}$x_\text{C}$ 
	+ {\color[rgb]{0.54,0.12,0.03}$b_\text{AB}$}$x_\text{A}x_\text{B}$ 
	+ {\color[rgb]{0.54,0.12,0.03}$b_\text{AC}$}$x_\text{A}x_\text{C}$ 
	+ {\color[rgb]{0.54,0.12,0.03}$b_\text{BC}$}$x_\text{B}x_\text{C}$ 
	+ {\color[rgb]{0.54,0.12,0.03}$b_\text{ABC}$}$x_\text{A}x_\text{B}x_\text{C}$}}

	\newcommand{\mw}{\color[rgb]{1,1,1}}
	\newcommand{\mm}{\color{lightgray}}
	\vspace{-0.8cm}
	{\LARGE
	\begin{flalign*}
			&{\mw =}\normalsize  \qquad\,\,\begin{matrix} \mm b_\text{0} & \mm b_\text{A} & \mm \,b_\text{B} & \mm \,b_\text{C}
			& \mm b_\text{AB} & \mm b_\text{AC} & \mm \hspace{-0.03cm}b_\text{BC} & \mm \hspace{-0.13cm}b_\text{ABC} 
		\end{matrix}
			\\
		\begin{pmatrix}\\y_2\\y_3\\\\y_5\\\\\\y_8\end{pmatrix} &= 
		\left(\begin{matrix}
			  &   &   &   &   &   &   &  \mw -  \\ 
			+ & + & - & - & - & - & + & + \\ 
			+ & - & + & - & - & + & - & + \\
			  &   &   &   &   &   &   &    \\ 
			+ & - & - & + & + & - & - & + \\ 
			  &   &   &   &   &   &   &    \\ 
			  &   &   &   &   &   &   &    \\ 
			+ & + & + & + & + & + & + & + \\
		 \end{matrix}\right)		 
		\begin{pmatrix}{\color[rgb]{0.54,0.12,0.03}b_0}\\
		{\color[rgb]{0.54,0.12,0.03} b_\text{A} }\\
		{\color[rgb]{0.54,0.12,0.03}b_\text{B}} \\
		{\color[rgb]{0.54,0.12,0.03}b_\text{C}} \\
		{\color[rgb]{0.54,0.12,0.03}b_\text{AB}} \\
		{\color[rgb]{0.54,0.12,0.03}b_\text{AC}} \\
		{\color[rgb]{0.54,0.12,0.03}b_\text{BC}} \\
		{\color[rgb]{0.54,0.12,0.03}b_\text{ABC}}\\
		 \end{pmatrix}\\
			 \mathbf{y} &= \mathbf{X}{\color[rgb]{0.54,0.12,0.03}\mathbf{b}}& % trailing "&" is required for flalign	 		
	\end{flalign*}
	}
\end{frame}

\begin{frame}\frametitle{Our predictive model \color[rgb]{0,0,0}{\scriptsize  $ \hat{y} = $ {\color[rgb]{0.54,0.12,0.03}$b_\text{0}$}  
	+ {\color[rgb]{0.54,0.12,0.03}$b_\text{A}$}$x_\text{A}$ 
	+ {\color[rgb]{0.54,0.12,0.03}$b_\text{B}$}$x_\text{B}$ 
	+ {\color[rgb]{0.54,0.12,0.03}$b_\text{C}$}$x_\text{C}$ 
	+ {\color[rgb]{0.54,0.12,0.03}$b_\text{AB}$}$x_\text{A}x_\text{B}$ 
	+ {\color[rgb]{0.54,0.12,0.03}$b_\text{AC}$}$x_\text{A}x_\text{C}$ 
	+ {\color[rgb]{0.54,0.12,0.03}$b_\text{BC}$}$x_\text{B}x_\text{C}$ 
	+ {\color[rgb]{0.54,0.12,0.03}$b_\text{ABC}$}$x_\text{A}x_\text{B}x_\text{C}$}}

	\newcommand{\mw}{\color[rgb]{1,1,1}}
	\newcommand{\mm}{\color{lightgray}}
	\vspace{-0.8cm}
	{\LARGE
	\begin{flalign*}
			&{\mw =}\normalsize  \qquad\,\,\begin{matrix} \mm b_\text{0} & \mm b_\text{A} & \mm \,b_\text{B} & \mm \,b_\text{C}
			& \mm b_\text{AB} & \mm b_\text{AC} & \mm \hspace{-0.03cm}b_\text{BC} & \mm \hspace{-0.13cm}b_\text{ABC} 
		\end{matrix}\\
		\begin{pmatrix}y_5\\y_2\\y_3\\\\\\\\\\y_8\end{pmatrix} &= 
		\left(\begin{matrix}
			+  & -  & -  & +  & +  & -  & -  & +  \\ 
			+  & +  & -  & -  & -  & -  & +  & +  \\ 
			+  & -  & +  & -  & -  & +  & -  & +  \\
			   &    &    &    &    &    &    &    \\ 
			   &    &    &    &    &    &    &  \mw-   \\ 
			   &    &    &    &    &    &    &    \\ 
			   &    &    &    &    &    &    &    \\ 
			+  & +  & +  & +  & +  & +  & +  & +  \\
		 \end{matrix}\right)		 
		\begin{pmatrix}{\color[rgb]{0.54,0.12,0.03}b_0}\\
		{\color[rgb]{0.54,0.12,0.03}b_\text{A} }\\
		{\color[rgb]{0.54,0.12,0.03}b_\text{B}} \\
		{\color[rgb]{0.54,0.12,0.03}b_\text{C}} \\
		{\color[rgb]{0.54,0.12,0.03}b_\text{AB}} \\
		{\color[rgb]{0.54,0.12,0.03}b_\text{AC}} \\
		{\color[rgb]{0.54,0.12,0.03}b_\text{BC}} \\
		{\color[rgb]{0.54,0.12,0.03}b_\text{ABC}}\\
		 \end{pmatrix}\\
			 \mathbf{y} &= \mathbf{X}{\color[rgb]{0.54,0.12,0.03}\mathbf{b}} & % trailing "&" is required for		
	\end{flalign*}
	}
\end{frame}

\begin{frame}\frametitle{Our predictive model \color[rgb]{0,0,0}{\scriptsize  $ \hat{y} = $ {\color[rgb]{0.54,0.12,0.03}$b_\text{0}$}  
	+ {\color[rgb]{0.54,0.12,0.03}$b_\text{A}$}$x_\text{A}$ 
	+ {\color[rgb]{0.54,0.12,0.03}$b_\text{B}$}$x_\text{B}$ 
	+ {\color[rgb]{0.54,0.12,0.03}$b_\text{C}$}$x_\text{C}$ 
	+ {\color[rgb]{0.54,0.12,0.03}$b_\text{AB}$}$x_\text{A}x_\text{B}$ 
	+ {\color[rgb]{0.54,0.12,0.03}$b_\text{AC}$}$x_\text{A}x_\text{C}$ 
	+ {\color[rgb]{0.54,0.12,0.03}$b_\text{BC}$}$x_\text{B}x_\text{C}$ 
	+ {\color[rgb]{0.54,0.12,0.03}$b_\text{ABC}$}$x_\text{A}x_\text{B}x_\text{C}$}}
	 
	\newcommand{\mw}{\color[rgb]{1,1,1}}
	\newcommand{\mm}{\color{lightgray}}
	\vspace{-0.8cm}
	{\LARGE
	\begin{flalign*}
		&{\mw =}\normalsize  \qquad\,\,\begin{matrix} \mm b_\text{0} & \mm b_\text{A} & \mm \,b_\text{B} & \mm \,b_\text{C}
		& \mm b_\text{AB} & \mm b_\text{AC} & \mm \hspace{-0.03cm}b_\text{BC} & \mm \hspace{-0.13cm}b_\text{ABC} 
	\end{matrix}
		\\
		\begin{pmatrix}y_5\\y_2\\y_3\\y_8\\\\\\\\\\\end{pmatrix} &= 
		\left(\begin{matrix}
			+  & -  & -  & +  & +  & -  & -  & +  \\ 
			+  & +  & -  & -  & -  & -  & +  & +  \\ 
			+  & -  & +  & -  & -  & +  & -  & +  \\
			+  & +  & +  & +  & +  & +  & +  & +  \\
			  &    &    &    &    &    &    &    \\ 
			  &    &    &    &    &    &    &  \mw-   \\ 
			  &    &    &    &    &    &    &    \\ 
			  &    &    &    &    &    &    &    \\ 			
		 \end{matrix}\right)		 
		\begin{pmatrix}{\color[rgb]{0.54,0.12,0.03}b_0}\\
		{\color[rgb]{0.54,0.12,0.03}b_\text{A} }\\
		{\color[rgb]{0.54,0.12,0.03}b_\text{B}} \\
		{\color[rgb]{0.54,0.12,0.03}b_\text{C}} \\
		{\color[rgb]{0.54,0.12,0.03}b_\text{AB}} \\
		{\color[rgb]{0.54,0.12,0.03}b_\text{AC}} \\
		{\color[rgb]{0.54,0.12,0.03}b_\text{BC}} \\
		{\color[rgb]{0.54,0.12,0.03}b_\text{ABC}}\\
		 \end{pmatrix}\\
			 \mathbf{y} &= \mathbf{X}{\color[rgb]{0.54,0.12,0.03}\mathbf{b}}& % trailing "&" is required for			 		
	\end{flalign*}
	}
\end{frame}

\begin{frame}\frametitle{Our predictive model \color[rgb]{0,0,0}{\scriptsize  $ \hat{y} = $ {\color[rgb]{0.54,0.12,0.03}$b_\text{0}$}  
	+ {\color[rgb]{0.54,0.12,0.03}$b_\text{A}$}$x_\text{A}$ 
	+ {\color[rgb]{0.54,0.12,0.03}$b_\text{B}$}$x_\text{B}$ 
	+ {\color[rgb]{0.54,0.12,0.03}$b_\text{C}$}$x_\text{C}$ 
	+ {\color[rgb]{0.54,0.12,0.03}$b_\text{AB}$}$x_\text{A}x_\text{B}$ 
	+ {\color[rgb]{0.54,0.12,0.03}$b_\text{AC}$}$x_\text{A}x_\text{C}$ 
	+ {\color[rgb]{0.54,0.12,0.03}$b_\text{BC}$}$x_\text{B}x_\text{C}$ 
	+ {\color[rgb]{0.54,0.12,0.03}$b_\text{ABC}$}$x_\text{A}x_\text{B}x_\text{C}$}}
	 
	\newcommand{\mw}{\color[rgb]{1,1,1}}
	\newcommand{\mm}{\color{lightgray}}
	\vspace{-0.8cm}
	{\LARGE
	\begin{flalign*}
		&{\mw =}\normalsize  \qquad\,\,\begin{matrix} \mm b_\text{0} & \mm b_\text{A} & \mm \,b_\text{B} & \mm \,b_\text{C}
		& \mm b_\text{AB} & \mm b_\text{AC} & \mm \hspace{-0.03cm}b_\text{BC} & \mm \hspace{-0.13cm}b_\text{ABC} 
	\end{matrix}
		\\
		\begin{pmatrix}y_5\\y_2\\y_3\\y_8\end{pmatrix} &= 
		\left(\begin{matrix}
			+  & -  & -  & +  & +  & -  & -  & +  \\ 
			+  & +  & -  & -  & -  & -  & +  & +  \\ 
			+  & -  & +  & -  & -  & +  & -  & +  \\
			+  & +  & +  & +  & +  & +  & +  & +  \\
		 \end{matrix}\right)	{	 \small
		\begin{pmatrix}{\color[rgb]{0.54,0.12,0.03}b_0}\\
		{\color[rgb]{0.54,0.12,0.03}b_\text{A} }\\
		{\color[rgb]{0.54,0.12,0.03}b_\text{B}} \\
		{\color[rgb]{0.54,0.12,0.03}b_\text{C}} \\
		{\color[rgb]{0.54,0.12,0.03}b_\text{AB}} \\
		{\color[rgb]{0.54,0.12,0.03}b_\text{AC}} \\
		{\color[rgb]{0.54,0.12,0.03}b_\text{BC}} \\
		{\color[rgb]{0.54,0.12,0.03}b_\text{ABC}}\\
		 \end{pmatrix}}\\
			 \mathbf{y} &= \mathbf{X}{\color[rgb]{0.54,0.12,0.03}\mathbf{b}}& % trailing "&" is required for			 		
	\end{flalign*}
	}
\end{frame}

\begin{frame}\frametitle{We have ``aliasing'' taking place here}
	\begin{columns}
		\column{0.48\textwidth}	
			 Who is \emph{Jorge Mario Bergoglio}?
		
		\onslide+<2->	{
		\column{0.48\textwidth}
			\includegraphics[width=.75\textwidth]{../4B/Supporting files/Wikipedia-Pope_Francis_among_the_people_at_St._Peter's_Square_-_12_May_2013.png}		

			{\scriptsize  \href{https://en.wikipedia.org/wiki/Pope\_Francis}{Wikipedia}}
			}
	\end{columns}
	
\end{frame}

\begin{frame}\frametitle{We have ``aliasing'' taking place here}
	\begin{columns}
		\column{0.48\textwidth}	
			 Who is \emph{Kevin George Dunn}?		

		\column{0.48\textwidth}
			\begin{itemize}
				\item	My email address
				\item	A username for a website
			\end{itemize}
	\end{columns}
	
\end{frame}

\begin{frame}\frametitle{Aliasing: when we have more than one name for the same thing}
	
	\vspace{1cm}
	What is aliased in this experimental design (i.e. which columns are the same)?
		
		\vspace{0.5cm}
		\begin{itemize}
			\item	\textbf{A=BC}
			
			\onslide+<2->	{
			\vspace{1cm}
			\item	\textbf{B=AC}
			
			\vspace{1cm}
			\item	\textbf{C=AB} (this was intentional: read from the ``tradeoff'' table)
			
			\vspace{1cm}
			\item	\textbf{ABC = Intercept} (the intercept is indicated as $b_0$)
			}
		\end{itemize}

\end{frame}

\begin{frame}\frametitle{Remove the aliases by collapsing identical columns}
	 
	\newcommand{\mw}{\color[rgb]{1,1,1}}
	\newcommand{\mm}{\color{lightgray}}
	\vspace{-0.8cm}
	{\LARGE
	\begin{flalign*}
		&{\mw =}\normalsize  \qquad\,\,\begin{matrix} \mm b_\text{0} & \mm b_\text{A} & \mm \,b_\text{B} & \mm \,b_\text{C}
		& \mm b_\text{AB} & \mm b_\text{AC} & \mm \hspace{-0.03cm}b_\text{BC} & \mm \hspace{-0.13cm}b_\text{ABC} 
	\end{matrix}
		\\
		\begin{pmatrix}y_5\\y_2\\y_3\\y_8\end{pmatrix} &= 
		\left(\begin{matrix}
			+  & -  & -  & +  & +  & -  & -  & +  \\ 
			+  & +  & -  & -  & -  & -  & +  & +  \\ 
			+  & -  & +  & -  & -  & +  & -  & +  \\
			+  & +  & +  & +  & +  & +  & +  & +  \\
		 \end{matrix}\right)	{	 \small
		\begin{pmatrix}{\color[rgb]{0.54,0.12,0.03}b_0}\\
		{\color[rgb]{0.54,0.12,0.03}b_\text{A} }\\
		{\color[rgb]{0.54,0.12,0.03}b_\text{B}} \\
		{\color[rgb]{0.54,0.12,0.03}b_\text{C}} \\
		{\color[rgb]{0.54,0.12,0.03}b_\text{AB}} \\
		{\color[rgb]{0.54,0.12,0.03}b_\text{AC}} \\
		{\color[rgb]{0.54,0.12,0.03}b_\text{BC}} \\
		{\color[rgb]{0.54,0.12,0.03}b_\text{ABC}}\\
		 \end{pmatrix}}& % trailing "&" is required for			 		
	\end{flalign*}
	}
	\begin{itemize}
		\item	4 equations (rows) in 8 unknowns {\color[rgb]{0.54,0.12,0.03}(the entries shown in the last matrix)}
		\onslide+<2->	{
			\item	a
			\item	s
			\item	d
		}
	\end{itemize}
\end{frame}

\begin{frame}\frametitle{Remove the aliases by collapsing identical columns}
	 
	\newcommand{\mw}{\color[rgb]{1,1,1}}
	\newcommand{\mm}{\color{lightgray}}
	\vspace{-0.8cm}
	{\LARGE
	\begin{flalign*}
		&{\mw =}\normalsize  \qquad\,\,\begin{matrix} \mm b_\text{0} + b_\text{ABC} & \mm b_\text{A} + b_\text{BC} & \mm \,b_\text{B} + b_\text{AC} & \mm \,\,b_\text{C}+ b_\text{AB} & \mm  & \mm \hspace{-0.03cm} & 
	\end{matrix}
		\\
		\begin{pmatrix}y_5\\y_2\\y_3\\y_8\end{pmatrix} &= 
		\left(\begin{matrix}
			+  & \qquad -  & \qquad -  & \qquad +  \\ 
			+  & \qquad +  & \qquad -  & \qquad -   \\ 
			+  & \qquad -  & \qquad +  & \qquad -   \\
			+  & \qquad +  & \qquad +  & \qquad +   \\
		 \end{matrix}\,\,\,\right)		 
		\begin{pmatrix}
		{\color[rgb]{0.54,0.12,0.03}\,\,b_0 \, + b_\text{ABC}}\\
		{\color[rgb]{0.54,0.12,0.03}b_\text{A} + b_\text{BC}}\\
		{\color[rgb]{0.54,0.12,0.03}b_\text{B} + b_\text{AC}} \\
		{\color[rgb]{0.54,0.12,0.03}b_\text{C} + b_\text{AB}}
		 \end{pmatrix}	 & % trailing "&" is required for			 		
	\end{flalign*}
	}
	\begin{itemize}
		\item	4 equations (rows) in 4 unknowns {\color[rgb]{0.54,0.12,0.03}(the entries in the last matrix)}
		\onslide+<2->	{
			\item	a
			\item	s
			\item	d
		}
	\end{itemize}
\end{frame}

\begin{frame}\frametitle{Remove the aliases by collapsing identical columns}
	 
	\newcommand{\mw}{\color[rgb]{1,1,1}}
	\newcommand{\mm}{\color{lightgray}}
	\vspace{-0.8cm}
	{\LARGE
	\begin{flalign*}
		&{\mw =}\normalsize  \qquad\,\,\begin{matrix} \mm b_\text{0} + b_\text{ABC} & \mm b_\text{A} + b_\text{BC} & \mm \,b_\text{B} + b_\text{AC} & \mm \,\,b_\text{C}+ b_\text{AB} & \mm  & \mm \hspace{-0.03cm} & 
	\end{matrix}
		\\
		\begin{pmatrix}y_5\\y_2\\y_3\\y_8\end{pmatrix} &= 
		\left(\begin{matrix}
			+  & \qquad -  & \qquad -  & \qquad +  \\ 
			+  & \qquad +  & \qquad -  & \qquad -   \\ 
			+  & \qquad -  & \qquad +  & \qquad -   \\
			+  & \qquad +  & \qquad +  & \qquad +   \\
		 \end{matrix}\,\,\,\right)		 
		\begin{pmatrix}
		{\color[rgb]{0.54,0.12,0.03}\,\,b_0 \, + b_\text{ABC}}\\
		{\color[rgb]{0.54,0.12,0.03}\hat{b}_\text{A}}   \\
		{\color[rgb]{0.54,0.12,0.03}b_\text{B} + b_\text{AC}} \\
		{\color[rgb]{0.54,0.12,0.03}b_\text{C} + b_\text{AB}}
		 \end{pmatrix}	 & % trailing "&" is required for			 		
	\end{flalign*}
	}
	\begin{itemize}
		\item	Let's call the merged coefficient ${\color[rgb]{0.54,0.12,0.03}\hat{b}_\text{A}} = b_\text{A} + b_\text{BC}$
		\onslide+<2->	{
			\item	a
			\item	s
			\item	d
		}
	\end{itemize}
\end{frame}

\begin{frame}\frametitle{Remove the aliases by collapsing identical columns}
	 
	\newcommand{\mw}{\color[rgb]{1,1,1}}
	\newcommand{\mm}{\color{lightgray}}
	\vspace{-0.8cm}
	{\LARGE
	\begin{flalign*}
		&{\mw =}\normalsize  \qquad\,\,\begin{matrix} \mm b_\text{0} + b_\text{ABC} & \mm b_\text{A} + b_\text{BC} & \mm \,b_\text{B} + b_\text{AC} & \mm \,\,b_\text{C}+ b_\text{AB} & \mm  & \mm \hspace{-0.03cm} & 
	\end{matrix}
		\\
		\begin{pmatrix}y_5\\y_2\\y_3\\y_8\end{pmatrix} &= 
		\left(\begin{matrix}
			+  & \qquad -  & \qquad -  & \qquad +  \\ 
			+  & \qquad +  & \qquad -  & \qquad -   \\ 
			+  & \qquad -  & \qquad +  & \qquad -   \\
			+  & \qquad +  & \qquad +  & \qquad +   \\
		 \end{matrix}\,\,\,\right)		 
		\begin{pmatrix}
		{\color[rgb]{0.54,0.12,0.03}\,\,b_0 \, + b_\text{ABC}}\\
		{\color[rgb]{0.54,0.12,0.03}\hat{b}_\text{A}}   \\
		{\color[rgb]{0.54,0.12,0.03}\hat{b}_\text{B}} \\
		{\color[rgb]{0.54,0.12,0.03}b_\text{C} + b_\text{AB}}
		 \end{pmatrix}	 & % trailing "&" is required for			 		
	\end{flalign*}
	}
	\begin{itemize}
		\item	Let's call the merged coefficient ${\color[rgb]{0.54,0.12,0.03}\hat{b}_\text{A}} = b_\text{A} + b_\text{BC}$
		\item	${\color[rgb]{0.54,0.12,0.03}\hat{b}_\text{B}} = b_\text{B} + b_\text{AC}$
		\onslide+<2->	{
			\item	s
			\item	d
		}
	\end{itemize}
\end{frame}

\begin{frame}\frametitle{Remove the aliases by collapsing identical columns}
	 
	\newcommand{\mw}{\color[rgb]{1,1,1}}
	\newcommand{\mm}{\color{lightgray}}
	\vspace{-0.8cm}
	{\LARGE
	\begin{flalign*}
		&{\mw =}\normalsize  \qquad\,\,\begin{matrix} \mm b_\text{0} + b_\text{ABC} & \mm b_\text{A} + b_\text{BC} & \mm \,b_\text{B} + b_\text{AC} & \mm \,\,b_\text{C}+ b_\text{AB} & \mm  & \mm \hspace{-0.03cm} & 
	\end{matrix}
		\\
		\begin{pmatrix}y_5\\y_2\\y_3\\y_8\end{pmatrix} &= 
		\left(\begin{matrix}
			+  & \qquad -  & \qquad -  & \qquad +  \\ 
			+  & \qquad +  & \qquad -  & \qquad -   \\ 
			+  & \qquad -  & \qquad +  & \qquad -   \\
			+  & \qquad +  & \qquad +  & \qquad +   \\
		 \end{matrix}\,\,\,\right)		 
		\begin{pmatrix}
		{\color[rgb]{0.54,0.12,0.03}\hat{b}_0}\\
		{\color[rgb]{0.54,0.12,0.03}\hat{b}_\text{A}}   \\
		{\color[rgb]{0.54,0.12,0.03}\hat{b}_\text{B}} \\
		{\color[rgb]{0.54,0.12,0.03}\hat{b}_\text{C}}
		 \end{pmatrix}	 & % trailing "&" is required for			 		
	\end{flalign*}
	}
	\begin{itemize}
		\item	${\color[rgb]{0.54,0.12,0.03}\hat{b}_\text{A}} = b_\text{A} + b_\text{BC}$
		\item	${\color[rgb]{0.54,0.12,0.03}\hat{b}_\text{B}} = b_\text{B} + b_\text{AC}$
		\item	${\color[rgb]{0.54,0.12,0.03}\hat{b}_\text{C}} = b_\text{C} + b_\text{AB}$
		\item	${\color[rgb]{0.54,0.12,0.03}\hat{b}_\text{0}} = b_\text{0} + b_\text{ABC}$		
		
	\end{itemize}
\end{frame}

\begin{frame}\frametitle{Now that we understand aliasing; how can we work with it in our system?}
	
	We'd like to take advantage of doing half the work, but still get the most benefit. But, 
	recall we showed that this will lead to a loss of some accuracy:
	
	\vspace{0.2cm}
	
	
	\vspace{-0.2cm}
	\begin{columns}[T]
		\column{0.35\textwidth}
			\begin{center}\textbf{Full factorial model}\end{center}
				\begin{align*}
					\hat{y} &= {\color{myOrange}11.25}\\
							&+ {\color{myOrange}6.25\,}x_\text{A}\\
							&+ {\color{blue}0.75\,}x_\text{B}\\
							&- {\color{myOrange}7.25\,}x_\text{C}\\
							&+ {\color{myOrange}0.25\,}x_\text{A}x_\text{B}\\
							&- {\color{blue}6.75\,}x_\text{A}x_\text{C}\\
							&- {\color{myOrange}0.25\,}x_\text{B}x_\text{C}\\
							&- {\color{myOrange}0.25\,}x_\text{A}x_\text{B}x_\text{C}		
				\end{align*}
		\column{0.02\textwidth}
			\vspace{1cm}
			\rule[3mm]{0.03cm}{65mm}
		\column{0.35\textwidth}
			\begin{center}\textbf{Fractional factorial model}\end{center}				
				\begin{align*}
					\hat{y} &= {\color{myOrange}11.0}\\
							&+ {\color{myOrange}6.0\,}x_\text{A}\\
							&- {\color{blue}6.0\,}x_\text{B}\\
							&- {\color{myOrange}7.0\,}x_\text{C}\\
							& \color{lightgray}+ \cancel{b_\text{AB}\,x_\text{A}x_\text{B}}\\
							& \color{lightgray}+ \cancel{b_\text{AC}\,x_\text{A}x_\text{C}}\\
							& \color{lightgray}+ \cancel{b_\text{BC}\,x_\text{B}x_\text{C}}\\
							& \color{lightgray}+ \cancel{b_\text{ABC}\,x_\text{A}x_\text{B}x_\text{C}}\\
				\end{align*}

	\end{columns}
\end{frame}

