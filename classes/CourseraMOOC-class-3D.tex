\begin{frame}\frametitle{Solar panel case study}
	
	\begin{columns}[T]
		\column{0.45\textwidth}
			\includegraphics[width=0.7\textwidth]{\imagedir/statistics/flickr-Box-Hunter-Hunter-cover-3056749047_1c1f633fcb_o.jpg}
			
			{\scriptsize (p. 230 in Box, Hunter and Hunter, 2$^\text{nd}$ ed)}
			
		\column{0.48\textwidth}
			\includegraphics[width=\textwidth]{\imagedir/doe/examples/solar-panel-mendelu-cz-website.png}
			
			
			\see{\href{http://yint.org/solar-panel-study}{http://yint.org/solar-panel-study}}
	\end{columns}
\end{frame}

\begin{frame}\frametitle{\includegraphics[width=0.3\textwidth]{\imagedir/doe/examples/advice-logo.png} when experimenting with computer simulations}
	\begin{columns}[T]
		\column{0.48\textwidth}
		
			\textbf{The same as regular experiments}:
			
			\vspace{12pt}
			\begin{itemize}
				\item	you must follow a systematic method
				\item	don't ``play around'' with the software: trial-and-error
			\end{itemize}
			
			%\begin{center}\rule[8mm]{4cm}{0.01cm}\end{center}
			
			\vspace{18pt}

			 \fbox{\parbox[b][7em][t]{\textwidth}{
			 	{\footnotesize
				 	Many experiments are simulations:
					\begin{itemize}
						\item	bridge/building design
						\item	chemical factory design
						\item	improve traffic light timing and queuing
			 			\item	test a stock market buy/sell strategy
			 		\end{itemize}}
			 } }
			 
				
		\column{0.01\textwidth}
			\rule[3mm]{0.01cm}{25mm}%
			
		\column{0.48\textwidth}
		\onslide+<2->{
			\textbf{Different to regular experiments:}
			\vspace{12pt}
			\begin{enumerate}
				\item	We can often run computer simulations in parallel
				\item	Computer experiments (mostly$^\ast$) are deterministic
					\begin{itemize}
						\item	i.e. if you repeat the experiments, you get the identical results
						\item	this indicates there are no disturbances that affect the outcome
						\item	this implies you do not need to randomize the order
						\item	or even repeat experiments!
						
					\end{itemize}
			\end{enumerate}
			{\scriptsize $^\ast$ {\emph{except those that have a random component}}}
		}
	\end{columns}
	
	
\end{frame}

\begin{frame}\frametitle{Solar panel case study}
	
	\begin{columns}[T]
		\column{0.45\textwidth}
			
			\includegraphics[width=\textwidth]{\imagedir/doe/examples/solar-panel-mendelu-cz-website.png}
			
			
			\href{http://yint.org/solar-panel-study}{http://yint.org/solar-panel-study}
		\column{0.45\textwidth}
			The factors are:
			\begin{itemize}
				\item	\textbf{A} = total daily \href{https://en.wikipedia.org/wiki/Insolation}{insolation} (sunlight received)
				\item	\textbf{B} = storage tank capacity
				\item	\textbf{C} = water flow rate
				\item	\textbf{D} = \href{https://en.wikipedia.org/wiki/Intermittent\_energy\_source }{solar intermittency}
			\end{itemize}
			\pause
			\vspace{12pt}
			The outcome variables were:
			\begin{itemize}
				\item	$y_1$ = collection efficiency
				\item	$y_2$ = energy delivery efficiency
			\end{itemize}
			\pause
			\vspace{12pt}
			Total experiments required = \pause $2^4 = 16$ 
			
	\end{columns}
\end{frame}

\begin{frame}\frametitle{R's automatic expansion of model terms}
	\texttt{lm(y {\raise.17ex\hbox{$\scriptstyle\mathtt{\sim}$}} A*B)}\\
		\qquad expands into: {\color{blue}\texttt{lm(y {\raise.17ex\hbox{$\scriptstyle\mathtt{\sim}$}} A + B + A*B)}}
		
	\vspace{24pt}
	\texttt{lm(y {\raise.17ex\hbox{$\scriptstyle\mathtt{\sim}$}} A*B*C)}\\
		\qquad ultimately expands into: {\color{blue}\texttt{lm(y {\raise.17ex\hbox{$\scriptstyle\mathtt{\sim}$}} A + B + C + A*B + A*C + B*C + A*B*C)}}
		
		\vspace{12pt}
		\qquad 
		This is because: \texttt{A*B*C} can be considered to be \texttt{(A*B)*C} \\
		\qquad which expands into \texttt{(A + B + A*B)*C = A*C + B*C + A*B*C}
		
		\vspace{6pt}
		\qquad  but \texttt{A*C} expands into \texttt{A + C + A*C}\\
		\qquad  and \texttt{B*C} expands into \texttt{B + C + B*C} \\
		\qquad which when all collected together gives the full model expansion above.
		
		
	
\end{frame}


{\usebackgroundtemplate{\vbox to \paperheight{\vfil\hbox to \paperwidth{\hfil  
    \includegraphics[width=0.95\paperwidth,trim=0 1.69cm 0 1.69cm, clip]
	{../3D/Slides/01Screen Shot 2014-07-21 at 22.26.44 .png}  \hfil}\vfil}}
\begin{frame}\frametitle{}
\end{frame}}

{\usebackgroundtemplate{\vbox to \paperheight{\vfil\hbox to \paperwidth{\hfil  
    \includegraphics[width=0.95\paperwidth,trim=0 1.69cm 0 1.69cm, clip]
	{../3D/Slides/02Screen Shot 2014-07-21 at 22.27.01 .png}  \hfil}\vfil}}
\begin{frame}\frametitle{}
\end{frame}}

{\usebackgroundtemplate{\vbox to \paperheight{\vfil\hbox to \paperwidth{\hfil  
    \includegraphics[width=0.95\paperwidth,trim=0 1.69cm 0 1.69cm, clip]
	{../3D/Slides/03Screen Shot 2014-07-21 at 22.27.05 .png}  \hfil}\vfil}}
\begin{frame}\frametitle{}
\end{frame}}

{\usebackgroundtemplate{\vbox to \paperheight{\vfil\hbox to \paperwidth{\hfil  
    \includegraphics[width=0.95\paperwidth,trim=0 1.69cm 0 1.69cm, clip]
	{../3D/Slides/04Screen Shot 2014-07-21 at 22.27.36 .png}  \hfil}\vfil}}
\begin{frame}\frametitle{}
\end{frame}}

\begin{frame}\frametitle{Advanced thinking: optimizing multiple objectives}
	\begin{center}
		{\color{myOrange}Consider the case where the aim is to maximize \textbf{both} $y_1$ and $y_2$.}
	\end{center}
	\vspace{-10pt}
	\begin{columns}[T]
		\column{0.01\textwidth}
		\column{0.35\textwidth}
			$y_1$ = ``collection efficiency'' 
			
			\includegraphics[width=1.1\textwidth]{\imagedir/doe/examples/y1-solar-panels.png}
		\column{0.05\textwidth}
		\column{0.45\textwidth}
			$y_2$ = ``energy delivery efficiency''
			\includegraphics[width=1.1\textwidth]{\imagedir/doe/examples/y2-solar-panels.png}
		\column{0.30\textwidth}	
			\vspace{12pt}
			\fbox{\parbox[b][8.5em][t]{.85\textwidth}{
				Specify at what levels (high or low) should we set the factors\\ \textbf{A}, \textbf{B}, \textbf{C}, and \textbf{D} to achieve a maximum in both $y_1$ \emph{and} in $y_2$.
			}}
	\end{columns}
\end{frame}

