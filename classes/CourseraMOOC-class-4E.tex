\begin{frame}\frametitle{}
	\begin{center}
		\includegraphics[width=.97\textwidth]{\imagedir/doe/disturbance-classification-MOOC.png}
	\end{center}
	%They should fall into one of these 3 categories.
\end{frame}

\begin{frame}\frametitle{Can you classify all the variables in your experimental system?}
	\begin{center}
		\includegraphics[width=.7\textwidth]{\imagedir/doe/disturbance-classification-MOOC.png}
	\end{center}
	They should fall into one of these 3 categories.
\end{frame}

\begin{frame}\frametitle{The ``CalApp'' example continues}
	
	\begin{columns}[t]
	
		\column{1\textwidth}
			\begin{tabulary}{\linewidth}{l|ll}\hline
				& \textbf{\relax Low level $-$} & \textbf{\relax High level $+$}\\ \hline  \\
				\textbf{A}: ``Promotion'' & 1 free in-app upgrade & 30-day trial of all features\\ \\
				\textbf{B}: ``Message'' & \parbox[t]{5.cm}{``CalApp has your schedule available at your fingertips, on any device.''} & \parbox[t]{5.5cm}{``CalApp features are configurable; only pay for the features you want.''} \\ \\
				\textbf{C}: ``Price'' & in-app purchase price is 89c &  in-app purchase price is 99c  \\& \\\hline
			\end{tabulary}
			
			
	\end{columns}
		
\end{frame}

\begin{frame}\frametitle{A subtlety: characteristics of a nuisance factor}
	\begin{columns}[T]
		\column{0.25\textwidth}

			\includegraphics[width=\textwidth]{../4E/Supporting files/flickr-6449132251_dd669e13dd_b-nuisance.jpg}

			\see{\href{https://secure.flickr.com/photos/nickwebb/6449132251}{Flickr: nickwebb}}
					
			
		\column{0.75\textwidth}
		
		\begin{itemize}
			\item	the factor does vary during your experiments
			\item	it is controllable, and it is measurable (just like a regular factor)
			\item	the factor is not the focus of your experiment

		\end{itemize}
		
			\onslide+<2->{
				

				\color{myOrange} How do we deal with a nuisance factor to avoid bias?
			}
			\onslide+<3->{
				\begin{itemize}
					\item	randomization will minimize the effect, but not eliminate it
					\item	your experiments should be successful, despite the nuisance factor
					\item	we use a process called {\color{purple} blocking}
				\end{itemize}
			}			
	\end{columns}
	
	\onslide+<4->{
		\vspace{0.5cm}
		
		\hbox{\hspace{-1.5em}
			\fbox{\parbox[b][3.5em][t]{1.04\textwidth}{
				{\color{myGreen} ``Does the process have to work successfully with different levels of the nuisance variable?''}\\
				\hbox{\hspace{1.5em}If \textbf{yes}: you must actively plan for blocking (that's the next topic)}\\
				\hbox{\hspace{1.5em}If \textbf{no}: that indicates you have good control over your system}\\
			}}
		}
		
	}
\end{frame}

\begin{frame}\frametitle{Some examples of nuisance factors}
	\begin{columns}[t]
		\column{0.25\textwidth}
			\textbf{CalApp}
			
			\includegraphics[width=\textwidth]{../4D/Supporting materials/4C-7-cell-phone-2830319467_1faaecc974_o-flickr.jpg}
			\\
			{\tiny{\href{https://secure.flickr.com/photos/williamhook/2830319467/}{Flickr: williamhook}}}
			
			\vspace{1cm}
			Apple or Android 
		
		
			\column{0.25\textwidth}
			\onslide+<2->{
				\textbf{Baking}
				\includegraphics[width=\textwidth]{../4E/Supporting files/flour-and-flour-cropped.png}
		
				\vspace{1.1cm}
				Brand of flour
			}
			
		\column{0.25\textwidth}
			\onslide+<3->{
				\textbf{Shift work}
				%\vspace{0.1cm}
				\includegraphics[width=\textwidth]{../4E/Supporting files/flickr-4140348113_f0efc8235b_z-clock.jpg}
				\\
				{\tiny{\href{https://secure.flickr.com/photos/leehaywood/4140348113}{Flickr: leehaywood}}}
				
				
				
				\vspace{0.8cm}
				Day shift or night shift
			}
		
		\column{0.25\textwidth}
			\onslide+<4->{
				\textbf{Gas mileage}
				
				%\vspace{-0.1cm}
				\includegraphics[width=\textwidth]{../4D/Supporting materials/4C-6-gas-mileage.png}
				
				\vspace{1.65cm}
				Driver 1 or driver 2
			}
			
	\end{columns}
			
\end{frame}

\begin{frame}\frametitle{The ``CalApp'' example continues}
	
	\begin{columns}[t]
	
		\column{1\textwidth}
			\begin{tabulary}{\linewidth}{l|ll}\hline
				& \textbf{\relax Low level $-$} & \textbf{\relax High level $+$}\\ \hline  \\
				\textbf{A}: ``Promotion'' & 1 free in-app upgrade & 30-day trial of all features\\ \\
				\textbf{B}: ``Message'' & \parbox[t]{5.cm}{``CalApp has your schedule available at your fingertips, on any device.''} & \parbox[t]{5.5cm}{``CalApp features are configurable; only pay for the features you want.''} \\ \\
				\textbf{C}: ``Price'' & in-app purchase price is 89c &  in-app purchase price is 99c  \\& \\\hline
			\end{tabulary}
			
			
	\end{columns}
	
	
\end{frame}

\begin{frame}\frametitle{Planning for blocking: {\color{myOrange} add a new factor column to the standard order table}}
	
	\newcommand{\white}{\color{white}}
	\begin{tabulary}{\linewidth}{ccccc}\hline 
		\multirow{2}{*}{\textbf{\relax Run }} & \textbf{\relax A } & \textbf{\relax B } & \textbf{\relax C } & \textbf{\relax D	}  \\
		 & \scriptsize ``Promotion'' & \scriptsize ``Message'' & \scriptsize ``Price'' & \scriptsize ``Apple or Android'' \\
		\hline 
		1 & \(-\) & \(-\) & \(-\) & \\
		2 & \(+\) & \(-\) & \(-\) & \\
		3 & \(-\) & \(+\) & \(-\) & \\
		4 & \(+\) & \(+\) & \(-\) & \\
		5 & \(-\) & \(-\) & \(+\) & \\
		6 & \(+\) & \(-\) & \(+\) & \\
		7 & \(-\) & \(+\) & \(+\) & \\
		8 & \(+\) & \(+\) & \(+\) & \\
		 \hline
	\end{tabulary}
	
	\begin{itemize}
	 	\item	Factor \textbf{D}: experiment with either  Android $(-)$ or Apple $(+)$ users \pause
	
	 	\item	This would require 16 experiments for a full factorial in factors \textbf{A}, \textbf{B}, \textbf{C}, and \textbf{D} \pause
	 	\item	but, we only want 8 experiments
	\end{itemize}
\end{frame}

\begin{frame}\frametitle{Planning for blocking: {\color{myOrange} add a new factor column to the standard order table}}
	
	\newcommand{\apple}{\scriptsize ~~\,Apple}
	\newcommand{\andrd}{\scriptsize Android}
	\begin{tabulary}{\linewidth}{ccccc}\hline 
		\multirow{2}{*}{\textbf{\relax Run }} & \textbf{\relax A } & \textbf{\relax B } & \textbf{\relax C } & \textbf{\relax D=ABC}  \\
		 & \scriptsize ``Promotion'' & \scriptsize ``Message'' & \scriptsize ``Price'' & \scriptsize ``Apple or Android'' \\
		\hline 
		1 & \(-\) & \(-\) & \(-\) & $-$ \andrd \\
		2 & \(+\) & \(-\) & \(-\) & $+$ \apple \\
		3 & \(-\) & \(+\) & \(-\) & $+$ \apple \\
		4 & \(+\) & \(+\) & \(-\) & $-$ \andrd \\
		5 & \(-\) & \(-\) & \(+\) & $+$ \apple \\
		6 & \(+\) & \(-\) & \(+\) & $-$ \andrd \\
		7 & \(-\) & \(+\) & \(+\) & $-$ \andrd \\
		8 & \(+\) & \(+\) & \(+\) & $+$ \apple \\
		 \hline
	\end{tabulary}
	
	\begin{itemize}
	 	\item	Factor \textbf{D}: experiment with either  Android $(-)$ or Apple $(+)$ users 
	 	\item	This would require 16 experiments for a full factorial in factors \textbf{A}, \textbf{B}, \textbf{C}, and \textbf{D}
	 	\item	but, we only want 8 experiments
	\end{itemize}
\end{frame}

\begin{frame}\frametitle{Visualizing the blocking procedure on a cube plot}
	
	\newcommand{\apple}{\scriptsize ~~\,Apple}
	\newcommand{\andrd}{\scriptsize Android}
	\begin{columns}[T]
		\column{0.5\textwidth}
			\begin{tabulary}{\linewidth}{ccccc}\hline 
				\multirow{2}{*}{\textbf{\relax Run }} & \textbf{\relax A } & \textbf{\relax B } & \textbf{\relax C } & \textbf{\relax D=ABC}  \\
				 & \scriptsize ``Promotion'' & \scriptsize ``Message'' & \scriptsize ``Price'' & \scriptsize ``Apple or Android'' \\
				\hline 
				1 & \(-\) & \(-\) & \(-\) & $-$ \andrd \\
				2 & \(+\) & \(-\) & \(-\) & $+$ \apple \\
				3 & \(-\) & \(+\) & \(-\) & $+$ \apple \\
				4 & \(+\) & \(+\) & \(-\) & $-$ \andrd \\
				5 & \(-\) & \(-\) & \(+\) & $+$ \apple \\
				6 & \(+\) & \(-\) & \(+\) & $-$ \andrd \\
				7 & \(-\) & \(+\) & \(+\) & $-$ \andrd \\
				8 & \(+\) & \(+\) & \(+\) & $+$ \apple \\
				 \hline
			\end{tabulary}
		\column{0.1\textwidth}
		\column{0.4\textwidth}
		
			\vspace{1cm}
			\centerline{\includegraphics[width=\textwidth]{\imagedir/doe/examples/half-fraction-in-3-factors-Apple-Android.png}}
			
	\end{columns}
\end{frame}

\begin{frame}\frametitle{Planning for blocking: {\color{myOrange} add a new factor column to the standard order table}}
	
	\newcommand{\apple}{\scriptsize ~~\,Apple}
	\newcommand{\andrd}{\scriptsize Android}
	\newcommand{\white}{\color{white}}
	\begin{tabulary}{\linewidth}{ccccccc}\hline 
		\multirow{2}{*}{\textbf{\relax Run }} & \textbf{\relax A } & \textbf{\relax B } & \textbf{\relax C } & \textbf{\relax D = ABC	} & \textbf{\relax Outcome, $y$	}\\
		 & \scriptsize ``Promotion'' & \scriptsize ``Message'' & \scriptsize ``Price'' & \scriptsize ``Apple or Android'' \\
		\hline 
		1 & \(-\) & \(-\) & \(-\) & $-$ \andrd & $y_{(1)} \white + g = \widetilde{y}_{(1)}$\\
		2 & \(+\) & \(-\) & \(-\) & $+$ \apple & $y_{(2)} \white + h = \mathring{y}_{(2)}$\\
		3 & \(-\) & \(+\) & \(-\) & $+$ \apple & $y_{(3)} \white + h = \mathring{y}_{(3)}$\\
		4 & \(+\) & \(+\) & \(-\) & $-$ \andrd & $y_{(4)} \white + g = \widetilde{y}_{(4)}$\\
		5 & \(-\) & \(-\) & \(+\) & $+$ \apple & $y_{(5)} \white + h = \mathring{y}_{(5)}$\\
		6 & \(+\) & \(-\) & \(+\) & $-$ \andrd & $y_{(6)} \white + g = \widetilde{y}_{(6)}$\\
		7 & \(-\) & \(+\) & \(+\) & $-$ \andrd & $y_{(7)} \white + g = \widetilde{y}_{(7)}$\\
		8 & \(+\) & \(+\) & \(+\) & $+$ \apple & $y_{(8)} \white + h = \mathring{y}_{(8)}$\\
		 \hline
	\end{tabulary}
	
	\vspace{0.5cm}
	
	\begin{itemize}
		\item	Android users: ${\white\widetilde{y}_{(i)} =}\,\, y_{(i)} + g$
		\item	Apple users:\,\,\,\,\, ${\white \mathring{y}_{(i)} =}\,\, y_{(i)} + h$
	\end{itemize}
	
\end{frame}

\begin{frame}\frametitle{Planning for blocking: {\color{myOrange} add a new factor column to the standard order table}}
	
	\newcommand{\apple}{\scriptsize ~~\,Apple}
	\newcommand{\andrd}{\scriptsize Android}
	\newcommand{\white}{}
	\begin{tabulary}{\linewidth}{ccccccc}\hline 
		\multirow{2}{*}{\textbf{\relax Run }} & \textbf{\relax A } & \textbf{\relax B } & \textbf{\relax C } & \textbf{\relax D = ABC	} & \textbf{\relax Outcome, $y$	}\\
		 & \scriptsize ``Promotion'' & \scriptsize ``Message'' & \scriptsize ``Price'' & \scriptsize ``Apple or Android'' \\
		\hline 
		1 & \(-\) & \(-\) & \(-\) & $-$ \andrd & $y_{(1)} + g = \widetilde{y}_{(1)}$\\
		2 & \(+\) & \(-\) & \(-\) & $+$ \apple & $y_{(2)} + h = \mathring{y}_{(2)}$\\
		3 & \(-\) & \(+\) & \(-\) & $+$ \apple & $y_{(3)} + h = \mathring{y}_{(3)}$\\
		4 & \(+\) & \(+\) & \(-\) & $-$ \andrd & $y_{(4)} + g = \widetilde{y}_{(4)}$\\
		5 & \(-\) & \(-\) & \(+\) & $+$ \apple & $y_{(5)} + h = \mathring{y}_{(5)}$\\
		6 & \(+\) & \(-\) & \(+\) & $-$ \andrd & $y_{(6)} + g = \widetilde{y}_{(6)}$\\
		7 & \(-\) & \(+\) & \(+\) & $-$ \andrd & $y_{(7)} + g = \widetilde{y}_{(7)}$\\
		8 & \(+\) & \(+\) & \(+\) & $+$ \apple & $y_{(8)} + h = \mathring{y}_{(8)}$\\
		 \hline
	\end{tabulary}
	
	\vspace{0.5cm}

	
	\begin{itemize}
		\item	Android users: ${\widetilde{y}_{(i)} =}\,\, y_{(i)} + g$
		\item	Apple users:\,\,\,\,\, ${\mathring{y}_{(i)} =}\,\, y_{(i)} + h$
	\end{itemize} 
	
\end{frame}

\begin{frame}\frametitle{Why does blocking like this work so successfully? (math alert!)}
	
	The main effect of \textbf{A}
	
	\newcommand{\mo}{\color{myOrange}}
	
	\begin{align*}
	\mathbf{A} &= \dfrac{1}{2}\left[ \dfrac{\left( \mathring{y}_{(8)} - \widetilde{y}_{(7)}\right)
										+   \left( \widetilde{y}_{(4)} - \mathring{y}_{(3)}\right)	
										+   \left( \widetilde{y}_{(6)} - \mathring{y}_{(5)}\right)
										+   \left( \mathring{y}_{(2)} - \widetilde{y}_{(1)}\right)	} {4}\right]\\
		\intertext{\color{myOrange}Notice two $+\widetilde{y}$ values and two $-\widetilde{y}$; and also two $+\mathring{y}$ values and two $-\mathring{y}$}	
	\onslide+<2->{
	\mathbf{A} &=  \dfrac{1}{8}\bigg[   \left( y_{(8)} + h - y_{(7)} - g\right)
									+  \left( y_{(4)} + g - y_{(3)} - h\right)	\bigg.\\
							   &\bigg.\qquad\qquad\qquad\qquad\qquad+ \left( y_{(6)} + g - y_{(5)} - h\right)
									+  \left( y_{(2)} + h - y_{(1)} - g\right)	\bigg]\\
		}
	\onslide+<3->{
		\intertext{which simplifies to}
	\mathbf{A} &= \dfrac{1}{8} \Big[ -y_{(1)} + y_{(2)} - y_{(3)} + y_{(4)} - y_{(5)} + y_{(6)} - y_{(7)} + y_{(8)}	\,\,\cancelto{0}{-2g + 2g} \,\, \cancelto{0}{-2h+2h}\,\, \Big]\\
	}
	\onslide+<4->{
		\mathbf{A} &= \text{pure effect of \textbf{A},  without bias}
	}
	\end{align*}
\end{frame}

\begin{frame}\frametitle{Why does blocking like this work so successfully? (math alert!)}
	
	\vspace{1cm}
	In a similar way, you can show all effects are estimated without bias:
	\begin{itemize}
		\item	\textbf{A}
		\item	\textbf{B}
		\item	\textbf{C}
		\item	\textbf{AB}
		\item	\textbf{AC}
		\item	\textbf{BC}
		
		\vspace{1cm}
		\item	except, for the effect of \textbf{ABC}:
			\begin{align*}
			\mathbf{ABC} &= \dfrac{1}{8} \Big[  \underbrace{-y_{(1)} + y_{(2)} +  y_{(3)} - y_{(4)} + y_{(5)} - y_{(6)} - y_{(7)} + y_{(8)}}_{\mathclap{\text{pure effect of \textbf{ABC}}}} \underbrace{\,\,\,\,{\color{red}-4g} \,\, {\color{red}+ 4h_{\color{white}(i)}}}_{\mathclap{\text{\emph{with} bias}}}	 \Big]\\	
			\\
			%\mathbf{ABC} &= \underbrace{\text{, }}_{\mathclap{\text{confounding with the blocking effect}}}
			\end{align*}
	\end{itemize}
\end{frame}

\begin{frame}\frametitle{Why does blocking like this work so successfully? (math alert!)}
	
	\vspace{1cm}
	In a similar way, you can show all effects are estimated without bias:
	\begin{itemize}
		\item	\textbf{A}
		\item	\textbf{B}
		\item	\textbf{C}
		\item	\textbf{AB}
		\item	\textbf{AC}
		\item	\textbf{BC}
		
		\vspace{1cm}
		\item	except, for the effect of \textbf{ABC}:
			\begin{align*}
			\mathbf{ABC} &= \dfrac{1}{8} \Big[  \underbrace{-y_{(1)} + y_{(2)} +  y_{(3)} - y_{(4)} + y_{(5)} - y_{(6)} - y_{(7)} + y_{(8)}}_{\mathclap{\text{pure effect of \textbf{ABC}}}} \underbrace{\,\,\,\,{\color{red}-4g} \,\, {\color{red}+ 4h_{\color{white}(i)}}}_{\mathclap{\text{confounded with the blocking effect}}}	 \Big]\\	
			\\
			\end{align*}
	\end{itemize}
\end{frame}

\begin{frame}\frametitle{More than two blocks are easily possible}
	
	\begin{columns}[t]
		\column{0.25\textwidth}
			\textbf{CalApp}
			
			\includegraphics[width=\textwidth]{../4D/Supporting materials/4C-7-cell-phone-2830319467_1faaecc974_o-flickr.jpg}
			\\
			{\tiny{\href{https://secure.flickr.com/photos/williamhook/2830319467/}{Flickr: williamhook}}}
			
			\vspace{1cm}
			Apple, Android, Blackberry
		
		
			\column{0.25\textwidth}
			\onslide+<2->{
				\textbf{Baking}
				\includegraphics[width=\textwidth]{../4E/Supporting files/flour-and-flour-cropped.png}
		
				\vspace{1.1cm}
				4 different brands of flour
			}
			
		\column{0.25\textwidth}
			\onslide+<3->{
				\textbf{Shift work}
				%\vspace{0.1cm}
				\includegraphics[width=\textwidth]{../4E/Supporting files/flickr-4140348113_f0efc8235b_z-clock.jpg}
				\\
				{\tiny{\href{https://secure.flickr.com/photos/leehaywood/4140348113}{Flickr: leehaywood}}}
				
				
				
				\vspace{0.8cm}
				Day shift\\
				Afternoon shift\\
				Night shift
			}
		
		\column{0.25\textwidth}
			\onslide+<4->{
				\textbf{Gas mileage}
				
				%\vspace{-0.1cm}
				\includegraphics[width=\textwidth]{../4D/Supporting materials/4C-6-gas-mileage.png}
				
				\vspace{1.65cm}
				Gasoline brand 1\\
				Gasoline brand 2\\
				Gasoline brand 3\\
			}
			
	\end{columns}
	
	
\end{frame}

\begin{frame}\frametitle{}
	\centerline{\includegraphics[width=\textwidth]{../4E/Supporting files/blocking-tables.jpg}}
\end{frame}

\begin{frame}\frametitle{General rule for blocking in two blocks}
	\begin{columns}[T]
		\column{0.45\textwidth}
			\begin{enumerate}
				\item	Write out your existing standard order table
				\item	Add an extra ``factor'' to your standard order table
				\item	Generate a half-fraction, using the trade-off table, on this new factor
				\item	Signs with $-$ are in one block; and signs with $+$ are in the other block
			\end{enumerate}
		
		\column{0.48\textwidth}
			\newcommand{\apple}{\scriptsize ~~\,Apple}
			\newcommand{\andrd}{\scriptsize Android}
			\begin{tabulary}{\linewidth}{cccc|c}\hline 
				\multirow{1}{*}{\textbf{\relax Run }} & \textbf{\relax A } & \textbf{\relax B } & \textbf{\relax C } & \textbf{\relax D=ABC}  \\
				\hline 
				1 & \(-\) & \(-\) & \(-\) & $-$ \andrd \\
				2 & \(+\) & \(-\) & \(-\) & $+$ \apple \\
				3 & \(-\) & \(+\) & \(-\) & $+$ \apple \\
				4 & \(+\) & \(+\) & \(-\) & $-$ \andrd \\
				5 & \(-\) & \(-\) & \(+\) & $+$ \apple \\
				6 & \(+\) & \(-\) & \(+\) & $-$ \andrd \\
				7 & \(-\) & \(+\) & \(+\) & $-$ \andrd \\
				8 & \(+\) & \(+\) & \(+\) & $+$ \apple \\
				 \hline
			\end{tabulary}
	\end{columns}
	
	
\end{frame}

