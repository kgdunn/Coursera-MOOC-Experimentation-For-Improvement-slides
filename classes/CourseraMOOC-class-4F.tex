\begin{comment}
		
	\begin{frame}\frametitle{\includegraphics[width=0.3\textwidth]{\imagedir/doe/examples/advice-logo.png}\,\, plan your experiments carefully ahead of time}
	\end{frame}
	
	\begin{columns}[T]
		\column{0.45\textwidth}
			\includegraphics[width=0.7\textwidth]{\imagedir/statistics/flicfcb_o.jpg}
		
			{\scriptsize (p. 230 in Box, Hunter and Hunter, 2$^\text{nd}$ ed)}
		
		\column{0.48\textwidth}
			\includegraphics[width=\textwidth]{\imagedir/doe/examples/solar-panel-mendelu-cz-website.png}
		
		
			\see{\href{http://yint.org/solar-panel-study}{http://yint.org/solar-panel-study}}
	\end{columns}
	
	\begin{center}\rule[8mm]{4cm}{0.01cm}\end{center}
	\rule[3mm]{0.01cm}{25mm}%
\end{comment}

\begin{frame}\frametitle{There's more: we never stop learning\emph{!}}
	\centerline{
	\fbox{\parbox[b][5.2em][t]{.7\textwidth}{
		{\color{myGreen} There is ongoing research on how to efficiently run fewer experiments. \\}\\
		We give some pointers to this research at the end of video 4F.\\
		\\
		{\color{myOrange}In this section we only cover established techniques.}
	}}
	}
\end{frame}	

\begin{frame}\frametitle{Cell-culture example: long duration runs; and many factors are possible}
	\newcommand{\white}{\color{white}}
	\begin{columns}[c]
		\column{0.5\textwidth} 
			\begin{enumerate}
				\item	\textbf{T}: the temperature profile
				\item	\textbf{D}: dissolved oxygen
				\item	\textbf{A}: agitation rate
				\item	\textbf{P}: pH
				\item	\textbf{S}: substrate type (A or B)
				%\onslide+<2->{
				%\item	\textbf{W}: water type (distilled or tap water)
				%\item	\textbf{M}: mixer type (axial or radial)
				%}
			\end{enumerate}
		
		\column{0.5\textwidth}
			{\color{blue} \small Industrial scale: {\color{white}y}}   
			
			\vspace{0.2cm}
			
			\centerline{\includegraphics[height=.7\textwidth]{../4E/Supporting material/flickr-493740898_b98113ae44_o-cell-culture-mod.png}}
			\see{\href{https://secure.flickr.com/photos/londonmatt/493740898}{Flickr: londonmatt}}
	\end{columns}

	\vfill
	At 10 days per cell culture, it would take $\approx$ 1 year for all $2^5 = 32$ runs
	
\end{frame}

\begin{frame}\frametitle{Cell-culture example: long duration runs; and many factors are possible}
	\begin{columns}[c]
		\column{0.5\textwidth}
			\begin{enumerate}
				\item	\textbf{T}: the temperature profile
				\item	\textbf{D}: dissolved oxygen
				\item	\textbf{A}: agitation rate
				\item	\textbf{P}: pH
				\item	\textbf{S}: substrate type (A or B)
				%\onslide+<2->{
				%\item	\textbf{W}: water type (distilled or tap water)
				%\item	\textbf{M}: mixer type (axial or radial)
				%}
			\end{enumerate}
		
		\column{0.5\textwidth}
			{\color{blue} \small Laboratory scale:} 
			
			\vspace{0.2cm}
			
			\centerline{\includegraphics[height=.7\textwidth]{../4E/Supporting material/flickr-3644661574_79d0810177_b-cell-culture-lab}}
			\see{\href{https://secure.flickr.com/photos/kaibara/3644661574}{Flickr: kaibara}}
	\end{columns}

	\vfill
	At 10 days per cell culture, it would take $\approx$ 1 year for all $2^5 = 32$ runs
	
\end{frame}

\begin{frame}\frametitle{Cell-culture example: long duration runs; and many factors are possible}
	\begin{columns}[c]
		\column{0.5\textwidth}
			\begin{enumerate}
				\item	\textbf{T}: the temperature profile
				\item	\textbf{D}: dissolved oxygen
				\item	\textbf{A}: agitation rate
				\item	\textbf{P}: pH
				\item	\textbf{S}: substrate type (A or B)
				%\onslide+<2->{
				%\item	\textbf{W}: water type (distilled or tap water)
				%\item	\textbf{M}: mixer type (axial or radial)
				%}
			\end{enumerate}
		
		\column{0.5\textwidth}
			{\color{blue} \small Laboratory equipment to control the culture:} 
			
			\vspace{0.2cm}
			
			\centerline{\includegraphics[height=.7\textwidth]{../4E/Supporting material/flickr-3815183191_98c6296080_b-cell-culture-instrumentation}}
			\see{\href{https://secure.flickr.com/photos/mjanicki/3815183191}{Flickr: mjanicki}}
	\end{columns}

	\vfill
	3 months available: {\color{myOrange} that corresponds to 9 experiments}.
	
\end{frame}

\begin{frame}\frametitle{Cell-culture example: long duration runs; and many factors are possible}
	\begin{columns}[c]
		\column{0.5\textwidth}
			\begin{enumerate}
				\item	\textbf{T}: the temperature profile
				\item	\textbf{D}: dissolved oxygen
				\item	$\cancel{\text{\textbf{A}: agitation rate}}$
				\item	$\cancel{\text{\textbf{P}: pH}}$
				\item	\textbf{S}: substrate type (A or B)
				%\onslide+<2->{
				%\item	\textbf{W}: water type (distilled or tap water)
				%\item	\textbf{M}: mixer type (axial or radial)
				%}
			\end{enumerate}
		
		\column{0.5\textwidth}
			{\color{blue} \small Laboratory equipment to control the culture:} 
			
			\vspace{0.2cm}
			
			\centerline{\includegraphics[height=.7\textwidth]{../4E/Supporting material/flickr-3815183191_98c6296080_b-cell-culture-instrumentation}}
			\see{\href{https://secure.flickr.com/photos/mjanicki/3815183191}{Flickr: mjanicki}}
	\end{columns}

	\vfill
	{\color{red} Don't do this:} remove factors in order to get a full factorial.
	
\end{frame}

\begin{frame}\frametitle{Cell-culture example: long duration runs; and many factors are possible}
	\begin{columns}[c]
		\column{0.5\textwidth}
			\begin{enumerate}
				\item	\textbf{T}: the temperature profile
				\item	\textbf{D}: dissolved oxygen
				\item	\textbf{A}: agitation rate
				\item	\textbf{P}: pH
				\item	\textbf{S}: substrate type (A or B)
				\item	\textbf{W}: water type (distilled or tap water)
				\item	\textbf{M}: mixer type (axial or radial)
		
			\end{enumerate}
		
		\column{0.5\textwidth}
			{\color{blue} \small Different types of mixers (impellers):} 
			
			\vspace{0.2cm}
			
			\centerline{\includegraphics[height=.7\textwidth]{../4E/Supporting material/Mixing_-_flusso_assiale_e_radiale-wikipedia.jpg}}
			
			\see{\href{https://en.wikipedia.org/wiki/File:Mixing_-_flusso_assiale_e_radiale.jpg}{Wikipedia}}
	\end{columns}

	\vfill
	3 months available: {\color{myOrange} that corresponds to 9 experiments}.
	
\end{frame}

\begin{frame}\frametitle{}
	\begin{columns}[T]
		\column{0.9\textwidth}
			\includegraphics[height=\textheight]{\imagedir/doe/DOE-trade-off-table-MOOC-resolution.png}
		
		\column{0.15\textwidth}
			
			\onslide+<2->{
				\vspace{2cm}
				{\Huge
					$2^{5-2}$
					%$2^{k-p}$
				} 
			 
				\vspace{2cm}
				\onslide+<3->{ 
					\begin{align*}
						\textbf{D} &= \textbf{AB}\\
						\textbf{E}\, &= \textbf{AC} 
					\end{align*}
				}
			}
	\end{columns}
	
\end{frame}

\begin{frame}\frametitle{}
	\begin{columns}[T]
		\column{0.9\textwidth}
			\includegraphics[height=\textheight]{\imagedir/doe/DOE-trade-off-table-MOOC-resolution.png}
		
		\column{0.15\textwidth}
			
			
				\vspace{2cm}
				{\Huge
					$2^{k-p}$
				} 
			 
				\vspace{2cm}
				\onslide+<3->{ 
					\begin{align*}
						\textbf{D} &= \textbf{AB}\\
						\textbf{E}\, &= \textbf{AC} 
					\end{align*}
				}
	\end{columns}
	
\end{frame}

\begin{frame}\frametitle{}
	\begin{columns}[T]
		\column{0.9\textwidth}
			\includegraphics[height=\textheight]{\imagedir/doe/DOE-trade-off-table-MOOC-resolution-half-marked.png}
		
		\column{0.15\textwidth}
			
			
				\vspace{2cm}
				{\Huge
					$2^{k-p}$\\
					
					\vspace{1cm}
					
				} $p=1$
			 
				
	\end{columns}
	
\end{frame}

\begin{frame}\frametitle{}
	\begin{columns}[T]
		\column{0.9\textwidth}
			\includegraphics[height=\textheight]{\imagedir/doe/DOE-trade-off-table-MOOC-resolution-quarter-marked.png}
		
		\column{0.15\textwidth}
			
			
				\vspace{2cm}
				{\Huge
					$2^{k-p}$\\
					
					\vspace{1cm}
					
				} $p=2$
			 
				
	\end{columns}
	
\end{frame}

\begin{frame}\frametitle{Cell-culture example: creating the fractional factorial design}
	
	\vspace{0.5cm}
	\begin{tabulary}{\linewidth}{ccccccc}
		\textbf{\relax Experiment} & \textbf{\relax A$^\ast$ } & \textbf{\relax B$\,\mathring{}$} & \textbf{\relax C$\,\mathring{}$ } & \onslide+<2->{\textbf{\relax D$\,\mathring{}$ = AB}} & \onslide+<2->{\textbf{\relax E$^\ast$ = AC}}\\ \cline{1-6}
		\textbf{1} & \(-\) & \(-\) & \(-\) & \onslide+<2->{\(+\)} & \onslide+<2->{\(+\)} \\
		\textbf{2} & \(+\) & \(-\) & \(-\) & \onslide+<2->{\(-\)} & \onslide+<2->{\(-\)} \\
		\textbf{3} & \(-\) & \(+\) & \(-\) & \onslide+<2->{\(-\)} & \onslide+<2->{\(+\)} \\
		\textbf{4} & \(+\) & \(+\) & \(-\) & \onslide+<2->{\(+\)} & \onslide+<2->{\(-\)} \\
		\textbf{5} & \(-\) & \(-\) & \(+\) & \onslide+<2->{\(+\)} & \onslide+<2->{\(-\)} \\
		\textbf{6} & \(+\) & \(-\) & \(+\) & \onslide+<2->{\(-\)} & \onslide+<2->{\(+\)} \\
		\textbf{7} & \(-\) & \(+\) & \(+\) & \onslide+<2->{\(-\)} & \onslide+<2->{\(-\)} \\
		\textbf{8} & \(+\) & \(+\) & \(+\) & \onslide+<2->{\(+\)} & \onslide+<2->{\(+\)} \\
		\onslide+<4->{\textbf{9} &  $-1$	& 0 		& 0 & 0 & +1		&\color{myOrange}$\longleftarrow$ this row is a baseline $^\#$} 
	\end{tabulary}
	
	\vspace{0.5cm}
	\begin{flalign*}
		^\ast       & \text{categorical factor}\\
		\mathring{}\,\, & \text{continuous factor} \\
		\onslide+<4->{\color{myOrange}^\#  & \text{this entry does \textbf{not} use the generators; it should be run first to establish a baseline} &}
	\end{flalign*}	
\end{frame}

\begin{frame}\frametitle{$^\text{\color{purple}Video 4B}$ Aliasing: when we have more than one name for the same thing}
	
	\vspace{1cm}
	What is aliased in this experimental design (i.e. which columns are the same)?
		
		\vspace{0.5cm}
		\begin{itemize}
			\item	\textbf{A=BC}
			
			\onslide+<2->	{
			\vspace{1cm}
			\item	\textbf{B=AC}
			
			\vspace{1cm}
			\item	\textbf{C=AB} 
			
			\vspace{1cm}
			\item	\textbf{ABC = Intercept} (the intercept is indicated as $b_0$)
			}
		\end{itemize}

\end{frame}

\begin{frame}\frametitle{Calculating the aliases using an interesting technique}
	
	\begin{columns}[T]
		\column{0.25\textwidth}
			\includegraphics[width=\textwidth]{../4E/Supporting material/generators.png}
		
		\column{0.04\textwidth}
		\column{0.75\textwidth}
		\vspace{-.5cm}
		{\LARGE
		
			\begin{flalign*}			
				\textbf{D} &= \textbf{AB}&\\
				\onslide+<2->{
					\textbf{D\,D} &= \textbf{AB\,D}&
				}
				\onslide+<3->{
					\intertext{{\color{purple}rule:} \textbf{AA=I}; \textbf{BB=I}; ... \textbf{DD=I}, \emph{etc}}
				}
				\onslide+<5->{
					\textbf{I} &= \textbf{ABD}&\\
				}
			\end{flalign*}
		}

	\end{columns}	
	
	\uncover<4>{
		\vspace{-1.8cm}
		If we have $ \textbf{A} = \begin{pmatrix}-1\\+1\\-1\\+1\\-1\\+1\\-1\\+1\\ \end{pmatrix} \text{then we can say}\,\, \textbf{AA} =  \begin{pmatrix}-1\\+1\\-1\\+1\\-1\\+1\\-1\\+1\\ \end{pmatrix} \begin{pmatrix}-1\\+1\\-1\\+1\\-1\\+1\\-1\\+1\\ \end{pmatrix}  =  \begin{pmatrix}+1\\+1\\+1\\+1\\+1\\+1\\+1\\+1\\ \end{pmatrix} = \textbf{I}$
	}
	
	
	
\end{frame}

\begin{frame}\frametitle{Calculating the aliases using an interesting technique}
	
	\begin{columns}[T]
		\column{0.25\textwidth}
			\includegraphics[width=\textwidth]{../4E/Supporting material/generators.png}
		
		\column{0.04\textwidth}
		\column{0.75\textwidth}
		{\huge
			\begin{flalign*}			
				\textbf{E} &= \textbf{AC}&\\
				%\onslide+<2->{
					\textbf{E\,E} &= \textbf{AC\,E}&
				%}
				%\onslide+<3->{
					\intertext{{\color{purple}rule:} \textbf{AA=I}; \textbf{BB=I}; ... \textbf{DD=I}, \emph{etc}}
				%}
				\onslide+<2->{
					\textbf{I} &= \textbf{ACE}&\\
				}
			\end{flalign*}
		}

	\end{columns}	
	\uncover<2>{
		{\color{myOrange}Multiple the left and right by the symbol that's on the left.}
	}
	
	
\end{frame}

\begin{frame}\frametitle{Approach to calculate the aliasing pattern}
	
	\begin{columns}[T]
		\column{0.58\textwidth}
			\begin{enumerate}
				\item	Read the generators from the table 
				\onslide+<2->{
					\item	Rearrange the generators as  $\textbf{I = \ldots}$
				}
				\onslide+<4->{
				 	\item	Form the {\color{purple}\textbf{defining relationship}} taking all combinations of the words, so that $\textbf{I = \ldots}$
				}	
				\onslide+<5->{
				 	\item	Ensure the defining relationship has $2^p$ words
				}
				\onslide+<16->{
					\item	Use the defining relationship to compute the aliasing pattern
				}
			\end{enumerate}
			
		\column{0.5\textwidth}
			\begin{enumerate}
				\item	$\textbf{D = AB}$  and $\textbf{E = AC}$ 
				\onslide+<2->{
					\item	$\textbf{I = ABD}$ and $\textbf{I = ACE}$ 
				}
				
				\onslide+<6->{
 					\item	$\textbf{I =}$ \onslide+<7->{$\textbf{ABD}$} \onslide+<8->{$\textbf{= ACE}$} \onslide+<9->{$\textbf{= (ABD)(ACE)}$} 
					\\ \vspace{0.4cm}
				}
				\onslide+<6->{
								\item	$p=2$\onslide+<15->{, and we have 4 words.}
				}
				\onslide+<16->{
					\item	See the next slides.
					
				}
			\end{enumerate}			
			
	\end{columns}
	\vspace{0.5cm}
	\onslide+<3-5>{
		{\color{purple}\textbf{Word}}: a collection of sequential factor letters
	}
	\\
	\onslide+<10->{$\textbf{(ABD)(ACE)}$}
	\onslide+<11->{$\textbf{ = AABCDE}$}
	\onslide+<12->{$\textbf{ = (AA)BCDE}$}
	\onslide+<13->{$\textbf{ = (I)BCDE}$}
	\onslide+<14->{$\textbf{ = BCDE}$} 	
\end{frame}

\begin{frame}\frametitle{Approach to calculate the aliasing pattern}
	
	\begin{columns}[T]
		\column{0.58\textwidth}
			\begin{enumerate}
				\item	Read the generators from the table 
					\item	Rearrange the generators as  $\textbf{I = \ldots}$
				 	\item	Form the {\color{purple}\textbf{defining relationship}} taking all combinations of the words, so that $\textbf{I = \ldots}$
				 	\item	Ensure the defining relationship has $2^p$ words
				\onslide+<3->{
					\item	Use the defining relationship to compute the aliasing pattern
				}
			\end{enumerate}
			
		\column{0.5\textwidth}
			\begin{enumerate}
				\item	$\textbf{D = AB}$  and $\textbf{E = AC}$ 
					\item	$\textbf{I = ABD}$ and $\textbf{I = ACE}$ 
 					\item	$\textbf{I =}$ $\textbf{ABD}$ $\textbf{= ACE}$ $\textbf{= BCDE}$
					\\ \vspace{0.4cm}
				
					\item	$p=2$\onslide+<2->{, and we have 4 words.}
				
				\onslide+<3->{
					\item	We will in the next video.
					
				}
			\end{enumerate}			
			
	\end{columns}
	\vspace{0.5cm}
	\onslide+<4>{
		{\color{purple}\textbf{Word}}: a collection of sequential factor letters
	}
	\\
	\onslide+<1-2>{
		$\textbf{(ABD)(ACE)}$
		$\textbf{ = AABCDE}$
		$\textbf{ = (AA)BCDE}$
		$\textbf{ = (I)BCDE}$
		$\textbf{ = BCDE}$	
	}
\end{frame}

\begin{frame}\frametitle{Example to try yourself: find the defining relationship}
	
	\vspace{0.5cm}
	You'd like to investigate 6 factors, and have a budget for 15 to 20 experiments.
	
	\vspace{0.5cm}
	\begin{columns}[T]
		\column{0.7\textwidth}
			\begin{enumerate}
				\item	Read the generators from the table 
				\item	Rearrange the generators as  $\textbf{I = \ldots}$
			 	\item	Form the {\color{purple}defining relationship} taking all combinations of the words, so that $\textbf{I = \ldots}$
			 	\item	Ensure the defining relationship has $2^p$ words
				\item	In the next video: use this defining relationship to compute the aliasing pattern
			\end{enumerate}
			
		\column{0.3\textwidth}

	\end{columns}

	
\end{frame}

\begin{frame}\frametitle{Example to try yourself: {\color{myOrange}solution} for finding the defining relationship}
	
	\vspace{0.5cm}
	You'd like to investigate 6 factors, and have a budget for 15 to 20 experiments.
	
	\vspace{0.5cm}
	\begin{columns}[T]
		\column{0.85\textwidth}
			\begin{enumerate}
				\item	Read the generators from the table for $k=6$, and 16 experiments
					\begin{itemize}
						\item	\textbf{E = ABC}	and  \textbf{F = BCD}
					\end{itemize}
				\item	Rearrange the generators as  $\textbf{I = \ldots}$
					\begin{itemize}
						\item	\textbf{I = ABCE}	and  \textbf{I = BCDF}
					\end{itemize}
			 	\item	Form the {\color{purple}defining relationship} taking all combinations of the words, so that $\textbf{I = \ldots}$
					\begin{itemize}
						\item	\textbf{I = ABCE = BCDF = A(BB)(CC)DEF = ADEF}
					\end{itemize}
			 	\item	Ensure the defining relationship has $2^p$ words
					\begin{itemize}
						\item	It does; since $p=2$ in this case
					\end{itemize} 
				\item	We will use this defining relationship in video 4F
			\end{enumerate}
			
		\column{0.3\textwidth}
			\centerline{\includegraphics[height=.7\textwidth]{../4E/Supporting material/ivq-final-solution.png}}

	\end{columns}

	
\end{frame}
