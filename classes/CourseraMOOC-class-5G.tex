\begin{frame}\frametitle{Before we get started: a quick look back at OFAT (COST)}
	
	OFAT is not recommended for a variety of reasons:
	\begin{columns}[T]
		\column{0.75\textwidth}
			\vspace{0.5cm}
			\begin{itemize}
				\item	With OFAT you are never really sure you are at the peak.
					\begin{itemize}
						\item	you keep iterating through all the factors
						\item	no conclusive indication that you converged
					\end{itemize}
				
		\onslide+<2->{
				\vspace{0.5cm}
				\item	OFAT is order-dependent (it is a lottery\emph{!})
		}
		\onslide+<3->{
				\vspace{0.5cm}
		
				\item	OFAT does not scale well 
					\begin{itemize}
						\item	For 3 or more factors: we often end up using more runs
						\item	For 2 factors: I find OFAT and RSM (response surface methods) use about the same number of
								runs
					\end{itemize}
		}
		\onslide+<4->{
				\vspace{0.5cm}
		
				\item	We don't learn about interactions; only about  main effects 
		}
			\end{itemize}
		\column{0.3\textwidth}
		
			\centerline{\includegraphics[width=\textwidth]{../5G/Supporting materials/COST-contours-shopping-extend-20.png}}
			\onslide+<2->{
				\centerline{\includegraphics[width=\textwidth]{../5G/Supporting materials/COST-contours-shopping-reorder-09.png}}
			}

	\end{columns}
	
	

\end{frame}
\begin{frame}\frametitle{}
	\centerline{\includegraphics[height=\textheight]{../5E/Supporting materials/RSM-26.png}}
\end{frame}
\begin{frame}\frametitle{}
	\centerline{\includegraphics[height=\textheight]{../5G/Supporting materials/COST-contours-shopping-extend-01.png}}
\end{frame}
\begin{frame}\frametitle{}
	\centerline{\includegraphics[height=\textheight]{../5G/Supporting materials/COST-contours-shopping-extend-02.png}}
\end{frame}
\begin{frame}\frametitle{}
	\centerline{\includegraphics[height=\textheight]{../5G/Supporting materials/COST-contours-shopping-extend-03.png}}
\end{frame}
\begin{frame}\frametitle{}
	\centerline{\includegraphics[height=\textheight]{../5G/Supporting materials/COST-contours-shopping-extend-04.png}}
\end{frame}
\begin{frame}\frametitle{}
	\centerline{\includegraphics[height=\textheight]{../5G/Supporting materials/COST-contours-shopping-extend-05.png}}
\end{frame} 
\begin{frame}\frametitle{}
	\centerline{\includegraphics[height=\textheight]{../5G/Supporting materials/COST-contours-shopping-extend-06.png}}
\end{frame}
\begin{frame}\frametitle{}
	\centerline{\includegraphics[height=\textheight]{../5G/Supporting materials/COST-contours-shopping-extend-07.png}}
\end{frame}
\begin{frame}\frametitle{}
	\centerline{\includegraphics[height=\textheight]{../5G/Supporting materials/COST-contours-shopping-extend-08.png}}
\end{frame}
\begin{frame}\frametitle{}
	\centerline{\includegraphics[height=\textheight]{../5G/Supporting materials/COST-contours-shopping-extend-09.png}}
\end{frame}
\begin{frame}\frametitle{}
	\centerline{\includegraphics[height=\textheight]{../5G/Supporting materials/COST-contours-shopping-extend-13.png}}
\end{frame}
\begin{frame}\frametitle{}
	\centerline{\includegraphics[height=\textheight]{../5G/Supporting materials/COST-contours-shopping-extend-14.png}}
\end{frame}
\begin{frame}\frametitle{}
	\centerline{\includegraphics[height=\textheight]{../5G/Supporting materials/COST-contours-shopping-extend-15.png}}
\end{frame}
\begin{frame}\frametitle{}
	\centerline{\includegraphics[height=\textheight]{../5G/Supporting materials/COST-contours-shopping-extend-16.png}}
\end{frame}
\begin{frame}\frametitle{}
	\centerline{\includegraphics[height=\textheight]{../5G/Supporting materials/COST-contours-shopping-extend-17.png}}
\end{frame}
\begin{frame}\frametitle{}
	\centerline{\includegraphics[height=\textheight]{../5G/Supporting materials/COST-contours-shopping-extend-18.png}}
\end{frame}
\begin{frame}\frametitle{}
	\centerline{\includegraphics[height=\textheight]{../5G/Supporting materials/COST-contours-shopping-extend-19.png}}
\end{frame}
\begin{frame}\frametitle{}
	\centerline{\includegraphics[height=\textheight]{../5G/Supporting materials/COST-contours-shopping-extend-20.png}}
\end{frame}
\begin{frame}\frametitle{}
	\centerline{\includegraphics[height=\textheight]{../5G/Supporting materials/COST-contours-shopping-reorder-01.png}}
\end{frame}
\begin{frame}\frametitle{}
	\centerline{\includegraphics[height=\textheight]{../5G/Supporting materials/COST-contours-shopping-reorder-02.png}}
\end{frame}
\begin{frame}\frametitle{}
	\centerline{\includegraphics[height=\textheight]{../5G/Supporting materials/COST-contours-shopping-reorder-03.png}}
\end{frame}
\begin{frame}\frametitle{}
	\centerline{\includegraphics[height=\textheight]{../5G/Supporting materials/COST-contours-shopping-reorder-04.png}}
\end{frame}
\begin{frame}\frametitle{}
	\centerline{\includegraphics[height=\textheight]{../5G/Supporting materials/COST-contours-shopping-reorder-05.png}}
\end{frame}
\begin{frame}\frametitle{}
	\centerline{\includegraphics[height=\textheight]{../5G/Supporting materials/COST-contours-shopping-reorder-06.png}}
\end{frame}
\begin{frame}\frametitle{}
	\centerline{\includegraphics[height=\textheight]{../5G/Supporting materials/COST-contours-shopping-reorder-07.png}}
\end{frame}
\begin{frame}\frametitle{}
	\centerline{\includegraphics[height=\textheight]{../5G/Supporting materials/COST-contours-shopping-reorder-08.png}}
\end{frame}
\begin{frame}\frametitle{}
	\centerline{\includegraphics[height=\textheight]{../5G/Supporting materials/COST-contours-shopping-reorder-09.png}}
\end{frame}
\begin{frame}\frametitle{}
	\centerline{\includegraphics[height=\textheight]{../5E/Supporting materials/RSM-26.png}}
\end{frame}
\begin{frame}\frametitle{}
	\centerline{\includegraphics[height=\textheight]{../5E/Supporting materials/RSM-27.png}}
\end{frame}
\begin{frame}\frametitle{}
	\centerline{\includegraphics[height=\textheight]{../5E/Supporting materials/RSM-28.png}}
\end{frame}
\begin{frame}\frametitle{}
	\centerline{\includegraphics[height=\textheight]{../5E/Supporting materials/RSM-29.png}}
\end{frame}
\begin{frame}\frametitle{}
	\centerline{\includegraphics[height=\textheight]{../5E/Supporting materials/RSM-30.png}}
\end{frame}
\begin{frame}\frametitle{}
	\centerline{\includegraphics[height=\textheight]{../5E/Supporting materials/RSM-31.png}}
\end{frame}
\begin{frame}\frametitle{}
	\centerline{\includegraphics[height=\textheight]{../5E/Supporting materials/RSM-32.png}}
\end{frame}
\begin{frame}\frametitle{}
	\centerline{\includegraphics[height=\textheight]{../5E/Supporting materials/RSM-33.png}}
\end{frame}
\begin{frame}\frametitle{}
	\centerline{\includegraphics[height=\textheight]{../5E/Supporting materials/RSM-34.png}}
\end{frame}
\begin{frame}\frametitle{}
	\centerline{\includegraphics[height=\textheight]{../5G/Supporting materials/no-interaction.png}}
\end{frame}
\begin{frame}\frametitle{}
	\centerline{\includegraphics[height=\textheight]{../5G/Supporting materials/with-interaction.png}}
\end{frame}

\begin{frame}\frametitle{The 3 models we have built so far}
	Factorial 1:  
	\[\color{blue}\hat{y} = 389.8  +134  x_\text{P} +  55 x_\text{T} -3.5 x_\text{P}x_\text{T} \]
	Factorial 2: 
	\[\color{myGreen} \hat{y} =  645.4  +47 x_\text{P} +  22.500 x_\text{T} -2.0 x_\text{P}x_\text{T} \]
	Factorial 3:
	\[\hat{y} = 723.6  -2.5 x_\text{P} +  7.500 x_\text{T} -1.5 x_\text{P}x_\text{T} \]
	
\end{frame}

\begin{frame}\frametitle{Taking the next step for experiment 15: {\color{myOrange}solution}}
	\begin{columns}[T]
		\column{0.2066\textwidth}
		
			\vspace{0.1cm}
			{\tiny 
				\begin{enumerate}
					\item	Pick change in coded units in one factor.
				\end{enumerate}
			 \par}
			 
			\onslide+<1->{
				{\tiny 
					\begin{enumerate}\setcounter{enumi}{1}
						\item	Find the ratios for\\ the other factor(s).
					\end{enumerate}
				
				\par}
			}
			
			\vspace{0.0cm}
			\onslide+<1->{
				{\tiny 
					\begin{enumerate}\setcounter{enumi}{2}
						\item	Calculate step size in coded units.
					\end{enumerate}
				
				\par}
			}
			
			\onslide+<1->{
				{\tiny 
					\begin{enumerate}\setcounter{enumi}{3}
						\item	Convert these to real-world \emph{changes}.
					\end{enumerate}
				
				\par}
			}
			
			\onslide+<1->{
				{\tiny 
					\begin{enumerate}\setcounter{enumi}{4}
						\item	Get the real-world location
						of the next experiment.
					\end{enumerate}
				
				\par}
			}
			
			
			\vspace{-0.2cm}
			\onslide+<1->{
				{\tiny 
					\begin{enumerate}\setcounter{enumi}{5}
						\item	Convert these back\\ to coded-units.
					\end{enumerate}
				
				\par}
			}
				
		\column{0.01\textwidth}
			\rule[3mm]{0.01cm}{60mm}%
			
			
		\column{0.4\textwidth}
			\centerline{\textbf{Price}}
			
			\vspace{-0.5cm}
			\onslide+<1->{
				\vspace{0.cm}
				\begin{align*}
					\Delta x_\text{P} &= {\color{blue} = \dfrac{b_\text{P}}{b_\text{T}} \times \Delta x_\text{T} }
				\end{align*}
			}
			
			\vspace{-1.15cm}
			\onslide+<1->{
				\vspace{0.cm}
				\begin{align*}
					\Delta x_\text{P} &= {\color{blue}  \dfrac{-2.5}{7.5} \times 2 = -\dfrac{2}{3}}
				\end{align*}
			}
			
			\vspace{-0.65cm}
			\onslide+<1->{
				\vspace{-0.6cm}
				\begin{align*} 
					\Delta \text{P} &= {\color{blue} -\dfrac{2}{3}\cdot\frac{1}{2}(0.36) = -\$0.12}
				\end{align*}
			}
			
			\vspace{-1.2cm}
			\onslide+<1->{
				\begin{align*} 
					\text{P}^{(15)} &= {\color{blue} \text{P}^{(10)} + \Delta \text{P} = \$1.63 - 0.12 = \$1.51}
				\end{align*}
			}
			
			\vspace{-1.5cm}
			\onslide+<1->{	
				\begin{align*} 
					x_\text{P}^{(15)} &={\color{blue}\dfrac{1.51-1.63}{\tfrac{1}{2}\cdot (0.36)} = -\dfrac{2}{3}}
				\end{align*}
			}
		
		\column{0.01\textwidth}
			\rule[3mm]{0.01cm}{56.5mm}%
			
		\column{0.4\textwidth}
			\centerline{\textbf{Throughput}}
			
			$\Delta x_\text{T} = 2$ (this was chosen)
			
			\vspace{1.3cm}
			$\Delta x_\text{T} = 2$ 
		
			\vspace{-0.85cm}
			\onslide+<1->{
				\begin{align*} 
					\Delta \text{T} &= {\color{blue} 6} \\
				\end{align*}
			}
			
			\vspace{-2cm}
			\onslide+<1->{
				\begin{align*} 
					 \text{T}^{(15)} &=  {\color{blue} \text{T}^{(10)} + \Delta \text{T}}\\
					 \text{T}^{(15)} &= 339 + 6 = 345\\
				\end{align*}
			}
			
			\vspace{-1.9cm}
			\onslide+<1->{	
				\begin{align*} 
					x_\text{T}^{(15)} &= {\color{blue} 2} 
				\end{align*}
			}
	\end{columns}
	
	\vspace{-0.5cm}
	\begin{columns}[T]
		\column{0.2\textwidth}

			\vspace{-0.0cm}
			\onslide+<1->{
				{\tiny 
					\begin{enumerate}\setcounter{enumi}{6}
						\item	Predict the next experiment's outcome.
					\end{enumerate}
				
				\par}
			}
			
			\vspace{0cm}
			\onslide+<2->{
				{\tiny 
					\begin{enumerate}\setcounter{enumi}{7}
						\item	Now run the next experiment, and record the values
					\end{enumerate}
				
				\par}
			}
			
		\column{0.01\textwidth}
			\rule[3mm]{0.01cm}{85mm}%
			
		\column{0.912\textwidth}
			
			\onslide+<1->{	
				\hrule
				\vspace{-0.2cm}
				\begin{align*}
					\hat{y}       &&=&& 723.6 &&-&& 2.5  x_\text{P} &&+&& 7.5 x_\text{T} &&-&& 1.5x_\text{P}x_\text{T}&& \\
					\hat{y}^{(15)} && \approx&& {\color{blue}\$ 742~\text{profit per hour}} \span\omit\span\omit\span\omit\span\omit\span\omit
				\end{align*}
			}
			
			\vspace{-1.6cm}
			\onslide+<2->{	
				\begin{align*}
					y^{(15)} &=  \color{blue} \$ 735 ~\text{profit per hour}
				\end{align*}
			}
	\end{columns}
	
\end{frame}

\begin{frame}\frametitle{Dealing with curvature when we suspect we are approaching an optimum}
	{\color{myOrange}We detect curvature in several ways (in practice, you will detect one or more of these)}
	
	\vspace{.7cm}
	\begin{enumerate}
		\item	interaction terms are comparable to main effects
	\end{enumerate}
	
	
	\begin{columns}[T]
		\column{0.33\textwidth}
			\centerline{\includegraphics[width=\textwidth]{../5G/Supporting materials/no-interaction.pdf}}
			\vspace{-0.5cm}
			\[ \hat{y} = -2.5 x_\text{P} +7.5 x_\text{T} + \cancelto{0}{0 \cdot x_\text{P}x_\text{T}} \]
		\column{0.33\textwidth}
			\centerline{\includegraphics[width=\textwidth]{../5G/Supporting materials/with-interaction.pdf}}
			\[\hat{y} = -2.5 x_\text{P} +  7.5 x_\text{T} \color{red}-1.5 x_\text{P}x_\text{T}\]
		\column{0.33\textwidth}
			\centerline{\includegraphics[width=\textwidth]{../5G/Supporting materials/system-with-a-saddle.pdf}}
			\vspace{-0.5cm}
			\[\hat{y} = -2.5 x_\text{P} +  5.35 x_\text{T} \color{red}- 7.5 x_\text{P}x_\text{T}\]
	\end{columns}
	
\end{frame}

\begin{frame}\frametitle{Dealing with curvature when we suspect we are approaching an optimum}
	{\color{myOrange}We detect curvature in several ways (in practice, you will detect one or more of these)}
	
	\vspace{.7cm}
	\begin{enumerate}\setcounter{enumi}{1}
		\item	differences in the {\color{myGreen} spread}$^\ast$ are becoming smaller and smaller
	\end{enumerate}
	
	\vspace{.3cm}
	$^\ast${\color{myGreen}\scriptsize can be crudely quantified as (highest outcome) $-$ (lowest outcome)}
	\vspace{.5cm}
	\begin{columns}[T]
		\column{0.33\textwidth}
			Factorial 1
			
			\vspace{.5cm}
			\centerline{\includegraphics[width=\textwidth]{../5G/Supporting materials/factorial-1.png}}
			
			spread = \$ 378
		\column{0.33\textwidth}
		
			\onslide+<2->{
				Factorial 2 
			
				\vspace{.5cm}
				\centerline{\includegraphics[width=\textwidth]{../5G/Supporting materials/factorial-2.png}}
			
				spread = \$ 139
			}
		\column{0.33\textwidth}
			\onslide+<3->{
				Factorial 3
				
				\vspace{.5cm}
				\centerline{\includegraphics[width=\textwidth]{../5G/Supporting materials/factorial-3.png}}
			
				\vspace{0.4cm}
				spread = \$ 20
			}
	\end{columns}
	
\end{frame}

\begin{frame}\frametitle{{\color{red} Aside}: If we are judging differences, we need an estimate of ``noise''}
	\begin{columns}[T]
		\column{0.5\textwidth}
			\centerline{\includegraphics[width=\textwidth]{../5E/Supporting materials/RSM-35.png}}
		\column{0.5\textwidth}
			\centerline{\includegraphics[width=\textwidth]{../5E/Supporting materials/RSM-36.png}}
	\end{columns}	
\end{frame}

\begin{frame}\frametitle{Dealing with curvature when we suspect we are approaching an optimum}
	{\color{myOrange}We detect curvature in several ways (in practice, you will detect one or more of these)}
	
	\vspace{.7cm}
	\begin{enumerate}\setcounter{enumi}{2}
		\item	we notice prediction errors with our model that indicate the surface is changing
	\end{enumerate}
	
	
	\vspace{.5cm}
	\begin{columns}[T]
		\column{0.33\textwidth}
			
			\onslide+<2->{
			
				\centerline{\includegraphics[width=\textwidth]{../5G/Supporting materials/flickr-rmkoske-2558420300_a9a3005dc7_o-burned-modified.jpg}}
				\see{\href{https://secure.flickr.com/photos/67146024@N00/2558420300/}{Flickr: rmkoske}}
			}
			
		\column{0.33\textwidth}
		
			\onslide+<2->{
			
				\vspace{.5cm}
				
			
			}
		\column{0.33\textwidth}
			\onslide+<3->{
				
				\centerline{\includegraphics[width=\textwidth]{../5G/Supporting materials/flickr-93653851-8514973597_3e189570e2_o-cliff.jpg}}
				\vspace{-1.2cm}
				\see{\href{https://secure.flickr.com/photos/93653851@N04/8514973597/}{Flickr: Digital Temi}}
			}
	\end{columns}
	
\end{frame}

\begin{frame}\frametitle{}
	\centerline{\includegraphics[height=\textheight]{../5E/Supporting materials/RSM-37.png}}
\end{frame}

\begin{frame}\frametitle{Dealing with curvature when we suspect we are approaching an optimum}
	{\color{myOrange}We detect curvature in several ways (in practice, you will detect one or more of these)}
	
	\vspace{.7cm}
	\begin{enumerate}\setcounter{enumi}{3}
		\item	we detect ``lack of fit'' in our empirical model
	\end{enumerate}
	
	\begin{columns}[T]
		\column{0.2\textwidth}

		\column{0.912\textwidth}
			
			\centerline{\includegraphics[height=0.7\textheight]{\imagedir/doe/lack-of-fit-illustration.png}}
	\end{columns}
	
\end{frame}

\begin{frame}\frametitle{Dealing with curvature when we suspect we are approaching an optimum}
	{\color{myOrange}We detect curvature in several ways (in practice, you will detect one or more of these)}
	
	\vspace{.7cm}
	\begin{enumerate}\setcounter{enumi}{3}
		\item	we detect ``lack of fit'' in our empirical model
	\end{enumerate}
	
	\begin{columns}[T]
		\column{0.35\textwidth}
			\onslide+<1->{
				\centerline{\includegraphics[width=\textwidth]{../5G/Supporting materials/factorial-1.png}}
				
				\scriptsize
				$\hat{y} = 390  +134  x_\text{P} +  55 x_\text{T} -3.5 x_\text{P}x_\text{T}$
				\normalsize
				
				\vspace{0.2cm}
				
					$\left.\begin{array}{c}
						y^{(0)} = \$ 407\\
						\hat{y}^{(0)} = \$ 390
					\end{array}\onslide+<2->{  \right\} }$
					 \onslide+<2->{
					 	{\scriptsize difference = \$17}
					} 
			}
			
		\column{0.01\textwidth}
			\rule[3mm]{0.01cm}{60mm}%
			
		\column{0.35\textwidth}
			\onslide+<3->{
				\centerline{\includegraphics[width=\textwidth]{../5G/Supporting materials/factorial-2.png}}
				
				\scriptsize
				$\hat{y} =  645  +47 x_\text{P} +  22.5 x_\text{T} -2.0 x_\text{P}x_\text{T}$
				\normalsize
				
				\vspace{0.2cm}
				
				
					$\left.\begin{array}{c}
						y^{(8)} = \$ 657\\
						\hat{y}^{(8)} = \$ 645
					\end{array}\right\}$ {\scriptsize difference = \$12}
			}
			
		\column{0.01\textwidth}
			\rule[3mm]{0.01cm}{60mm}%
			
		\column{0.37\textwidth}
			\onslide+<4->{
				\centerline{\includegraphics[width=\textwidth]{../5G/Supporting materials/factorial-3-with-multiple-centers.png}}
				
				\vspace{0.4cm}
				\scriptsize
				$\hat{y} = 724  -2.5 x_\text{P} +  7.5 x_\text{T} -1.5 x_\text{P}x_\text{T} $
				\normalsize
				
				\vspace{0.2cm}
				
				$\left.\begin{array}{c}
					y^{(\text{mid})} = \$ 734\\
					\hat{y}^{(\text{mid})} = \$ 724
				\end{array}\right\}$ {\scriptsize difference = \$10}
			}
	\end{columns}
\end{frame}

\begin{frame}\frametitle{The one-dimensional equivalent of what we are seeing in two dimensions}
	
	{\color{myOrange}We detect curvature in several ways (in practice, you will detect one or more of these)}
	
	\vspace{.7cm}
	\begin{enumerate}\setcounter{enumi}{3}
		\item	we detect ``lack of fit'' in our empirical model
	\end{enumerate}
	
	\begin{columns}[T]
		\column{0.68\textwidth}
			\centerline{\includegraphics[width=5cm]{\imagedir/doe/lack-of-fit-illustration-linear-region.png}}
		\column{0.01\textwidth}
			\rule[3mm]{0.01cm}{60mm}%
		\column{0.32\textwidth}
			\centerline{\includegraphics[width=5cm]{\imagedir/doe/lack-of-fit-illustration.png}}
	\end{columns}	
\end{frame}

\begin{frame}\frametitle{Dealing with curvature when we suspect we are approaching an optimum}
	{\color{myOrange}We detect curvature in several ways (in practice, you will detect one or more of these)}
	
	\vspace{.7cm}
	\begin{enumerate}
		\item	interaction terms are comparable to main effects
		\item	differences in the spread are becoming smaller and smaller
		\item	we notice prediction errors with our model that indicate the surface is changing
		\item	we detect ``lack of fit''
		\item	confidence intervals show some model coefficients are not significant
	\end{enumerate}
	
\end{frame}

\begin{frame}\frametitle{Improving the model's prediction ability by adding quadratic terms}
	Current situation, displaying lack of fit:
	
	\centerline{\includegraphics[height=.8\textheight]{\imagedir/doe/lack-of-fit-illustration-nonlinear.png}}
\end{frame}

\begin{frame}\frametitle{Improving the model's prediction ability by adding quadratic terms}
	Add specially placed points:
	
	\centerline{\includegraphics[height=.8\textheight]{\imagedir/doe/lack-of-fit-illustration-nonlinear-extra-points.png}}
\end{frame}

\begin{frame}\frametitle{Improving the model's prediction ability by adding quadratic terms}
	And fit a quadratic model now:
	
	\centerline{\includegraphics[height=.8\textheight]{\imagedir/doe/lack-of-fit-illustration-nonlinear-extra-points-quadratic.png}}
\end{frame}

\begin{frame}\frametitle{Adding experiments so we can fit quadratic terms: where do we put them?}
	\begin{columns}[T]
		\column{0.5\textwidth}
			\centerline{\includegraphics[width=\textwidth]{\imagedir/doe/central-composite-design-MOOC-FCD-points-no-persective.png}}
		\column{0.5\textwidth}			
			\centerline{\includegraphics[width=\textwidth]{\imagedir/doe/central-composite-design-MOOC-CCD-points-no-persective.png}}
			\vspace{-0.5cm}
			\[\alpha = 1.41\]
	\end{columns}
\end{frame}

\begin{frame}\frametitle{Adding experiments so we can fit quadratic terms: where do we put them?}
	\begin{columns}[T]
		\column{0.5\textwidth}
			\centerline{\includegraphics[width=\textwidth]{\imagedir/doe/central-composite-design-MOOC-FCD-points.png}}
		\column{0.5\textwidth}			
			\centerline{\includegraphics[width=\textwidth]{\imagedir/doe/central-composite-design-MOOC-CCD-points.png}}
			
			\vspace{-0.5cm}
			\[\alpha = 1.41\]
	\end{columns}
\end{frame}

\begin{frame}\frametitle{Adding experiments so we can fit quadratic terms: where do we put them?}
	\begin{columns}[T]
		\column{0.5\textwidth}
		
			\vspace{1cm}
			
			\begin{itemize}
				\item	Run the factorial points first
				\onslide+<2->{
					\item	Then run the star (axial) points after$^\ast$
				}
				\onslide+<2->{
					\Large
					\[\alpha  = \left(2^k\right)^{0.25}\]
					
			
					\begin{itemize}
						\onslide+<2->{\item	$k = 2$; then $\alpha = \sqrt{2} \approx 1.41$ }
						\onslide+<2->{\item	$k = 3$; then $\alpha = 2^{\tfrac{3}{4}} \approx 1.68$}
					\end{itemize}
					\normalsize
				}
				\onslide+<2->{
					\item	Run several center points randomly, during the above two stages
				}
			\end{itemize}
			
			\onslide+<2->{
				\vspace{0cm}
				\tiny
				$^\ast$ Astute viewers will wonder about confounding disturbances, since these runs are not randomized.
			
				You can show that these {\color[rgb]{0,0.5,1}\textbf{two}} {\color[rgb]{0.5,0, 0.5}\textbf{groups}} of experiments are blocked.
			}
			
			
		\column{0.5\textwidth}			
			\centerline{\includegraphics[width=\textwidth]{\imagedir/doe/central-composite-design-MOOC-anim-01.png}}
			
			\vspace{-0.5cm}
			\[\alpha = 1.41\]
	\end{columns}
\end{frame}

\begin{frame}\frametitle{Adding experiments so we can fit quadratic terms: where do we put them?}
	\begin{columns}[T]
		\column{0.5\textwidth}
		
			\vspace{1cm}
			
			\begin{itemize}
				\item	Run the factorial points first
				\onslide+<1->{
					\item	Then run the star (axial) points after$^\ast$
				}
				\onslide+<1->{
					\Large
					\[\alpha  = \left(2^k\right)^{0.25}\]
					
			
					\begin{itemize}
						\onslide+<2->{\item	$k = 2$; then $\alpha = \sqrt{2} \approx 1.41$ }
						\onslide+<3->{\item	$k = 3$; then $\alpha = 2^{\tfrac{3}{4}} \approx 1.68$}
					\end{itemize}
					\normalsize
				}
				\onslide+<4->{
					\item	Run several center points randomly, during the above two stages
				}
			\end{itemize}
			
			\onslide+<4->{
				\vspace{0cm}
				\tiny
				$^\ast$ Astute viewers will wonder about confounding disturbances, since these runs are not randomized.
			
				You can show that these {\color[rgb]{0,0.5,1}\textbf{two}} {\color[rgb]{0.5,0, 0.5}\textbf{groups}} of experiments are blocked.
			}
			
			
		\column{0.5\textwidth}			
			\centerline{\includegraphics[width=\textwidth]{\imagedir/doe/central-composite-design-MOOC-anim-02.png}}
			
			\vspace{-0.5cm}
			\[\alpha = 1.41\]
	\end{columns}
\end{frame}

\begin{frame}\frametitle{Adding experiments so we can fit quadratic terms: where do we put them?}
	\begin{columns}[T]
		\column{0.5\textwidth}
		
			\vspace{1cm}
			
			\begin{itemize}
				\item	Run the factorial points first
				\onslide+<1->{
					\item	Then run the star (axial) points after$^\ast$
				}
				\onslide+<1->{
					\Large
					\[\alpha  = \left(2^k\right)^{0.25}\]
					
			
					\begin{itemize}
						\onslide+<1->{\item	$k = 2$; then $\alpha = \sqrt{2} \approx 1.41$ }
						\onslide+<1->{\item	$k = 3$; then $\alpha = 2^{\tfrac{3}{4}} \approx 1.68$}
					\end{itemize}
					\normalsize
				}
				\onslide+<2->{
					\item	Run several center points randomly, during the above two stages
				}
			\end{itemize}
			
			\onslide+<2->{
				\vspace{0cm}
				\tiny
				$^\ast$ Astute viewers will wonder about confounding disturbances, since these runs are not randomized.
			
				You can show that these {\color[rgb]{0,0.5,1}\textbf{two}} {\color[rgb]{0.5,0, 0.5}\textbf{groups}} of experiments are blocked.
			}
			
			
		\column{0.5\textwidth}			
			\centerline{\includegraphics[width=\textwidth]{\imagedir/doe/central-composite-design-MOOC-3-factors.png}}
			
			\vspace{-0.5cm}
			\[\alpha = 1.68\]
	\end{columns}
\end{frame}

\begin{frame}\frametitle{Adding experiments so we can fit quadratic terms: where do we put them?}
	\begin{columns}[T]
		\column{0.5\textwidth}
		
			\vspace{1cm}
			
			\begin{itemize}
				\item	Run the factorial points first
				\onslide+<1->{
					\item	Then run the star (axial) points after$^\ast$
				}
				\onslide+<1->{
					\Large
					\[\alpha  = \left(2^k\right)^{0.25}\]
					
			
					\begin{itemize}
						\onslide+<1->{\item	$k = 2$; then $\alpha = \sqrt{2} \approx 1.41$ }
						\onslide+<1->{\item	$k = 3$; then $\alpha = 2^{\tfrac{3}{4}} \approx 1.68$}
					\end{itemize}
					\normalsize
				}
				\onslide+<1->{
					\item	Run several center points randomly, during the above two stages
				}
			\end{itemize}
			
			\onslide+<2->{
				\vspace{0cm}
				\tiny
				$^\ast$ Astute viewers will wonder about confounding disturbances, since these runs are not randomized.
			
				You can show that these {\color[rgb]{0,0.5,1}\textbf{two}} {\color[rgb]{0.5,0, 0.5}\textbf{groups}} of experiments are blocked.
			}
			
			
		\column{0.5\textwidth}			
			\centerline{\includegraphics[width=\textwidth]{\imagedir/doe/central-composite-design-MOOC-anim-03.png}}
			
			\vspace{-0.5cm}
			\[\alpha = 1.41\]
	\end{columns}
\end{frame}

\begin{frame}\frametitle{Adding experiments so we can fit quadratic terms: where do we put them?}
	\begin{columns}[T]
		\column{0.5\textwidth}
		
			\vspace{1cm}
			
			\begin{itemize}
				\item	Run the factorial points first
				\onslide+<1->{
					\item	Then run the star (axial) points after$^\ast$
				}
				\onslide+<1->{
					\Large
					\[\alpha  = \left(2^k\right)^{0.25}\]
					
			
					\begin{itemize}
						\onslide+<1->{\item	$k = 2$; then $\alpha = \sqrt{2} \approx 1.41$ }
						\onslide+<1->{\item	$k = 3$; then $\alpha = 2^{\tfrac{3}{4}} \approx 1.68$}
					\end{itemize}
					\normalsize
				}
				\onslide+<1->{
					\item	Run several center points randomly, during the above two stages
				}
			\end{itemize}
			
			\onslide+<1->{
				\vspace{0cm}
				\tiny
				$^\ast$ Astute viewers will wonder about confounding disturbances, since these runs are not randomized.
			
				You can show that these {\color[rgb]{0,0.5,1}\textbf{two}} {\color[rgb]{0.5,0, 0.5}\textbf{groups}} of experiments are blocked.
			}
			
			
		\column{0.5\textwidth}			
			\centerline{\includegraphics[width=\textwidth]{\imagedir/doe/central-composite-design-MOOC-anim-04.png}}
			
			\vspace{-0.5cm}
			\[\alpha = 1.41\]
	\end{columns}
\end{frame}

\begin{frame}\frametitle{Adding experiments so we can fit quadratic terms: where do we put them?}
	\begin{columns}[T]
		\column{0.5\textwidth}
		
			\vspace{1cm}
			
			\begin{itemize}
				\item	Run the factorial points first
				\onslide+<1->{
					\item	Then run the star (axial) points after$^\ast$
				}
				\onslide+<1->{
					\Large
					\[\alpha  = \left(2^k\right)^{0.25}\]
					
			
					\begin{itemize}
						\onslide+<1->{\item	$k = 2$; then $\alpha = \sqrt{2} \approx 1.41$ }
						\onslide+<1->{\item	$k = 3$; then $\alpha = 2^{\tfrac{3}{4}} \approx 1.68$}
					\end{itemize}
					\normalsize
				}
				\onslide+<1->{
					\item	Run several center points randomly, during the above two stages
				}
			\end{itemize}
			
			\onslide+<1->{
				\vspace{0cm}
				\tiny
				$^\ast$ Astute viewers will wonder about confounding disturbances, since these runs are not randomized.
			
				You can show that these {\color[rgb]{0,0.5,1}\textbf{two}} {\color[rgb]{0.5,0, 0.5}\textbf{groups}} of experiments are blocked.
			}
			
			
		\column{0.5\textwidth}			
			\centerline{\includegraphics[width=\textwidth]{\imagedir/doe/central-composite-design-MOOC-anim-05.png}}
			
			\vspace{-0.5cm}
			\[\alpha = 1.41\]
	\end{columns}
\end{frame}

\begin{frame}\frametitle{}
	\centerline{\includegraphics[height=\textheight]{../5E/Supporting materials/RSM-38.png}}
\end{frame}
\begin{frame}\frametitle{}
	\centerline{\includegraphics[height=\textheight]{../5E/Supporting materials/RSM-39.png}}
\end{frame}
\begin{frame}\frametitle{}
	\centerline{\includegraphics[height=\textheight]{../5E/Supporting materials/RSM-40.png}}
\end{frame}
\begin{frame}\frametitle{}
	\centerline{\includegraphics[height=\textheight]{../5E/Supporting materials/RSM-41.png}}
\end{frame}
\begin{frame}\frametitle{}
	\centerline{\includegraphics[height=\textheight]{../5E/Supporting materials/RSM-42.png}}
\end{frame}
\begin{frame}\frametitle{List of all experiments}
	\begin{tabulary}{\linewidth}{c|cc|cc|c|c|cc}
		\textbf{\relax Experiment} & \textbf{\relax P } & \textbf{\relax T} & \textbf{\relax $x_\text{P}$} & \textbf{\relax $x_\text{T}$} & \textbf{\relax $\hat{y}$} & \textbf{\relax Actual = $y$} & \textbf{\relax Type } \\ \hline
			10 & ~~1.63 & 339 & $~0$ & $~0$ &  & \$ 732 &~Center \\
			11 & ~~1.45 & 336 & $-1$ & $-1$ &  & ~~715  &~Factorial \\
			12 & ~~1.81 & 336 & $+1$ & $-1$ &  & ~~713  &~Factorial \\
			13 & ~~1.45 & 342 & $-1$ & $+1$ &  & ~~733  &~Factorial \\ 
			14 & ~~1.81 & 342 & $+1$ & $+1$ &  & ~~725  &~Factorial \\
			15 & ~~1.51 & 345 & $-\tfrac{2}{3}$ & $+2$ & \$ 742 & ~~735  & Steepest path \\ \hline
			16 & ~~1.63 & 339 & $~0$ & $~0$ &  & \$ 733 &~Center \\
			17 & ~~1.63 & 339 & $~0$ & $~0$ &  & ~~737 &~Center \\ \hline
			\onslide+<2->{18 & ~~1.63 & 343 & $~0$ & $+1.41$ &  & \onslide+<6->{\$ 738} &~Star} \\ 
			\onslide+<3->{19 & ~~1.38 & 339 & $-1.41$ & $0$ &  & \onslide+<6->{~~717} &~Star \\ 
				20 & ~~1.63 & 335 & $0$ & $-1.41$ &  & \onslide+<6->{~~721} &~Star \\ 
				21 & ~~1.88 & 339 & $+1.41$ & $0$ &  & \onslide+<6->{~~710} &~Star}
			\onslide+<4->{\\ \hline
				22 & ~~1.63 & 339 & $~0$ & $~0$ &  & \onslide+<6->{\$ 735} &~Center} 
			\onslide+<7->{\\ \hline
				23 & ~~ &  &  &  &  &  &~Optimum?} 
			\end{tabulary}
\end{frame}
\begin{frame}\frametitle{Results of the 12 runs: (factorial + star + center). Are we nearly there?}
	\[\hat{y} = 734.23 -2.5x_\text{P}    +    6.97  x_\text{T}    -10.6  x^2_\text{P}     -2.5  x^2_\text{T}     -1.5x_\text{P}x_\text{T}\]
	\begin{columns}[T]
		\column{0.6\textwidth}
			\begin{enumerate}
				\onslide+<2->{
					\item	We get good predictions at the center,
							indicating a ``small lack of fit''
							\begin{itemize}
								\item	$\hat{y}^{(\text{center})} = \$ 734$
								\item	$y^{(\text{center})}_\text{actual} = \$ 734.25$
							\end{itemize}
				}
				\onslide+<3->{
					\vspace{0.5cm}
					\item	We have good predictions of other points in the region
						\centerline{\includegraphics[width=.8\textwidth]{../5G/Supporting materials/software-prediction.png}}
						\vspace{-0.3cm}
						\begin{itemize}
							\item	$\hat{y}^{(15)} = \$ 737$
							\item	$y^{(15)}_\text{actual} = \$ 735$
						\end{itemize}
					}
			\end{enumerate}
			
		\column{0.01\textwidth}
			\rule[3mm]{0.01cm}{60mm}%
		
		\column{0.4\textwidth}	
			Output from the R software:

			\centerline{\includegraphics[width=\textwidth]{../5G/Supporting materials/R-software-output.png}}

	\end{columns}
	
	
\end{frame}
\begin{frame}\frametitle{}
	\centerline{\includegraphics[height=\textheight]{../5E/Supporting materials/RSM-43.png}}
\end{frame}
\begin{frame}\frametitle{}
	\centerline{\includegraphics[height=\textheight]{../5E/Supporting materials/RSM-44.png}}
\end{frame}
\begin{frame}\frametitle{}
	\centerline{\includegraphics[height=\textheight]{../5E/Supporting materials/RSM-45.png}}
\end{frame}
\begin{frame}\frametitle{}
	\centerline{\includegraphics[height=\textheight]{../5E/Supporting materials/RSM-46.png}}
\end{frame}
\begin{frame}\frametitle{Finding the optimum analytically (with differentiation)}
	\[\hat{y} =	 734  -2.5x_\text{P}    +    6.97  x_\text{T}    -10.6  x^2_\text{P}     -2.5  x^2_\text{T}     -1.5x_\text{P}x_\text{T}	 \]

	\onslide+<2->{
		\begin{align*}
			\dfrac{\partial \hat{y}}{\partial x_\text{P}} &&=&& -2.5 &&-&& 21.2x_\text{P} &&-&& 1.5x_\text{T} &&=&& 0 \\
			\dfrac{\partial \hat{y}}{\partial x_\text{T}} &&=&& 6.97 &&-&& 1.5x_\text{P} &&-&& 5.0x_\text{T} &&=&& 0 \\
		\end{align*}
	}
	\vspace{-0.5cm}
	\onslide+<3->{
		\[
		\begin{pmatrix}-21.2 && -1.5 \\ \\ -1.5 && -5.0\end{pmatrix}\begin{pmatrix}x_\text{P} \\ \\ x_\text{T}\end{pmatrix} = \begin{pmatrix}2.5 \\ \\ -6.97\end{pmatrix}
		\]
	}
	\vspace{0.4cm}
	\onslide+<4->{
		\begin{align*}
			x^{(23)}_\text{P} &= -0.22 &&& \onslide+<5->{\text{P}^{(23)}& = -0.22 \cdot \tfrac{1}{2}(0.36) +1.63 = {\color{red}\$1.59}}\\
			x^{(23)}_\text{T} &= 1.46 &&&  \onslide+<5->{\text{T}^{(23)} &= 1.46\cdot \tfrac{1}{2}(6) +339 \approx {\color{red} 343~\text{parts per hour}}} \\
		\end{align*}
	}
	
\end{frame}
\begin{frame}\frametitle{}
	\centerline{\includegraphics[height=\textheight]{../5E/Supporting materials/RSM-47.png}}
\end{frame}
\begin{frame}\frametitle{}
	\centerline{\includegraphics[height=\textheight]{../5E/Supporting materials/RSM-48.png}}
\end{frame}
\begin{frame}\frametitle{}
	\centerline{\includegraphics[height=\textheight]{../5E/Supporting materials/RSM-49.png}}
\end{frame}
\begin{frame}\frametitle{}
	\centerline{\includegraphics[height=\textheight]{../5E/Supporting materials/RSM-50.png}}
\end{frame}
\begin{frame}\frametitle{}
	\centerline{\includegraphics[height=\textheight]{../5E/Supporting materials/RSM-51.png}}
\end{frame}
\begin{frame}\frametitle{}
	\centerline{\includegraphics[height=\textheight]{../5E/Supporting materials/RSM-52.png}}
\end{frame}
\begin{frame}\frametitle{}
	\centerline{\includegraphics[height=\textheight]{../5E/Supporting materials/RSM-53.png}}
\end{frame}
\begin{frame}\frametitle{}
	\centerline{\includegraphics[height=\textheight]{../5E/Supporting materials/RSM-54.png}}
\end{frame}
\begin{frame}\frametitle{}
	\centerline{\includegraphics[height=\textheight]{../5E/Supporting materials/RSM-55.png}}
\end{frame}
\begin{frame}\frametitle{}
	\centerline{\includegraphics[height=\textheight]{../5E/Supporting materials/RSM-56.png}}
\end{frame}
\begin{frame}\frametitle{}
	\centerline{\includegraphics[height=\textheight]{../5E/Supporting materials/RSM-57.png}}
\end{frame}
\begin{frame}\frametitle{}
	\centerline{\includegraphics[height=\textheight]{../5E/Supporting materials/RSM-58.png}}
\end{frame}
\begin{frame}\frametitle{}
	\centerline{\includegraphics[height=\textheight]{../5E/Supporting materials/RSM-59.png}}
\end{frame}
\begin{frame}\frametitle{}
	\centerline{\includegraphics[height=\textheight]{../5E/Supporting materials/RSM-60.png}}
\end{frame}
