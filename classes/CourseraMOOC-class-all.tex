

% /Users/kevindunn/Dropbox/Coursera/Media/All-course-slides/classes/CourseraMOOC-class-1A.tex

\begin{frame}\frametitle{}
	\begin{itemize}
		\item	\LARGE an {\color{purple}objective} to improve
		\item	\LARGE multiple {\color{purple}factors} that we can adjust
	\end{itemize}
\end{frame}

\begin{frame}\frametitle{Google's YouTube experiment to maximize new sign-ups}
	\begin{center}
		\Huge{\href{http://yint.org/youtube-experiment}{http://yint.org/youtube-experiment}}
	\end{center}	
\end{frame}

\begin{frame}\frametitle{Ghassan Marajaba: strength of concrete veneer}
	\begin{center}
		\includegraphics[width=0.48\textwidth]{\imagedir/doe/examples/Ghassan-Marjaba-1.png}
		\includegraphics[width=0.48\textwidth]{\imagedir/doe/examples/Ghassan-Marjaba-2.png}
	\end{center}
\end{frame}

\begin{frame}\frametitle{\href{http://yint.org/concrete-strength}{http://yint.org/concrete-strength}}
	\begin{center}
		\includegraphics[width=\textwidth]{\imagedir/doe/examples/Concrete-Lafarge.png}
	\end{center}
\end{frame}

\begin{frame}\frametitle{Please make a video about your improvement!}
	{\Huge \href{http://yint.org/my-experiment}{http://yint.org/my-experiment}
	
	\vspace{24pt}	
	Why don't you upload a short video showing others in the class what you want to experiment on?	
	}
\end{frame}


% /Users/kevindunn/Dropbox/Coursera/Media/All-course-slides/classes/CourseraMOOC-class-1B.tex

\begin{frame}\frametitle{Some important terminology we will use all the time}
	
	\begin{columns}[T]
		\column{0.75\textwidth}
			\textbf{{\color{purple} Outcome}}
				\begin{itemize}
					\item	What we measure after the experiment is finished  \pause
					\item	It is the aspect you are interested in improving.\pause
			 
				\end{itemize}
		\column{0.3\textwidth}
		
			\centerline{\includegraphics[width=\textwidth]{\imagedir/doe/measure-4904403417_93baa750a6-flickr.jpg}}

	\end{columns}
	
			
	\vspace{24pt}
	\textbf{{\color{purple} Factors}}
		\begin{itemize}
			\item	Things which you actively change to influence the outcome.
			\item	We typically change 2, 3, 4, or many more factors. \pause
			\item	Don't fixate on changing 1 factor at a time.
		\end{itemize}
\end{frame}

\begin{frame}\frametitle{}
	
	\begin{columns}[c]
		\column{0.5\textwidth}
			\centerline{\includegraphics[width=0.8\textwidth]{../1B/Supporting-material/plant.png}}
		
		\column{0.5\textwidth}
			Various outcomes are possible in your experiment:
			
			\begin{itemize}
				\item	height of the plant
				\item	average length of leaves
				\item	the number of flowers
			\end{itemize}
			
			\vspace{12pt}
			These are examples of numeric measurements (quantitative).

	\end{columns}
\end{frame}

\begin{frame}\frametitle{}
	
	\begin{columns}[c]
		\column{0.5\textwidth}
			\centerline{\includegraphics[width=0.8\textwidth]{../1B/Supporting-material/plant.png}}
		
		\column{0.5\textwidth}
			%Different outcomes are possible:
			
			\begin{itemize}
				\item	colour of the flower
			\end{itemize}
			
			\vspace{12pt}
			This is a qualitative measurement.
			
			(We use qualitative outcomes infrequently)

	\end{columns}
\end{frame}

\begin{frame}\frametitle{}
	
	\textbf{{\color{purple} Outcome}} \onslide+<5->{ = \textbf{{\color{purple} Response}}}
		\begin{itemize}
			\item	What we measure after the experiment is finished  
			\item	It is the aspect you are interested in improving.
		\end{itemize}
	
	\vspace{12pt}
	\pause
	
	\textbf{{\color{purple} Objective}}
	
	
		\qquad combine the {\color{purple} outcome} with ``a desire to \emph{adjust} the outcome''
	\vspace{24pt}
	\pause
	
	{\textbf{{Various examples of ``objectives''}}}
	
		\begin{itemize}
			\item	maximize $(\uparrow)$ the profit
			\item	maximize $(\uparrow)$ the height of the plant
			\item	minimize $(\downarrow)$ pollution
			\item	minimize $(\downarrow)$ energy used to produce a product
		\end{itemize} 	
		
	\pause
		\vspace{12pt}
		But sometimes the objective is ``the same as before'' $(=)$
\end{frame}

\begin{frame}\frametitle{}
	
	\textbf{{\color{purple} Factors}} \onslide+<6->{ = \textbf{{\color{purple} Variables}}}
	\pause
		
		\begin{columns}[T]
			\column{0.33\textwidth}
				\onslide+<2->{\centerline{\includegraphics[width=\textwidth]{../1B/Supporting-material/water.png}}}
				
				
			\column{0.33\textwidth}
				\onslide+<3->{\centerline{\includegraphics[width=\textwidth]{../1B/Supporting-material/fertilizer.png}}}
				
				
			\column{0.33\textwidth}
				\onslide+<4->{\centerline{\includegraphics[width=\textwidth]{../1B/Supporting-material/soil.png}}}
			
		\end{columns}
		
		
	\vspace{24pt}
	
	\onslide+<5->{
		\textbf{{\color{purple} Types of factors}}
	
	
		\qquad \emph{numeric} factors (quantitative) can be measured and adjusted to different levels
		
			\qquad \qquad 
		\vspace{12pt}
	
		\qquad \emph{categorical} factors (qualitative) are simply selected for their characteristic
	}
	
	
\end{frame}

{\usebackgroundtemplate{\vbox to \paperheight{\vfil\hbox to \paperwidth{\hfil  
    \includegraphics[width=0.95\paperwidth, clip]
	{../1B/Slides/01Screen Shot 2015-08-06 at 22.40.09 .png}  \hfil}\vfil}}
\begin{frame}\frametitle{}
\end{frame}}

{\usebackgroundtemplate{\vbox to \paperheight{\vfil\hbox to \paperwidth{\hfil  
    \includegraphics[width=0.95\paperwidth, clip]
	{../1B/Slides/02Screen Shot 2015-08-06 at 22.40.32 .png}  \hfil}\vfil}}
\begin{frame}\frametitle{}
\end{frame}}

{\usebackgroundtemplate{\vbox to \paperheight{\vfil\hbox to \paperwidth{\hfil  
    \includegraphics[width=0.95\paperwidth, clip]
	{../1B/Slides/03Screen Shot 2015-08-06 at 22.40.40 .png}  \hfil}\vfil}}
\begin{frame}\frametitle{}
\end{frame}}




% /Users/kevindunn/Dropbox/Coursera/Media/All-course-slides/classes/CourseraMOOC-class-1C.tex

\begin{frame}\frametitle{}
	\LARGE \textbf{{\color{purple} Outcome = Response}}
	
	
	
	\begin{flushright}
		\includegraphics[width=0.5\textwidth]{\imagedir/doe/measure-4904403417_93baa750a6-flickr.jpg}
		
		\see{\href{https://www.flickr.com/photos/free-stock/4904403417/}{Flickr: free-stock}}
	\end{flushright}
\end{frame}

\begin{frame}\frametitle{}
	\LARGE \textbf{{\color{purple} Factors = Variables}}
	
	
	
	\begin{flushright}
		\includegraphics[width=0.45\textwidth]{../1C/Supporting-materials/1B-fertilizer.png}
		\includegraphics[width=0.45\textwidth]{../1C/Supporting-materials/1B-typeofsoil.png}

		Amount of soil	\hfill Type of soil 
		
		(Numeric)	\hfill (Categorical)
	\end{flushright}
\end{frame}

\begin{frame}\frametitle{Experiment for this class}
	
	{\LARGE\textbf{{\color{purple} Outcome}}}
	
		\qquad 	{\color{purple} Profit} made in one day when selling a specific product
		
	\vspace{24pt}
	\pause
	
	{\LARGE\textbf{{\color{purple} Our outcome today}}}
	
		\qquad 	Profit = total income $-$ total expenses


	
\end{frame}

\begin{frame}\frametitle{Experiment for this class}
	
	{\LARGE\textbf{{\color{purple} Objective}}}
	
	
		\qquad combine the {\color{purple} outcome} with a desire to \emph{adjust} the outcome
		
	\vspace{24pt}
	\pause
	
	{\LARGE\textbf{{\color{purple} Our objective today}}}
	
		\qquad 	to \emph{maximize} profit
\end{frame}

\begin{frame}\frametitle{Experiment for this class}
	
	{\LARGE\textbf{{\color{purple} Factors}}}
	
	\vspace{24pt}
		\qquad 1. the amount of light in the store
		
	\vspace{12pt}
	
		\pause
		\qquad 2. the price of the product
	
\end{frame}

\begin{frame}\frametitle{Always try to make a prediction}
	
	{\LARGE\textbf{{\color{purple} How will price affect the outcome (profit)?}}}
	\pause
	
	\vspace{24pt}
		\qquad Does a higher price increase the profit?
		\pause
	
	\vspace{12pt}	
		\qquad Does a higher price decrease the profit?
		
		\pause
		
	\vspace{12pt}
		\qquad Does a higher price actually affect profit?
		
		\pause
		
	\vspace{24pt}
	\pause
	
	{\LARGE\textbf{{\color{purple} How will the lighting amount affect the outcome?}}}
\end{frame}

{\usebackgroundtemplate{\vbox to \paperheight{\vfil\hbox to \paperwidth{\hfil  
    \includegraphics[width=0.95\paperwidth, clip]
	{../1C/Slides/01Screen Shot 2015-08-06 at 22.34.17 .png}  \hfil}\vfil}}
\begin{frame}\frametitle{}
\end{frame}}

{\usebackgroundtemplate{\vbox to \paperheight{\vfil\hbox to \paperwidth{\hfil  
    \includegraphics[width=0.95\paperwidth, clip]
	{../1C/Slides/02Screen Shot 2015-08-06 at 22.34.42 .png}  \hfil}\vfil}}
\begin{frame}\frametitle{}
\end{frame}}

{\usebackgroundtemplate{\vbox to \paperheight{\vfil\hbox to \paperwidth{\hfil  
    \includegraphics[width=0.95\paperwidth, clip]
	{../1C/Slides/03Screen Shot 2015-08-06 at 22.35.01 .png}  \hfil}\vfil}}
\begin{frame}\frametitle{}
\end{frame}}

{\usebackgroundtemplate{\vbox to \paperheight{\vfil\hbox to \paperwidth{\hfil  
    \includegraphics[width=0.95\paperwidth, clip]
	{../1C/Slides/04Screen Shot 2015-08-06 at 22.35.11 .png}  \hfil}\vfil}}
\begin{frame}\frametitle{}
\end{frame}}

{\usebackgroundtemplate{\vbox to \paperheight{\vfil\hbox to \paperwidth{\hfil  
    \includegraphics[width=0.95\paperwidth, clip]
	{../1C/Slides/05Screen Shot 2015-08-06 at 22.35.22 .png}  \hfil}\vfil}}
\begin{frame}\frametitle{}
\end{frame}}

{\usebackgroundtemplate{\vbox to \paperheight{\vfil\hbox to \paperwidth{\hfil  
    \includegraphics[width=0.95\paperwidth, clip]
	{../1C/Slides/06Screen Shot 2015-08-06 at 22.35.37 .png}  \hfil}\vfil}}
\begin{frame}\frametitle{}
\end{frame}}


% /Users/kevindunn/Dropbox/Coursera/Media/All-course-slides/classes/CourseraMOOC-class-1D.tex


\begin{frame}\frametitle{}
	
	{\LARGE\textbf{{\color{purple} Numeric factors, both continuous}}}
	
	\vspace{24pt}
		\qquad 1. the amount of light in the store: \quad\, 50\% or 75\% on the dimmer
		
	\vspace{12pt}
	
		\pause
		\qquad 2. the price of the product: \qquad\qquad\, \$7.79 or \$8.49
	
\end{frame}

{\usebackgroundtemplate{\vbox to \paperheight{\vfil\hbox to \paperwidth{\hfil  
    \includegraphics[width=0.95\paperwidth, clip]
	{../1D/Slides/01Screen Shot 2015-08-06 at 22.44.08 .png}  \hfil}\vfil}}
\begin{frame}\frametitle{}
\end{frame}}

{\usebackgroundtemplate{\vbox to \paperheight{\vfil\hbox to \paperwidth{\hfil  
    \includegraphics[width=0.95\paperwidth, clip]
	{../1D/Slides/02Screen Shot 2015-08-06 at 22.43.45 .png}  \hfil}\vfil}}
\begin{frame}\frametitle{}
\end{frame}}



% /Users/kevindunn/Dropbox/Coursera/Media/All-course-slides/classes/CourseraMOOC-class-2A.tex

\begin{frame}\frametitle{Two-factor experiments: a recap from the last module}
    \includegraphics[width=\textwidth]{\imagedir/doe/two-types-of-variables.png}
\end{frame}

\begin{frame}\frametitle{Two-factor experiments: a recap from the last module}
    \includegraphics[width=\textwidth]{\imagedir/doe/two-types-of-variables-expanded.png}
\end{frame}

{\usebackgroundtemplate{\vbox to \paperheight{\vfil\hbox to \paperwidth{\hfil  
    \includegraphics[width=\paperwidth]
	{../2A/Supporting files/flickr-4137738962_064e1f4604_o.jpg}  \hfil}\vfil}}
\begin{frame}\frametitle{}
	\vspace{-5cm}
	\mode<handout>{
		\see{Flickr: ed\_welker \href{https://www.flickr.com/photos/ed_welker/4137738962/}{4137738962}}
	}
\end{frame}}

\begin{frame}\frametitle{A systematic approach}
	 \begin{enumerate}
	 	\item	What's my outcome? \pause
	 	\item	What's my objective? \pause
	 	\item	Which factors? \pause
		\item	At what levels?\pause
	 	\item	Plan the experiment \pause
	 	\item	Implement the experiment\pause
	 	\item	Analyze the results\pause
		\item	Repeat (if required)
	 \end{enumerate}
	 \vspace{5cm}
\end{frame}

{\usebackgroundtemplate{\vbox to \paperheight{\vfil\hbox to \paperwidth{\hfil  
    \includegraphics[width=\paperwidth]
	{../2A/Supporting files/flickr-13193465325_b35360c4b8_o.jpg}  \hfil}\vfil}}
\begin{frame}\frametitle{}
	
	{\color{white}
		\textbf{Outcome}
		
		\qquad Number of popped corn
		
		\vspace{18pt}
		
		\textbf{Objective}
		
		\qquad Maximizing number of popped corn
	

		\vspace{12pt}
		\onslide+<2->{
			This is equivalent to:

			``\emph{minimize the number of unpopped corn}''
		}

	}
	\vspace{4cm}
	\mode<handout>{
		\see{Flickr: booleansplit \href{https://www.flickr.com/photos/booleansplit/13193465325/}{13193465325}}
	}
\end{frame}}

{\usebackgroundtemplate{\vbox to \paperheight{\vfil\hbox to \paperwidth{\hfil  
    \includegraphics[width=0.95\paperwidth,trim=0 1.63cm 0 1.63cm, clip]
	{../2A/Slides/04Screen Shot 2014-07-15 at 10.14.11 .png}  \hfil}\vfil}}
\begin{frame}\frametitle{}
\end{frame}}

{\usebackgroundtemplate{\vbox to \paperheight{\vfil\hbox to \paperwidth{\hfil  
    \includegraphics[width=0.95\paperwidth,trim=0 1.63cm 0 1.63cm, clip]
	{../2A/Slides/05Screen Shot 2014-07-15 at 10.14.20 .png}  \hfil}\vfil}}
\begin{frame}\frametitle{}
\end{frame}}

{\usebackgroundtemplate{\vbox to \paperheight{\vfil\hbox to \paperwidth{\hfil  
    \includegraphics[width=0.95\paperwidth,trim=0 1.63cm 0 1.63cm, clip]
	{../2A/Slides/06Screen Shot 2014-07-15 at 10.14.48 .png}  \hfil}\vfil}}
\begin{frame}\frametitle{}
\end{frame}}

\begin{frame}\frametitle{}
	
     \includegraphics[width=0.5\textwidth]{\imagedir/doe/examples/advice-logo.png}
	 \pause
	 \vspace{2cm}
	 \begin{center}
		\color{purple}
		 \Large Always run experiments
		 in random order
	 \end{center}
     \vspace{2cm}
\end{frame}

\begin{frame}\frametitle{Various options for selecting the random order of experiments}
	Let's assume you need 8 random numbers:
	\begin{enumerate}
		\item	write numbers 1, 2, ... 8 on pieces of paper / cards
		\item	spreadsheets: use the following code (ignore any duplicates)
			\begin{varblock}[5cm]{}
				\texttt{=1 + INT( 8 * RAND() )}
			\end{varblock}

		\item	some spreadsheets have a special function
			\begin{varblock}[5cm]{}
				\texttt{=RANDBETWEEN(1, 8)}
				\begin{center}
					% FIGURE GOT DELETED
					% \includegraphics[width=0.4\textwidth]{\imagedir/doe/screenshots/randbetween.png}
				\end{center}
			\end{varblock}

		\item	In R (statistical software)
			\begin{varblock}[5cm]{}
				\texttt{sample(8)}
			\end{varblock}
	\end{enumerate}
\end{frame}

{\usebackgroundtemplate{\vbox to \paperheight{\vfil\hbox to \paperwidth{\hfil  
    \includegraphics[width=0.95\paperwidth,trim=0 1.63cm 0 1.63cm, clip]
	{../2A/Slides/08Screen Shot 2014-07-15 at 10.15.11.png}  \hfil}\vfil}}
\begin{frame}\frametitle{}
\end{frame}}

{\usebackgroundtemplate{\vbox to \paperheight{\vfil\hbox to \paperwidth{\hfil  
    \includegraphics[width=0.95\paperwidth,trim=0 1.63cm 0 1.63cm, clip]
	{../2A/Slides/09Screen Shot 2014-07-15 at 10.15.29 .png}  \hfil}\vfil}}
\begin{frame}\frametitle{}
\end{frame}}

{\usebackgroundtemplate{\vbox to \paperheight{\vfil\hbox to \paperwidth{\hfil  
    \includegraphics[width=0.95\paperwidth,trim=0 1.63cm 0 1.63cm, clip]
	{../2A/Slides/10Screen Shot 2014-07-15 at 10.15.35 .png}  \hfil}\vfil}}
\begin{frame}\frametitle{}
\end{frame}}

{\usebackgroundtemplate{\vbox to \paperheight{\vfil\hbox to \paperwidth{\hfil  
    \includegraphics[width=0.95\paperwidth,trim=0 1.63cm 0 1.63cm, clip]
	{../2A/Slides/11Screen Shot 2014-07-15 at 10.15.48 .png}  \hfil}\vfil}}
\begin{frame}\frametitle{}
\end{frame}}

{\usebackgroundtemplate{\vbox to \paperheight{\vfil\hbox to \paperwidth{\hfil  
    \includegraphics[width=0.95\paperwidth,trim=0 1.63cm 0 1.63cm, clip]
	{../2A/Slides/12Screen Shot 2014-07-15 at 10.16.00 .png}  \hfil}\vfil}}
\begin{frame}\frametitle{}
\end{frame}}

{\usebackgroundtemplate{\vbox to \paperheight{\vfil\hbox to \paperwidth{\hfil  
    \includegraphics[width=0.95\paperwidth,trim=0 1.63cm 0 1.63cm, clip]
	{../2A/Slides/13Screen Shot 2014-07-15 at 10.16.06 .png}  \hfil}\vfil}}
\begin{frame}\frametitle{}
\end{frame}}

{\usebackgroundtemplate{\vbox to \paperheight{\vfil\hbox to \paperwidth{\hfil  
    \includegraphics[width=0.95\paperwidth,trim=0 1.63cm 0 1.63cm, clip]
	{../2A/Slides/14Screen Shot 2014-07-15 at 10.16.20 .png}  \hfil}\vfil}}
\begin{frame}\frametitle{}
\end{frame}}

{\usebackgroundtemplate{\vbox to \paperheight{\vfil\hbox to \paperwidth{\hfil  
    \includegraphics[width=0.95\paperwidth,trim=0 1.63cm 0 1.63cm, clip]
	{../2A/Slides/15Screen Shot 2014-07-15 at 10.16.28 .png}  \hfil}\vfil}}
\begin{frame}\frametitle{}
\end{frame}}

{\usebackgroundtemplate{\vbox to \paperheight{\vfil\hbox to \paperwidth{\hfil  
    \includegraphics[width=0.95\paperwidth,trim=0 1.63cm 0 1.63cm, clip]
	{../2A/Slides/17Screen Shot 2014-07-15 at 10.16.42 .png}  \hfil}\vfil}}
\begin{frame}\frametitle{}
\end{frame}}

\begin{frame}\frametitle{Simple visualization of these factorial designs are powerful!}
	\begin{columns}[T]
		\column{0.5\textwidth}
			1. Tables show numeric trends
			
			\vspace{2pt}
		    \includegraphics[width=0.6\textwidth]{../2A/Slides/popcorn-example-table.png}
			
			
			2.	Cube plots indicate important factors
			
			\includegraphics[width=0.55\textwidth]{../2A/Slides/popcorn-example-cube.png}
			
		\column{0.01\textwidth}
			\rule[3mm]{0.01cm}{85mm}%
			
		\column{0.5\textwidth}
			3.	Contour plots show where to move next
			
			\includegraphics[width=0.55\textwidth]{../2A/Slides/popcorn-example-contour.png}
			
			4.	Interaction plots show synergies [next...]
			
			\includegraphics[width=0.55\textwidth]{../2A/Slides/popcorn-example-interaction.png}
			
	\end{columns}
	
	
\end{frame}



% /Users/kevindunn/Dropbox/Coursera/Media/All-course-slides/classes/CourseraMOOC-class-2B.tex

{\usebackgroundtemplate{\vbox to \paperheight{\vfil\hbox to \paperwidth{\hfil  
    \includegraphics[width=0.95\paperwidth,trim=0 1.65cm 0 1.65cm, clip]
	{../2B/Slides/01Screen Shot 2014-07-15 at 10.41.58 .png}  \hfil}\vfil}}
\begin{frame}\frametitle{}
\end{frame}}

{\usebackgroundtemplate{\vbox to \paperheight{\vfil\hbox to \paperwidth{\hfil  
    \includegraphics[width=0.95\paperwidth,trim=0 1.65cm 0 1.65cm, clip]
	{../2B/Slides/02Screen Shot 2014-07-15 at 10.47.34 .png}  \hfil}\vfil}}
\begin{frame}\frametitle{}
\end{frame}}

{\usebackgroundtemplate{\vbox to \paperheight{\vfil\hbox to \paperwidth{\hfil  
    \includegraphics[width=0.95\paperwidth,trim=0 1.65cm 0 1.65cm, clip]
	{../2B/Slides/03Screen Shot 2014-07-15 at 10.42.28 .png}  \hfil}\vfil}}
\begin{frame}\frametitle{}
\end{frame}}

{\usebackgroundtemplate{\vbox to \paperheight{\vfil\hbox to \paperwidth{\hfil  
    \includegraphics[width=0.95\paperwidth,trim=0 1.65cm 0 1.65cm, clip]
	{../2B/Slides/04Screen Shot 2014-07-15 at 10.42.39 .png}  \hfil}\vfil}}
\begin{frame}\frametitle{}
\end{frame}}

{\usebackgroundtemplate{\vbox to \paperheight{\vfil\hbox to \paperwidth{\hfil  
    \includegraphics[width=0.95\paperwidth,trim=0 1.65cm 0 1.65cm, clip]
	{../2B/Slides/05Screen Shot 2014-07-15 at 10.42.54 .png}  \hfil}\vfil}}
\begin{frame}\frametitle{}
\end{frame}}

{\usebackgroundtemplate{\vbox to \paperheight{\vfil\hbox to \paperwidth{\hfil  
    \includegraphics[width=0.95\paperwidth,trim=0 1.65cm 0 1.65cm, clip]
	{../2B/Slides/06Screen Shot 2014-07-15 at 10.43.10 .png}  \hfil}\vfil}}
\begin{frame}\frametitle{}
\end{frame}}

{\usebackgroundtemplate{\vbox to \paperheight{\vfil\hbox to \paperwidth{\hfil  
    \includegraphics[width=0.95\paperwidth,trim=0 1.65cm 0 1.65cm, clip]
	{../2B/Slides/07Screen Shot 2014-07-15 at 10.43.14 .png}  \hfil}\vfil}}
\begin{frame}\frametitle{}
\end{frame}}

{\usebackgroundtemplate{\vbox to \paperheight{\vfil\hbox to \paperwidth{\hfil  
    \includegraphics[width=0.95\paperwidth,trim=0 1.65cm 0 1.65cm, clip]
	{../2B/Slides/08Screen Shot 2014-07-15 at 10.43.17 .png}  \hfil}\vfil}}
\begin{frame}\frametitle{}
\end{frame}}

{\usebackgroundtemplate{\vbox to \paperheight{\vfil\hbox to \paperwidth{\hfil  
    \includegraphics[width=0.95\paperwidth,trim=0 1.65cm 0 1.65cm, clip]
	{../2B/Slides/09Screen Shot 2014-07-15 at 10.43.19 .png}  \hfil}\vfil}}
\begin{frame}\frametitle{}
\end{frame}}

{\usebackgroundtemplate{\vbox to \paperheight{\vfil\hbox to \paperwidth{\hfil  
    \includegraphics[width=0.95\paperwidth,trim=0 1.65cm 0 1.65cm, clip]
	{../2B/Slides/10Screen Shot 2014-07-15 at 10.43.24 .png}  \hfil}\vfil}}
\begin{frame}\frametitle{}
\end{frame}}

{\usebackgroundtemplate{\vbox to \paperheight{\vfil\hbox to \paperwidth{\hfil  
    \includegraphics[width=0.95\paperwidth,trim=0 1.65cm 0 1.65cm, clip]
	{../2B/Slides/11Screen Shot 2014-07-15 at 10.43.31 .png}  \hfil}\vfil}}
\begin{frame}\frametitle{}
\end{frame}}

{\usebackgroundtemplate{\vbox to \paperheight{\vfil\hbox to \paperwidth{\hfil  
    \includegraphics[width=0.95\paperwidth,trim=0 1.65cm 0 1.65cm, clip]
	{../2B/Slides/12Screen Shot 2014-07-15 at 10.43.34 .png}  \hfil}\vfil}}
\begin{frame}\frametitle{}
\end{frame}}

{\usebackgroundtemplate{\vbox to \paperheight{\vfil\hbox to \paperwidth{\hfil  
    \includegraphics[width=0.95\paperwidth,trim=0 1.65cm 0 1.65cm, clip]
	{../2B/Slides/13Screen Shot 2014-07-15 at 10.43.40 .png}  \hfil}\vfil}}
\begin{frame}\frametitle{}
\end{frame}}

{\usebackgroundtemplate{\vbox to \paperheight{\vfil\hbox to \paperwidth{\hfil  
    \includegraphics[width=0.95\paperwidth,trim=0 1.65cm 0 1.65cm, clip]
	{../2B/Slides/14Screen Shot 2014-07-15 at 10.43.50 .png}  \hfil}\vfil}}
\begin{frame}\frametitle{}
\end{frame}}

{\usebackgroundtemplate{\vbox to \paperheight{\vfil\hbox to \paperwidth{\hfil  
    \includegraphics[width=0.95\paperwidth,trim=0 1.65cm 0 1.65cm, clip]
	{../2B/Slides/15Screen Shot 2014-07-15 at 10.47.01 .png}  \hfil}\vfil}}
\begin{frame}\frametitle{}
\end{frame}}


% /Users/kevindunn/Dropbox/Coursera/Media/All-course-slides/classes/CourseraMOOC-class-2C.tex

{\usebackgroundtemplate{\vbox to \paperheight{\vfil\hbox to \paperwidth{\hfil  
    \includegraphics[width=0.95\paperwidth,trim=0 1.69cm 0 1.69cm, clip]
	{../2C/Slides/01Screen Shot 2014-07-15 at 11.35.02 .png}  \hfil}\vfil}}
\begin{frame}\frametitle{}
\end{frame}}

\begin{frame}\frametitle{Understanding interactions in the water-soap example}
	\begin{center}
		\includegraphics[width=\textwidth]{\imagedir/doe/examples/hand-washing-flickr-7008312299_2fcf07309c_k.jpg}
	\end{center}
	\vspace{-4cm}
	\see{\href{https://secure.flickr.com/photos/usdagov/7008312299/}{Flickr}}
\end{frame}

\begin{frame}\frametitle{Understanding interactions in the water-soap example}
	\begin{itemize}
		\item	\textbf{Using soap} works better with warm water (instead of cold water)

			\makebox[14cm][c]{We say: ``the effect of warm water enhances the effect of soap''}
		
		\vspace{24pt}
		\pause
		\item	\textbf{Warm water} works better with soap (instead of no soap)
			\makebox[14cm][c]{We say: ``the effect of soap is enhanced by using warm water''}
			
	\end{itemize}
	\pause
	\vspace{24pt}
	This interaction works in our favour.
\end{frame}

\begin{frame}\frametitle{The definition of ``interaction''}
	\begin{exampleblock}{}
		The effect of one factor (for example, \textbf{A}) depends on the value, or the level, of another factor (\textbf{B}, for example).
	\end{exampleblock}
\end{frame}

\begin{frame}\frametitle{Interactions are symmetrical (\textbf{AB} = \textbf{BA})}
	These two alternatives provide the same effect: \vspace{24pt}
	\begin{itemize}
		\item	Using \textbf{(soap)} together with \textbf{(warm water)} 
		\vspace{12pt}
		\item	Using \textbf{(warm water)} together with \textbf{(soap)}  
	\end{itemize}
\end{frame}

\begin{frame}\frametitle{Interactions don't always benefit us}
	
	Interactions can cancel out the effect of an improvement. 
	
	\vspace{24pt}
	{\color{myOrange} 	\emph{Interactions are important and do exist in real systems!}}
\end{frame}

\begin{frame}\frametitle{Ginger biscuits outcome variables}
	Three outcomes were measured:
	\begin{enumerate}
		\item	Taste		
		\item	Break strength (breakability)
		\item	Breakability after 1 week (measures ``freshness'' of biscuit)
	\end{enumerate}
\end{frame}

\begin{frame}\frametitle{\includegraphics[width=0.3\textwidth]{\imagedir/doe/examples/advice-logo.png}}
	\begin{exampleblock}{Always measure as many things as you can during your experiments}
		\begin{itemize}
			\item	Even things you believe are not too important now
			\item	We will learn more about disturbances later, these can affect your experiments.
		\end{itemize}
	\end{exampleblock}
	\vspace{24pt}\pause
	{\color{myOrange} 	\emph{Experiments are expensive to repeat!}}
\end{frame}

\begin{frame}\frametitle{Back to the ginger biscuits: taste!}
	\begin{itemize}
		\item	Try to choose outcome factors that are \emph{not} subjective.
		\item	But sometimes we have no choice.
	\end{itemize}
\end{frame}

\begin{frame}\frametitle{A recipe for ginger biscuits}
	\begin{columns}[t]
		\column{0.70\textwidth}
			\begin{itemize}
				\item	150 grams margarine
				\item	195 grams packed brown sugar
				\item	1 egg
				\item		\textbf{B = 55 grams molasses or 55 grams of honey}
				\item	250 grams all-purpose flour
				\item	3 grams ground ginger
				\item	8 grams baking soda
				\item	3 grams salt
				\item	7 grams minced ginger
			\end{itemize}
		\column{0.30\textwidth}
		
			\includegraphics[width=.95\textwidth]{\imagedir/doe/examples/ginger-root-flickr-177639720_204e947bfd_o.jpg}
			\see{\href{https://secure.flickr.com/photos/notafish/177639720}{Flickr}}
			
			\includegraphics[width=.95\textwidth]{\imagedir/doe/examples/diced-ginger-root-flickr-4447387517_602233756d_o.jpg}
			\see{\href{https://secure.flickr.com/photos/75001512@N00/4447387517}{Flickr}}
			\\
			
			\emph{minced ginger}
	\end{columns}
		
	Baking time: 10 minutes at 350°C.
	
	\vspace{12pt}
	\textbf{A = bake for 8 minutes or 14 minutes}
\end{frame}

\begin{frame}\frametitle{Factors considered}	
	\begin{columns}[b]
		\column{0.40\textwidth}
			\begin{itemize}
				\item	\textbf{A: baking duration}
					\begin{itemize}
						\item	8 minutes
						\item	14 minutes
					\end{itemize}
					
				\vspace{24pt}
				\item	\textbf{B: sugar type}
				\begin{itemize}
					\item	honey
					\item	molasses
				\end{itemize}
			\end{itemize}
		\column{0.60\textwidth}
			\includegraphics[width=.95\textwidth]{\imagedir/doe/examples/KGD-Molasses-honey.jpg}
	\end{columns}
\end{frame}

{\usebackgroundtemplate{\vbox to \paperheight{\vfil\hbox to \paperwidth{\hfil  
    \includegraphics[width=0.95\paperwidth,trim=0 1.69cm 0 1.69cm, clip]
	{../2C/Slides/13Screen Shot 2014-07-15 at 11.35.53 .png}  \hfil}\vfil}}
\begin{frame}\frametitle{}
\end{frame}}

{\usebackgroundtemplate{\vbox to \paperheight{\vfil\hbox to \paperwidth{\hfil  
    \includegraphics[width=0.95\paperwidth,trim=0 1.69cm 0 1.69cm, clip]
	{../2C/Slides/14Screen Shot 2014-07-15 at 11.35.59 .png}  \hfil}\vfil}}
\begin{frame}\frametitle{}
\end{frame}}

{\usebackgroundtemplate{\vbox to \paperheight{\vfil\hbox to \paperwidth{\hfil  
    \includegraphics[width=0.95\paperwidth,trim=0 1.69cm 0 1.69cm, clip]
	{../2C/Slides/15Screen Shot 2014-07-15 at 11.36.03 .png}  \hfil}\vfil}}
\begin{frame}\frametitle{}
\end{frame}}

{\usebackgroundtemplate{\vbox to \paperheight{\vfil\hbox to \paperwidth{\hfil  
    \includegraphics[width=0.95\paperwidth,trim=0 1.69cm 0 1.69cm, clip]
	{../2C/Slides/16Screen Shot 2014-07-15 at 11.36.06 .png}  \hfil}\vfil}}
\begin{frame}\frametitle{}
\end{frame}}

{\usebackgroundtemplate{\vbox to \paperheight{\vfil\hbox to \paperwidth{\hfil  
    \includegraphics[width=0.95\paperwidth,trim=0 1.69cm 0 1.69cm, clip]
	{../2C/Slides/17Screen Shot 2014-07-15 at 11.41.59 .png}  \hfil}\vfil}}
\begin{frame}\frametitle{}
\end{frame}}

{\usebackgroundtemplate{\vbox to \paperheight{\vfil\hbox to \paperwidth{\hfil  
    \includegraphics[width=0.95\paperwidth,trim=0 1.69cm 0 1.69cm, clip]
	{../2C/Slides/18Screen Shot 2014-07-15 at 11.36.21 .png}  \hfil}\vfil}}
\begin{frame}\frametitle{}
\end{frame}}

{\usebackgroundtemplate{\vbox to \paperheight{\vfil\hbox to \paperwidth{\hfil  
    \includegraphics[width=0.95\paperwidth,trim=0 1.69cm 0 1.69cm, clip]
	{../2C/Slides/19Screen Shot 2014-07-15 at 11.36.32 .png}  \hfil}\vfil}}
\begin{frame}\frametitle{}
\end{frame}}

{\usebackgroundtemplate{\vbox to \paperheight{\vfil\hbox to \paperwidth{\hfil  
    \includegraphics[width=0.95\paperwidth,trim=0 1.69cm 0 1.69cm, clip]
	{../2C/Slides/20Screen Shot 2014-07-15 at 11.37.51 .png}  \hfil}\vfil}}
\begin{frame}\frametitle{}
\end{frame}}

{\usebackgroundtemplate{\vbox to \paperheight{\vfil\hbox to \paperwidth{\hfil  
    \includegraphics[width=0.95\paperwidth,trim=0 1.69cm 0 1.69cm, clip]
	{../2C/Slides/21Screen Shot 2014-07-15 at 11.37.54 .png}  \hfil}\vfil}}
\begin{frame}\frametitle{}
\end{frame}}

{\usebackgroundtemplate{\vbox to \paperheight{\vfil\hbox to \paperwidth{\hfil  
    \includegraphics[width=0.95\paperwidth,trim=0 1.69cm 0 1.69cm, clip]
	{../2C/Slides/22Screen Shot 2014-07-15 at 11.38.37 .png}  \hfil}\vfil}}
\begin{frame}\frametitle{}
\end{frame}}

{\usebackgroundtemplate{\vbox to \paperheight{\vfil\hbox to \paperwidth{\hfil  
    \includegraphics[width=0.95\paperwidth,trim=0 1.69cm 0 1.69cm, clip]
	{../2C/Slides/23Screen Shot 2014-07-15 at 11.38.42 .png}  \hfil}\vfil}}
\begin{frame}\frametitle{}
\end{frame}}

{\usebackgroundtemplate{\vbox to \paperheight{\vfil\hbox to \paperwidth{\hfil  
    \includegraphics[width=0.95\paperwidth,trim=0 1.69cm 0 1.69cm, clip]
	{../2C/Slides/24Screen Shot 2014-07-15 at 11.38.52 .png}  \hfil}\vfil}}
\begin{frame}\frametitle{}
\end{frame}}

{\usebackgroundtemplate{\vbox to \paperheight{\vfil\hbox to \paperwidth{\hfil  
    \includegraphics[width=0.95\paperwidth,trim=0 1.69cm 0 1.69cm, clip]
	{../2C/Slides/25Screen Shot 2014-07-15 at 11.39.01 .png}  \hfil}\vfil}}
\begin{frame}\frametitle{}
\end{frame}}

{\usebackgroundtemplate{\vbox to \paperheight{\vfil\hbox to \paperwidth{\hfil  
    \includegraphics[width=0.95\paperwidth,trim=0 1.69cm 0 1.69cm, clip]
	{../2C/Slides/26Screen Shot 2014-07-15 at 11.39.07 .png}  \hfil}\vfil}}
\begin{frame}\frametitle{}
\end{frame}}

{\usebackgroundtemplate{\vbox to \paperheight{\vfil\hbox to \paperwidth{\hfil  
    \includegraphics[width=0.95\paperwidth,trim=0 1.69cm 0 1.69cm, clip]
	{../2C/Slides/27Screen Shot 2014-07-15 at 11.39.10 .png}  \hfil}\vfil}}
\begin{frame}\frametitle{}
\end{frame}}

{\usebackgroundtemplate{\vbox to \paperheight{\vfil\hbox to \paperwidth{\hfil  
    \includegraphics[width=0.95\paperwidth,trim=0 1.69cm 0 1.69cm, clip]
	{../2C/Slides/28Screen Shot 2014-07-15 at 11.39.15 .png}  \hfil}\vfil}}
\begin{frame}\frametitle{}
\end{frame}}

{\usebackgroundtemplate{\vbox to \paperheight{\vfil\hbox to \paperwidth{\hfil  
    \includegraphics[width=0.95\paperwidth,trim=0 1.69cm 0 1.69cm, clip]
	{../2C/Slides/29Screen Shot 2014-07-15 at 11.39.21 .png}  \hfil}\vfil}}
\begin{frame}\frametitle{}
\end{frame}}

{\usebackgroundtemplate{\vbox to \paperheight{\vfil\hbox to \paperwidth{\hfil  
    \includegraphics[width=0.95\paperwidth,trim=0 1.69cm 0 1.69cm, clip]
	{../2C/Slides/30Screen Shot 2014-07-15 at 11.39.30 .png}  \hfil}\vfil}}
\begin{frame}\frametitle{}
\end{frame}}

{\usebackgroundtemplate{\vbox to \paperheight{\vfil\hbox to \paperwidth{\hfil  
    \includegraphics[width=0.95\paperwidth,trim=0 1.69cm 0 1.69cm, clip]
	{../2C/Slides/31Screen Shot 2014-07-15 at 11.39.32 .png}  \hfil}\vfil}}
\begin{frame}\frametitle{}
\end{frame}}

{\usebackgroundtemplate{\vbox to \paperheight{\vfil\hbox to \paperwidth{\hfil  
    \includegraphics[width=0.95\paperwidth,trim=0 1.69cm 0 1.69cm, clip]
	{../2C/Slides/32Screen Shot 2014-07-15 at 11.39.34 .png}  \hfil}\vfil}}
\begin{frame}\frametitle{}
\end{frame}}

{\usebackgroundtemplate{\vbox to \paperheight{\vfil\hbox to \paperwidth{\hfil  
    \includegraphics[width=0.95\paperwidth,trim=0 1.69cm 0 1.69cm, clip]
	{../2C/Slides/33Screen Shot 2014-07-15 at 11.39.35 .png}  \hfil}\vfil}}
\begin{frame}\frametitle{}
\end{frame}}

{\usebackgroundtemplate{\vbox to \paperheight{\vfil\hbox to \paperwidth{\hfil  
    \includegraphics[width=0.95\paperwidth,trim=0 1.69cm 0 1.69cm, clip]
	{../2C/Slides/34Screen Shot 2014-07-15 at 11.39.38 .png}  \hfil}\vfil}}
\begin{frame}\frametitle{}
\end{frame}}

{\usebackgroundtemplate{\vbox to \paperheight{\vfil\hbox to \paperwidth{\hfil  
    \includegraphics[width=0.95\paperwidth,trim=0 1.69cm 0 1.69cm, clip]
	{../2C/Slides/35Screen Shot 2014-07-15 at 11.39.50 .png}  \hfil}\vfil}}
\begin{frame}\frametitle{}
\end{frame}}

\begin{frame}\frametitle{\includegraphics[width=0.3\textwidth]{\imagedir/doe/examples/advice-logo.png}}
	\begin{exampleblock}{Always interpret your results}
		\vspace{12pt}
		Critically think about your results by asking:
		\begin{itemize}
			\item	what did I learn?
			\item	what next experiments should I be doing?
		\end{itemize} 
	\end{exampleblock}
	\vspace{24pt}\pause
	{\color{myOrange} 	\emph{Experiments are expensive to repeat!}}
\end{frame}

{\usebackgroundtemplate{\vbox to \paperheight{\vfil\hbox to \paperwidth{\hfil  
    \includegraphics[width=0.95\paperwidth,trim=0 1.69cm 0 1.69cm, clip]
	{../2C/Slides/37Screen Shot 2014-07-15 at 11.40.10 .png}  \hfil}\vfil}}
\begin{frame}\frametitle{}
\end{frame}}

{\usebackgroundtemplate{\vbox to \paperheight{\vfil\hbox to \paperwidth{\hfil  
    \includegraphics[width=0.95\paperwidth,trim=0 1.69cm 0 1.69cm, clip]
	{../2C/Slides/38Screen Shot 2014-07-15 at 11.40.16 .jpg}  \hfil}\vfil}}
\begin{frame}\frametitle{}
\end{frame}}

{\usebackgroundtemplate{\vbox to \paperheight{\vfil\hbox to \paperwidth{\hfil  
    \includegraphics[width=0.95\paperwidth,trim=0 1.69cm 0 1.69cm, clip]
	{../2C/Slides/39Screen Shot 2014-07-15 at 11.40.21 .jpg}  \hfil}\vfil}}
\begin{frame}\frametitle{}
\end{frame}}










% /Users/kevindunn/Dropbox/Coursera/Media/All-course-slides/classes/CourseraMOOC-class-2D.tex

%\begin{frame}\frametitle{Waste water treatment example: analysis of the data by hand}
%	\begin{tabular}{|l|l|>{\centering\arraybackslash}m{1cm}|>{\centering\arraybackslash}m{1cm}|>{\centering\arraybackslash}m{1cm}|>{\centering\arraybackslash}m{3cm}|}\hline
%	
%		Standard order & Actual order & $C$ & $T$ & $S$ & Outcome [lbs] \\ \hline
%		&&&&&\\&&&&&\\&&&&&\\&&&&&\\&&&&&\\&&&&&\\&&&&&\\&&&&&\\&&&&&\\&&&&&\\&&&&&\\&&&&&\\\hline
%	\end{tabular}
%\end{frame}

{\usebackgroundtemplate{\vbox to \paperheight{\vfil\hbox to \paperwidth{\hfil  
    \includegraphics[width=0.95\paperwidth,trim=0 1.69cm 0 1.69cm, clip]
	{../2D/Slides/01Screen Shot 2014-07-15 at 12.28.28 .png}  \hfil}\vfil}}
\begin{frame}\frametitle{}
\end{frame}}

{\usebackgroundtemplate{\vbox to \paperheight{\vfil\hbox to \paperwidth{\hfil  
    \includegraphics[width=0.95\paperwidth,trim=0 1.69cm 0 1.69cm, clip]
	{../2D/Slides/02Screen Shot 2014-07-15 at 12.29.26 .png}  \hfil}\vfil}}
\begin{frame}\frametitle{}
\end{frame}}

{\usebackgroundtemplate{\vbox to \paperheight{\vfil\hbox to \paperwidth{\hfil  
    \includegraphics[width=0.95\paperwidth,trim=0 1.69cm 0 1.69cm, clip]
	{../2D/Slides/03Screen Shot 2014-07-15 at 12.29.29 .png}  \hfil}\vfil}}
\begin{frame}\frametitle{}
\end{frame}}

{\usebackgroundtemplate{\vbox to \paperheight{\vfil\hbox to \paperwidth{\hfil  
    \includegraphics[width=0.95\paperwidth,trim=0 1.69cm 0 1.69cm, clip]
	{../2D/Slides/04Screen Shot 2014-07-15 at 12.29.45 .png}  \hfil}\vfil}}
\begin{frame}\frametitle{}
\end{frame}}

{\usebackgroundtemplate{\vbox to \paperheight{\vfil\hbox to \paperwidth{\hfil  
    \includegraphics[width=0.95\paperwidth,trim=0 1.69cm 0 1.69cm, clip]
	{../2D/Slides/05Screen Shot 2014-07-15 at 12.30.02 .png}  \hfil}\vfil}}
\begin{frame}\frametitle{}
\end{frame}}

{\usebackgroundtemplate{\vbox to \paperheight{\vfil\hbox to \paperwidth{\hfil  
    \includegraphics[width=0.95\paperwidth,trim=0 1.69cm 0 1.69cm, clip]
	{../2D/Slides/06Screen Shot 2014-07-15 at 12.30.12 .png}  \hfil}\vfil}}
\begin{frame}\frametitle{}
\end{frame}}

{\usebackgroundtemplate{\vbox to \paperheight{\vfil\hbox to \paperwidth{\hfil  
    \includegraphics[width=0.95\paperwidth,trim=0 1.69cm 0 1.69cm, clip]
	{../2D/Slides/07Screen Shot 2014-07-15 at 12.30.52 .png}  \hfil}\vfil}}
\begin{frame}\frametitle{}
\end{frame}}

{\usebackgroundtemplate{\vbox to \paperheight{\vfil\hbox to \paperwidth{\hfil  
    \includegraphics[width=0.95\paperwidth,trim=0 1.69cm 0 1.69cm, clip]
	{../2D/Slides/08Screen Shot 2014-07-15 at 12.30.59 .png}  \hfil}\vfil}}
\begin{frame}\frametitle{}
\end{frame}}

{\usebackgroundtemplate{\vbox to \paperheight{\vfil\hbox to \paperwidth{\hfil  
    \includegraphics[width=0.95\paperwidth,trim=0 1.69cm 0 1.69cm, clip]
	{../2D/Slides/09Screen Shot 2014-07-15 at 12.31.20 .png}  \hfil}\vfil}}
\begin{frame}\frametitle{}
\end{frame}}

{\usebackgroundtemplate{\vbox to \paperheight{\vfil\hbox to \paperwidth{\hfil  
    \includegraphics[width=0.95\paperwidth,trim=0 1.69cm 0 1.69cm, clip]
	{../2D/Slides/10Screen Shot 2014-07-15 at 12.31.32 .png}  \hfil}\vfil}}
\begin{frame}\frametitle{}
\end{frame}}

{\usebackgroundtemplate{\vbox to \paperheight{\vfil\hbox to \paperwidth{\hfil  
    \includegraphics[width=0.95\paperwidth,trim=0 1.69cm 0 1.69cm, clip]
	{../2D/Slides/11Screen Shot 2014-07-15 at 12.32.27 .png}  \hfil}\vfil}}
\begin{frame}\frametitle{}
\end{frame}}

{\usebackgroundtemplate{\vbox to \paperheight{\vfil\hbox to \paperwidth{\hfil  
    \includegraphics[width=0.95\paperwidth,trim=0 1.69cm 0 1.69cm, clip]
	{../2D/Slides/12Screen Shot 2014-07-15 at 12.32.52 .png}  \hfil}\vfil}}
\begin{frame}\frametitle{}
\end{frame}}

{\usebackgroundtemplate{\vbox to \paperheight{\vfil\hbox to \paperwidth{\hfil  
    \includegraphics[width=0.95\paperwidth,trim=0 1.69cm 0 1.69cm, clip]
	{../2D/Slides/13Screen Shot 2014-07-15 at 12.33.09 .png}  \hfil}\vfil}}
\begin{frame}\frametitle{}
\end{frame}}

{\usebackgroundtemplate{\vbox to \paperheight{\vfil\hbox to \paperwidth{\hfil  
    \includegraphics[width=0.95\paperwidth,trim=0 1.69cm 0 1.69cm, clip]
	{../2D/Slides/14Screen Shot 2014-07-15 at 12.33.28 .png}  \hfil}\vfil}}
\begin{frame}\frametitle{}
\end{frame}}

{\usebackgroundtemplate{\vbox to \paperheight{\vfil\hbox to \paperwidth{\hfil  
    \includegraphics[width=0.95\paperwidth,trim=0 1.69cm 0 1.69cm, clip]
	{../2D/Slides/15Screen Shot 2014-07-15 at 12.33.55 .png}  \hfil}\vfil}}
\begin{frame}\frametitle{}
\end{frame}}

{\usebackgroundtemplate{\vbox to \paperheight{\vfil\hbox to \paperwidth{\hfil  
    \includegraphics[width=0.95\paperwidth,trim=0 1.69cm 0 1.69cm, clip]
	{../2D/Slides/16Screen Shot 2014-07-15 at 12.34.19 .png}  \hfil}\vfil}}
\begin{frame}\frametitle{}
\end{frame}}

{\usebackgroundtemplate{\vbox to \paperheight{\vfil\hbox to \paperwidth{\hfil  
    \includegraphics[width=0.95\paperwidth,trim=0 1.69cm 0 1.69cm, clip]
	{../2D/Slides/17Screen Shot 2014-07-15 at 12.34.21 .png}  \hfil}\vfil}}
\begin{frame}\frametitle{}
\end{frame}}

{\usebackgroundtemplate{\vbox to \paperheight{\vfil\hbox to \paperwidth{\hfil  
    \includegraphics[width=0.95\paperwidth,trim=0 1.69cm 0 1.69cm, clip]
	{../2D/Slides/18Screen Shot 2014-07-15 at 12.35.04 .png}  \hfil}\vfil}}
\begin{frame}\frametitle{}
\end{frame}}

{\usebackgroundtemplate{\vbox to \paperheight{\vfil\hbox to \paperwidth{\hfil  
    \includegraphics[width=0.95\paperwidth,trim=0 1.69cm 0 1.69cm, clip]
	{../2D/Slides/19Screen Shot 2014-07-15 at 12.35.10 .jpg}  \hfil}\vfil}}
\begin{frame}\frametitle{}
\end{frame}}

{\usebackgroundtemplate{\vbox to \paperheight{\vfil\hbox to \paperwidth{\hfil  
    \includegraphics[width=0.95\paperwidth,trim=0 1.69cm 0 1.69cm, clip]
	{../2D/Slides/20Screen Shot 2014-07-15 at 12.35.18 .png}  \hfil}\vfil}}
\begin{frame}\frametitle{}
\end{frame}}

{\usebackgroundtemplate{\vbox to \paperheight{\vfil\hbox to \paperwidth{\hfil  
    \includegraphics[width=0.95\paperwidth,trim=0 1.69cm 0 1.69cm, clip]
	{../2D/Slides/21Screen Shot 2014-07-15 at 12.35.41 .png}  \hfil}\vfil}}
\begin{frame}\frametitle{}
\end{frame}}

{\usebackgroundtemplate{\vbox to \paperheight{\vfil\hbox to \paperwidth{\hfil  
    \includegraphics[width=0.95\paperwidth,trim=0 1.69cm 0 1.69cm, clip]
	{../2D/Slides/22Screen Shot 2014-07-15 at 12.35.52 .png}  \hfil}\vfil}}
\begin{frame}\frametitle{}
\end{frame}}

{\usebackgroundtemplate{\vbox to \paperheight{\vfil\hbox to \paperwidth{\hfil  
    \includegraphics[width=0.95\paperwidth,trim=0 1.69cm 0 1.69cm, clip]
	{../2D/Slides/23Screen Shot 2014-07-15 at 12.36.03 .png}  \hfil}\vfil}}
\begin{frame}\frametitle{}
\end{frame}}

{\usebackgroundtemplate{\vbox to \paperheight{\vfil\hbox to \paperwidth{\hfil  
    \includegraphics[width=0.95\paperwidth,trim=0 1.69cm 0 1.69cm, clip]
	{../2D/Slides/24Screen Shot 2014-07-15 at 12.36.08 .png}  \hfil}\vfil}}
\begin{frame}\frametitle{}
\end{frame}}

{\usebackgroundtemplate{\vbox to \paperheight{\vfil\hbox to \paperwidth{\hfil  
    \includegraphics[width=0.95\paperwidth,trim=0 1.69cm 0 1.69cm, clip]
	{../2D/Slides/25Screen Shot 2014-07-15 at 12.36.34 .png}  \hfil}\vfil}}
\begin{frame}\frametitle{}
\end{frame}}

{\usebackgroundtemplate{\vbox to \paperheight{\vfil\hbox to \paperwidth{\hfil  
    \includegraphics[width=0.95\paperwidth,trim=0 1.69cm 0 1.69cm, clip]
	{../2D/Slides/26Screen Shot 2014-07-15 at 12.36.47 .png}  \hfil}\vfil}}
\begin{frame}\frametitle{}
\end{frame}}

{\usebackgroundtemplate{\vbox to \paperheight{\vfil\hbox to \paperwidth{\hfil  
    \includegraphics[width=0.95\paperwidth,trim=0 1.69cm 0 1.69cm, clip]
	{../2D/Slides/27Screen Shot 2014-07-15 at 12.37.00 .png}  \hfil}\vfil}}
\begin{frame}\frametitle{}
\end{frame}}

{\usebackgroundtemplate{\vbox to \paperheight{\vfil\hbox to \paperwidth{\hfil  
    \includegraphics[width=0.95\paperwidth,trim=0 1.69cm 0 1.69cm, clip]
	{../2D/Slides/28Screen Shot 2014-07-15 at 12.37.12 .png}  \hfil}\vfil}}
\begin{frame}\frametitle{}
\end{frame}}

{\usebackgroundtemplate{\vbox to \paperheight{\vfil\hbox to \paperwidth{\hfil  
    \includegraphics[width=0.95\paperwidth,trim=0 1.69cm 0 1.69cm, clip]
	{../2D/Slides/29Screen Shot 2014-07-15 at 12.37.36 .jpg}  \hfil}\vfil}}
\begin{frame}\frametitle{}
\end{frame}}

% /Users/kevindunn/Dropbox/Coursera/Media/All-course-slides/classes/CourseraMOOC-class-3A.tex

{\usebackgroundtemplate{\vbox to \paperheight{\vfil\hbox to \paperwidth{\hfil  
    \includegraphics[width=\paperwidth]
	{../2A/Supporting files/flickr-4137738962_064e1f4604_o.jpg}  \hfil}\vfil}}
\begin{frame}\frametitle{}
	\vspace{-5cm}
	%\mode<handout>{
		\see{Flickr: ed\_welker \href{https://www.flickr.com/photos/ed_welker/4137738962/}{4137738962}}
	%}
\end{frame}}

{\usebackgroundtemplate{\vbox to \paperheight{\vfil\hbox to \paperwidth{\hfil  
    \includegraphics[width=\paperwidth]
	{../2A/Supporting files/flickr-13193465325_b35360c4b8_o.jpg}  \hfil}\vfil}}
\begin{frame}\frametitle{}
	
	{\color{white}
		\textbf{Outcome}
		
		\qquad Number of popped corn
		
		\vspace{18pt}
		
		\textbf{Objective}
		
		\qquad Maximizing number of popped corn
	

		\vspace{12pt}
		\onslide+<2->{
			This is equivalent to:

			``\emph{minimize the number of unpopped corn}''
		}

	}
	\vspace{4cm}
	%\mode<handout>{
		\see{Flickr: booleansplit \href{https://www.flickr.com/photos/booleansplit/13193465325/}{13193465325}}
	%}
\end{frame}}

{\usebackgroundtemplate{\vbox to \paperheight{\vfil\hbox to \paperwidth{\hfil  
    \includegraphics[width=0.95\paperwidth,trim=0 1.69cm 0 1.69cm, clip]
	{../3A/Slides/02Screen Shot 2014-07-16 at 07.12.39 .png}  \hfil}\vfil}}
\begin{frame}\frametitle{}
\end{frame}}

{\usebackgroundtemplate{\vbox to \paperheight{\vfil\hbox to \paperwidth{\hfil  
    \includegraphics[width=0.95\paperwidth,trim=0 1.69cm 0 1.69cm, clip]
	{../3A/Slides/03Screen Shot 2014-07-16 at 07.12.50 .png}  \hfil}\vfil}}
\begin{frame}\frametitle{}
\end{frame}}

{\usebackgroundtemplate{\vbox to \paperheight{\vfil\hbox to \paperwidth{\hfil  
    \includegraphics[width=0.95\paperwidth,trim=0 1.69cm 0 1.69cm, clip]
	{../3A/Slides/04Screen Shot 2014-07-16 at 07.12.52 .png}  \hfil}\vfil}}
\begin{frame}\frametitle{}
\end{frame}}

{\usebackgroundtemplate{\vbox to \paperheight{\vfil\hbox to \paperwidth{\hfil  
    \includegraphics[width=0.95\paperwidth,trim=0 1.69cm 0 1.69cm, clip]
	{../3A/Slides/05Screen Shot 2014-07-16 at 07.12.54 .png}  \hfil}\vfil}}
\begin{frame}\frametitle{}
\end{frame}}

{\usebackgroundtemplate{\vbox to \paperheight{\vfil\hbox to \paperwidth{\hfil  
    \includegraphics[width=0.95\paperwidth,trim=0 1.69cm 0 1.69cm, clip]
	{../3A/Slides/06Screen Shot 2014-07-16 at 07.12.58 .png}  \hfil}\vfil}}
\begin{frame}\frametitle{}
\end{frame}}

{\usebackgroundtemplate{\vbox to \paperheight{\vfil\hbox to \paperwidth{\hfil  
    \includegraphics[width=0.95\paperwidth,trim=0 1.69cm 0 1.69cm, clip]
	{../3A/Slides/07Screen Shot 2014-07-16 at 07.14.28 .png}  \hfil}\vfil}}
\begin{frame}\frametitle{}
\end{frame}}

{\usebackgroundtemplate{\vbox to \paperheight{\vfil\hbox to \paperwidth{\hfil  
    \includegraphics[width=0.95\paperwidth,trim=0 1.69cm 0 1.69cm, clip]
	{../3A/Slides/08Screen Shot 2014-07-16 at 07.14.54 .png}  \hfil}\vfil}}
\begin{frame}\frametitle{}
\end{frame}}

{\usebackgroundtemplate{\vbox to \paperheight{\vfil\hbox to \paperwidth{\hfil  
    \includegraphics[width=0.95\paperwidth,trim=0 1.69cm 0 1.69cm, clip]
	{../3A/Slides/09Screen Shot 2014-07-16 at 07.14.58 .png}  \hfil}\vfil}}
\begin{frame}\frametitle{}
\end{frame}}

{\usebackgroundtemplate{\vbox to \paperheight{\vfil\hbox to \paperwidth{\hfil  
    \includegraphics[width=0.95\paperwidth,trim=0 1.69cm 0 1.69cm, clip]
	{../3A/Slides/10Screen Shot 2014-07-16 at 07.15.02 .png}  \hfil}\vfil}}
\begin{frame}\frametitle{}
\end{frame}}

{\usebackgroundtemplate{\vbox to \paperheight{\vfil\hbox to \paperwidth{\hfil  
    \includegraphics[width=0.95\paperwidth,trim=0 1.69cm 0 1.69cm, clip]
	{../3A/Slides/11Screen Shot 2014-07-16 at 07.15.18 .png}  \hfil}\vfil}}
\begin{frame}\frametitle{}
\end{frame}}

{\usebackgroundtemplate{\vbox to \paperheight{\vfil\hbox to \paperwidth{\hfil  
    \includegraphics[width=0.95\paperwidth,trim=0 1.69cm 0 1.69cm, clip]
	{../3A/Slides/12Screen Shot 2014-07-16 at 07.15.24 .png}  \hfil}\vfil}}
\begin{frame}\frametitle{}
\end{frame}}

{\usebackgroundtemplate{\vbox to \paperheight{\vfil\hbox to \paperwidth{\hfil  
    \includegraphics[width=0.95\paperwidth,trim=0 1.69cm 0 1.69cm, clip]
	{../3A/Slides/13Screen Shot 2014-07-16 at 07.15.42 .png}  \hfil}\vfil}}
\begin{frame}\frametitle{}
\end{frame}}

{\usebackgroundtemplate{\vbox to \paperheight{\vfil\hbox to \paperwidth{\hfil  
    \includegraphics[width=0.95\paperwidth,trim=0 1.69cm 0 1.69cm, clip]
	{../3A/Slides/14Screen Shot 2014-07-16 at 07.15.54 .png}  \hfil}\vfil}}
\begin{frame}\frametitle{}
\end{frame}}

% /Users/kevindunn/Dropbox/Coursera/Media/All-course-slides/classes/CourseraMOOC-class-3B.tex

\begin{frame}\frametitle{Software installation required for this course}
	\begin{columns}[b]
		\column{0.5\textwidth}
			
			\begin{exampleblock}{\color{red}Step one}
				
				\vspace{10pt}
				\centerline{\includegraphics[width=\textwidth]{\imagedir/statistics/R-project-website-16-July-2015.png}}
			\end{exampleblock}
			
			\vfill
			
			\href{http://yint.org/R}{http://yint.org/R}
		
		\column{0.50\textwidth}
		
			\begin{exampleblock}{\color{red}Step two}
				\centerline{\includegraphics[width=\textwidth]{\imagedir/statistics/RStudio-website-16-July-2015.png}}
			\end{exampleblock}
			
			\vfill
			\href{http://yint.org/RStudio}{http://yint.org/RStudio}
	\end{columns}
	
	\vspace{6pt}
	\small
	The above links will redirect to the most up-to-date website. Alternatively, please search for\\ ``\texttt{R project}'' and ``\texttt{RStudio}''.
		
\end{frame}

\begin{frame}\frametitle{Software alternatives to R}

	\begin{itemize}
		\item	Excel
		\item	Python
		\item	Minitab
		\item	MATLAB
		\item	SAS
		\item	JMP
	\end{itemize}
\end{frame}

{\usebackgroundtemplate{\vbox to \paperheight{\vfil\hbox to \paperwidth{\hfil  
    \includegraphics[width=0.95\paperwidth,trim=0 1.69cm 0 1.69cm, clip]
	{../3B/Slides/14Screen Shot 2015-08-10 at 10.23.58 .png}  \hfil}\vfil}}
\begin{frame}\frametitle{}
\end{frame}}

{\usebackgroundtemplate{\vbox to \paperheight{\vfil\hbox to \paperwidth{\hfil  
    \includegraphics[width=0.95\paperwidth,trim=0 1.69cm 0 1.69cm, clip]
	{../3B/Slides/16Screen Shot 2015-08-10 at 10.24.40 .png}  \hfil}\vfil}}
\begin{frame}\frametitle{}
\end{frame}}


{\usebackgroundtemplate{\vbox to \paperheight{\vfil\hbox to \paperwidth{\hfil  
    \includegraphics[width=0.95\paperwidth,trim=0 1.69cm 0 1.69cm, clip]
	{../3B/Slides/17Screen Shot 2014-07-16 at 07.18.03 .png}  \hfil}\vfil}}
\begin{frame}\frametitle{}
\end{frame}}

{\usebackgroundtemplate{\vbox to \paperheight{\vfil\hbox to \paperwidth{\hfil  
    \includegraphics[width=0.95\paperwidth, clip]
	{../3B/Slides/19Screen Shot 2015-08-10 at 10.25.09 .png}  \hfil}\vfil}}
\begin{frame}\frametitle{}
\end{frame}}


% /Users/kevindunn/Dropbox/Coursera/Media/All-course-slides/classes/CourseraMOOC-class-3C.tex

{\usebackgroundtemplate{\vbox to \paperheight{\vfil\hbox to \paperwidth{\hfil  
    \includegraphics[width=0.95\paperwidth,trim=0 1.69cm 0 1.69cm, clip]
	{../3C/Slides/01Screen Shot 2014-07-23 at 11.16.20 .png}  \hfil}\vfil}}
\begin{frame}\frametitle{}
\end{frame}}

{\usebackgroundtemplate{\vbox to \paperheight{\vfil\hbox to \paperwidth{\hfil  
    \includegraphics[width=0.95\paperwidth,trim=0 1.69cm 0 1.69cm, clip]
	{../3C/Slides/02Screen Shot 2014-07-23 at 11.16.43 .png}  \hfil}\vfil}}
\begin{frame}\frametitle{}
\end{frame}}

{\usebackgroundtemplate{\vbox to \paperheight{\vfil\hbox to \paperwidth{\hfil  
    \includegraphics[width=0.95\paperwidth,trim=0 1.69cm 0 1.69cm, clip]
	{../3C/Slides/05Screen Shot 2015-08-10 at 10.42.40 .png}  \hfil}\vfil}}
\begin{frame}\frametitle{}
\end{frame}}

{\usebackgroundtemplate{\vbox to \paperheight{\vfil\hbox to \paperwidth{\hfil  
    \includegraphics[width=0.95\paperwidth,trim=0 1.69cm 0 1.69cm, clip]
	{../3C/Slides/07Screen Shot 2015-08-10 at 10.44.54 .png}  \hfil}\vfil}}
\begin{frame}\frametitle{}
\end{frame}}

{\usebackgroundtemplate{\vbox to \paperheight{\vfil\hbox to \paperwidth{\hfil  
    \includegraphics[width=0.95\paperwidth,trim=0 1.69cm 0 1.69cm, clip]
	{../3C/Slides/08Screen Shot 2014-07-28 at 08.56.24 .png}  \hfil}\vfil}}
\begin{frame}\frametitle{}
\end{frame}}

{\usebackgroundtemplate{\vbox to \paperheight{\vfil\hbox to \paperwidth{\hfil  
    \includegraphics[width=0.95\paperwidth,trim=0 1.69cm 0 1.69cm, clip]
	{../3C/Slides/09Screen Shot 2014-07-28 at 08.56.26 .png}  \hfil}\vfil}}
\begin{frame}\frametitle{}
\end{frame}}

{\usebackgroundtemplate{\vbox to \paperheight{\vfil\hbox to \paperwidth{\hfil  
    \includegraphics[width=0.95\paperwidth,trim=0 1.69cm 0 1.69cm, clip]
	{../3C/Slides/10Screen Shot 2015-08-10 at 10.46.38 .png}  \hfil}\vfil}}
\begin{frame}\frametitle{}
\end{frame}}

\begin{frame}\frametitle{There are many good reasons for writing code for your data analysis}
	\includegraphics[width=0.3\textwidth]{\imagedir/doe/examples/advice-logo.png}
	\begin{enumerate}
		\item	It creates a record of your work, detailing the exact steps

		\vspace{24pt}
		\item	It is traceable: showing the data source, and how you subsequently analyzed it
			\begin{itemize}
				\item	menu driven software is not traceable in this way
			\end{itemize}

		\vspace{24pt}
		\item	You can save it, and share it with colleagues
			\begin{itemize}
				\item	they can reproduce your steps exactly
			\end{itemize}
	\end{enumerate}
\end{frame}

{\usebackgroundtemplate{\vbox to \paperheight{\vfil\hbox to \paperwidth{\hfil  
    \includegraphics[width=0.95\paperwidth,trim=0 1.69cm 0 1.69cm, clip]
	{../3C/Slides/12Screen Shot 2015-08-10 at 10.49.02 .png}  \hfil}\vfil}}
\begin{frame}\frametitle{}
\end{frame}}

{\usebackgroundtemplate{\vbox to \paperheight{\vfil\hbox to \paperwidth{\hfil  
    \includegraphics[width=0.95\paperwidth,trim=0 1.69cm 0 1.69cm, clip]
	{../3C/Slides/13Screen Shot 2015-08-10 at 10.49.36 .png}  \hfil}\vfil}}
\begin{frame}\frametitle{}
\end{frame}}

{\usebackgroundtemplate{\vbox to \paperheight{\vfil\hbox to \paperwidth{\hfil  
    \includegraphics[width=0.95\paperwidth,trim=0 1.69cm 0 1.69cm, clip]
	{../3C/Slides/14Screen Shot 2015-08-10 at 10.51.53 .png}  \hfil}\vfil}}
\begin{frame}\frametitle{}
\end{frame}}

{\usebackgroundtemplate{\vbox to \paperheight{\vfil\hbox to \paperwidth{\hfil  
    \includegraphics[width=0.95\paperwidth,trim=0 1.69cm 0 1.69cm, clip]
	{../3C/Slides/15Screen Shot 2015-08-10 at 10.51.57 .png}  \hfil}\vfil}}
\begin{frame}\frametitle{}
\end{frame}}

{\usebackgroundtemplate{\vbox to \paperheight{\vfil\hbox to \paperwidth{\hfil  
    \includegraphics[width=0.95\paperwidth,trim=0 1.69cm 0 1.69cm, clip]
	{../3C/Slides/16Screen Shot 2015-08-10 at 10.52.45 .png}  \hfil}\vfil}}
\begin{frame}\frametitle{}
\end{frame}}

{\usebackgroundtemplate{\vbox to \paperheight{\vfil\hbox to \paperwidth{\hfil  
    \includegraphics[width=0.95\paperwidth,trim=0 1.69cm 0 1.69cm, clip]
	{../3C/Slides/17Screen Shot 2015-08-10 at 10.52.56 .png}  \hfil}\vfil}}
\begin{frame}\frametitle{}
\end{frame}}

{\usebackgroundtemplate{\vbox to \paperheight{\vfil\hbox to \paperwidth{\hfil  
    \includegraphics[width=0.95\paperwidth,trim=0 1.69cm 0 1.69cm, clip]
	{../3C/Slides/18Screen Shot 2015-08-10 at 10.53.30 .png}  \hfil}\vfil}}
\begin{frame}\frametitle{}
\end{frame}}

{\usebackgroundtemplate{\vbox to \paperheight{\vfil\hbox to \paperwidth{\hfil  
    \includegraphics[width=0.95\paperwidth,trim=0 1.69cm 0 1.69cm, clip]
	{../3C/Slides/19Screen Shot 2015-08-10 at 10.53.35 .png}  \hfil}\vfil}}
\begin{frame}\frametitle{}
\end{frame}}

{\usebackgroundtemplate{\vbox to \paperheight{\vfil\hbox to \paperwidth{\hfil  
    \includegraphics[width=0.95\paperwidth,trim=0 1.69cm 0 1.69cm, clip]
	{../3C/Slides/20Screen Shot 2015-08-10 at 10.53.41 .png}  \hfil}\vfil}}
\begin{frame}\frametitle{}
\end{frame}}

{\usebackgroundtemplate{\vbox to \paperheight{\vfil\hbox to \paperwidth{\hfil  
    \includegraphics[width=0.95\paperwidth,trim=0 1.69cm 0 1.69cm, clip]
	{../3C/Slides/21Screen Shot 2015-08-10 at 10.53.44 .png}  \hfil}\vfil}}
\begin{frame}\frametitle{}
	\vfill
	\vspace{72pt}
	Click here: \href{http://yint.org/3C}{http://yint.org/3C}
\end{frame}}




% /Users/kevindunn/Dropbox/Coursera/Media/All-course-slides/classes/CourseraMOOC-class-3D.tex

\begin{frame}\frametitle{Solar panel case study}
	
	\begin{columns}[T]
		\column{0.45\textwidth}
			\includegraphics[width=0.7\textwidth]{\imagedir/statistics/flickr-Box-Hunter-Hunter-cover-3056749047_1c1f633fcb_o.jpg}
			
			{\scriptsize (p. 230 in Box, Hunter and Hunter, 2$^\text{nd}$ ed)}
			
		\column{0.48\textwidth}
			\includegraphics[width=\textwidth]{\imagedir/doe/examples/solar-panel-mendelu-cz-website.png}
			
			
			\see{\href{http://yint.org/solar-panel-study}{http://yint.org/solar-panel-study}}
	\end{columns}
\end{frame}

\begin{frame}\frametitle{\includegraphics[width=0.3\textwidth]{\imagedir/doe/examples/advice-logo.png} when experimenting with computer simulations}
	\begin{columns}[T]
		\column{0.48\textwidth}
		
			\textbf{The same as regular experiments}:
			
			\vspace{12pt}
			\begin{itemize}
				\item	you must follow a systematic method
				\item	don't ``play around'' with the software: trial-and-error
			\end{itemize}
			
			%\begin{center}\rule[8mm]{4cm}{0.01cm}\end{center}
			
			\vspace{18pt}

			 \fbox{\parbox[b][7em][t]{\textwidth}{
			 	{\footnotesize
				 	Many experiments are simulations:
					\begin{itemize}
						\item	bridge/building design
						\item	chemical factory design
						\item	improve traffic light timing and queuing
			 			\item	test a stock market buy/sell strategy
			 		\end{itemize}}
			 } }
			 
				
		\column{0.01\textwidth}
			\rule[3mm]{0.01cm}{25mm}%
			
		\column{0.48\textwidth}
		\onslide+<2->{
			\textbf{Different to regular experiments:}
			\vspace{12pt}
			\begin{enumerate}
				\item	We can often run computer simulations in parallel
				\item	Computer experiments (mostly$^\ast$) are deterministic
					\begin{itemize}
						\item	i.e. if you repeat the experiments, you get the identical results
						\item	this indicates there are no disturbances that affect the outcome
						\item	this implies you do not need to randomize the order
						\item	or even repeat experiments!
						
					\end{itemize}
			\end{enumerate}
			{\scriptsize $^\ast$ {\emph{except those that have a random component}}}
		}
	\end{columns}
	
	
\end{frame}

\begin{frame}\frametitle{Solar panel case study}
	
	\begin{columns}[T]
		\column{0.45\textwidth}
			
			\includegraphics[width=\textwidth]{\imagedir/doe/examples/solar-panel-mendelu-cz-website.png}
			
			
			\href{http://yint.org/solar-panel-study}{http://yint.org/solar-panel-study}
		\column{0.45\textwidth}
			The factors are:
			\begin{itemize}
				\item	\textbf{A} = total daily \href{https://en.wikipedia.org/wiki/Insolation}{insolation} (sunlight received)
				\item	\textbf{B} = storage tank capacity
				\item	\textbf{C} = water flow rate
				\item	\textbf{D} = \href{https://en.wikipedia.org/wiki/Intermittent\_energy\_source }{solar intermittency}
			\end{itemize}
			\pause
			\vspace{12pt}
			The outcome variables were:
			\begin{itemize}
				\item	$y_1$ = collection efficiency
				\item	$y_2$ = energy delivery efficiency
			\end{itemize}
			\pause
			\vspace{12pt}
			Total experiments required = \pause $2^4 = 16$ 
			
	\end{columns}
\end{frame}

{\usebackgroundtemplate{\vbox to \paperheight{\vfil\hbox to \paperwidth{\hfil  
    \includegraphics[width=0.95\paperwidth,trim=0 1.69cm 0 1.69cm, clip]
	{../3D/Slides/01Screen Shot 2015-08-10 at 13.58.53 .png}  \hfil}\vfil}}
\begin{frame}\frametitle{}
\end{frame}}
{\usebackgroundtemplate{\vbox to \paperheight{\vfil\hbox to \paperwidth{\hfil  
    \includegraphics[width=0.95\paperwidth,trim=0 1.69cm 0 1.69cm, clip]
	{../3D/Slides/02Screen Shot 2015-08-10 at 13.59.12 .png}  \hfil}\vfil}}
\begin{frame}\frametitle{}
\end{frame}}
{\usebackgroundtemplate{\vbox to \paperheight{\vfil\hbox to \paperwidth{\hfil  
    \includegraphics[width=0.95\paperwidth,trim=0 1.69cm 0 1.69cm, clip]
	{../3D/Slides/03Screen Shot 2015-08-10 at 13.59.31 .png}  \hfil}\vfil}}
\begin{frame}\frametitle{}
\end{frame}}
{\usebackgroundtemplate{\vbox to \paperheight{\vfil\hbox to \paperwidth{\hfil  
    \includegraphics[width=0.95\paperwidth,trim=0 1.69cm 0 1.69cm, clip]
	{../3D/Slides/04Screen Shot 2015-08-10 at 13.59.44 .png}  \hfil}\vfil}}
\begin{frame}\frametitle{}
\end{frame}}
{\usebackgroundtemplate{\vbox to \paperheight{\vfil\hbox to \paperwidth{\hfil  
    \includegraphics[width=0.95\paperwidth,trim=0 1.69cm 0 1.69cm, clip]
	{../3D/Slides/05Screen Shot 2015-08-10 at 13.59.55 .png}  \hfil}\vfil}}
\begin{frame}\frametitle{}
\end{frame}}
{\usebackgroundtemplate{\vbox to \paperheight{\vfil\hbox to \paperwidth{\hfil  
    \includegraphics[width=0.95\paperwidth,trim=0 1.69cm 0 1.69cm, clip]
	{../3D/Slides/06Screen Shot 2015-08-10 at 14.00.07 .png}  \hfil}\vfil}}
\begin{frame}\frametitle{}
\end{frame}}
{\usebackgroundtemplate{\vbox to \paperheight{\vfil\hbox to \paperwidth{\hfil  
    \includegraphics[width=0.95\paperwidth,trim=0 1.69cm 0 1.69cm, clip]
	{../3D/Slides/07Screen Shot 2015-08-10 at 14.00.11 .png}  \hfil}\vfil}}
\begin{frame}\frametitle{}
\end{frame}}

\begin{frame}\frametitle{R's automatic expansion of model terms}
	\texttt{lm(y {\raise.17ex\hbox{$\scriptstyle\mathtt{\sim}$}} A*B)}\\
		\qquad expands into: {\color{blue}\texttt{lm(y {\raise.17ex\hbox{$\scriptstyle\mathtt{\sim}$}} A + B + A*B)}}
		
	\vspace{24pt}
	\texttt{lm(y {\raise.17ex\hbox{$\scriptstyle\mathtt{\sim}$}} A*B*C)}\\
		\qquad ultimately expands into: {\color{blue}\texttt{lm(y {\raise.17ex\hbox{$\scriptstyle\mathtt{\sim}$}} A + B + C + A*B + A*C + B*C + A*B*C)}}
		
		\vspace{12pt}
		\qquad 
		This is because: \texttt{A*B*C} can be considered to be \texttt{(A*B)*C} \\
		\qquad which expands into \texttt{(A + B + A*B)*C = A*C + B*C + A*B*C}
		
		\vspace{6pt}
		\qquad  but \texttt{A*C} expands into \texttt{A + C + A*C}\\
		\qquad  and \texttt{B*C} expands into \texttt{B + C + B*C} \\
		\qquad which when all collected together gives the full model expansion above.
\end{frame}

{\usebackgroundtemplate{\vbox to \paperheight{\vfil\hbox to \paperwidth{\hfil  
    \includegraphics[width=0.95\paperwidth,trim=0 1.69cm 0 1.69cm, clip]
	{../3D/Slides/08Screen Shot 2015-08-10 at 14.00.25 .png}  \hfil}\vfil}}
\begin{frame}\frametitle{}
\end{frame}}
{\usebackgroundtemplate{\vbox to \paperheight{\vfil\hbox to \paperwidth{\hfil  
    \includegraphics[width=0.95\paperwidth,trim=0 1.69cm 0 1.69cm, clip]
	{../3D/Slides/09Screen Shot 2015-08-10 at 14.00.33 .png}  \hfil}\vfil}}
\begin{frame}\frametitle{}
\end{frame}}
{\usebackgroundtemplate{\vbox to \paperheight{\vfil\hbox to \paperwidth{\hfil  
    \includegraphics[width=0.95\paperwidth,trim=0 1.69cm 0 1.69cm, clip]
	{../3D/Slides/10Screen Shot 2015-08-10 at 14.00.38 .png}  \hfil}\vfil}}
\begin{frame}\frametitle{}
\end{frame}}
{\usebackgroundtemplate{\vbox to \paperheight{\vfil\hbox to \paperwidth{\hfil  
    \includegraphics[width=0.95\paperwidth,trim=0 1.69cm 0 1.69cm, clip]
	{../3D/Slides/11Screen Shot 2015-08-10 at 14.00.46 .png}  \hfil}\vfil}}
\begin{frame}\frametitle{}
\end{frame}}
{\usebackgroundtemplate{\vbox to \paperheight{\vfil\hbox to \paperwidth{\hfil  
    \includegraphics[width=0.95\paperwidth,trim=0 1.69cm 0 1.69cm, clip]
	{../3D/Slides/12Screen Shot 2015-08-10 at 14.01.18 .png}  \hfil}\vfil}}
\begin{frame}\frametitle{}
\end{frame}}
{\usebackgroundtemplate{\vbox to \paperheight{\vfil\hbox to \paperwidth{\hfil  
    \includegraphics[width=0.95\paperwidth,trim=0 1.69cm 0 1.69cm, clip]
	{../3D/Slides/13Screen Shot 2015-08-10 at 14.01.43 .png}  \hfil}\vfil}}
\begin{frame}\frametitle{}
\end{frame}}


\begin{frame}\frametitle{Advanced thinking: optimizing multiple objectives}
	\begin{center}
		{\color{myOrange}Consider the case where the aim is to maximize \textbf{both} $y_1$ and $y_2$.}
	\end{center}
	\vspace{-10pt}
	\begin{columns}[T]
		\column{0.01\textwidth}
		\column{0.35\textwidth}
			$y_1$ = ``collection efficiency'' 
			
			\includegraphics[width=1.1\textwidth]{\imagedir/doe/examples/y1-solar-panels.png}
		\column{0.05\textwidth}
		\column{0.45\textwidth}
			$y_2$ = ``energy delivery efficiency''
			\includegraphics[width=1.1\textwidth]{\imagedir/doe/examples/y2-solar-panels.png}
		\column{0.30\textwidth}	
			\vspace{12pt}
			\fbox{\parbox[b][8.5em][t]{.85\textwidth}{
				Specify at what levels (high or low) should we set the factors\\ \textbf{A}, \textbf{B}, \textbf{C}, and \textbf{D} to achieve a maximum in both $y_1$ \emph{and} in $y_2$.
			}}
	\end{columns}
\end{frame}



% /Users/kevindunn/Dropbox/Coursera/Media/All-course-slides/classes/CourseraMOOC-class-4A.tex


% NOTE: THIS FIRST SECTION IS ALL COMMENTED OUT

\begin{comment}
	\begin{columns}[T]
		\column{0.45\textwidth}
			\includegraphics[width=0.7\textwidth]{\imagedir/statistics/flicfcb_o.jpg}
		
			{\scriptsize (p. 230 in Box, Hunter and Hunter, 2$^\text{nd}$ ed)}
		
		\column{0.48\textwidth}
			\includegraphics[width=\textwidth]{\imagedir/doe/examples/solar-panel-mendelu-cz-website.png}
		
		
			\see{\href{http://yint.org/solar-panel-study}{http://yint.org/solar-panel-study}}
	\end{columns}
	
	\begin{center}\rule[8mm]{4cm}{0.01cm}\end{center}
	\rule[3mm]{0.01cm}{25mm}%

\begin{frame}\frametitle{}

		Assuming 6 hours and \$150 per experiment:
		
		\begin{tabular}{cccc}\hline 
			\textsf{\relax Number of } & \textsf{\relax Total  } & \textsf{\relax Cost of all} & \textsf{\relax Time to } \\
			\textsf{\relax factors} & \textsf{\relax experiments} & \textsf{\relax experiments} & \textsf{\relax run experiments}\\	\hline \hline
			\onslide+<1->{
				& & & \vspace{-.5cm} \\}
			\onslide+<2->{
				2 & 4 & \$300 & 1 day\\}
			\onslide+<3->{
				3 & 8 & \$600 & 2 days\\}
			\onslide+<4->{
				4 & 16 & \$1,200 & 4 days\\}
			\onslide+<5->{
				5 & 32 & \$2,400 & 8 days\\}
			\onslide+<6->{
				6 & 64 & \$4,800 & 16 days\\}
			\onslide+<7->{
				7 & 128 & \$9,600 & 32 days\\}
		\end{tabular}

	

	
\end{frame}

\begin{frame}\frametitle{There are $2^k$ model parameters in a full-factorial: not all are meaningful!}
	\vspace{6pt}
	For $k=4$ factors:
	\vspace{-6pt}
		{\scriptsize
		\begin{align*}
			\hat{y} &= {\color{myOrange}b_0}\\
					&+ {\color{myOrange}b_\text{A}\,}x_\text{A}\\
					&+ {\color{myOrange}b_\text{B}\,}x_\text{B}\\
					&+ {\color{myOrange}b_\text{C}\,}x_\text{C}\\
					&+ {\color{myOrange}b_\text{D}\,}x_\text{D}\\
					&+ {\color{myOrange}b_\text{AB}\,}x_\text{A}x_\text{B}\\
					&+ {\color{myOrange}b_\text{AC}\,}x_\text{A}x_\text{C}\\
					&+ {\color{myOrange}b_\text{BC}\,}x_\text{B}x_\text{C}\\
					&+ {\color{myOrange}b_\text{AD}\,}x_\text{A}x_\text{D}\\
					&+ {\color{myOrange}b_\text{BD}\,}x_\text{B}x_\text{D}\\
					&+ {\color{myOrange}b_\text{CD}\,}x_\text{C}x_\text{D}\\
					&+ {\color{myOrange}b_\text{ABC}\,}x_\text{A}x_\text{B}x_\text{C}\\
					&+ {\color{myOrange}b_\text{ABD}\,}x_\text{A}x_\text{B}x_\text{D}\\
					&+ {\color{myOrange}b_\text{ACD}\,}x_\text{A}x_\text{C}x_\text{D}\\
					&+ {\color{myOrange}b_\text{BCD}\,}x_\text{B}x_\text{C}x_\text{D}\\
					&+ {\color{myOrange}b_\text{ABCD}\,}x_\text{A}x_\text{B}x_\text{C}x_\text{D}
		\end{align*}
		}
	

\end{frame}

\begin{frame}\frametitle{Most real systems exhibit minor interactions; main effects usually dominate}
	\begin{columns}[T]
		\column{0.45\textwidth}
			\includegraphics[width=\textwidth]{\imagedir/doe/examples/chemical-conversion-pareto.png}
			
			{\tiny p. 200 in Box, Hunter and Hunter, 2$^\text{nd}$ ed}
			
		\column{0.45\textwidth}
			\includegraphics[width=\textwidth]{\imagedir/doe/examples/metal-removal-pareto.png}
			
			{\tiny Metal removal from wastewater; McMaster student project}
			
	\end{columns}
	
\end{frame}

\begin{frame}\frametitle{Core assumption regarding fractional factorials}
	\begin{columns}
		\column{0.65\textwidth}
			{\scriptsize
			\begin{align*}
				\hat{y} &= {\color{myOrange}b_0}\\
						&+ {\color{myOrange}b_\text{A}\,}x_\text{A}\\
						&+ {\color{myOrange}b_\text{B}\,}x_\text{B}\\
						&+ {\color{myOrange}b_\text{C}\,}x_\text{C}\\
						&+ {\color{myOrange}b_\text{D}\,}x_\text{D}\\
						&+ {\color{myOrange}b_\text{AB}\,}x_\text{A}x_\text{B}\\
						&+ {\color{myOrange}b_\text{AC}\,}x_\text{A}x_\text{C}\\
						&+ {\color{myOrange}b_\text{BC}\,}x_\text{B}x_\text{C}\\
						&+ {\color{myOrange}b_\text{AD}\,}x_\text{A}x_\text{D}\\
						&+ {\color{myOrange}b_\text{BD}\,}x_\text{B}x_\text{D}\\
						&+ {\color{myOrange}b_\text{CD}\,}x_\text{C}x_\text{D}\\
						&+ {\color{myOrange}b_\text{ABC}\,}x_\text{A}x_\text{B}x_\text{C}\\
						&+ {\color{myOrange}b_\text{ABD}\,}x_\text{A}x_\text{B}x_\text{D}\\
						&+ {\color{myOrange}b_\text{ACD}\,}x_\text{A}x_\text{C}x_\text{D}\\
						&+ {\color{myOrange}b_\text{BCD}\,}x_\text{B}x_\text{C}x_\text{D}\\
						&+ {\color{myOrange}b_\text{ABCD}\,}x_\text{A}x_\text{B}x_\text{C}x_\text{D}
			\end{align*}
			}
			
		\column{0.45\textwidth}
			\pause
			The main effects and some two factor interactions are often the only parameters of interest
			
			\vspace{36pt}
			The higher order interactions can safely be ignored
			\begin{itemize}
				\item	Now it is an assumption, but it's reasonable in many cases
				\item	The cost of obtaining them can be prohibitive
			\end{itemize}
			
	\end{columns}
\end{frame}

\begin{frame}\frametitle{If you'd like to do half the work, which 4 experiments would you pick?}
	\begin{columns}
		\column{0.65\textwidth}
			\begin{center}
				\includegraphics[width=.9\textwidth]{\imagedir/doe/half-fraction-in-3-factors-MOOC-all-8.png}
			\end{center}
			
		\column{0.45\textwidth}
			\begin{tabulary}{\linewidth}{|c||c|c|c|}\hline 
				\textsf{\relax Experiment } & \textbf{\relax A } & \textbf{\relax B } & \textbf{\relax C } \\
				\hline 1 & \(-\) & \(-\) & \(-\) \\
				\hline 2 & \(+\) & \(-\) & \(-\) \\
				\hline 3 & \(-\) & \(+\) & \(-\) \\
				\hline 4 & \(+\) & \(+\) & \(-\) \\
				\hline 5 & \(-\) & \(-\) & \(+\) \\
				\hline 6 & \(+\) & \(-\) & \(+\) \\
				\hline 7 & \(-\) & \(+\) & \(+\) \\
				\hline 8 & \(+\) & \(+\) & \(+\) \\
				\hline
			\end{tabulary}
	\end{columns}	
\end{frame}

\begin{frame}\frametitle{If you'd like to do half the work, which 4 experiments would you pick?}
	\begin{columns}
		\column{0.65\textwidth}
			\begin{center}
				\includegraphics[width=.9\textwidth]{\imagedir/doe/half-fraction-in-3-factors-MOOC-front-4.png}
			\end{center}
			
		\column{0.45\textwidth}
			\begin{tabulary}{\linewidth}{|c||c|c|c|}\hline 
				\textsf{\relax Experiment } & \textbf{\relax A } & \textbf{\relax B } & \textbf{\relax C } \\
				\hline \color{myOrange} \textbf{1} & \(-\) & \(-\) & \(-\) \\
				\hline \color{myOrange} \textbf{2} & \(+\) & \(-\) & \(-\) \\
				\hline \color{myOrange} \textbf{3} & \(-\) & \(+\) & \(-\) \\
				\hline \color{myOrange} \textbf{4} & \(+\) & \(+\) & \(-\) \\
				\hline 5 & \(-\) & \(-\) & \(+\) \\
				\hline 6 & \(+\) & \(-\) & \(+\) \\
				\hline 7 & \(-\) & \(+\) & \(+\) \\
				\hline 8 & \(+\) & \(+\) & \(+\) \\
				\hline
			\end{tabulary}
	\end{columns}	
\end{frame}

\begin{frame}\frametitle{If you'd like to do half the work, which 4 experiments would you pick?}
	\begin{columns}
		\column{0.65\textwidth}
			\begin{center}
				\includegraphics[width=.9\textwidth]{\imagedir/doe/half-fraction-in-3-factors-MOOC-middle-4.png}
			\end{center}
			
		\column{0.45\textwidth}
			\begin{tabulary}{\linewidth}{|c||c|c|c|}\hline 
				\textsf{\relax Experiment } & \textbf{\relax A } & \textbf{\relax B } & \textbf{\relax C } \\
				\hline 1 & \(-\) & \(-\) & \(-\) \\
				\hline 2 & \(+\) & \(-\) & \(-\) \\
				\hline \color{myOrange} \textbf{3} & \(-\) & \(+\) & \(-\) \\
				\hline \color{myOrange} \textbf{4} & \(+\) & \(+\) & \(-\) \\
				\hline \color{myOrange} \textbf{5} & \(-\) & \(-\) & \(+\) \\
				\hline \color{myOrange} \textbf{6} & \(+\) & \(-\) & \(+\) \\
				\hline 7 & \(-\) & \(+\) & \(+\) \\
				\hline 8 & \(+\) & \(+\) & \(+\) \\
				\hline
			\end{tabulary}
	\end{columns}	
\end{frame}

\begin{frame}\frametitle{If you'd like to do half the work, which 4 experiments would you pick?}
	\begin{columns}
		\column{0.65\textwidth}
			\begin{center}
				\includegraphics[width=.9\textwidth]{\imagedir/doe/half-fraction-in-3-factors-MOOC-optimum-4.png}
			\end{center}
			
		\column{0.45\textwidth}
			\begin{tabulary}{\linewidth}{|c||c|c|c|}\hline 
				\textsf{\relax Experiment } & \textbf{\relax A } & \textbf{\relax B } & \textbf{\relax C } \\
				\hline 1 & \(-\) & \(-\) & \(-\) \\
				\hline \color{myOrange} \textbf{2} & \(+\) & \(-\) & \(-\) \\
				\hline \color{myOrange} \textbf{3} & \(-\) & \(+\) & \(-\) \\
				\hline 4 & \(+\) & \(+\) & \(-\) \\
				\hline \color{myOrange} \textbf{5} & \(-\) & \(-\) & \(+\) \\
				\hline 6 & \(+\) & \(-\) & \(+\) \\
				\hline 7 & \(-\) & \(+\) & \(+\) \\
				\hline \color{myOrange} \textbf{8} & \(+\) & \(+\) & \(+\) \\
				\hline
			\end{tabulary}
	\end{columns}	
\end{frame}

\begin{frame}\frametitle{If you'd like to do half the work, which 4 experiments would you pick?}
	\begin{columns}
		\column{0.65\textwidth}
			\begin{center}
				\includegraphics[width=.9\textwidth]{\imagedir/doe/half-fraction-in-3-factors-MOOC-optimum-4.png}
			\end{center}
			
		\column{0.45\textwidth}
			\begin{tabulary}{\linewidth}{|c||c|c|c|}\hline 
				\textsf{\relax Experiment } & \textbf{\relax A } & \textbf{\relax B } & \textbf{\relax C } \\
				\hline \color{blue} \textbf{1} & \(-\) & \(-\) & \(-\) \\
				\hline 2 & \(+\) & \(-\) & \(-\) \\
				\hline 3 & \(-\) & \(+\) & \(-\) \\
				\hline \color{blue} \textbf{4} & \(+\) & \(+\) & \(-\) \\
				\hline 5 & \(-\) & \(-\) & \(+\) \\
				\hline \color{blue} \textbf{6} & \(+\) & \(-\) & \(+\) \\
				\hline \color{blue} \textbf{7} & \(-\) & \(+\) & \(+\) \\
				\hline 8 & \(+\) & \(+\) & \(+\) \\
				\hline
			\end{tabulary}
	\end{columns}	
\end{frame}

\begin{frame}\frametitle{If you'd like to do half the work, which 4 experiments would you pick?}
	\begin{columns}
		\column{0.65\textwidth}
			\begin{center}
				\includegraphics[width=.9\textwidth]{\imagedir/doe/half-fraction-in-3-factors-MOOC-collapse-transition.png}
			\end{center}
			
		\column{0.45\textwidth}
			\begin{tabulary}{\linewidth}{|c||c|c|c|}\hline 
				\textsf{\relax Experiment } & \textbf{\relax A } & \textbf{\relax B } & \textbf{\relax C } \\
				\hline 1 & \(-\) & \(-\) & \(-\) \\
				\hline \color{myOrange} \textbf{2} & \(+\) & \(-\) & \(-\) \\
				\hline \color{myOrange} \textbf{3} & \(-\) & \(+\) & \(-\) \\
				\hline 4 & \(+\) & \(+\) & \(-\) \\
				\hline \color{myOrange} \textbf{5} & \(-\) & \(-\) & \(+\) \\
				\hline 6 & \(+\) & \(-\) & \(+\) \\
				\hline 7 & \(-\) & \(+\) & \(+\) \\
				\hline \color{myOrange} \textbf{8} & \(+\) & \(+\) & \(+\) \\
				\hline
			\end{tabulary}
	\end{columns}	
\end{frame}

\begin{frame}\frametitle{If you'd like to do half the work, which 4 experiments would you pick?}
	\begin{columns}
		\column{0.65\textwidth}
			\begin{center}
				\includegraphics[width=.9\textwidth]{\imagedir/doe/half-fraction-in-3-factors-MOOC-collapse-low.png}
			\end{center}
			
		\column{0.45\textwidth}
			\begin{tabulary}{\linewidth}{|c||c|c|c|}\hline 
				\textsf{\relax Experiment } & \textbf{\relax A } & \textbf{\relax B } & \textbf{\relax C } \\
				\hline 1 & \(-\) & \(-\) & \(-\) \\
				\hline \color{myOrange} \textbf{2} & \(+\) & \(-\) & \(-\) \\
				\hline \color{myOrange} \textbf{3} & \(-\) & \(+\) & \(-\) \\
				\hline 4 & \(+\) & \(+\) & \(-\) \\
				\hline \color{myOrange} \textbf{5} & \(-\) & \(-\) & \(+\) \\
				\hline 6 & \(+\) & \(-\) & \(+\) \\
				\hline 7 & \(-\) & \(+\) & \(+\) \\
				\hline \color{myOrange} \textbf{8} & \(+\) & \(+\) & \(+\) \\
				\hline
			\end{tabulary}
	\end{columns}	
\end{frame}

\begin{frame}\frametitle{If you'd like to do half the work, which 4 experiments would you pick?}
	\begin{columns}
		\column{0.65\textwidth}
			\begin{center}
				\includegraphics[width=.9\textwidth]{\imagedir/doe/half-fraction-in-3-factors-MOOC-collapse-high.png}
			\end{center}
			
		\column{0.45\textwidth}
			\begin{tabulary}{\linewidth}{|c||c|c|c|}\hline 
				\textsf{\relax Experiment } & \textbf{\relax A } & \textbf{\relax B } & \textbf{\relax C } \\
				\hline 1 & \(-\) & \(-\) & \(-\) \\
				\hline \color{myOrange} \textbf{2} & \(+\) & \(-\) & \(-\) \\
				\hline \color{myOrange} \textbf{3} & \(-\) & \(+\) & \(-\) \\
				\hline 4 & \(+\) & \(+\) & \(-\) \\
				\hline \color{myOrange} \textbf{5} & \(-\) & \(-\) & \(+\) \\
				\hline 6 & \(+\) & \(-\) & \(+\) \\
				\hline 7 & \(-\) & \(+\) & \(+\) \\
				\hline \color{myOrange} \textbf{8} & \(+\) & \(+\) & \(+\) \\
				\hline
			\end{tabulary}
	\end{columns}	
\end{frame}

\begin{frame}\frametitle{If you'd like to do half the work, which 4 experiments would you pick?}
	\begin{columns}
		\column{0.65\textwidth}
			\begin{center}
				\includegraphics[width=.9\textwidth]{\imagedir/doe/half-fraction-in-3-factors-MOOC-collapse-both.png}
			\end{center}
			
		\column{0.45\textwidth}
			\begin{tabulary}{\linewidth}{|c||c|c|c|}\hline 
				\textsf{\relax Experiment } & \textbf{\relax A } & \textbf{\relax B } & \textbf{\relax C } \\
				\hline 1 & \(-\) & \(-\) & \(-\) \\
				\hline \color{myOrange} \textbf{2} & \(+\) & \(-\) & \(-\) \\
				\hline \color{myOrange} \textbf{3} & \(-\) & \(+\) & \(-\) \\
				\hline 4 & \(+\) & \(+\) & \(-\) \\
				\hline \color{myOrange} \textbf{5} & \(-\) & \(-\) & \(+\) \\
				\hline 6 & \(+\) & \(-\) & \(+\) \\
				\hline 7 & \(-\) & \(+\) & \(+\) \\
				\hline \color{myOrange} \textbf{8} & \(+\) & \(+\) & \(+\) \\
				\hline
			\end{tabulary}
	\end{columns}	
\end{frame}

\begin{frame}\frametitle{If you'd like to do half the work, which 4 experiments would you pick?}
	\begin{columns}
		\column{0.65\textwidth}
			\begin{center}
				\includegraphics[width=.9\textwidth]{\imagedir/doe/half-fraction-in-3-factors-MOOC-collapse-3-in.png}
			\end{center}
			
		\column{0.45\textwidth}
			\begin{tabulary}{\linewidth}{|c||c|c|c|}\hline 
				\textsf{\relax Experiment } & \textbf{\relax A } & \textbf{\relax B } & \textbf{\relax C } \\
				\hline 1 & \(-\) & \(-\) & \(-\) \\
				\hline \color{myOrange} \textbf{2} & \(+\) & \(-\) & \(-\) \\
				\hline \color{myOrange} \textbf{3} & \(-\) & \(+\) & \(-\) \\
				\hline 4 & \(+\) & \(+\) & \(-\) \\
				\hline \color{myOrange} \textbf{5} & \(-\) & \(-\) & \(+\) \\
				\hline 6 & \(+\) & \(-\) & \(+\) \\
				\hline 7 & \(-\) & \(+\) & \(+\) \\
				\hline \color{myOrange} \textbf{8} & \(+\) & \(+\) & \(+\) \\
				\hline
			\end{tabulary}
	\end{columns}	
\end{frame}

\begin{frame}\frametitle{If you'd like to do half the work, which 4 experiments would you pick?}
	\begin{columns}
		\column{0.65\textwidth}
			\begin{center}
				\includegraphics[width=.9\textwidth]{\imagedir/doe/half-fraction-in-3-factors-MOOC-collapse-6-in.png}
			\end{center}
			
		\column{0.45\textwidth}
			\begin{tabulary}{\linewidth}{|c||c|c|c|}\hline 
				\textsf{\relax Experiment } & \textbf{\relax A } & \textbf{\relax B } & \textbf{\relax C } \\
				\hline 1 & \(-\) & \(-\) & \(-\) \\
				\hline \color{myOrange} \textbf{2} & \(+\) & \(-\) & \(-\) \\
				\hline \color{myOrange} \textbf{3} & \(-\) & \(+\) & \(-\) \\
				\hline 4 & \(+\) & \(+\) & \(-\) \\
				\hline \color{myOrange} \textbf{5} & \(-\) & \(-\) & \(+\) \\
				\hline 6 & \(+\) & \(-\) & \(+\) \\
				\hline 7 & \(-\) & \(+\) & \(+\) \\
				\hline \color{myOrange} \textbf{8} & \(+\) & \(+\) & \(+\) \\
				\hline
			\end{tabulary}
	\end{columns}	
\end{frame}

\begin{frame}\frametitle{If you'd like to do half the work, which 4 experiments would you pick?}
	\begin{columns}
		\column{0.65\textwidth}
			\begin{center}
				\includegraphics[width=.9\textwidth]{\imagedir/doe/half-fraction-in-3-factors-MOOC-collapse-total.png}
			\end{center}
			
		\column{0.45\textwidth}
			\begin{tabulary}{\linewidth}{|c||c|c|c|}\hline 
				\textsf{\relax Experiment } & \textbf{\relax A } & \textbf{\relax B } & \textbf{\relax C } \\
				\hline 1 & \(-\) & \(-\) & \(-\) \\
				\hline \color{myOrange} \textbf{2} & \(+\) & \(-\) & \(-\) \\
				\hline \color{myOrange} \textbf{3} & \(-\) & \(+\) & \(-\) \\
				\hline 4 & \(+\) & \(+\) & \(-\) \\
				\hline \color{myOrange} \textbf{5} & \(-\) & \(-\) & \(+\) \\
				\hline 6 & \(+\) & \(-\) & \(+\) \\
				\hline 7 & \(-\) & \(+\) & \(+\) \\
				\hline \color{myOrange} \textbf{8} & \(+\) & \(+\) & \(+\) \\
				\hline
			\end{tabulary}
	\end{columns}	
\end{frame}

\begin{frame}\frametitle{If you'd like to do half the work, which 4 experiments would you pick?}
	\begin{columns}
		\column{0.65\textwidth}
			\begin{center}
				\includegraphics[width=.9\textwidth]{\imagedir/doe/half-fraction-in-3-factors-MOOC-collapse-total-rotate.png}
			\end{center}
			
		\column{0.45\textwidth}
			\begin{tabulary}{\linewidth}{|c||c|c|c|}\hline 
				\textsf{\relax Experiment } & \textbf{\relax A } & \textbf{\relax B } & \textbf{\relax C } \\
				\hline 1 & \(-\) & \(-\) & \(-\) \\
				\hline \color{myOrange} \textbf{2} & \(+\) & \(-\) & \(-\) \\
				\hline \color{myOrange} \textbf{3} & \(-\) & \(+\) & \(-\) \\
				\hline 4 & \(+\) & \(+\) & \(-\) \\
				\hline \color{myOrange} \textbf{5} & \(-\) & \(-\) & \(+\) \\
				\hline 6 & \(+\) & \(-\) & \(+\) \\
				\hline 7 & \(-\) & \(+\) & \(+\) \\
				\hline \color{myOrange} \textbf{8} & \(+\) & \(+\) & \(+\) \\
				\hline
			\end{tabulary}
	\end{columns}	
\end{frame}

\begin{frame}\frametitle{If you'd like to do half the work, which 4 experiments would you pick?}
	\begin{columns}
		\column{0.65\textwidth}
			\begin{center}
				\includegraphics[width=.9\textwidth]{\imagedir/doe/half-fraction-in-3-factors-MOOC-collapse-total-rotate-labelled.png}
			\end{center}
			
		\column{0.45\textwidth}
			\begin{tabulary}{\linewidth}{|c||c|c|c|}\hline 
				\textsf{\relax Experiment } & \textbf{\relax A } & \textbf{\relax B } & \textbf{\relax C } \\
				\hline 1 & \(-\) & \(-\) & \(-\) \\
				\hline \color{myOrange} \textbf{2} & \(+\) & \(-\) & \(-\) \\
				\hline \color{myOrange} \textbf{3} & \(-\) & \(+\) & \(-\) \\
				\hline 4 & \(+\) & \(+\) & \(-\) \\
				\hline \color{myOrange} \textbf{5} & \(-\) & \(-\) & \(+\) \\
				\hline 6 & \(+\) & \(-\) & \(+\) \\
				\hline 7 & \(-\) & \(+\) & \(+\) \\
				\hline \color{myOrange} \textbf{8} & \(+\) & \(+\) & \(+\) \\
				\hline
			\end{tabulary}
	\end{columns}	
\end{frame}

\begin{frame}\frametitle{There are embedded full factorials inside the fractional factorial}
	\begin{columns}[T]
		\column{0.55\textwidth}
			\includegraphics[width=\textwidth]{\imagedir/doe/projectivity-of-a-half-fraction-in-3-factors-MOOC.png}
		
		\column{0.35\textwidth}
			
			
			\vspace{4cm}
			If we choose our runs in a smart way, then fractional factorials will collapse to full factorials if an effect is insignificant.
	\end{columns}
	
\end{frame}

\begin{frame}\frametitle{If you'd like to do half the work, which 4 experiments would you pick?}
	\begin{columns}
		\column{0.65\textwidth}
			\begin{center}
				\includegraphics[width=.9\textwidth]{\imagedir/doe/half-fraction-in-3-factors-MOOC-optimum-4.png}
			\end{center}
			
		\column{0.55\textwidth}
			\begin{tabulary}{\linewidth}{|c||c|c|c|}\hline 
				\textsf{\relax Experiment } & \textbf{\relax A } & \textbf{\relax B } & \textbf{\relax C } \\
				\hline 1 & \(-\) & \(-\) & \(-\) \\
				\hline \color{myOrange} \textbf{2} & \(+\) & \(-\) & \(-\) \\
				\hline \color{myOrange} \textbf{3} & \(-\) & \(+\) & \(-\) \\
				\hline 4 & \(+\) & \(+\) & \(-\) \\
				\hline \color{myOrange} \textbf{5} & \(-\) & \(-\) & \(+\) \\
				\hline 6 & \(+\) & \(-\) & \(+\) \\
				\hline 7 & \(-\) & \(+\) & \(+\) \\
				\hline \color{myOrange} \textbf{8} & \(+\) & \(+\) & \(+\) \\
				\hline
			\end{tabulary}
			
			\vspace{1cm}
			{\color{myOrange} We will consider the 4 open circles next.}
	\end{columns}	

\end{frame}

\begin{frame}\frametitle{If you'd like to do half the work, which 4 experiments would you pick?}
	\begin{columns}
		\column{0.65\textwidth}
			\begin{center}
				\includegraphics[width=.9\textwidth]{\imagedir/doe/half-fraction-in-3-factors-MOOC-optimum-4.png}
			\end{center}
			
		\column{0.55\textwidth}
			\begin{tabulary}{\linewidth}{|c||c|c|c|}\hline 
				\textsf{\relax Experiment } & \textbf{\relax A } & \textbf{\relax B } & \textbf{\relax C } \\
				\hline \color{lightgray} 1 & \color{lightgray} \(-\) & \color{lightgray}\(-\) & \color{lightgray}\(-\) \\
				\hline \color{myOrange} \textbf{2} & \(+\) & \(-\) & \(-\) \\
				\hline \color{myOrange} \textbf{3} & \(-\) & \(+\) & \(-\) \\
				\hline \color{lightgray}4 & \color{lightgray}\(+\) & \color{lightgray}\(+\) & \color{lightgray}\(-\) \\
				\hline \color{myOrange} \textbf{5} & \(-\) & \(-\) & \(+\) \\
				\hline \color{lightgray}6 & \color{lightgray}\(+\) & \color{lightgray}\(-\) & \color{lightgray}\(+\) \\
				\hline \color{lightgray}7 & \color{lightgray}\(-\) & \color{lightgray}\(+\) & \color{lightgray}\(+\) \\
				\hline \color{myOrange} \textbf{8} & \(+\) & \(+\) & \(+\) \\
				\hline
			\end{tabulary}
			
			\vspace{1cm}
			{\color{myOrange} We will consider the 4 open circles next.}
	\end{columns}	

\end{frame}

\begin{frame}\frametitle{Model comparison between the full and fractional factorials}
	\begin{columns}[T]
		\column{0.35\textwidth}
			\begin{center}\textbf{Full factorial model}\end{center}
			
				%Coefficients:
				%  (Intercept)    C            T            S          C:T          C:S          T:S        C:T:S  
				%11.25         6.25         0.75        -7.25         0.25        -6.75        -0.25        -0.25  
				
				\begin{align*}
					\hat{y} &= {\color{myOrange}11.25}\\
							&+ {\color{myOrange}6.25\,}x_\text{A}\\
							&+ {\color{blue}0.75\,}x_\text{B}\\
							&- {\color{myOrange}7.25\,}x_\text{C}\\
							&+ {\color{myOrange}0.25\,}x_\text{A}x_\text{B}\\
							&- {\color{blue}6.75\,}x_\text{A}x_\text{C}\\
							&- {\color{myOrange}0.25\,}x_\text{B}x_\text{C}\\
							&- {\color{myOrange}0.25\,}x_\text{A}x_\text{B}x_\text{C}		
				\end{align*}
			
		\onslide+<2->	{
		\column{0.02\textwidth}
			\rule[3mm]{0.03cm}{65mm}
		\column{0.35\textwidth}
			\begin{center}\textbf{Fractional factorial model}\end{center}
				
				\begin{align*}
					\hat{y} &= {\color{myOrange}11.0}\\
							&+ {\color{myOrange}6.0\,}x_\text{A}\\
							&- {\color{blue}6.0\,}x_\text{B}\\
							&- {\color{myOrange}7.0\,}x_\text{C}\\
							& \color{lightgray}+ \cancel{b_\text{AB}\,x_\text{A}x_\text{B}}\\
							& \color{lightgray}+ \cancel{b_\text{AC}\,x_\text{A}x_\text{C}}\\
							& \color{lightgray}+ \cancel{b_\text{BC}\,x_\text{B}x_\text{C}}\\
							& \color{lightgray}+ \cancel{b_\text{ABC}\,x_\text{A}x_\text{B}x_\text{C}}\\
				\end{align*}
				}
	\end{columns}
\end{frame}

\begin{frame}\frametitle{The mathematics behind a fractional factorial}
	\begin{columns}
		\column{0.65\textwidth}
			\begin{center}
				\includegraphics[width=.9\textwidth]{\imagedir/doe/half-fraction-in-3-factors-MOOC-optimum-4.png}
			\end{center}
			
		\column{0.55\textwidth}
			{\fontsize{1cm}{2em}\selectfont  $2^3 = 8$}
	
			\pause
			\vspace{1cm}
			{\fontsize{1cm}{2em}\selectfont  $\dfrac{2^3}{2} = 4$}
	
			\pause
			\vspace{1cm}
			{\fontsize{1cm}{2em}\selectfont  $2^{3-1} = 2^2 = 4$}
	\end{columns}	

\end{frame}

\begin{frame}\frametitle{Setting up the half-fraction in 3 factors}
	\begin{columns}
		\column{0.65\textwidth}
			\begin{center}
				\includegraphics[width=.9\textwidth]{\imagedir/doe/half-fraction-in-3-factors-MOOC-no-labels.png}
			\end{center}
			
		\onslide+<2->	{
		\column{0.55\textwidth}
			\begin{tabulary}{\linewidth}{|c||c|c|}\hline 
				\textsf{\relax Experiment } & \textbf{\relax A } & \textbf{\relax B } \\ %& \textbf{\relax C } \\
				\hline \textbf{1} & \(-\) & \(-\) \\%& \(-\) \\
				\hline \textbf{2} & \(+\) & \(-\) \\%& \(-\) \\
				\hline \textbf{3} & \(-\) & \(+\) \\%& \(+\) \\
				\hline \textbf{4} & \(+\) & \(+\) \\%& \(+\) \\
				\hline
			\end{tabulary}
		}
	\end{columns}	
\end{frame}

\begin{frame}\frametitle{Setting up the half-fraction in 3 factors}
	\begin{columns}
		\column{0.65\textwidth}
			\begin{center}
				\includegraphics[width=.9\textwidth]{\imagedir/doe/half-fraction-in-3-factors-MOOC-no-labels.png}
			\end{center}
			
		\column{0.55\textwidth}
			\begin{tabulary}{\linewidth}{|c||c|c|c|}\hline 
				\textsf{\relax Experiment } & \textbf{\relax A } & \textbf{\relax B } & \textbf{\relax C = AB } \\
				\hline \textbf{1} & \(-\) & \(-\) & \((-)(-) = +\) \\
				\hline \textbf{2} & \(+\) & \(-\) & \((+)(-) = -\) \\
				\hline \textbf{3} & \(-\) & \(+\) & \((-)(+) = -\) \\
				\hline \textbf{4} & \(+\) & \(+\) & \((+)(+) = +\) \\
				\hline
			\end{tabulary}
			
			%\pause
		%\vspace{12pt}
			%Cost for this: \$40,000
	\end{columns}
\end{frame}

\begin{frame}\frametitle{\includegraphics[width=0.3\textwidth]{\imagedir/doe/examples/advice-logo.png} on when fractional-factorials are suitable}
	\textbf{Screening} is when you evaluate a new system
	\begin{itemize}
		\item	lab-scale exploration
		\item	making a new product
		\item	troubleshooting a problem to isolate major causes
	\end{itemize}
	
	\pause
	\vspace{1cm}
	\textbf{Optimization}: where you need that prediction accuracy
	\begin{itemize}
		\item	avoid optimizing prematurely
		\item	a less-fractionated design is used for optimization (more on this later)
	\end{itemize}
	
\end{frame}

\begin{frame}\frametitle{Quote from George Box}
	\begin{columns}[T]
		\column{0.45\textwidth}
			\includegraphics[width=0.7\textwidth]{\imagedir/doe/GeorgeEPBox-wikipedia.png}
		
			{\tiny \href{https://en.wikiquote.org/wiki/George\_E.\_P.\_Box}{Wikipedia}}
		
		\column{0.48\textwidth}
			``In an ongoing investigation, a rough rule is that only a portion (say 25\%) of the experimental effort and budget should be invested in the first design.''
	\end{columns}	
\end{frame}

\begin{frame}\frametitle{In the next class ...}
	We learn how to create half-fractions for any general system.
	
	\vspace{2cm}
	For example, how did we get $\mathbf{C = AB}$?
\end{frame}
\end{comment}

\begin{frame}\frametitle{}

	{\LARGE 
	
	\begin{tabular}{cccc}\hline 
	\textsf{\relax Number of } & \textsf{\relax Total  } & \textsf{\relax Cost of all} & \textsf{\relax Time to } \\
	\textsf{\relax factors} & \textsf{\relax experiments} & \textsf{\relax experiments} & \textsf{\relax run experiments}\\	\hline \hline
	\onslide+<1->{
	& & & \vspace{-.5cm} \\}
	\onslide+<2->{
	2 & 4 & \$300 & 1 day\\}
	\onslide+<3->{
	3 & 8 & \$600 & 2 days\\}
	\onslide+<4->{
	4 & 16 & \$1,200 & 4 days\\}
	\onslide+<5->{
	5 & 32 & \$2,400 & 8 days\\}
	\onslide+<6->{
	6 & 64 & \$4,800 & 16 days\\}
	\onslide+<7->{
	7 & 128 & \$9,600 & 32 days\\}
	\end{tabular}
	
	}
	\onslide+<2->{
	assuming 6 hours and \$150 per experiment}
	
\end{frame}

\begin{frame}\frametitle{There are $2^k$ model parameters in a full-factorial: not all are meaningful!}
	\vspace{6pt}
	For $k=4$ factors:
	\vspace{-6pt}
		{\scriptsize
		\begin{align*}
			\hat{y} &= {\color{myOrange}b_0}\\
					&+ {\color{myOrange}b_\text{A}\,}x_\text{A}\\
					&+ {\color{myOrange}b_\text{B}\,}x_\text{B}\\
					&+ {\color{myOrange}b_\text{C}\,}x_\text{C}\\
					&+ {\color{myOrange}b_\text{D}\,}x_\text{D}\\
					&+ {\color{myOrange}b_\text{AB}\,}x_\text{A}x_\text{B}\\
					&+ {\color{myOrange}b_\text{AC}\,}x_\text{A}x_\text{C}\\
					&+ {\color{myOrange}b_\text{BC}\,}x_\text{B}x_\text{C}\\
					&+ {\color{myOrange}b_\text{AD}\,}x_\text{A}x_\text{D}\\
					&+ {\color{myOrange}b_\text{BD}\,}x_\text{B}x_\text{D}\\
					&+ {\color{myOrange}b_\text{CD}\,}x_\text{C}x_\text{D}\\
					&+ {\color{myOrange}b_\text{ABC}\,}x_\text{A}x_\text{B}x_\text{C}\\
					&+ {\color{myOrange}b_\text{ABD}\,}x_\text{A}x_\text{B}x_\text{D}\\
					&+ {\color{myOrange}b_\text{ACD}\,}x_\text{A}x_\text{C}x_\text{D}\\
					&+ {\color{myOrange}b_\text{BCD}\,}x_\text{B}x_\text{C}x_\text{D}\\
					&+ {\color{myOrange}b_\text{ABCD}\,}x_\text{A}x_\text{B}x_\text{C}x_\text{D}
		\end{align*}
		}
	

\end{frame}

\begin{frame}\frametitle{Most real systems exhibit minor interactions; main effects usually dominate}
	\begin{columns}[T]
		\column{0.45\textwidth}
			\includegraphics[width=\textwidth]{\imagedir/doe/examples/chemical-conversion-pareto.png}
			
			{\tiny p. 200 in Box, Hunter and Hunter, 2$^\text{nd}$ ed}
			
		\column{0.45\textwidth}
			\includegraphics[width=\textwidth]{\imagedir/doe/examples/metal-removal-pareto.png}
			
			{\tiny Metal removal from wastewater; McMaster student project}
			
	\end{columns}
	
\end{frame}

\begin{frame}\frametitle{Core assumption regarding fractional factorials}
	\begin{columns}
		\column{0.65\textwidth}
			{\scriptsize
			\begin{align*}
				\hat{y} &= {\color{myOrange}b_0}\\
						&+ {\color{myOrange}b_\text{A}\,}x_\text{A}\\
						&+ {\color{myOrange}b_\text{B}\,}x_\text{B}\\
						&+ {\color{myOrange}b_\text{C}\,}x_\text{C}\\
						&+ {\color{myOrange}b_\text{D}\,}x_\text{D}\\
						&+ {\color{myOrange}b_\text{AB}\,}x_\text{A}x_\text{B}\\
						&+ {\color{myOrange}b_\text{AC}\,}x_\text{A}x_\text{C}\\
						&+ {\color{myOrange}b_\text{BC}\,}x_\text{B}x_\text{C}\\
						&+ {\color{myOrange}b_\text{AD}\,}x_\text{A}x_\text{D}\\
						&+ {\color{myOrange}b_\text{BD}\,}x_\text{B}x_\text{D}\\
						&+ {\color{myOrange}b_\text{CD}\,}x_\text{C}x_\text{D}\\
						&+ {\color{myOrange}b_\text{ABC}\,}x_\text{A}x_\text{B}x_\text{C}\\
						&+ {\color{myOrange}b_\text{ABD}\,}x_\text{A}x_\text{B}x_\text{D}\\
						&+ {\color{myOrange}b_\text{ACD}\,}x_\text{A}x_\text{C}x_\text{D}\\
						&+ {\color{myOrange}b_\text{BCD}\,}x_\text{B}x_\text{C}x_\text{D}\\
						&+ {\color{myOrange}b_\text{ABCD}\,}x_\text{A}x_\text{B}x_\text{C}x_\text{D}
			\end{align*}
			}
			
		\column{0.45\textwidth}
			\pause
			The main effects and some two factor interactions are often the only parameters of interest
			
			\vspace{36pt}
			The higher order interactions can safely be ignored
			\begin{itemize}
				\item	Now it is an assumption, but it's reasonable in many cases
				\item	The cost of obtaining them can be prohibitive
			\end{itemize}
			
	\end{columns}
\end{frame}

\begin{frame}\frametitle{If you'd like to do half the work, which 4 experiments would you pick?}
	\begin{columns}
		\column{0.65\textwidth}
			\begin{center}
				\includegraphics[width=.9\textwidth]{\imagedir/doe/half-fraction-in-3-factors-MOOC-all-8.png}
			\end{center}
			
		\column{0.45\textwidth}
			\begin{tabulary}{\linewidth}{|c||c|c|c|}\hline 
				\textsf{\relax Experiment } & \textbf{\relax A } & \textbf{\relax B } & \textbf{\relax C } \\
				\hline 1 & \(-\) & \(-\) & \(-\) \\
				\hline 2 & \(+\) & \(-\) & \(-\) \\
				\hline 3 & \(-\) & \(+\) & \(-\) \\
				\hline 4 & \(+\) & \(+\) & \(-\) \\
				\hline 5 & \(-\) & \(-\) & \(+\) \\
				\hline 6 & \(+\) & \(-\) & \(+\) \\
				\hline 7 & \(-\) & \(+\) & \(+\) \\
				\hline 8 & \(+\) & \(+\) & \(+\) \\
				\hline
			\end{tabulary}
	\end{columns}	
\end{frame}

\begin{frame}\frametitle{If you'd like to do half the work, which 4 experiments would you pick?}
	\begin{columns}
		\column{0.65\textwidth}
			\begin{center}
				\includegraphics[width=.9\textwidth]{\imagedir/doe/half-fraction-in-3-factors-MOOC-front-4.png}
			\end{center}
			
		\column{0.45\textwidth}
			\begin{tabulary}{\linewidth}{|c||c|c|c|}\hline 
				\textsf{\relax Experiment } & \textbf{\relax A } & \textbf{\relax B } & \textbf{\relax C } \\
				\hline \color{myOrange} \textbf{1} & \(-\) & \(-\) & \(-\) \\
				\hline \color{myOrange} \textbf{2} & \(+\) & \(-\) & \(-\) \\
				\hline \color{myOrange} \textbf{3} & \(-\) & \(+\) & \(-\) \\
				\hline \color{myOrange} \textbf{4} & \(+\) & \(+\) & \(-\) \\
				\hline 5 & \(-\) & \(-\) & \(+\) \\
				\hline 6 & \(+\) & \(-\) & \(+\) \\
				\hline 7 & \(-\) & \(+\) & \(+\) \\
				\hline 8 & \(+\) & \(+\) & \(+\) \\
				\hline
			\end{tabulary}
	\end{columns}	
\end{frame}

\begin{frame}\frametitle{If you'd like to do half the work, which 4 experiments would you pick?}
	\begin{columns}
		\column{0.65\textwidth}
			\begin{center}
				\includegraphics[width=.9\textwidth]{\imagedir/doe/half-fraction-in-3-factors-MOOC-middle-4.png}
			\end{center}
			
		\column{0.45\textwidth}
			\begin{tabulary}{\linewidth}{|c||c|c|c|}\hline 
				\textsf{\relax Experiment } & \textbf{\relax A } & \textbf{\relax B } & \textbf{\relax C } \\
				\hline 1 & \(-\) & \(-\) & \(-\) \\
				\hline 2 & \(+\) & \(-\) & \(-\) \\
				\hline \color{myOrange} \textbf{3} & \(-\) & \(+\) & \(-\) \\
				\hline \color{myOrange} \textbf{4} & \(+\) & \(+\) & \(-\) \\
				\hline \color{myOrange} \textbf{5} & \(-\) & \(-\) & \(+\) \\
				\hline \color{myOrange} \textbf{6} & \(+\) & \(-\) & \(+\) \\
				\hline 7 & \(-\) & \(+\) & \(+\) \\
				\hline 8 & \(+\) & \(+\) & \(+\) \\
				\hline
			\end{tabulary}
	\end{columns}	
\end{frame}

\begin{frame}\frametitle{If you'd like to do half the work, which 4 experiments would you pick?}
	\begin{columns}
		\column{0.65\textwidth}
			\begin{center}
				\includegraphics[width=.9\textwidth]{\imagedir/doe/half-fraction-in-3-factors-MOOC-optimum-4.png}
			\end{center}
			
		\column{0.45\textwidth}
			\begin{tabulary}{\linewidth}{|c||c|c|c|}\hline 
				\textsf{\relax Experiment } & \textbf{\relax A } & \textbf{\relax B } & \textbf{\relax C } \\
				\hline 1 & \(-\) & \(-\) & \(-\) \\
				\hline \color{myOrange} \textbf{2} & \(+\) & \(-\) & \(-\) \\
				\hline \color{myOrange} \textbf{3} & \(-\) & \(+\) & \(-\) \\
				\hline 4 & \(+\) & \(+\) & \(-\) \\
				\hline \color{myOrange} \textbf{5} & \(-\) & \(-\) & \(+\) \\
				\hline 6 & \(+\) & \(-\) & \(+\) \\
				\hline 7 & \(-\) & \(+\) & \(+\) \\
				\hline \color{myOrange} \textbf{8} & \(+\) & \(+\) & \(+\) \\
				\hline
			\end{tabulary}
	\end{columns}	
\end{frame}

\begin{frame}\frametitle{If you'd like to do half the work, which 4 experiments would you pick?}
	\begin{columns}
		\column{0.65\textwidth}
			\begin{center}
				\includegraphics[width=.9\textwidth]{\imagedir/doe/half-fraction-in-3-factors-MOOC-optimum-4.png}
			\end{center}
			
		\column{0.45\textwidth}
			\begin{tabulary}{\linewidth}{|c||c|c|c|}\hline 
				\textsf{\relax Experiment } & \textbf{\relax A } & \textbf{\relax B } & \textbf{\relax C } \\
				\hline \color{blue} \textbf{1} & \(-\) & \(-\) & \(-\) \\
				\hline 2 & \(+\) & \(-\) & \(-\) \\
				\hline 3 & \(-\) & \(+\) & \(-\) \\
				\hline \color{blue} \textbf{4} & \(+\) & \(+\) & \(-\) \\
				\hline 5 & \(-\) & \(-\) & \(+\) \\
				\hline \color{blue} \textbf{6} & \(+\) & \(-\) & \(+\) \\
				\hline \color{blue} \textbf{7} & \(-\) & \(+\) & \(+\) \\
				\hline 8 & \(+\) & \(+\) & \(+\) \\
				\hline
			\end{tabulary}
	\end{columns}	
\end{frame}

\begin{frame}\frametitle{If you'd like to do half the work, which 4 experiments would you pick?}
	\begin{columns}
		\column{0.65\textwidth}
			\begin{center}
				\includegraphics[width=.9\textwidth]{\imagedir/doe/half-fraction-in-3-factors-MOOC-collapse-6-in.png}
			\end{center}
			
		\column{0.45\textwidth}
			\begin{tabulary}{\linewidth}{|c||c|c|c|}\hline 
				\textsf{\relax Experiment } & \textbf{\relax A } & \textbf{\relax B } & \textbf{\relax C } \\
				\hline 1 & \(-\) & \(-\) & \(-\) \\
				\hline \color{myOrange} \textbf{2} & \(+\) & \(-\) & \(-\) \\
				\hline \color{myOrange} \textbf{3} & \(-\) & \(+\) & \(-\) \\
				\hline 4 & \(+\) & \(+\) & \(-\) \\
				\hline \color{myOrange} \textbf{5} & \(-\) & \(-\) & \(+\) \\
				\hline 6 & \(+\) & \(-\) & \(+\) \\
				\hline 7 & \(-\) & \(+\) & \(+\) \\
				\hline \color{myOrange} \textbf{8} & \(+\) & \(+\) & \(+\) \\
				\hline
			\end{tabulary}
	\end{columns}	
\end{frame}

\begin{frame}\frametitle{If you'd like to do half the work, which 4 experiments would you pick?}
	\begin{columns}
		\column{0.65\textwidth}
			\begin{center}
				\includegraphics[width=.9\textwidth]{\imagedir/doe/half-fraction-in-3-factors-MOOC-collapse-total.png}
			\end{center}
			
		\column{0.45\textwidth}
			\begin{tabulary}{\linewidth}{|c||c|c|c|}\hline 
				\textsf{\relax Experiment } & \textbf{\relax A } & \textbf{\relax B } & \textbf{\relax C } \\
				\hline 1 & \(-\) & \(-\) & \(-\) \\
				\hline \color{myOrange} \textbf{2} & \(+\) & \(-\) & \(-\) \\
				\hline \color{myOrange} \textbf{3} & \(-\) & \(+\) & \(-\) \\
				\hline 4 & \(+\) & \(+\) & \(-\) \\
				\hline \color{myOrange} \textbf{5} & \(-\) & \(-\) & \(+\) \\
				\hline 6 & \(+\) & \(-\) & \(+\) \\
				\hline 7 & \(-\) & \(+\) & \(+\) \\
				\hline \color{myOrange} \textbf{8} & \(+\) & \(+\) & \(+\) \\
				\hline
			\end{tabulary}
	\end{columns}	
\end{frame}

\begin{frame}\frametitle{If you'd like to do half the work, which 4 experiments would you pick?}
	\begin{columns}
		\column{0.65\textwidth}
			\begin{center}
				\includegraphics[width=.9\textwidth]{\imagedir/doe/half-fraction-in-3-factors-MOOC-collapse-total-rotate.png}
			\end{center}
			
		\column{0.45\textwidth}
			\begin{tabulary}{\linewidth}{|c||c|c|c|}\hline 
				\textsf{\relax Experiment } & \textbf{\relax A } & \textbf{\relax B } & \textbf{\relax C } \\
				\hline 1 & \(-\) & \(-\) & \(-\) \\
				\hline \color{myOrange} \textbf{2} & \(+\) & \(-\) & \(-\) \\
				\hline \color{myOrange} \textbf{3} & \(-\) & \(+\) & \(-\) \\
				\hline 4 & \(+\) & \(+\) & \(-\) \\
				\hline \color{myOrange} \textbf{5} & \(-\) & \(-\) & \(+\) \\
				\hline 6 & \(+\) & \(-\) & \(+\) \\
				\hline 7 & \(-\) & \(+\) & \(+\) \\
				\hline \color{myOrange} \textbf{8} & \(+\) & \(+\) & \(+\) \\
				\hline
			\end{tabulary}
	\end{columns}	
\end{frame}

\begin{frame}\frametitle{If you'd like to do half the work, which 4 experiments would you pick?}
	\begin{columns}
		\column{0.65\textwidth}
			\begin{center}
				\includegraphics[width=.9\textwidth]{\imagedir/doe/half-fraction-in-3-factors-MOOC-collapse-total-rotate-labelled.png}
			\end{center}
			
		\column{0.45\textwidth}
			\begin{tabulary}{\linewidth}{|c||c|c|c|}\hline 
				\textsf{\relax Experiment } & \textbf{\relax A } & \textbf{\relax B } & \textbf{\relax C } \\
				\hline 1 & \(-\) & \(-\) & \(-\) \\
				\hline \color{myOrange} \textbf{2} & \(+\) & \(-\) & \(-\) \\
				\hline \color{myOrange} \textbf{3} & \(-\) & \(+\) & \(-\) \\
				\hline 4 & \(+\) & \(+\) & \(-\) \\
				\hline \color{myOrange} \textbf{5} & \(-\) & \(-\) & \(+\) \\
				\hline 6 & \(+\) & \(-\) & \(+\) \\
				\hline 7 & \(-\) & \(+\) & \(+\) \\
				\hline \color{myOrange} \textbf{8} & \(+\) & \(+\) & \(+\) \\
				\hline
			\end{tabulary}
	\end{columns}	
\end{frame}

\begin{frame}\frametitle{There are embedded full factorials inside the fractional factorial}
	\begin{columns}[T]
		\column{0.55\textwidth}
			\includegraphics[width=\textwidth]{\imagedir/doe/projectivity-of-a-half-fraction-in-3-factors-MOOC.png}
		
		\column{0.35\textwidth}
			
			
			\vspace{4cm}
			If we choose our runs in a smart way, then fractional factorials will collapse to full factorials if an effect is insignificant.
	\end{columns}
	
\end{frame}

\begin{frame}\frametitle{If you'd like to do half the work, which 4 experiments would you pick?}
	\begin{columns}
		\column{0.65\textwidth}
			\begin{center}
				\includegraphics[width=.9\textwidth]{\imagedir/doe/half-fraction-in-3-factors-MOOC-optimum-4.png}
			\end{center}
			
		\column{0.55\textwidth}
			\begin{tabulary}{\linewidth}{|c||c|c|c|}\hline 
				\textsf{\relax Experiment } & \textbf{\relax A } & \textbf{\relax B } & \textbf{\relax C } \\
				\hline \color{lightgray} 1 & \color{lightgray} \(-\) & \color{lightgray}\(-\) & \color{lightgray}\(-\) \\
				\hline \color{myOrange} \textbf{2} & \(+\) & \(-\) & \(-\) \\
				\hline \color{myOrange} \textbf{3} & \(-\) & \(+\) & \(-\) \\
				\hline \color{lightgray}4 & \color{lightgray}\(+\) & \color{lightgray}\(+\) & \color{lightgray}\(-\) \\
				\hline \color{myOrange} \textbf{5} & \(-\) & \(-\) & \(+\) \\
				\hline \color{lightgray}6 & \color{lightgray}\(+\) & \color{lightgray}\(-\) & \color{lightgray}\(+\) \\
				\hline \color{lightgray}7 & \color{lightgray}\(-\) & \color{lightgray}\(+\) & \color{lightgray}\(+\) \\
				\hline \color{myOrange} \textbf{8} & \(+\) & \(+\) & \(+\) \\
				\hline
			\end{tabulary}
			
			\vspace{1cm}
			{\color{myOrange} Only consider the 4 open circles.}
	\end{columns}	

\end{frame}

\begin{frame}\frametitle{Model comparison between the full and fractional factorials}
	\begin{columns}[T]
		\column{0.35\textwidth}
			\begin{center}\textbf{Full factorial model}\end{center}
			
				%Coefficients:
				%  (Intercept)    C            T            S          C:T          C:S          T:S        C:T:S  
				%11.25         6.25         0.75        -7.25         0.25        -6.75        -0.25        -0.25  
				
				\begin{align*}
					\hat{y} &= {\color{myOrange}11.25}\\
							&+ {\color{myOrange}6.25\,}x_\text{A}\\
							&+ {\color{blue}0.75\,}x_\text{B}\\
							&- {\color{myOrange}7.25\,}x_\text{C}\\
							&+ {\color{myOrange}0.25\,}x_\text{A}x_\text{B}\\
							&- {\color{blue}6.75\,}x_\text{A}x_\text{C}\\
							&- {\color{myOrange}0.25\,}x_\text{B}x_\text{C}\\
							&- {\color{myOrange}0.25\,}x_\text{A}x_\text{B}x_\text{C}		
				\end{align*}
			
		\onslide+<2->	{
		\column{0.02\textwidth}
			\rule[3mm]{0.03cm}{65mm}
		\column{0.35\textwidth}
			\begin{center}\textbf{Fractional factorial model}\end{center}
				
				\begin{align*}
					\hat{y} &= {\color{myOrange}11.0}\\
							&+ {\color{myOrange}6.0\,}x_\text{A}\\
							&- {\color{blue}6.0\,}x_\text{B}\\
							&- {\color{myOrange}7.0\,}x_\text{C}\\
							& \color{lightgray}+ \cancel{b_\text{AB}\,x_\text{A}x_\text{B}}\\
							& \color{lightgray}+ \cancel{b_\text{AC}\,x_\text{A}x_\text{C}}\\
							& \color{lightgray}+ \cancel{b_\text{BC}\,x_\text{B}x_\text{C}}\\
							& \color{lightgray}+ \cancel{b_\text{ABC}\,x_\text{A}x_\text{B}x_\text{C}}\\
				\end{align*}
				}
	\end{columns}
\end{frame}

\begin{frame}\frametitle{The mathematics behind a fractional factorial}
	\begin{columns}
		\column{0.65\textwidth}
			\begin{center}
				\includegraphics[width=.9\textwidth]{\imagedir/doe/half-fraction-in-3-factors-MOOC-optimum-4.png}
			\end{center}
			
		\column{0.55\textwidth}
			{\fontsize{1cm}{2em}\selectfont  $2^3 = 8$}
	
			\pause
			\vspace{1cm}
			{\fontsize{1cm}{2em}\selectfont  $\dfrac{2^3}{2} = 4$}
	
			\pause
			\vspace{1cm}
			{\fontsize{1cm}{2em}\selectfont  $2^{3-1} = 2^2 = 4$}
	\end{columns}	

\end{frame}

\begin{frame}\frametitle{Setting up the half-fraction in 3 factors}
	\begin{columns}
		\column{0.65\textwidth}
			\begin{center}
				\includegraphics[width=.9\textwidth]{\imagedir/doe/half-fraction-in-3-factors-MOOC-no-labels.png}
			\end{center}
			
		\onslide+<2->	{
		\column{0.55\textwidth}
			\begin{tabulary}{\linewidth}{|c||c|c|}\hline 
				\textsf{\relax Experiment } & \textbf{\relax A } & \textbf{\relax B } \\ %& \textbf{\relax C } \\
				\hline \textbf{1} & \(-\) & \(-\) \\%& \(-\) \\
				\hline \textbf{2} & \(+\) & \(-\) \\%& \(-\) \\
				\hline \textbf{3} & \(-\) & \(+\) \\%& \(+\) \\
				\hline \textbf{4} & \(+\) & \(+\) \\%& \(+\) \\
				\hline
			\end{tabulary}
		}
	\end{columns}	
\end{frame}

\begin{frame}\frametitle{Setting up the half-fraction in 3 factors}
	\begin{columns}
		\column{0.65\textwidth}
			\begin{center}
				\includegraphics[width=.9\textwidth]{\imagedir/doe/half-fraction-in-3-factors-MOOC-no-labels.png}
			\end{center}
			
		\column{0.55\textwidth}
			\begin{tabulary}{\linewidth}{|c||c|c|c|}\hline 
				\textsf{\relax Experiment } & \textbf{\relax A } & \textbf{\relax B } & \textbf{\relax C = AB } \\
				\hline \textbf{1} & \(-\) & \(-\) & \((-)(-) = +\) \\
				\hline \textbf{2} & \(+\) & \(-\) & \((+)(-) = -\) \\
				\hline \textbf{3} & \(-\) & \(+\) & \((-)(+) = -\) \\
				\hline \textbf{4} & \(+\) & \(+\) & \((+)(+) = +\) \\
				\hline
			\end{tabulary}
			
			%\pause
		%\vspace{12pt}
			%Cost for this: \$40,000
	\end{columns}
\end{frame}

\begin{frame}\frametitle{\includegraphics[width=0.3\textwidth]{\imagedir/doe/examples/advice-logo.png} on when fractional-factorials are suitable}
	\textbf{Screening} is when you evaluate a new system
	\begin{itemize}
		\item	lab-scale exploration
		\item	making a new product
		\item	troubleshooting a problem to isolate major causes
	\end{itemize}
	
	\pause
	\vspace{1cm}
	\textbf{Optimization}: where you need that prediction accuracy
	\begin{itemize}
		\item	avoid optimizing prematurely
		\item	a less-fractionated design is used for optimization (more on this later)
	\end{itemize}
	
\end{frame}

\begin{frame}\frametitle{Quote from George Box}
	\begin{columns}[T]
		\column{0.45\textwidth}
			\includegraphics[width=0.7\textwidth]{\imagedir/doe/GeorgeEPBox-wikipedia.png}
		
			{\tiny \href{https://en.wikiquote.org/wiki/George\_E.\_P.\_Box}{Wikipedia}}
		
		\column{0.48\textwidth}
			``In an ongoing investigation, a rough rule is that only a portion (say 25\%) of the experimental effort and budget should be invested in the first design.''
	\end{columns}	
\end{frame}

\begin{frame}\frametitle{In the next class ...}
	We learn how to create half-fractions for any general system.
	
	\vspace{2cm}
	For example, how did we get $\mathbf{C = AB}$?
\end{frame}



% /Users/kevindunn/Dropbox/Coursera/Media/All-course-slides/classes/CourseraMOOC-class-4B.tex

% \begin{comment}
%
%
% 	\begin{frame}\frametitle{\includegraphics[width=0.3\textwidth]{\imagedir/doe/examples/advice-logo.png}\,\, plan your experiments carefully ahead of time}
% 	\end{frame}
%
% 	\begin{columns}[T]
% 		\column{0.45\textwidth}
% 			\includegraphics[width=0.7\textwidth]{\imagedir/statistics/flicfcb_o.jpg}
%
% 			{\scriptsize (p. 230 in Box, Hunter and Hunter, 2$^\text{nd}$ ed)}
%
% 		\column{0.48\textwidth}
% 			\includegraphics[width=\textwidth]{\imagedir/doe/examples/solar-panel-mendelu-cz-website.png}
%
%
% 			\see{\href{http://yint.org/solar-panel-study}{http://yint.org/solar-panel-study}}
% 	\end{columns}
%
% 	\begin{center}\rule[8mm]{4cm}{0.01cm}\end{center}
% 	\rule[3mm]{0.01cm}{25mm}%
%
% \end{comment}

\begin{frame}\frametitle{}
	\begin{center}
	\includegraphics[height=.9\textheight]{\imagedir/doe/DOE-trade-off-table-no-annotation.png}
	\end{center}
\end{frame}

\begin{frame}\frametitle{}
	\begin{center}
	\includegraphics[height=.9\textheight]{\imagedir/doe/DOE-trade-off-table.png}
	\end{center}
\end{frame}

\begin{frame}\frametitle{}
	\begin{center}
	\includegraphics[height=.9\textheight]{\imagedir/doe/DOE-trade-off-table-cell1.png}
	\end{center}
\end{frame}

\begin{frame}\frametitle{For your own case: be prepared to remap your letters ... temporarily}
	\begin{columns}[b]
		\column{0.45\textwidth}
			\emph{Original factor names}
			
			\begin{itemize}
				\item	\textbf{C}: chemical variable
				\item	\textbf{T}: temperature variable
				\item	\textbf{S}: stirring speed variable
			\end{itemize}
			
			\vspace{1cm}
			\textbf{$+$S = CT}
			
		\onslide+<3->{
			\vspace{1cm}
			\textbf{$-$S = CT}
		}
			
		\column{0.45\textwidth}
		\onslide+<2->{
			\emph{Factor names in the table}
			
			\begin{itemize}
				\item	\textbf{A}
				\item	\textbf{B}
				\item	\textbf{C}
			\end{itemize}
			
			\vspace{1cm}
			\textbf{$+$C = AB}
		}
		
		\onslide+<3->{	
			\vspace{1cm}
			\textbf{$-$C = AB}
		}
	\end{columns}	
\end{frame}

\begin{frame}\frametitle{Setting up the half-fraction in 3 factors: using the complementary half}
	\begin{columns}
		\column{0.6\textwidth}
			\begin{center}
				\includegraphics[width=.9\textwidth]{\imagedir/doe/half-fraction-in-3-factors-MOOC-no-labels.png}
			\end{center}
			
		\column{0.55\textwidth}
			\begin{tabulary}{\linewidth}{|c||c|c|c|}\hline 
				\textsf{\relax Experiment } & \textbf{\relax A } & \textbf{\relax B } & \textbf{\relax C = $-$AB } \\
				\hline \textbf{1} & \(-\) & \(-\) & \(-(-)(-) = -\) \\
				\hline \textbf{2} & \(+\) & \(-\) & \(-(+)(-) = +\) \\
				\hline \textbf{3} & \(-\) & \(+\) & \(-(-)(+) = +\) \\
				\hline \textbf{4} & \(+\) & \(+\) & \(-(+)(+) = -\) \\
				\hline
			\end{tabulary}
			
			\small
			\vspace{1cm}
			\textbf{\relax $-$C = AB} can be rewritten as \textbf{\relax C = $-$AB}
			
			\vspace{1cm}
			Notice how the 4 runs generated with \textbf{\relax C = $-$AB} correspond to the closed circles.

			\pause
			\vspace{1cm}
			The 4 runs generated with \textbf{\relax C = $+$AB} correspond\\
			to the open circles.
			
	\end{columns}
\end{frame}

\begin{frame}\frametitle{{\large Creating and understanding fractional factorials from a half-fraction example}}
	\begin{columns}
		\column{0.65\textwidth}
			\begin{center}
				\includegraphics[width=.9\textwidth]{\imagedir/doe/half-fraction-in-3-factors-MOOC-optimum-4.png}
			\end{center}
			
		\onslide+<2->	{
		\column{0.55\textwidth}
			\begin{tabulary}{\linewidth}{cccl}\hline 
				\textsf{\relax Experiment } & \textbf{\relax A } & \textbf{\relax B } & \textbf{\relax C } \\
				\hline 
				1 & \(-\) & \(-\) & \(-\) \\
				\color{myOrange} \textbf{2} & \(+\) & \(-\) & \(-\) \\
				\color{myOrange} \textbf{3} & \(-\) & \(+\) & \(-\) \\
				4 & \(+\) & \(+\) & \(-\) \\
				\color{myOrange} \textbf{5} & \(-\) & \(-\) & \(+\) \\
				6 & \(+\) & \(-\) & \(+\) \\
				7 & \(-\) & \(+\) & \(+\) \\
				\color{myOrange} \textbf{8} & \(+\) & \(+\) & \(+\) \\  \hline
			\end{tabulary}
		}
			
	\end{columns}	
\end{frame}

\begin{frame}\frametitle{{\large Creating and understanding fractional factorials from a half-fraction example}}
	\begin{columns}
		\column{0.65\textwidth}
			\begin{center}
				\includegraphics[width=.9\textwidth]{\imagedir/doe/half-fraction-in-3-factors-MOOC-optimum-4.png}
			\end{center}

		\column{0.55\textwidth}
			\begin{tabulary}{\linewidth}{cccl}\hline 
				\textsf{\relax Experiment } & \textbf{\relax A } & \textbf{\relax B } & \textbf{\relax C } \\
				\hline 
				 & & & \\
				\color{myOrange} \textbf{2} & \(+\) & \(-\) & \(-\) \\
				\color{myOrange} \textbf{3} & \(-\) & \(+\) & \(-\) \\
				 & & & \\
				\color{myOrange} \textbf{5} & \(-\) & \(-\) & \(+\) \\
				 & & & \\
				 & & & \\
				\color{myOrange} \textbf{8} & \(+\) & \(+\) & \(+\) \\ \hline
			\end{tabulary}
	\end{columns}	
\end{frame}

\begin{frame}\frametitle{{\large Creating and understanding fractional factorials from a half-fraction example}}
	\begin{columns}
		\column{0.65\textwidth}
			\begin{center}
				\includegraphics[width=.9\textwidth]{\imagedir/doe/half-fraction-in-3-factors-MOOC-optimum-4.png}
			\end{center}

		\column{0.55\textwidth}
			\begin{tabulary}{\linewidth}{cccl}\hline 
				\textsf{\relax Experiment } & \textbf{\relax A } & \textbf{\relax B } & \textbf{\relax C } \\
				\hline 
				\color{myOrange} \textbf{5} & \(-\) & \(-\) & \(+\) \\
				\color{myOrange} \textbf{2} & \(+\) & \(-\) & \(-\) \\
				\color{myOrange} \textbf{3} & \(-\) & \(+\) & \(-\) \\
				 & & & \\
				 & & & \\
				 & & & \\
				 & & & \\
				\color{myOrange} \textbf{8} & \(+\) & \(+\) & \(+\) \\  \hline
			\end{tabulary}
	\end{columns}	
\end{frame}

\begin{frame}\frametitle{{\large Creating and understanding fractional factorials from a half-fraction example}}
	\begin{columns}
		\column{0.65\textwidth}
			\begin{center}
				\includegraphics[width=.9\textwidth]{\imagedir/doe/half-fraction-in-3-factors-MOOC-optimum-4.png}
			\end{center}

		\column{0.55\textwidth}
			\begin{tabulary}{\linewidth}{cccl}\hline 
				\textsf{\relax Experiment } & \textbf{\relax A } & \textbf{\relax B } & \textbf{\relax C } \\
				\hline 
				 \color{myOrange} \textbf{5} & \(-\) & \(-\) & \(+\) \\
				 \color{myOrange} \textbf{2} & \(+\) & \(-\) & \(-\) \\
				 \color{myOrange} \textbf{3} & \(-\) & \(+\) & \(-\) \\
				 \color{myOrange} \textbf{8} & \(+\) & \(+\) & \(+\) \\ \hline
				  & & & \\
				  & & & \\
				  & & & \\
				  & & & 
			\end{tabulary}
	\end{columns}	
\end{frame}

\begin{frame}\frametitle{{\large Creating and understanding fractional factorials from a half-fraction example}}
	\begin{columns}
		\column{0.65\textwidth}
			\begin{center}
				\includegraphics[width=.9\textwidth]{\imagedir/doe/half-fraction-in-3-factors-MOOC-optimum-4.png}
			\end{center}

		\column{0.55\textwidth}
			\begin{tabulary}{\linewidth}{cccl}\hline 
				\textsf{\relax Experiment } & \textbf{\relax A } & \textbf{\relax B } & \textbf{\relax C = AB } \\
				\hline 
				 \color{myOrange} \textbf{5} & \(-\) & \(-\) & \(+\) \\
				 \color{myOrange} \textbf{2} & \(+\) & \(-\) & \(-\) \\
				 \color{myOrange} \textbf{3} & \(-\) & \(+\) & \(-\) \\
				 \color{myOrange} \textbf{8} & \(+\) & \(+\) & \(+\) \\ \hline
				  & & & \\
				  & & & \\
				  & & & \\
				  & & & 
			\end{tabulary}
	\end{columns}	
\end{frame}

\begin{frame}\frametitle{Our predictive model \color[rgb]{0,0,0}{\scriptsize  $ \hat{y} = $ {\color[rgb]{0.54,0.12,0.03}$b_\text{0}$}  
	+ {\color[rgb]{0.54,0.12,0.03}$b_\text{C}$}$x_\text{C}$ 
	+ {\color[rgb]{0.54,0.12,0.03}$b_\text{T}$}$x_\text{T}$ 
	+ {\color[rgb]{0.54,0.12,0.03}$b_\text{S}$}$x_\text{S}$ 
	+ {\color[rgb]{0.54,0.12,0.03}$b_\text{CT}$}$x_\text{C}x_\text{T}$ 
	+ {\color[rgb]{0.54,0.12,0.03}$b_\text{CS}$}$x_\text{C}x_\text{S}$ 
	+ {\color[rgb]{0.54,0.12,0.03}$b_\text{TS}$}$x_\text{T}x_\text{S}$ 
	+ {\color[rgb]{0.54,0.12,0.03}$b_\text{CTS}$}$x_\text{C}x_\text{T}x_\text{S}$}}  
	
	% The CTS one	
	
	\newcommand{\mw}{\color[rgb]{1,1,1}}
	\newcommand{\mm}{\color{lightgray}}
		
	\vspace{-0.8cm}
	 {\LARGE
	\begin{flalign*}
				&{\mw =}\normalsize  \qquad\,\,\begin{matrix} \mm \,b_\text{0} & \mm \quad b_\text{C} & \mm \quad b_\text{T} & \mm \quad b_\text{S}
				& \mm \,\,\, b_\text{CT} & \mm \,\,\, b_\text{CS} & \mm \,\,\,b_\text{TS} & \mm \,\,b_\text{CTS} 
			\end{matrix}
				\\
		\begin{pmatrix}y_1\\y_2\\y_3\\y_4\\y_5\\y_6\\y_7\\y_8\end{pmatrix} &= 
		\left(\begin{matrix}
			+1 & -1 & -1 & -1 & +1 & +1 & +1 & -1 \\ 
			+1 & +1 & -1 & -1 & -1 & -1 & +1 & +1 \\ 
			+1 & -1 & +1 & -1 & -1 & +1 & -1 & +1 \\
			+1 & +1 & +1 & -1 & +1 & -1 & -1 & -1 \\
			+1 & -1 & -1 & +1 & +1 & -1 & -1 & +1 \\ 
			+1 & +1 & -1 & +1 & -1 & +1 & -1 & -1 \\ 
			+1 & -1 & +1 & +1 & -1 & -1 & +1 & -1 \\
			+1 & +1 & +1 & +1 & +1 & +1 & +1 & +1 \\
		\end{matrix}\right)	
		\begin{pmatrix}{\color[rgb]{0.54,0.12,0.03}b_0}\\
		{\color[rgb]{0.54,0.12,0.03}b_\text{C} }\\
		{\color[rgb]{0.54,0.12,0.03}b_\text{T}} \\
		{\color[rgb]{0.54,0.12,0.03}b_\text{S}} \\
		{\color[rgb]{0.54,0.12,0.03}b_\text{CT}} \\
		{\color[rgb]{0.54,0.12,0.03}b_\text{CS}} \\
		{\color[rgb]{0.54,0.12,0.03}b_\text{TS}} \\
		{\color[rgb]{0.54,0.12,0.03}b_\text{CTS}}\\
		 \end{pmatrix}	 \\
		 \mathbf{y} &= \mathbf{X}{\color[rgb]{0.54,0.12,0.03}\mathbf{b}}	 		
	\end{flalign*}
	}
\end{frame}

\begin{frame}\frametitle{Our predictive model \color[rgb]{0,0,0}{\scriptsize  $ \hat{y} = $ {\color[rgb]{0.54,0.12,0.03}$b_\text{0}$}  
	+ {\color[rgb]{0.54,0.12,0.03}$b_\text{A}$}$x_\text{A}$ 
	+ {\color[rgb]{0.54,0.12,0.03}$b_\text{B}$}$x_\text{B}$ 
	+ {\color[rgb]{0.54,0.12,0.03}$b_\text{C}$}$x_\text{C}$ 
	+ {\color[rgb]{0.54,0.12,0.03}$b_\text{AB}$}$x_\text{A}x_\text{B}$ 
	+ {\color[rgb]{0.54,0.12,0.03}$b_\text{AC}$}$x_\text{A}x_\text{C}$ 
	+ {\color[rgb]{0.54,0.12,0.03}$b_\text{BC}$}$x_\text{B}x_\text{C}$ 
	+ {\color[rgb]{0.54,0.12,0.03}$b_\text{ABC}$}$x_\text{A}x_\text{B}x_\text{C}$}}   % The ABC, with all the "-1" and "+1"
	
	
	\newcommand{\mw}{\color[rgb]{1,1,1}}
	\newcommand{\mm}{\color{lightgray}}
		
	\vspace{-0.8cm}
	{\LARGE
	\begin{flalign*}
			&{\mw =}\normalsize  \qquad\,\,\begin{matrix} \mm \,b_\text{0} & \mm \quad b_\text{A} & \mm \quad b_\text{B} & \mm \quad b_\text{C}
			& \mm \,\,\, b_\text{AB} & \mm \,\,\, b_\text{AC} & \mm \,\,\,b_\text{BC} & \mm \,\,b_\text{ABC} 
		\end{matrix}
			\\ 
		\begin{pmatrix}y_1\\y_2\\y_3\\y_4\\y_5\\y_6\\y_7\\y_8\end{pmatrix} &= 
		\left(\begin{matrix}
			+1 & -1 & -1 & -1 & +1 & +1 & +1 & -1 \\ 
			+1 & +1 & -1 & -1 & -1 & -1 & +1 & +1 \\ 
			+1 & -1 & +1 & -1 & -1 & +1 & -1 & +1 \\
			+1 & +1 & +1 & -1 & +1 & -1 & -1 & -1 \\
			+1 & -1 & -1 & +1 & +1 & -1 & -1 & +1 \\ 
			+1 & +1 & -1 & +1 & -1 & +1 & -1 & -1 \\ 
			+1 & -1 & +1 & +1 & -1 & -1 & +1 & -1 \\
			+1 & +1 & +1 & +1 & +1 & +1 & +1 & +1 \\
		 \end{matrix}\right)	
		\begin{pmatrix}{\color[rgb]{0.54,0.12,0.03}b_0}\\
		{\color[rgb]{0.54,0.12,0.03}b_\text{A} }\\
		{\color[rgb]{0.54,0.12,0.03}b_\text{B}} \\
		{\color[rgb]{0.54,0.12,0.03}b_\text{C}} \\
		{\color[rgb]{0.54,0.12,0.03}b_\text{AB}} \\
		{\color[rgb]{0.54,0.12,0.03}b_\text{AC}} \\
		{\color[rgb]{0.54,0.12,0.03}b_\text{BC}} \\
		{\color[rgb]{0.54,0.12,0.03}b_\text{ABC}}\\
		 \end{pmatrix}\\
			 \mathbf{y} &= \mathbf{X}{\color[rgb]{0.54,0.12,0.03}\mathbf{b}}	 		
	\end{flalign*}
	}	
\end{frame}

\begin{frame}\frametitle{Our predictive model \color[rgb]{0,0,0}{\scriptsize  $ \hat{y} = $ {\color[rgb]{0.54,0.12,0.03}$b_\text{0}$}  
	+ {\color[rgb]{0.54,0.12,0.03}$b_\text{A}$}$x_\text{A}$ 
	+ {\color[rgb]{0.54,0.12,0.03}$b_\text{B}$}$x_\text{B}$ 
	+ {\color[rgb]{0.54,0.12,0.03}$b_\text{C}$}$x_\text{C}$ 
	+ {\color[rgb]{0.54,0.12,0.03}$b_\text{AB}$}$x_\text{A}x_\text{B}$ 
	+ {\color[rgb]{0.54,0.12,0.03}$b_\text{AC}$}$x_\text{A}x_\text{C}$ 
	+ {\color[rgb]{0.54,0.12,0.03}$b_\text{BC}$}$x_\text{B}x_\text{C}$ 
	+ {\color[rgb]{0.54,0.12,0.03}$b_\text{ABC}$}$x_\text{A}x_\text{B}x_\text{C}$}}  % The ABC, now just with signs
	
	\newcommand{\mw}{\color[rgb]{1,1,1}}
	\newcommand{\mm}{\color{lightgray}}
	
	\vspace{-0.8cm}
	{\LARGE
	\begin{flalign*}
			&{\mw =}\normalsize  \qquad\,\,\begin{matrix} \mm b_\text{0} & \mm b_\text{A} & \mm \,b_\text{B} & \mm \,b_\text{C}
			& \mm b_\text{AB} & \mm b_\text{AC} & \mm \hspace{-0.03cm}b_\text{BC} & \mm \hspace{-0.13cm}b_\text{ABC} 
		\end{matrix}
			\\
		\begin{pmatrix}y_1\\y_2\\y_3\\y_4\\y_5\\y_6\\y_7\\y_8\end{pmatrix} &= 
		\left(\begin{matrix}
			+ & - & - & - & + & + & + & - \\ 
			+ & + & - & - & - & - & + & + \\ 
			+ & - & + & - & - & + & - & + \\
			+ & + & + & - & + & - & - & - \\
			+ & - & - & + & + & - & - & + \\ 
			+ & + & - & + & - & + & - & - \\ 
			+ & - & + & + & - & - & + & - \\
			+ & + & + & + & + & + & + & + \\
		 \end{matrix}\right)	
		\begin{pmatrix}{\color[rgb]{0.54,0.12,0.03}b_0}\\
		{\color[rgb]{0.54,0.12,0.03}b_\text{A} }\\
		{\color[rgb]{0.54,0.12,0.03}b_\text{B}} \\
		{\color[rgb]{0.54,0.12,0.03}b_\text{C}} \\
		{\color[rgb]{0.54,0.12,0.03}b_\text{AB}} \\
		{\color[rgb]{0.54,0.12,0.03}b_\text{AC}} \\
		{\color[rgb]{0.54,0.12,0.03}b_\text{BC}} \\
		{\color[rgb]{0.54,0.12,0.03}b_\text{ABC}}\\
		 \end{pmatrix}\\
			 \mathbf{y} &= \mathbf{X}{\color[rgb]{0.54,0.12,0.03}\mathbf{b}} & % trailing "&" is required for flalign
	\end{flalign*}
	}	
\end{frame}

\begin{frame}\frametitle{Our predictive model \color[rgb]{0,0,0}{\scriptsize  $ \hat{y} = $ {\color[rgb]{0.54,0.12,0.03}$b_\text{0}$}  
	+ {\color[rgb]{0.54,0.12,0.03}$b_\text{A}$}$x_\text{A}$ 
	+ {\color[rgb]{0.54,0.12,0.03}$b_\text{B}$}$x_\text{B}$ 
	+ {\color[rgb]{0.54,0.12,0.03}$b_\text{C}$}$x_\text{C}$ 
	+ {\color[rgb]{0.54,0.12,0.03}$b_\text{AB}$}$x_\text{A}x_\text{B}$ 
	+ {\color[rgb]{0.54,0.12,0.03}$b_\text{AC}$}$x_\text{A}x_\text{C}$ 
	+ {\color[rgb]{0.54,0.12,0.03}$b_\text{BC}$}$x_\text{B}x_\text{C}$ 
	+ {\color[rgb]{0.54,0.12,0.03}$b_\text{ABC}$}$x_\text{A}x_\text{B}x_\text{C}$}}

	\newcommand{\mw}{\color[rgb]{1,1,1}}
	\newcommand{\mm}{\color{lightgray}}
	\vspace{-0.8cm}
	{\LARGE
	\begin{flalign*}
			&{\mw =}\normalsize  \qquad\,\,\begin{matrix} \mm b_\text{0} & \mm b_\text{A} & \mm \,b_\text{B} & \mm \,b_\text{C}
			& \mm b_\text{AB} & \mm b_\text{AC} & \mm \hspace{-0.03cm}b_\text{BC} & \mm \hspace{-0.13cm}b_\text{ABC} 
		\end{matrix}
			\\
		\begin{pmatrix}\\y_2\\y_3\\\\y_5\\\\\\y_8\end{pmatrix} &= 
		\left(\begin{matrix}
			  &   &   &   &   &   &   &  \mw -  \\ 
			+ & + & - & - & - & - & + & + \\ 
			+ & - & + & - & - & + & - & + \\
			  &   &   &   &   &   &   &    \\ 
			+ & - & - & + & + & - & - & + \\ 
			  &   &   &   &   &   &   &    \\ 
			  &   &   &   &   &   &   &    \\ 
			+ & + & + & + & + & + & + & + \\
		 \end{matrix}\right)		 
		\begin{pmatrix}{\color[rgb]{0.54,0.12,0.03}b_0}\\
		{\color[rgb]{0.54,0.12,0.03} b_\text{A} }\\
		{\color[rgb]{0.54,0.12,0.03}b_\text{B}} \\
		{\color[rgb]{0.54,0.12,0.03}b_\text{C}} \\
		{\color[rgb]{0.54,0.12,0.03}b_\text{AB}} \\
		{\color[rgb]{0.54,0.12,0.03}b_\text{AC}} \\
		{\color[rgb]{0.54,0.12,0.03}b_\text{BC}} \\
		{\color[rgb]{0.54,0.12,0.03}b_\text{ABC}}\\
		 \end{pmatrix}\\
			 \mathbf{y} &= \mathbf{X}{\color[rgb]{0.54,0.12,0.03}\mathbf{b}}& % trailing "&" is required for flalign	 		
	\end{flalign*}
	}
\end{frame}

\begin{frame}\frametitle{Our predictive model \color[rgb]{0,0,0}{\scriptsize  $ \hat{y} = $ {\color[rgb]{0.54,0.12,0.03}$b_\text{0}$}  
	+ {\color[rgb]{0.54,0.12,0.03}$b_\text{A}$}$x_\text{A}$ 
	+ {\color[rgb]{0.54,0.12,0.03}$b_\text{B}$}$x_\text{B}$ 
	+ {\color[rgb]{0.54,0.12,0.03}$b_\text{C}$}$x_\text{C}$ 
	+ {\color[rgb]{0.54,0.12,0.03}$b_\text{AB}$}$x_\text{A}x_\text{B}$ 
	+ {\color[rgb]{0.54,0.12,0.03}$b_\text{AC}$}$x_\text{A}x_\text{C}$ 
	+ {\color[rgb]{0.54,0.12,0.03}$b_\text{BC}$}$x_\text{B}x_\text{C}$ 
	+ {\color[rgb]{0.54,0.12,0.03}$b_\text{ABC}$}$x_\text{A}x_\text{B}x_\text{C}$}}

	\newcommand{\mw}{\color[rgb]{1,1,1}}
	\newcommand{\mm}{\color{lightgray}}
	\vspace{-0.8cm}
	{\LARGE
	\begin{flalign*}
			&{\mw =}\normalsize  \qquad\,\,\begin{matrix} \mm b_\text{0} & \mm b_\text{A} & \mm \,b_\text{B} & \mm \,b_\text{C}
			& \mm b_\text{AB} & \mm b_\text{AC} & \mm \hspace{-0.03cm}b_\text{BC} & \mm \hspace{-0.13cm}b_\text{ABC} 
		\end{matrix}\\
		\begin{pmatrix}y_5\\y_2\\y_3\\\\\\\\\\y_8\end{pmatrix} &= 
		\left(\begin{matrix}
			+  & -  & -  & +  & +  & -  & -  & +  \\ 
			+  & +  & -  & -  & -  & -  & +  & +  \\ 
			+  & -  & +  & -  & -  & +  & -  & +  \\
			   &    &    &    &    &    &    &    \\ 
			   &    &    &    &    &    &    &  \mw-   \\ 
			   &    &    &    &    &    &    &    \\ 
			   &    &    &    &    &    &    &    \\ 
			+  & +  & +  & +  & +  & +  & +  & +  \\
		 \end{matrix}\right)		 
		\begin{pmatrix}{\color[rgb]{0.54,0.12,0.03}b_0}\\
		{\color[rgb]{0.54,0.12,0.03}b_\text{A} }\\
		{\color[rgb]{0.54,0.12,0.03}b_\text{B}} \\
		{\color[rgb]{0.54,0.12,0.03}b_\text{C}} \\
		{\color[rgb]{0.54,0.12,0.03}b_\text{AB}} \\
		{\color[rgb]{0.54,0.12,0.03}b_\text{AC}} \\
		{\color[rgb]{0.54,0.12,0.03}b_\text{BC}} \\
		{\color[rgb]{0.54,0.12,0.03}b_\text{ABC}}\\
		 \end{pmatrix}\\
			 \mathbf{y} &= \mathbf{X}{\color[rgb]{0.54,0.12,0.03}\mathbf{b}} & % trailing "&" is required for		
	\end{flalign*}
	}
\end{frame}

\begin{frame}\frametitle{Our predictive model \color[rgb]{0,0,0}{\scriptsize  $ \hat{y} = $ {\color[rgb]{0.54,0.12,0.03}$b_\text{0}$}  
	+ {\color[rgb]{0.54,0.12,0.03}$b_\text{A}$}$x_\text{A}$ 
	+ {\color[rgb]{0.54,0.12,0.03}$b_\text{B}$}$x_\text{B}$ 
	+ {\color[rgb]{0.54,0.12,0.03}$b_\text{C}$}$x_\text{C}$ 
	+ {\color[rgb]{0.54,0.12,0.03}$b_\text{AB}$}$x_\text{A}x_\text{B}$ 
	+ {\color[rgb]{0.54,0.12,0.03}$b_\text{AC}$}$x_\text{A}x_\text{C}$ 
	+ {\color[rgb]{0.54,0.12,0.03}$b_\text{BC}$}$x_\text{B}x_\text{C}$ 
	+ {\color[rgb]{0.54,0.12,0.03}$b_\text{ABC}$}$x_\text{A}x_\text{B}x_\text{C}$}}
	 
	\newcommand{\mw}{\color[rgb]{1,1,1}}
	\newcommand{\mm}{\color{lightgray}}
	\vspace{-0.8cm}
	{\LARGE
	\begin{flalign*}
		&{\mw =}\normalsize  \qquad\,\,\begin{matrix} \mm b_\text{0} & \mm b_\text{A} & \mm \,b_\text{B} & \mm \,b_\text{C}
		& \mm b_\text{AB} & \mm b_\text{AC} & \mm \hspace{-0.03cm}b_\text{BC} & \mm \hspace{-0.13cm}b_\text{ABC} 
	\end{matrix}
		\\
		\begin{pmatrix}y_5\\y_2\\y_3\\y_8\\\\\\\\\\\end{pmatrix} &= 
		\left(\begin{matrix}
			+  & -  & -  & +  & +  & -  & -  & +  \\ 
			+  & +  & -  & -  & -  & -  & +  & +  \\ 
			+  & -  & +  & -  & -  & +  & -  & +  \\
			+  & +  & +  & +  & +  & +  & +  & +  \\
			  &    &    &    &    &    &    &    \\ 
			  &    &    &    &    &    &    &  \mw-   \\ 
			  &    &    &    &    &    &    &    \\ 
			  &    &    &    &    &    &    &    \\ 			
		 \end{matrix}\right)		 
		\begin{pmatrix}{\color[rgb]{0.54,0.12,0.03}b_0}\\
		{\color[rgb]{0.54,0.12,0.03}b_\text{A} }\\
		{\color[rgb]{0.54,0.12,0.03}b_\text{B}} \\
		{\color[rgb]{0.54,0.12,0.03}b_\text{C}} \\
		{\color[rgb]{0.54,0.12,0.03}b_\text{AB}} \\
		{\color[rgb]{0.54,0.12,0.03}b_\text{AC}} \\
		{\color[rgb]{0.54,0.12,0.03}b_\text{BC}} \\
		{\color[rgb]{0.54,0.12,0.03}b_\text{ABC}}\\
		 \end{pmatrix}\\
			 \mathbf{y} &= \mathbf{X}{\color[rgb]{0.54,0.12,0.03}\mathbf{b}}& % trailing "&" is required for			 		
	\end{flalign*}
	}
\end{frame}

\begin{frame}\frametitle{Our predictive model \color[rgb]{0,0,0}{\scriptsize  $ \hat{y} = $ {\color[rgb]{0.54,0.12,0.03}$b_\text{0}$}  
	+ {\color[rgb]{0.54,0.12,0.03}$b_\text{A}$}$x_\text{A}$ 
	+ {\color[rgb]{0.54,0.12,0.03}$b_\text{B}$}$x_\text{B}$ 
	+ {\color[rgb]{0.54,0.12,0.03}$b_\text{C}$}$x_\text{C}$ 
	+ {\color[rgb]{0.54,0.12,0.03}$b_\text{AB}$}$x_\text{A}x_\text{B}$ 
	+ {\color[rgb]{0.54,0.12,0.03}$b_\text{AC}$}$x_\text{A}x_\text{C}$ 
	+ {\color[rgb]{0.54,0.12,0.03}$b_\text{BC}$}$x_\text{B}x_\text{C}$ 
	+ {\color[rgb]{0.54,0.12,0.03}$b_\text{ABC}$}$x_\text{A}x_\text{B}x_\text{C}$}}
	 
	\newcommand{\mw}{\color[rgb]{1,1,1}}
	\newcommand{\mm}{\color{lightgray}}
	\vspace{-0.8cm}
	{\LARGE
	\begin{flalign*}
		&{\mw =}\normalsize  \qquad\,\,\begin{matrix} \mm b_\text{0} & \mm b_\text{A} & \mm \,b_\text{B} & \mm \,b_\text{C}
		& \mm b_\text{AB} & \mm b_\text{AC} & \mm \hspace{-0.03cm}b_\text{BC} & \mm \hspace{-0.13cm}b_\text{ABC} 
	\end{matrix}
		\\
		\begin{pmatrix}y_5\\y_2\\y_3\\y_8\end{pmatrix} &= 
		\left(\begin{matrix}
			+  & -  & -  & +  & +  & -  & -  & +  \\ 
			+  & +  & -  & -  & -  & -  & +  & +  \\ 
			+  & -  & +  & -  & -  & +  & -  & +  \\
			+  & +  & +  & +  & +  & +  & +  & +  \\
		 \end{matrix}\right)	{	 \small
		\begin{pmatrix}{\color[rgb]{0.54,0.12,0.03}b_0}\\
		{\color[rgb]{0.54,0.12,0.03}b_\text{A} }\\
		{\color[rgb]{0.54,0.12,0.03}b_\text{B}} \\
		{\color[rgb]{0.54,0.12,0.03}b_\text{C}} \\
		{\color[rgb]{0.54,0.12,0.03}b_\text{AB}} \\
		{\color[rgb]{0.54,0.12,0.03}b_\text{AC}} \\
		{\color[rgb]{0.54,0.12,0.03}b_\text{BC}} \\
		{\color[rgb]{0.54,0.12,0.03}b_\text{ABC}}\\
		 \end{pmatrix}}\\
			 \mathbf{y} &= \mathbf{X}{\color[rgb]{0.54,0.12,0.03}\mathbf{b}}& % trailing "&" is required for			 		
	\end{flalign*}
	}
\end{frame}

\begin{frame}\frametitle{We have ``aliasing'' taking place here}
	\begin{columns}
		\column{0.48\textwidth}	
			 Who is \emph{Jorge Mario Bergoglio}?
		
		\onslide+<2->	{
		\column{0.48\textwidth}
			\includegraphics[width=.75\textwidth]{../4B/Supporting files/Wikipedia-Pope_Francis_among_the_people_at_St._Peter's_Square_-_12_May_2013.png}		

			{\scriptsize  \href{https://en.wikipedia.org/wiki/Pope\_Francis}{Wikipedia}}
			}
	\end{columns}
	
\end{frame}

\begin{frame}\frametitle{We have ``aliasing'' taking place here}
	\begin{columns}
		\column{0.48\textwidth}	
			 Who is \emph{Kevin George Dunn}?		

		\column{0.48\textwidth}
			\begin{itemize}
				\item	My email address
				\item	A username for a website
			\end{itemize}
	\end{columns}
	
\end{frame}

\begin{frame}\frametitle{Aliasing: when we have more than one name for the same thing}
	
	\vspace{1cm}
	What is aliased in this experimental design (i.e. which columns are the same)?
		
		\vspace{0.5cm}
		\begin{itemize}
			\item	\textbf{A=BC}
			
			\onslide+<2->	{
			\vspace{1cm}
			\item	\textbf{B=AC}
			
			\vspace{1cm}
			\item	\textbf{C=AB} (this was intentional: read from the ``tradeoff'' table)
			
			\vspace{1cm}
			\item	\textbf{ABC = Intercept} (the intercept is indicated as $b_0$)
			}
		\end{itemize}

\end{frame}

\begin{frame}\frametitle{Remove the aliases by collapsing identical columns}
	 
	\newcommand{\mw}{\color[rgb]{1,1,1}}
	\newcommand{\mm}{\color{lightgray}}
	\vspace{-0.8cm}
	{\LARGE
	\begin{flalign*}
		&{\mw =}\normalsize  \qquad\,\,\begin{matrix} \mm b_\text{0} & \mm b_\text{A} & \mm \,b_\text{B} & \mm \,b_\text{C}
		& \mm b_\text{AB} & \mm b_\text{AC} & \mm \hspace{-0.03cm}b_\text{BC} & \mm \hspace{-0.13cm}b_\text{ABC} 
	\end{matrix}
		\\
		\begin{pmatrix}y_5\\y_2\\y_3\\y_8\end{pmatrix} &= 
		\left(\begin{matrix}
			+  & -  & -  & +  & +  & -  & -  & +  \\ 
			+  & +  & -  & -  & -  & -  & +  & +  \\ 
			+  & -  & +  & -  & -  & +  & -  & +  \\
			+  & +  & +  & +  & +  & +  & +  & +  \\
		 \end{matrix}\right)	{	 \small
		\begin{pmatrix}{\color[rgb]{0.54,0.12,0.03}b_0}\\
		{\color[rgb]{0.54,0.12,0.03}b_\text{A} }\\
		{\color[rgb]{0.54,0.12,0.03}b_\text{B}} \\
		{\color[rgb]{0.54,0.12,0.03}b_\text{C}} \\
		{\color[rgb]{0.54,0.12,0.03}b_\text{AB}} \\
		{\color[rgb]{0.54,0.12,0.03}b_\text{AC}} \\
		{\color[rgb]{0.54,0.12,0.03}b_\text{BC}} \\
		{\color[rgb]{0.54,0.12,0.03}b_\text{ABC}}\\
		 \end{pmatrix}}& % trailing "&" is required for			 		
	\end{flalign*}
	}
	\begin{itemize}
		\item	4 equations (rows) in 8 unknowns {\color[rgb]{0.54,0.12,0.03}(the entries shown in the last matrix)}
		\onslide+<2->	{
			\item	a
			\item	s
			\item	d
		}
	\end{itemize}
\end{frame}

\begin{frame}\frametitle{Remove the aliases by collapsing identical columns}
	 
	\newcommand{\mw}{\color[rgb]{1,1,1}}
	\newcommand{\mm}{\color{lightgray}}
	\vspace{-0.8cm}
	{\LARGE
	\begin{flalign*}
		&{\mw =}\normalsize  \qquad\,\,\begin{matrix} \mm b_\text{0} + b_\text{ABC} & \mm b_\text{A} + b_\text{BC} & \mm \,b_\text{B} + b_\text{AC} & \mm \,\,b_\text{C}+ b_\text{AB} & \mm  & \mm \hspace{-0.03cm} & 
	\end{matrix}
		\\
		\begin{pmatrix}y_5\\y_2\\y_3\\y_8\end{pmatrix} &= 
		\left(\begin{matrix}
			+  & \qquad -  & \qquad -  & \qquad +  \\ 
			+  & \qquad +  & \qquad -  & \qquad -   \\ 
			+  & \qquad -  & \qquad +  & \qquad -   \\
			+  & \qquad +  & \qquad +  & \qquad +   \\
		 \end{matrix}\,\,\,\right)		 
		\begin{pmatrix}
		{\color[rgb]{0.54,0.12,0.03}\,\,b_0 \, + b_\text{ABC}}\\
		{\color[rgb]{0.54,0.12,0.03}b_\text{A} + b_\text{BC}}\\
		{\color[rgb]{0.54,0.12,0.03}b_\text{B} + b_\text{AC}} \\
		{\color[rgb]{0.54,0.12,0.03}b_\text{C} + b_\text{AB}}
		 \end{pmatrix}	 & % trailing "&" is required for			 		
	\end{flalign*}
	}
	\begin{itemize}
		\item	4 equations (rows) in 4 unknowns {\color[rgb]{0.54,0.12,0.03}(the entries in the last matrix)}
		\onslide+<2->	{
			\item	a
			\item	s
			\item	d
		}
	\end{itemize}
\end{frame}

\begin{frame}\frametitle{Remove the aliases by collapsing identical columns}
	 
	\newcommand{\mw}{\color[rgb]{1,1,1}}
	\newcommand{\mm}{\color{lightgray}}
	\vspace{-0.8cm}
	{\LARGE
	\begin{flalign*}
		&{\mw =}\normalsize  \qquad\,\,\begin{matrix} \mm b_\text{0} + b_\text{ABC} & \mm b_\text{A} + b_\text{BC} & \mm \,b_\text{B} + b_\text{AC} & \mm \,\,b_\text{C}+ b_\text{AB} & \mm  & \mm \hspace{-0.03cm} & 
	\end{matrix}
		\\
		\begin{pmatrix}y_5\\y_2\\y_3\\y_8\end{pmatrix} &= 
		\left(\begin{matrix}
			+  & \qquad -  & \qquad -  & \qquad +  \\ 
			+  & \qquad +  & \qquad -  & \qquad -   \\ 
			+  & \qquad -  & \qquad +  & \qquad -   \\
			+  & \qquad +  & \qquad +  & \qquad +   \\
		 \end{matrix}\,\,\,\right)		 
		\begin{pmatrix}
		{\color[rgb]{0.54,0.12,0.03}\,\,b_0 \, + b_\text{ABC}}\\
		{\color[rgb]{0.54,0.12,0.03}\hat{b}_\text{A}}   \\
		{\color[rgb]{0.54,0.12,0.03}b_\text{B} + b_\text{AC}} \\
		{\color[rgb]{0.54,0.12,0.03}b_\text{C} + b_\text{AB}}
		 \end{pmatrix}	 & % trailing "&" is required for			 		
	\end{flalign*}
	}
	\begin{itemize}
		\item	Let's call the merged coefficient ${\color[rgb]{0.54,0.12,0.03}\hat{b}_\text{A}} = b_\text{A} + b_\text{BC}$
		\onslide+<2->	{
			\item	a
			\item	s
			\item	d
		}
	\end{itemize}
\end{frame}

\begin{frame}\frametitle{Remove the aliases by collapsing identical columns}
	 
	\newcommand{\mw}{\color[rgb]{1,1,1}}
	\newcommand{\mm}{\color{lightgray}}
	\vspace{-0.8cm}
	{\LARGE
	\begin{flalign*}
		&{\mw =}\normalsize  \qquad\,\,\begin{matrix} \mm b_\text{0} + b_\text{ABC} & \mm b_\text{A} + b_\text{BC} & \mm \,b_\text{B} + b_\text{AC} & \mm \,\,b_\text{C}+ b_\text{AB} & \mm  & \mm \hspace{-0.03cm} & 
	\end{matrix}
		\\
		\begin{pmatrix}y_5\\y_2\\y_3\\y_8\end{pmatrix} &= 
		\left(\begin{matrix}
			+  & \qquad -  & \qquad -  & \qquad +  \\ 
			+  & \qquad +  & \qquad -  & \qquad -   \\ 
			+  & \qquad -  & \qquad +  & \qquad -   \\
			+  & \qquad +  & \qquad +  & \qquad +   \\
		 \end{matrix}\,\,\,\right)		 
		\begin{pmatrix}
		{\color[rgb]{0.54,0.12,0.03}\,\,b_0 \, + b_\text{ABC}}\\
		{\color[rgb]{0.54,0.12,0.03}\hat{b}_\text{A}}   \\
		{\color[rgb]{0.54,0.12,0.03}\hat{b}_\text{B}} \\
		{\color[rgb]{0.54,0.12,0.03}b_\text{C} + b_\text{AB}}
		 \end{pmatrix}	 & % trailing "&" is required for			 		
	\end{flalign*}
	}
	\begin{itemize}
		\item	Let's call the merged coefficient ${\color[rgb]{0.54,0.12,0.03}\hat{b}_\text{A}} = b_\text{A} + b_\text{BC}$
		\item	${\color[rgb]{0.54,0.12,0.03}\hat{b}_\text{B}} = b_\text{B} + b_\text{AC}$
		\onslide+<2->	{
			\item	s
			\item	d
		}
	\end{itemize}
\end{frame}

\begin{frame}\frametitle{Remove the aliases by collapsing identical columns}
	 
	\newcommand{\mw}{\color[rgb]{1,1,1}}
	\newcommand{\mm}{\color{lightgray}}
	\vspace{-0.8cm}
	{\LARGE
	\begin{flalign*}
		&{\mw =}\normalsize  \qquad\,\,\begin{matrix} \mm b_\text{0} + b_\text{ABC} & \mm b_\text{A} + b_\text{BC} & \mm \,b_\text{B} + b_\text{AC} & \mm \,\,b_\text{C}+ b_\text{AB} & \mm  & \mm \hspace{-0.03cm} & 
	\end{matrix}
		\\
		\begin{pmatrix}y_5\\y_2\\y_3\\y_8\end{pmatrix} &= 
		\left(\begin{matrix}
			+  & \qquad -  & \qquad -  & \qquad +  \\ 
			+  & \qquad +  & \qquad -  & \qquad -   \\ 
			+  & \qquad -  & \qquad +  & \qquad -   \\
			+  & \qquad +  & \qquad +  & \qquad +   \\
		 \end{matrix}\,\,\,\right)		 
		\begin{pmatrix}
		{\color[rgb]{0.54,0.12,0.03}\hat{b}_0}\\
		{\color[rgb]{0.54,0.12,0.03}\hat{b}_\text{A}}   \\
		{\color[rgb]{0.54,0.12,0.03}\hat{b}_\text{B}} \\
		{\color[rgb]{0.54,0.12,0.03}\hat{b}_\text{C}}
		 \end{pmatrix}	 & % trailing "&" is required for			 		
	\end{flalign*}
	}
	\begin{itemize}
		\item	${\color[rgb]{0.54,0.12,0.03}\hat{b}_\text{A}} = b_\text{A} + b_\text{BC}$
		\item	${\color[rgb]{0.54,0.12,0.03}\hat{b}_\text{B}} = b_\text{B} + b_\text{AC}$
		\item	${\color[rgb]{0.54,0.12,0.03}\hat{b}_\text{C}} = b_\text{C} + b_\text{AB}$
		\item	${\color[rgb]{0.54,0.12,0.03}\hat{b}_\text{0}} = b_\text{0} + b_\text{ABC}$		
		
	\end{itemize}
\end{frame}

\begin{frame}\frametitle{Now that we understand aliasing; how can we work with it in our system?}
	
	We'd like to take advantage of doing half the work, but still get the most benefit. But, 
	recall we showed that this will lead to a loss of some accuracy:
	
	\vspace{0.2cm}
	
	
	\vspace{-0.2cm}
	\begin{columns}[T]
		\column{0.35\textwidth}
			\begin{center}\textbf{Full factorial model}\end{center}
				\begin{align*}
					\hat{y} &= {\color{myOrange}11.25}\\
							&+ {\color{myOrange}6.25\,}x_\text{A}\\
							&+ {\color{blue}0.75\,}x_\text{B}\\
							&- {\color{myOrange}7.25\,}x_\text{C}\\
							&+ {\color{myOrange}0.25\,}x_\text{A}x_\text{B}\\
							&- {\color{blue}6.75\,}x_\text{A}x_\text{C}\\
							&- {\color{myOrange}0.25\,}x_\text{B}x_\text{C}\\
							&- {\color{myOrange}0.25\,}x_\text{A}x_\text{B}x_\text{C}		
				\end{align*}
		\column{0.02\textwidth}
			\vspace{1cm}
			\rule[3mm]{0.03cm}{65mm}
		\column{0.35\textwidth}
			\begin{center}\textbf{Fractional factorial model}\end{center}				
				\begin{align*}
					\hat{y} &= {\color{myOrange}11.0}\\
							&+ {\color{myOrange}6.0\,}x_\text{A}\\
							&- {\color{blue}6.0\,}x_\text{B}\\
							&- {\color{myOrange}7.0\,}x_\text{C}\\
							& \color{lightgray}+ \cancel{b_\text{AB}\,x_\text{A}x_\text{B}}\\
							& \color{lightgray}+ \cancel{b_\text{AC}\,x_\text{A}x_\text{C}}\\
							& \color{lightgray}+ \cancel{b_\text{BC}\,x_\text{B}x_\text{C}}\\
							& \color{lightgray}+ \cancel{b_\text{ABC}\,x_\text{A}x_\text{B}x_\text{C}}\\
				\end{align*}

	\end{columns}
\end{frame}



% /Users/kevindunn/Dropbox/Coursera/Media/All-course-slides/classes/CourseraMOOC-class-4C.tex

\begin{frame}\frametitle{Now that we understand aliasing; how can we work with it in our system?}
	
	We'd like to take advantage of doing half the work, but still get the most benefit. But, 
	recall we showed that this will lead to a loss of some accuracy:
	
	\vspace{0.2cm}
	
	
	\vspace{-0.2cm}
	\begin{columns}[T]
		\column{0.35\textwidth}
			\begin{center}\textbf{Full factorial model}\end{center}
				\begin{align*}
					\hat{y} &= {\color{myOrange}11.25}\\
							&+ {\color{myOrange}6.25\,}x_\text{A}\\
							&+ {\color{blue}0.75\,}x_\text{B}\\
							&- {\color{myOrange}7.25\,}x_\text{C}\\
							&+ {\color{myOrange}0.25\,}x_\text{A}x_\text{B}\\
							&- {\color{blue}6.75\,}x_\text{A}x_\text{C}\\
							&- {\color{myOrange}0.25\,}x_\text{B}x_\text{C}\\
							&- {\color{myOrange}0.25\,}x_\text{A}x_\text{B}x_\text{C}		
				\end{align*}
		\column{0.02\textwidth}
			\vspace{1cm}
			\rule[3mm]{0.03cm}{65mm}
		\column{0.35\textwidth}
			\begin{center}\textbf{Fractional factorial model}\end{center}				
				\begin{align*}
					\hat{y} &= {\color{myOrange}11.0}\\
							&+ {\color{myOrange}6.0\,}x_\text{A}\\
							&- {\color{blue}6.0\,}x_\text{B}\\
							&- {\color{myOrange}7.0\,}x_\text{C}\\
							& \color{lightgray}+ \cancel{b_\text{AB}\,x_\text{A}x_\text{B}}\\
							& \color{lightgray}+ \cancel{b_\text{AC}\,x_\text{A}x_\text{C}}\\
							& \color{lightgray}+ \cancel{b_\text{BC}\,x_\text{B}x_\text{C}}\\
							& \color{lightgray}+ \cancel{b_\text{ABC}\,x_\text{A}x_\text{B}x_\text{C}}\\
				\end{align*}

	\end{columns}
\end{frame}

\begin{frame}\frametitle{Example: treating water (again)}
	
	\begin{columns}[T]
		\column{0.65\textwidth}
			We have identified 3 factors to investigate
			\begin{enumerate}
				\item	Water treatment chemical used
				\item	Temperature of the treatment
				\item	Stirring speed
			\end{enumerate}
		
			\vspace{1cm}
			\emph{Budget}: for 4, maybe 5, experiments
			
			\onslide+<2->{
				\vspace{1cm}
				Other criteria:
				\begin{itemize}
					\item	We don't expect interactions between temperature, and stirring speed
					\onslide+<3->{
						\item	We want a good (i.e. unbiased) estimate of the chemical effect
					}
				\end{itemize}
			}
			
		
		\column{0.38\textwidth}
			\includegraphics[width=\textwidth]{../4B/Supporting files/177506595-thinkstock-reduced.jpg}
		
		
			
	\end{columns}
\end{frame}

\begin{frame}\frametitle{Example: treating water (again)}
	
	\begin{columns}[T]
		\column{0.70\textwidth}
			Now we choose to assign the factor letters: \textbf{A}, \textbf{B}, and \textbf{C}
			
			\vspace{0.25cm}
			\begin{itemize} 
				\item	\textbf{A}: Temperature of the treatment
				\item	\textbf{B}: Stirring speed				
				\item	\textbf{C}: Water treatment chemical used

			\end{itemize}
			
			\vspace{0.5cm}
			Reasons for this choice of letters (assignment):
			\begin{itemize}
				\item	We don't expect interaction between temperature (\textbf{A}), and stirring speed (\textbf{B}). That implies $b_\text{AB} \approx 0$
				\item	We want a clear, unbiased estimate of the chemical effect (\textbf{C}). So an unbiased $b_\text{C}$ is desired.
				\item	We know that $\hat{b}_\text{C}$ will be the estimated chemical effect in a half-fraction. It will
					 	be confounded:
				\begin{itemize}
					\item	$\hat{b}_\text{C}  = b_\text{C} + b_\text{AB}  = b_\text{C} + \cancelto{\approx 0}{b_\text{AB}} \qquad$ so then $\hat{b}_C \approx b_\text{C}$
				\end{itemize}
			\end{itemize}
		
		\column{0.35\textwidth}
			\includegraphics[width=\textwidth]{../4B/Supporting files/177506595-thinkstock-reduced.jpg}
	\end{columns}
\end{frame}

\begin{frame}\frametitle{\includegraphics[width=0.3\textwidth]{\imagedir/doe/examples/advice-logo.png}\,\, plan your experiments carefully ahead of time}
	
	\vspace{24pt}
	{\color{myOrange} 	\emph{Experiments are expensive to run!}}
	

	\begin{exampleblock}{}
		\vspace{12pt}
		\begin{itemize}
			\item	so plan not only the experiments, but also \textbf{how you will analyze} the results
			\item	you don't need outcome values ($y$-values) to do this
			
			\vspace{1cm}

			\item	Re-allocate your letters if you don't like the confounding
		\end{itemize} 
	\end{exampleblock}
	
\end{frame}

\begin{frame}\frametitle{}
	%\vspace{-0.5cm}
	\includegraphics[height=.9\textheight]{\imagedir/doe/DOE-trade-off-table-half-fractions.png}
	
	\vspace{-10pt}
	Notice where all the half fractions lie in the table. \\
	Do you observe the recurring pattern?
\end{frame}

\begin{frame}\frametitle{\includegraphics[width=0.3\textwidth]{\imagedir/doe/examples/advice-logo.png}\,\, the utility of half-fractions}
	
	\vspace{24pt}
	Four factors: requires 16 runs for a full factorial. Why not start with a half fraction?
	
	\vspace{24pt}
	\begin{itemize}
		\item	Start with 8 runs
		\item	Then come back and do the other 8 later on
	\end{itemize}
	
	\pause
	\vspace{24pt}
	{\color{myOrange} Coming up later in this module: we will see quarter, or even $\tfrac{1}{8}^\text{th}$ fractions, applied in the same way.}
	
\end{frame}

\begin{frame}\frametitle{Demonstration on an example we can visualize: a 3-factor system}
	\begin{columns}
		\column{0.6\textwidth}
			\begin{center}
				\includegraphics[width=.9\textwidth]{\imagedir/doe/half-fraction-in-3-factors-MOOC-no-labels.png}
			\end{center}
			
		\column{0.55\textwidth}
		
			{\color{blue}Start with a half fraction $^\ast$}
			\begin{tabulary}{\linewidth}{cccl}\hline 
				\textsf{\relax Experiment } & \textbf{\relax A } & \textbf{\relax B } & \textbf{\relax C = AB} \\
				\hline 
				\textbf{1} & \(-\) & \(-\) & \(+\) \\
				\textbf{2} & \(+\) & \(-\) & \(-\) \\
				\textbf{3} & \(-\) & \(+\) & \(-\) \\
				\textbf{4} & \(+\) & \(+\) & \(+\) \\  \hline
			\end{tabulary}
			
			{\color{myOrange}These are the 4 runs with the open circles.}

			\small
			\vspace{0.5cm}
			
			
			\onslide+<2->{
			{\color{blue}\emph{Later on} ... come back and complete the complementary half-fraction.$^\ast$}
			
			\begin{tabulary}{\linewidth}{cccl}\hline 
				\textsf{\relax Experiment } & \textbf{\relax A } & \textbf{\relax B } & \textbf{\relax C = $-$AB } \\
				\hline 
				\textbf{5} & \(-\) & \(-\) & \(-\) \\
				\textbf{6} & \(+\) & \(-\) & \(+\) \\
				\textbf{7} & \(-\) & \(+\) & \(+\) \\
				\textbf{8} & \(+\) & \(+\) & \(-\) \\ \hline
			\end{tabulary}
			
			{\color{myOrange}These are the 4 runs with the closed circles.}
			}
			
			\vspace{6pt}
			{\color{blue}$^\ast$\scriptsize {Always randomize within the group of 4!}}
	\end{columns}
\end{frame}

\begin{frame}\frametitle{}
	%\vspace{-0.5cm}
	\includegraphics[height=.9\textheight]{\imagedir/doe/DOE-trade-off-table-half-fractions.png}
	
	\vspace{-10pt}
	We will see how entries in this table are good building blocks.\\
	Consider how you might use it for your experiments.
\end{frame}

\begin{frame}\frametitle{Half fractions are  one of the experimental building blocks available to you}
	
	\begin{columns}[C]
		\column{0.3\textwidth}
		
	
			Initial groups of experiments can be build on, and extended later.
			
		\column{0.7\textwidth}
			
				\centerline{\includegraphics[width=0.5\textwidth]{../4G/Supporting material/flickr-6479064129_25ce3bb07f_o-building-block.png} \quad
				\see{\href{https://secure.flickr.com/photos/rahego/6479064129}{Flickr: rahego}}
				}
		
				
		
	\end{columns}
\end{frame}

% /Users/kevindunn/Dropbox/Coursera/Media/All-course-slides/classes/CourseraMOOC-class-4D.tex


% IGNORE THIS COMMENT
\begin{comment}
		
	\begin{frame}\frametitle{\includegraphics[width=0.3\textwidth]{\imagedir/doe/examples/advice-logo.png}\,\, plan your experiments carefully ahead of time}
	\end{frame}
	
	\begin{columns}[T]
		\column{0.45\textwidth}
			\includegraphics[width=0.7\textwidth]{\imagedir/statistics/flicfcb_o.jpg}
		
			{\scriptsize (p. 230 in Box, Hunter and Hunter, 2$^\text{nd}$ ed)}
		
		\column{0.48\textwidth}
			\includegraphics[width=\textwidth]{\imagedir/doe/examples/solar-panel-mendelu-cz-website.png}
		
		
			\see{\href{http://yint.org/solar-panel-study}{http://yint.org/solar-panel-study}}
	\end{columns}
	
	\begin{center}\rule[8mm]{4cm}{0.01cm}\end{center}
	\rule[3mm]{0.01cm}{25mm}%

\begin{frame}\frametitle{The variables affecting our experimental system}
	We may classify them in several ways:
	
	\begin{itemize}
		\item	those we know about, and those that are unknown
			\pause
		\item	variables we can control, and uncontrolled variables: \emph{we'll consider these today}
			\pause
		\item	some we can measure, and others we cannot measure: \emph{we'll consider these today}
		
	\end{itemize}	
\end{frame}

\begin{frame}\frametitle{Controllable and measurable variables and the need for randomization}
	Use a familiar example: {\color{myOrange} ginger biscuits}
	\vspace{0.25cm}
	\begin{columns}[T]
		\column{0.4\textwidth}
			We have 3 factors:
			\begin{enumerate}
				\item	\textbf{A}: baking temperature
				\item	\textbf{B}: amount of baking soda
				\item	\textbf{C}: baking time
			\end{enumerate}
			
			
			\vspace{1cm}
			
			\includegraphics[width=0.8\textwidth]{../4D/Supporting materials/4C-2-ginger-biscuits-3242839562_10c30d1aa3_o.jpg}
		
		
			\see{\href{https://secure.flickr.com/photos/babbagecabbage/3242839562}{Flickr: babbagecabbage}}
			
		
		\column{0.58\textwidth}
		
			\onslide+<2->{
			
				Order of experiments$^\ast$:
			
				\begin{tabulary}{\linewidth}{ccccc}\hline 
					\textbf{\relax Experiment } & \textbf{\relax A } & \textbf{\relax B } & \textbf{\relax C } & \textbf{\relax Time finished} \\
					\hline 
					\textbf{5} & \(-\) & \(-\) & \(+\) & 08:32\\
					\textbf{1} & \(-\) & \(-\) & \(-\) & 09:46\\
					\textbf{8} & \(+\) & \(+\) & \(+\) & 10:50\\
					\textbf{3} & \(-\) & \(+\) & \(-\) & 12:05\\
					\textbf{6} & \(+\) & \(-\) & \(+\) & 13:16\\
					\textbf{2} & \(+\) & \(-\) & \(-\) & 14:30\\
					\textbf{4} & \(+\) & \(+\) & \(-\) & 15:57\\
					\textbf{7} & \(-\) & \(+\) & \(+\) & 17:09\\
					
					 \hline
				\end{tabulary}
				
				\vspace{0.2cm}
				{\scriptsize $^\ast$ note the randomized order}
			}
	\end{columns}	
\end{frame}

\begin{frame}\frametitle{Disturbances are uncontrolled and unmeasured variables}
	
	\begin{columns}[T]
		\column{0.45\textwidth}
			Any potential impact on our system which
			\begin{itemize}
				\item	is not controlled, and
		
				\item	is not measured
			\end{itemize}
			
			\vspace{1cm}
			In the ginger biscuit example, it might include:
			\begin{itemize}
				\item	ambient humidity
				\item	ambient temperature
				\item	impurities in the ingredients
				\item	other examples ...
			\end{itemize}
		
		\column{0.48\textwidth}
			\includegraphics[width=\textwidth]{../4D/Supporting materials/4C-1-tiredness.jpg}
	\end{columns}
\end{frame}

\begin{frame}\frametitle{``Control'' and ``Measure'': let's clarify what we mean by those terms}
	
	{\color{myOrange}Given enough resources (usually money), we can control and measure most things.}
	
	\begin{columns}[T]
		\column{0.49\textwidth}
			\begin{center}\textbf{Control}\end{center}
				
			We can not actively control these:
			\begin{itemize}
				\item	ambient humidity \onslide+<2->{(unless indoors)}
				\item	ambient temperature \onslide+<2->{(unless indoors)}
			\end{itemize}
			
			\vspace{1cm}
			Practically, it might be expensive to control certain variables. For example: 
			
			impurities in the ingredients. I could buy enough
			ingredients, mix them all up, and split them into 8 pieces, so all
			8 experiments have the same ``impurity''.
		
		\column{0.48\textwidth}
			\onslide+<3->{
				\begin{center}\textbf{Measure}\end{center}
			
				Some variables are too expensive or unreliable to measure.
			
				\onslide+<4->{
					\begin{center}
						\includegraphics[width=.6\textwidth]{../4D/Supporting materials/5937291945_4e961c8baa_o-flickr-humidity-modified.png}
						\see{\href{https://secure.flickr.com/photos/sidelong/5937291945}{Flickr: sidelong}}
					\end{center}
				}
			}
			
	\end{columns}
	
\end{frame}

\begin{frame}\frametitle{Uncontrolled and unmeasurable variables can negate all your hard work}
	

	\begin{columns}[T]
		\column{0.6\textwidth}
			%Order of experiments$^\ast$:
			
			\vspace{1cm}
			\begin{tabulary}{\linewidth}{ccccc}\hline 
				\textbf{\relax Experiment } & \textbf{\relax A } & \textbf{\relax B } & \textbf{\relax C } & \textbf{\relax Time finished} \\
				\hline 
				\textbf{1} & \(-\) & \(-\) & \(-\) & 08:32\\
				\textbf{2} & \(+\) & \(-\) & \(-\) & 09:46\\
				\textbf{3} & \(-\) & \(+\) & \(-\) & 10:50\\
				\textbf{4} & \(+\) & \(+\) & \(-\) & 12:05\\
				\textbf{5} & \(-\) & \(-\) & \(+\) & 13:16\\
				\textbf{6} & \(+\) & \(-\) & \(+\) & 14:30\\
				\textbf{7} & \(-\) & \(+\) & \(+\) & 15:57\\
				\textbf{8} & \(+\) & \(+\) & \(+\) & 17:09\\
				 \hline
			\end{tabulary}
			
			\vspace{0.2cm}
			{\scriptsize $^\ast$ note the experiments are in \emph{standard order this time}}

		\column{0.38\textwidth}
	\end{columns}	
\end{frame}

\begin{frame}\frametitle{Take a moment to think about this ...}
	\begin{exampleblock}{}
		\emph{Pause the video}: what problems do you think might be caused by this confounding?
		
		\vspace{1cm}
		\begin{itemize}
			\item	Recall what the term ``confounding'' means (from the previous video)
			\item	Guess what will happen when you analyze the outcome variable, $y$
			\item	Now resume the video ...
			
		\end{itemize}
	\end{exampleblock}
\end{frame}

\begin{frame}\frametitle{\includegraphics[width=0.3\textwidth]{\imagedir/doe/examples/advice-logo.png} when experimenting with computer simulations}
	\begin{columns}[T]
		\column{0.48\textwidth}
		
			\textbf{The same as regular experiments}:
			
			\vspace{12pt}
			\begin{itemize}
				\item	you must follow a systematic method
				\item	don't ``play around'' with the software: trial-and-error
			\end{itemize}
			
			\begin{center}\rule[8mm]{4cm}{0.01cm}\end{center}
		\column{0.01\textwidth}
			\rule[3mm]{0.01cm}{25mm}%
			
		\column{0.48\textwidth}
			\textbf{Different to regular experiments:}
			\vspace{12pt}
			\begin{enumerate}
				\item	We can often run computer simulations in parallel
				\item	Computer experiments (mostly$^\ast$) are deterministic
					\begin{itemize}
						\item	i.e. if you repeat the experiments, you get the identical results
						\item	this indicates there are no disturbances that affect the outcome
						\item	this implies you do not need to randomize the order
						\item	or even repeat experiments!
						
					\end{itemize}
			\end{enumerate}
			{\scriptsize $^\ast$ {\emph{except those that use random number generators}}}
	\end{columns}
	
	\vspace{-90pt}

	 \fbox{\parbox[b][8em][t]{0.45\textwidth}{
	 	{\footnotesize
		 	Many experiments are run this \\way these days:
			\begin{itemize}
				\item	bridge/building design
				\item	chemical factory design
				\item	traffic light timing and queuing
	 			\item	stock market buy/sell strategy\\ evaluations \tiny{(see \href{http://yint.org/trading-expt}{ http://yint.org/trading-expt})}
	 		\end{itemize}
		}
	 } }
\end{frame}

\begin{frame}\frametitle{Experiments that could be problematic without randomization}
	\begin{columns}[T]
		\column{0.33\textwidth}
			\includegraphics[width=\textwidth]{../4D/Supporting materials/4C-5-Italian.png}
		
		\column{0.33\textwidth}
			\includegraphics[width=\textwidth]{../4D/Supporting materials/4C-4-memorize.png}
		
		\column{0.33\textwidth}
			\includegraphics[width=\textwidth]{../4D/Supporting materials/4C-3-gym.png}
			
	\end{columns}
\end{frame}

\begin{frame}\frametitle{Other systems experience deterioration over time: we must randomize to counteract this}
	\begin{columns}[T]
		\column{0.35\textwidth}
			Gas mileage in a vehicle: 
			
			\includegraphics[width=\textwidth]{../4D/Supporting materials/4C-6-gas-mileage.png}
			
			\vspace{2.5cm}
			{\scriptsize (Question: how would you describe the variable: ``\textbf{R} = \emph{rain on the road}\,''?)}
		
		\column{0.33\textwidth}
			
			\onslide+<2->{
				Ginger biscuit baking experiment: 
			
				\includegraphics[width=\textwidth]{../4C/Supporting materials/4C-1-tiredness.jpg}
			}
		
		\column{0.33\textwidth}
		
			\onslide+<3->{
				\includegraphics[width=.8\textwidth]{../4D/Supporting materials/7629039140_c2e2cde0c8_o-flickr-rusting.jpg}
				
				\vspace{-0.52cm}
				{\tiny \href{https://secure.flickr.com/photos/mikecogh/7629039140}{Flickr: mikecogh}}
				
				
				{\scriptsize 
					This is a slow moving disturbance: we must account for it though if we run experiments that span several months, or years.
				}
			}
			
	\end{columns}
\end{frame}

\begin{frame}\frametitle{\includegraphics[width=0.3\textwidth]{\imagedir/doe/examples/advice-logo.png}\,\, Always randomize!}
	
	\begin{exampleblock}{Why do we randomize?}
		So when we analyze the effect of the factors -- we can be almost certain they are not confound by unmeasured, and uncontrolled disturbances.
	\end{exampleblock}
	
\end{frame}

\begin{frame}\frametitle{\includegraphics[width=0.25\textwidth]{\imagedir/doe/examples/advice-logo.png}\,\,disturbances which can be measured, should be recorded}
	
	\begin{tabulary}{\linewidth}{c|ccc|cc|c}\hline 
		& \multicolumn{3}{c|}{\textbf{\relax Factors}} & \multicolumn{2}{c|}{\textbf{\relax Measured disturbances}}  & \textbf{\relax Outcome variable}                      \\
		& \multicolumn{3}{c|}{$\overbrace{ \hspace{2.2cm}}{}$}  & \multicolumn{2}{c|}{$\overbrace{ \hspace{4cm}}{}$} & $\overbrace{\hspace{3cm}}{}$ \\
		\textbf{\relax Experiment$^\ast$} & \textbf{\relax A } & \textbf{\relax B } & \textbf{\relax C } & \textbf{\relax T}emperature & \textbf{\relax H}umidity & \textbf{\relax $y$ = breakability}\\
		\hline 
		\textbf{1} & \(-\) & \(-\) & \(-\) & 23 & 32 \\
		\textbf{2} & \(+\) & \(-\) & \(-\) & 21 & 56 \\
		\textbf{3} & \(-\) & \(+\) & \(-\) & 24 & 24 \\
		\textbf{4} & \(+\) & \(+\) & \(-\) & 22 & 24 \\
		\textbf{5} & \(-\) & \(-\) & \(+\) & 23 & 30 \\
		\textbf{6} & \(+\) & \(-\) & \(+\) & 22 & 54 \\
		\textbf{7} & \(-\) & \(+\) & \(+\) & 23 & 36 \\
		\textbf{8} & \(+\) & \(+\) & \(+\) & 24 & 24 \\
		\textbf{9} &  &  &  &  & \\
		$\vdots$   &  &  &  &  & \\ \hline
	\end{tabulary}
	
	\vspace{0.05cm}
	{\scriptsize $^\ast$ Experiments were run in random order; but are reported in standard order}
	
	\onslide+<2->{
		\vspace{0.25cm}
		{\small ``{\color{purple} measured disturbances}'' = ``variables which are measured, but not controlled'' = ``{\color{purple}  covariates}''}
	}	
\end{frame}

\begin{frame}\frametitle{Using the covariate information}
	
	
	\begin{tabulary}{\linewidth}{c|ccc|cc|c}\hline 
		& \multicolumn{3}{c|}{\textbf{\relax Factors}} & \multicolumn{2}{c|}{\textbf{\relax Measured disturbances}}  & \textbf{\relax Outcome variable}                      \\
		& \multicolumn{3}{c|}{$\overbrace{ \hspace{2.2cm}}{}$}  & \multicolumn{2}{c|}{$\overbrace{ \hspace{4cm}}{}$} & $\overbrace{\hspace{3cm}}{}$ \\
		\textbf{\relax Experiment$^\ast$} & \textbf{\relax A } & \textbf{\relax B } & \textbf{\relax C } & \textbf{\relax T}emperature & \textbf{\relax H}umidity & \textbf{\relax $y$ = breakability}\\
		\hline 
		\textbf{1} & \(-\) & \(-\) & \(-\) & 23 & 32 & 4 \\
		\textbf{2} & \(+\) & \(-\) & \(-\) & 21 & 56 & 5 \\
		\textbf{3} & \(-\) & \(+\) & \(-\) & 24 & 24 & 5 \\
		\textbf{4} & \(+\) & \(+\) & \(-\) & 22 & 24 & 6 \\
		\textbf{5} & \(-\) & \(-\) & \(+\) & 23 & 77 & 3 \\
		\textbf{6} & \(+\) & \(-\) & \(+\) & 22 & 54 & 8 \\
		\textbf{7} & \(-\) & \(+\) & \(+\) & 23 & 36 & 6 \\
		\textbf{8} & \(+\) & \(+\) & \(+\) & 24 & 24 & 9 \\
		\textbf{9} & $0$   & $0$   & $0$   & 23 & 39 & 5 \\
		$\vdots$   &  &  &  &  & \\ \hline
	\end{tabulary}
	
	\onslide+<2->{
		\vspace{0.25cm}
		$\hat{y} = b_0 + b_\text{A}x_\text{A} + b_\text{B}x_\text{B} + b_\text{C}x_\text{C} +  b_\text{AB}x_\text{A}x_\text{B} + \dots +\underbrace{ \color{myOrange} b_\text{T}\,x_\text{T} \,\, + \,\, b_\text{H}\,x_\text{H}}_{\mathclap{\text{covariate terms are added to the model$^\ast$}}}$
		
		
		
		%\begin{flushright}
		{\tiny $^\ast$ but you will require more than 8 experiments to build this model}
		%\end{flushright}	
	}
\end{frame}

\begin{frame}\frametitle{Cellphone app example: ``CalApp''}	
	\begin{columns}[T]
		\column{0.25\textwidth}
			\includegraphics[width=\textwidth]{../4D/Supporting materials/4C-7-cell-phone-2830319467_1faaecc974_o-flickr.jpg}
			\\
			\tiny{\href{https://secure.flickr.com/photos/williamhook/2830319467/}{Flickr: williamhook}}
			
		\column{0.80\textwidth}
		
			The app has various upgradable features, called ``in-app purchases''
			
			\begin{itemize}
				\item	sync-to-other-devices
				\item	text message reminders
				\item	integrate with desktop calendar
			\end{itemize}
			
			\vspace{1cm}
			
			
		\onslide+<2->{	
		   	 \fbox{\parbox[b][5em][t]{0.65\textwidth}{
			 	Your marketing idea for experimenting: 
				\begin{itemize}
					\item	each test group has 2000 people
					\item	calculate the percentage using the app after 60 days; that's your outcome, $y$
				\end{itemize}
		   	 } }
		}
		 
	\end{columns}	
\end{frame}

\begin{frame}\frametitle{Cellphone app example: ``CalApp''}	
	
			{\color{myOrange} The factors you are actively testing} {\small (confirm whether they are controllable)}
			\vspace{12pt}
			
			\begin{tabulary}{\linewidth}{l|ll}\hline
				& \textbf{\relax Low level $-$} & \textbf{\relax High level $+$}\\ \hline  \\
				\textbf{A}: ``Promotion'' & 1 free in-app upgrade & 30-day trial of all features\\ \\
				\textbf{B}: ``Message'' & \parbox[t]{5.cm}{``CalApp has your schedule available at your fingertips, on any device.''} & \parbox[t]{5.5cm}{``CalApp features are configurable; only pay for the features you want.''} \\ \\
				\textbf{C}: ``Price'' & in-app purchase price is 89c &  in-app purchase price is 99c  \\& \\\hline
			\end{tabulary}	
\end{frame}

\begin{frame}\frametitle{Cellphone app example: test your understanding}
	
	
	Are these ``disturbances'' (\emph{not measured, not controlled}), or \\
	\qquad\quad\,\,\,\,\,\, ``covariates'' \,\,\,\, (\emph{measured, but not controlled}), or \\
	\qquad\quad\,\,\,\,\,\, ``neither of these'':
	
	\vspace{0.5cm}
	
	\begin{itemize}
		\item	\textbf{E}: smartphone user's age
		\item	\textbf{N}: smartphone user's gender
		\item	\textbf{S}: smartphone user's connection speed (e.g. cell, or wifi)
		\item	\textbf{R}: amount of free memory (RAM) on the device
		\item	\textbf{F}: whether the advert/message is delivered via ad network G, or ad network H
		\item	\textbf{D}: if the user's phone is Android or Apple		
	\end{itemize}
	
	\vspace{0.5cm}
	
	Participate in the forums and share your opinion at \href{http://yint.org/cal-app}{http://yint.org/cal-app}
\end{frame}

\end{comment}

\begin{frame}\frametitle{The variables affecting our experimental system}
	We may classify them in several ways:
	
	\begin{itemize}
		\item	those we know about, and those that are unknown
			\pause
		\item	variables we can control, and uncontrolled variables: \emph{we'll consider these today}
			\pause
		\item	some we can measure, and others we cannot measure: \emph{we'll consider these today}
		
	\end{itemize}	
\end{frame}

\begin{frame}\frametitle{Controllable and measurable variables and the need for randomization}
	Use a familiar example: {\color{myOrange} ginger biscuits}
	\vspace{0.25cm}
	\begin{columns}[T]
		\column{0.4\textwidth}
			We have 3 factors:
			\begin{enumerate}
				\item	\textbf{A}: baking temperature
				\item	\textbf{B}: amount of baking soda
				\item	\textbf{C}: baking time
			\end{enumerate}
			
			
			\vspace{1cm}
			
			\includegraphics[width=0.8\textwidth]{../4D/Supporting materials/4C-2-ginger-biscuits-3242839562_10c30d1aa3_o.jpg}
		
		
			\see{\href{https://secure.flickr.com/photos/babbagecabbage/3242839562}{Flickr: babbagecabbage}}
			
		
		\column{0.58\textwidth}
		
			\onslide+<2->{
			
				Order of experiments$^\ast$:
			
				\begin{tabulary}{\linewidth}{ccccc}\hline 
					\textbf{\relax Experiment } & \textbf{\relax A } & \textbf{\relax B } & \textbf{\relax C } & \textbf{\relax Time finished} \\
					\hline 
					\textbf{5} & \(-\) & \(-\) & \(+\) & 08:32\\
					\textbf{1} & \(-\) & \(-\) & \(-\) & 09:46\\
					\textbf{8} & \(+\) & \(+\) & \(+\) & 10:50\\
					\textbf{3} & \(-\) & \(+\) & \(-\) & 12:05\\
					\textbf{6} & \(+\) & \(-\) & \(+\) & 13:16\\
					\textbf{2} & \(+\) & \(-\) & \(-\) & 14:30\\
					\textbf{4} & \(+\) & \(+\) & \(-\) & 15:57\\
					\textbf{7} & \(-\) & \(+\) & \(+\) & 17:09\\
					
					 \hline
				\end{tabulary}
				
				\vspace{0.2cm}
				{\scriptsize $^\ast$ note the randomized order}
			}
			
			\vspace{0.75cm}
			%{\scriptsize See forum postings at \href{http://yint.org/forum-randomization}{http://yint.org/forum-randomization}}
	\end{columns}	
\end{frame}

\begin{frame}\frametitle{Disturbances are uncontrolled and unmeasured variables}
	
	\begin{columns}[T]
		\column{0.45\textwidth}
			Any potential impact on our system which
			\begin{itemize}
				\item	is not controlled, and
		
				\item	is not measured
			\end{itemize}
			
			\vspace{1cm}
			In the ginger biscuit example, it might include:
			\begin{itemize}
				\item	ambient humidity
				\item	ambient temperature
				\item	impurities in the ingredients
				\item	other examples ...
			\end{itemize}
		
		\column{0.48\textwidth}
			\includegraphics[width=\textwidth]{../4D/Supporting materials/4C-1-tiredness.jpg}
	\end{columns}
\end{frame}

\begin{frame}\frametitle{``Control'' and ``Measure'': let's clarify what we mean by those terms}
	
	{\color{myOrange}Given enough resources (usually money), we can control and measure most things.}
	
	\begin{columns}[T]
		\column{0.49\textwidth}
			\begin{center}\textbf{Control}\end{center}
				
			We can not actively control these:
			\begin{itemize}
				\item	ambient humidity \onslide+<2->{(unless indoors)}
				\item	ambient temperature \onslide+<2->{(unless indoors)}
			\end{itemize}
			
			\vspace{1cm}
			%Practically, it might be expensive to control certain variables. For example: 
			
			%impurities in the ingredients. I could buy enough
			%ingredients, mix them all up, and split them into 8 pieces, so all
			%8 experiments have the same ``impurity''.
		
		\column{0.48\textwidth}
			\onslide+<3->{
				\begin{center}\textbf{Measure}\end{center}
			
				Some variables are too expensive or unreliable to measure.
			
				\onslide+<4->{
					\begin{center}
						\includegraphics[width=.6\textwidth]{../4D/Supporting materials/5937291945_4e961c8baa_o-flickr-humidity-modified.png}
						\see{\href{https://secure.flickr.com/photos/sidelong/5937291945}{Flickr: sidelong}}
					\end{center}
				}
			}
			
	\end{columns}
	
\end{frame}

\begin{frame}\frametitle{Uncontrolled and unmeasurable variables can negate all your hard work}
	

	\begin{columns}[T]
		\column{0.6\textwidth}
			%Order of experiments$^\ast$:
			
			\vspace{1cm}
			\begin{tabulary}{\linewidth}{ccccc}\hline 
				\textbf{\relax Experiment } & \textbf{\relax A } & \textbf{\relax B } & \textbf{\relax C } & \textbf{\relax Time finished} \\
				\hline 
				\textbf{1} & \(-\) & \(-\) & \(-\) & 08:32\\
				\textbf{2} & \(+\) & \(-\) & \(-\) & 09:46\\
				\textbf{3} & \(-\) & \(+\) & \(-\) & 10:50\\
				\textbf{4} & \(+\) & \(+\) & \(-\) & 12:05\\
				\textbf{5} & \(-\) & \(-\) & \(+\) & 13:16\\
				\textbf{6} & \(+\) & \(-\) & \(+\) & 14:30\\
				\textbf{7} & \(-\) & \(+\) & \(+\) & 15:57\\
				\textbf{8} & \(+\) & \(+\) & \(+\) & 17:09\\
				 \hline
			\end{tabulary}
			
			\vspace{0.2cm}
			{\scriptsize $^\ast$ note the experiments are in \emph{standard order this time}}

		\column{0.38\textwidth}
	\end{columns}	
\end{frame}

\begin{frame}\frametitle{Take a moment to think about this ...}
	\begin{exampleblock}{}
		\emph{Pause the video}: what problems do you think might be caused by this confounding?
		
		\vspace{1cm}
		\begin{itemize}
			\item	Recall what the term ``confounding'' means (from the previous video)
			\item	Guess what will happen when you analyze the outcome variable, $y$
			\item	Now resume the video ...
			
		\end{itemize}
	\end{exampleblock}
\end{frame}

\begin{frame}\frametitle{\includegraphics[width=0.3\textwidth]{\imagedir/doe/examples/advice-logo.png} when experimenting with computer simulations}
	\begin{columns}[T]
		\column{0.48\textwidth}
		
			\textbf{The same as regular experiments}:
			
			\vspace{12pt}
			\begin{itemize}
				\item	you must follow a systematic method
				\item	don't ``play around'' with the software: trial-and-error
			\end{itemize}
			
			%\begin{center}\rule[8mm]{4cm}{0.01cm}\end{center}
			
			\vspace{18pt}

			 \fbox{\parbox[b][7em][t]{\textwidth}{
			 	{\footnotesize
				 	Many experiments are simulations:
					\begin{itemize}
						\item	bridge/building design
						\item	chemical factory design
						\item	improve traffic light timing and queuing
			 			\item	test a stock market buy/sell strategy
			 		\end{itemize}}
			 } }
			 
				
		\column{0.01\textwidth}
			\rule[3mm]{0.01cm}{25mm}%
			
		\column{0.48\textwidth}
		\onslide+<2->{
			\textbf{Different to regular experiments:}
			\vspace{12pt}
			\begin{enumerate}
				\item	We can often run computer simulations in parallel
				\item	Computer experiments (mostly$^\ast$) are deterministic
					\begin{itemize}
						\item	i.e. if you repeat the experiments, you get the identical results
						\item	this indicates there are no disturbances that affect the outcome
						\item	this implies you do not need to randomize the order
						\item	or even repeat experiments!
						
					\end{itemize}
			\end{enumerate}
			{\scriptsize $^\ast$ {\emph{except those that have a random component}}}
		}
	\end{columns}
\end{frame}

\begin{frame}\frametitle{Experiments that could be problematic without randomization}
	\begin{columns}[T]
		\column{0.33\textwidth}
			\includegraphics[width=\textwidth]{../4D/Supporting materials/4C-5-Italian.png}
		
		\column{0.33\textwidth}
			\includegraphics[width=\textwidth]{../4D/Supporting materials/4C-4-memorize.png}
		
		\column{0.33\textwidth}
			\includegraphics[width=\textwidth]{../4D/Supporting materials/4C-3-gym.png}
			
	\end{columns}
\end{frame}

\begin{frame}\frametitle{Other systems experience deterioration over time: we must randomize to counteract this}
	\begin{columns}[T]
		\column{0.35\textwidth}
			Gas mileage in a vehicle: 
			
			\includegraphics[width=\textwidth]{../4D/Supporting materials/4C-6-gas-mileage.png}
			
			\vspace{2.5cm}
			{\scriptsize (Question: how would you describe the variable: ``\textbf{R} = \emph{rain on the road}\,''?)}
		
		\column{0.33\textwidth}
			
			\onslide+<2->{
				Ginger biscuit baking experiment: 
			
				\includegraphics[width=\textwidth]{../4D/Supporting materials/4C-1-tiredness.jpg}
			}
		
		\column{0.33\textwidth}
		
			\onslide+<3->{
				\includegraphics[width=.8\textwidth]{../4D/Supporting materials/7629039140_c2e2cde0c8_o-flickr-rusting.jpg}
				
				\vspace{-0.52cm}
				{\tiny \href{https://secure.flickr.com/photos/mikecogh/7629039140}{Flickr: mikecogh}}
				
				
				{\scriptsize 
					This is a slow moving disturbance: we must account for it though if we run experiments that span several months, or years.
				}
			}
			
	\end{columns}
\end{frame}

\begin{frame}\frametitle{\includegraphics[width=0.3\textwidth]{\imagedir/doe/examples/advice-logo.png}\,\, Always randomize!}
	
	\begin{exampleblock}{Why do we randomize?}
		So when we analyze the effect of the factors -- we can be almost certain they are not confound by unmeasured, and uncontrolled disturbances.
	\end{exampleblock}
	
\end{frame}

\begin{frame}\frametitle{\includegraphics[width=0.25\textwidth]{\imagedir/doe/examples/advice-logo.png}\,\,disturbances which can be measured, should be recorded}
	
	\begin{tabulary}{\linewidth}{c|ccc|cc|c}\hline 
		& \multicolumn{3}{c|}{\textbf{\relax Factors}} & \multicolumn{2}{c|}{\textbf{\relax Measured disturbances}}  & \textbf{\relax Outcome variable}                      \\
		& \multicolumn{3}{c|}{$\overbrace{ \hspace{2.2cm}}{}$}  & \multicolumn{2}{c|}{$\overbrace{ \hspace{4cm}}{}$} & $\overbrace{\hspace{3cm}}{}$ \\
		\textbf{\relax Experiment$^\ast$} & \textbf{\relax A } & \textbf{\relax B } & \textbf{\relax C } & \textbf{\relax T}emperature & \textbf{\relax H}umidity & \textbf{\relax $y$ = breakability}\\
		\hline 
		\textbf{1} & \(-\) & \(-\) & \(-\) & 23 & 32 \\
		\textbf{2} & \(+\) & \(-\) & \(-\) & 21 & 56 \\
		\textbf{3} & \(-\) & \(+\) & \(-\) & 24 & 24 \\
		\textbf{4} & \(+\) & \(+\) & \(-\) & 22 & 24 \\
		\textbf{5} & \(-\) & \(-\) & \(+\) & 23 & 30 \\
		\textbf{6} & \(+\) & \(-\) & \(+\) & 22 & 54 \\
		\textbf{7} & \(-\) & \(+\) & \(+\) & 23 & 36 \\
		\textbf{8} & \(+\) & \(+\) & \(+\) & 24 & 24 \\
		\textbf{9} &  &  &  &  & \\
		$\vdots$   &  &  &  &  & \\ \hline
	\end{tabulary}
	
	\vspace{0.05cm}
	{\scriptsize $^\ast$ Experiments were run in random order; but are reported in standard order}
	
	\onslide+<2->{
		\vspace{0.25cm}
		{\small ``{\color{purple} measured disturbances}'' = ``variables which are measured, but not controlled'' = ``{\color{purple}  covariates}''}
	}	
\end{frame}

\begin{frame}\frametitle{Using the covariate information}
	
	
	\begin{tabulary}{\linewidth}{c|ccc|cc|c}\hline 
		& \multicolumn{3}{c|}{\textbf{\relax Factors}} & \multicolumn{2}{c|}{\textbf{\relax Measured disturbances}}  & \textbf{\relax Outcome variable}                      \\
		& \multicolumn{3}{c|}{$\overbrace{ \hspace{2.2cm}}{}$}  & \multicolumn{2}{c|}{$\overbrace{ \hspace{4cm}}{}$} & $\overbrace{\hspace{3cm}}{}$ \\
		\textbf{\relax Experiment$^\ast$} & \textbf{\relax A } & \textbf{\relax B } & \textbf{\relax C } & \textbf{\relax T}emperature & \textbf{\relax H}umidity & \textbf{\relax $y$ = breakability}\\
		\hline 
		\textbf{1} & \(-\) & \(-\) & \(-\) & 23 & 32 & 4 \\
		\textbf{2} & \(+\) & \(-\) & \(-\) & 21 & 56 & 5 \\
		\textbf{3} & \(-\) & \(+\) & \(-\) & 24 & 24 & 5 \\
		\textbf{4} & \(+\) & \(+\) & \(-\) & 22 & 24 & 6 \\
		\textbf{5} & \(-\) & \(-\) & \(+\) & 23 & 77 & 3 \\
		\textbf{6} & \(+\) & \(-\) & \(+\) & 22 & 54 & 8 \\
		\textbf{7} & \(-\) & \(+\) & \(+\) & 23 & 36 & 6 \\
		\textbf{8} & \(+\) & \(+\) & \(+\) & 24 & 24 & 9 \\
		\textbf{9} & $0$   & $0$   & $0$   & 23 & 39 & 5 \\
		$\vdots$   &  &  &  &  & \\ \hline
	\end{tabulary}
	
	\onslide+<2->{
		\vspace{0.25cm}
		$\hat{y} = b_0 + b_\text{A}x_\text{A} + b_\text{B}x_\text{B} + b_\text{C}x_\text{C} +  b_\text{AB}x_\text{A}x_\text{B} + \dots +\underbrace{ \color{myOrange} b_\text{T}\,x_\text{T} \,\, + \,\, b_\text{H}\,x_\text{H}}_{\mathclap{\text{covariate terms are added to the model$^\ast$}}}$
		
		
		
		%\begin{flushright}
		{\tiny $^\ast$ but you will require more than 8 experiments to build this model}
		%\end{flushright}	
	}
\end{frame}

\begin{frame}\frametitle{Cellphone app example: ``CalApp''}	
	\begin{columns}[T]
		\column{0.25\textwidth}
			\includegraphics[width=\textwidth]{../4D/Supporting materials/4C-7-cell-phone-2830319467_1faaecc974_o-flickr.jpg}
			\\
			\tiny{\href{https://secure.flickr.com/photos/williamhook/2830319467/}{Flickr: williamhook}}
			
		\column{0.80\textwidth}
		
			The app has various upgradable features, called ``in-app purchases''
			
			\begin{itemize}
				\item	sync-to-other-devices
				\item	text message reminders
				\item	integrate with desktop calendar
			\end{itemize}
			
			\vspace{1cm}
			
			
		\onslide+<2->{	
		   	 \fbox{\parbox[b][5em][t]{0.65\textwidth}{
			 	Your marketing idea for experimenting: 
				\begin{itemize}
					\item	each test group has 2000 people
					\item	calculate the percentage using the app after 60 days; that's your outcome, $y$
				\end{itemize}
		   	 } }
		}
		
		\vspace{0.5cm}
		{\scriptsize Read the Harvard Business Review article at \href{http://yint.org/hbr-article}{http://yint.org/hbr-article}}
		 
	\end{columns}	
\end{frame}

\begin{frame}\frametitle{Cellphone app example: ``CalApp''}	
	
			{\color{myOrange} The factors you are actively testing} {\small (confirm whether they are controllable)}
			\vspace{12pt}
			
			\begin{tabulary}{\linewidth}{l|ll}\hline
				& \textbf{\relax Low level $-$} & \textbf{\relax High level $+$}\\ \hline  \\
				\textbf{A}: ``Promotion'' & 1 free in-app upgrade & 30-day trial of all features\\ \\
				\textbf{B}: ``Message'' & \parbox[t]{5.cm}{``CalApp has your schedule available at your fingertips, on any device.''} & \parbox[t]{5.5cm}{``CalApp features are configurable; only pay for the features you want.''} \\ \\
				\textbf{C}: ``Price'' & in-app purchase price is 89c &  in-app purchase price is 99c  \\& \\\hline
			\end{tabulary}	
\end{frame}

\begin{frame}\frametitle{Cellphone app example: test your understanding}
	
	
	Are these ``disturbances'' (\emph{not measured, not controlled}), or \\
	\qquad\quad\,\,\,\,\,\, ``covariates'' \,\,\,\, (\emph{measured, but not controlled}), or \\
	\qquad\quad\,\,\,\,\,\, ``neither of these'':
	
	\vspace{0.5cm}
	
	\begin{itemize}
		\item	\textbf{E}: smartphone user's age
		\item	\textbf{N}: smartphone user's gender
		\item	\textbf{S}: smartphone user's connection speed (e.g. cell, or wifi)
		\item	\textbf{R}: amount of free memory (RAM) on the device
		\item	\textbf{F}: whether the advert/message is delivered via ad network G, or ad network H
		\item	\textbf{D}: if the user's phone is Android or Apple		
	\end{itemize}
	
	\vspace{0.5cm}
	
	Participate in the forums and share your opinion at \href{http://yint.org/cal-app}{http://yint.org/cal-app}
\end{frame}

% /Users/kevindunn/Dropbox/Coursera/Media/All-course-slides/classes/CourseraMOOC-class-4E.tex

\begin{frame}\frametitle{}
	\begin{center}
		\includegraphics[width=.97\textwidth]{\imagedir/doe/disturbance-classification-MOOC.png}
	\end{center}
	%They should fall into one of these 3 categories.
\end{frame}

\begin{frame}\frametitle{Can you classify all the variables in your experimental system?}
	\begin{center}
		\includegraphics[width=.7\textwidth]{\imagedir/doe/disturbance-classification-MOOC.png}
	\end{center}
	They should fall into one of these 3 categories.
\end{frame}

\begin{frame}\frametitle{The ``CalApp'' example continues}
	
	\begin{columns}[t]
	
		\column{1\textwidth}
			\begin{tabulary}{\linewidth}{l|ll}\hline
				& \textbf{\relax Low level $-$} & \textbf{\relax High level $+$}\\ \hline  \\
				\textbf{A}: ``Promotion'' & 1 free in-app upgrade & 30-day trial of all features\\ \\
				\textbf{B}: ``Message'' & \parbox[t]{5.cm}{``CalApp has your schedule available at your fingertips, on any device.''} & \parbox[t]{5.5cm}{``CalApp features are configurable; only pay for the features you want.''} \\ \\
				\textbf{C}: ``Price'' & in-app purchase price is 89c &  in-app purchase price is 99c  \\& \\\hline
			\end{tabulary}
			
			
	\end{columns}
		
\end{frame}

\begin{frame}\frametitle{A subtlety: characteristics of a nuisance factor}
	\begin{columns}[T]
		\column{0.25\textwidth}

			\includegraphics[width=\textwidth]{../4E/Supporting files/flickr-6449132251_dd669e13dd_b-nuisance.jpg}

			\see{\href{https://secure.flickr.com/photos/nickwebb/6449132251}{Flickr: nickwebb}}
					
			
		\column{0.75\textwidth}
		
		\begin{itemize}
			\item	the factor does vary during your experiments
			\item	it is controllable, and it is measurable (just like a regular factor)
			\item	the factor is not the focus of your experiment

		\end{itemize}
		
			\onslide+<2->{
				

				\color{myOrange} How do we deal with a nuisance factor to avoid bias?
			}
			\onslide+<3->{
				\begin{itemize}
					\item	randomization will minimize the effect, but not eliminate it
					\item	your experiments should be successful, despite the nuisance factor
					\item	we use a process called {\color{purple} blocking}
				\end{itemize}
			}			
	\end{columns}
	
	\onslide+<4->{
		\vspace{0.5cm}
		
		\hbox{\hspace{-1.5em}
			\fbox{\parbox[b][3.5em][t]{1.04\textwidth}{
				{\color{myGreen} ``Does the process have to work successfully with different levels of the nuisance variable?''}\\
				\hbox{\hspace{1.5em}If \textbf{yes}: you must actively plan for blocking (that's the next topic)}\\
				\hbox{\hspace{1.5em}If \textbf{no}: that indicates you have good control over your system}\\
			}}
		}
		
	}
\end{frame}

\begin{frame}\frametitle{Some examples of nuisance factors}
	\begin{columns}[t]
		\column{0.25\textwidth}
			\textbf{CalApp}
			
			\includegraphics[width=\textwidth]{../4D/Supporting materials/4C-7-cell-phone-2830319467_1faaecc974_o-flickr.jpg}
			\\
			{\tiny{\href{https://secure.flickr.com/photos/williamhook/2830319467/}{Flickr: williamhook}}}
			
			\vspace{1cm}
			Apple or Android 
		
		
			\column{0.25\textwidth}
			\onslide+<2->{
				\textbf{Baking}
				\includegraphics[width=\textwidth]{../4E/Supporting files/flour-and-flour-cropped.png}
		
				\vspace{1.1cm}
				Brand of flour
			}
			
		\column{0.25\textwidth}
			\onslide+<3->{
				\textbf{Shift work}
				%\vspace{0.1cm}
				\includegraphics[width=\textwidth]{../4E/Supporting files/flickr-4140348113_f0efc8235b_z-clock.jpg}
				\\
				{\tiny{\href{https://secure.flickr.com/photos/leehaywood/4140348113}{Flickr: leehaywood}}}
				
				
				
				\vspace{0.8cm}
				Day shift or night shift
			}
		
		\column{0.25\textwidth}
			\onslide+<4->{
				\textbf{Gas mileage}
				
				%\vspace{-0.1cm}
				\includegraphics[width=\textwidth]{../4D/Supporting materials/4C-6-gas-mileage.png}
				
				\vspace{1.65cm}
				Driver 1 or driver 2
			}
			
	\end{columns}
			
\end{frame}

\begin{frame}\frametitle{The ``CalApp'' example continues}
	
	\begin{columns}[t]
	
		\column{1\textwidth}
			\begin{tabulary}{\linewidth}{l|ll}\hline
				& \textbf{\relax Low level $-$} & \textbf{\relax High level $+$}\\ \hline  \\
				\textbf{A}: ``Promotion'' & 1 free in-app upgrade & 30-day trial of all features\\ \\
				\textbf{B}: ``Message'' & \parbox[t]{5.cm}{``CalApp has your schedule available at your fingertips, on any device.''} & \parbox[t]{5.5cm}{``CalApp features are configurable; only pay for the features you want.''} \\ \\
				\textbf{C}: ``Price'' & in-app purchase price is 89c &  in-app purchase price is 99c  \\& \\\hline
			\end{tabulary}
			
			
	\end{columns}
	
	
\end{frame}

\begin{frame}\frametitle{Planning for blocking: {\color{myOrange} add a new factor column to the standard order table}}
	
	\newcommand{\white}{\color{white}}
	\begin{tabulary}{\linewidth}{ccccc}\hline 
		\multirow{2}{*}{\textbf{\relax Run }} & \textbf{\relax A } & \textbf{\relax B } & \textbf{\relax C } & \textbf{\relax D	}  \\
		 & \scriptsize ``Promotion'' & \scriptsize ``Message'' & \scriptsize ``Price'' & \scriptsize ``Apple or Android'' \\
		\hline 
		1 & \(-\) & \(-\) & \(-\) & \\
		2 & \(+\) & \(-\) & \(-\) & \\
		3 & \(-\) & \(+\) & \(-\) & \\
		4 & \(+\) & \(+\) & \(-\) & \\
		5 & \(-\) & \(-\) & \(+\) & \\
		6 & \(+\) & \(-\) & \(+\) & \\
		7 & \(-\) & \(+\) & \(+\) & \\
		8 & \(+\) & \(+\) & \(+\) & \\
		 \hline
	\end{tabulary}
	
	\begin{itemize}
	 	\item	Factor \textbf{D}: experiment with either  Android $(-)$ or Apple $(+)$ users \pause
	
	 	\item	This would require 16 experiments for a full factorial in factors \textbf{A}, \textbf{B}, \textbf{C}, and \textbf{D} \pause
	 	\item	but, we only want 8 experiments
	\end{itemize}
\end{frame}

\begin{frame}\frametitle{Planning for blocking: {\color{myOrange} add a new factor column to the standard order table}}
	
	\newcommand{\apple}{\scriptsize ~~\,Apple}
	\newcommand{\andrd}{\scriptsize Android}
	\begin{tabulary}{\linewidth}{ccccc}\hline 
		\multirow{2}{*}{\textbf{\relax Run }} & \textbf{\relax A } & \textbf{\relax B } & \textbf{\relax C } & \textbf{\relax D=ABC}  \\
		 & \scriptsize ``Promotion'' & \scriptsize ``Message'' & \scriptsize ``Price'' & \scriptsize ``Apple or Android'' \\
		\hline 
		1 & \(-\) & \(-\) & \(-\) & $-$ \andrd \\
		2 & \(+\) & \(-\) & \(-\) & $+$ \apple \\
		3 & \(-\) & \(+\) & \(-\) & $+$ \apple \\
		4 & \(+\) & \(+\) & \(-\) & $-$ \andrd \\
		5 & \(-\) & \(-\) & \(+\) & $+$ \apple \\
		6 & \(+\) & \(-\) & \(+\) & $-$ \andrd \\
		7 & \(-\) & \(+\) & \(+\) & $-$ \andrd \\
		8 & \(+\) & \(+\) & \(+\) & $+$ \apple \\
		 \hline
	\end{tabulary}
	
	\begin{itemize}
	 	\item	Factor \textbf{D}: experiment with either  Android $(-)$ or Apple $(+)$ users 
	 	\item	This would require 16 experiments for a full factorial in factors \textbf{A}, \textbf{B}, \textbf{C}, and \textbf{D}
	 	\item	but, we only want 8 experiments
	\end{itemize}
\end{frame}

\begin{frame}\frametitle{Visualizing the blocking procedure on a cube plot}
	
	\newcommand{\apple}{\scriptsize ~~\,Apple}
	\newcommand{\andrd}{\scriptsize Android}
	\begin{columns}[T]
		\column{0.5\textwidth}
			\begin{tabulary}{\linewidth}{ccccc}\hline 
				\multirow{2}{*}{\textbf{\relax Run }} & \textbf{\relax A } & \textbf{\relax B } & \textbf{\relax C } & \textbf{\relax D=ABC}  \\
				 & \scriptsize ``Promotion'' & \scriptsize ``Message'' & \scriptsize ``Price'' & \scriptsize ``Apple or Android'' \\
				\hline 
				1 & \(-\) & \(-\) & \(-\) & $-$ \andrd \\
				2 & \(+\) & \(-\) & \(-\) & $+$ \apple \\
				3 & \(-\) & \(+\) & \(-\) & $+$ \apple \\
				4 & \(+\) & \(+\) & \(-\) & $-$ \andrd \\
				5 & \(-\) & \(-\) & \(+\) & $+$ \apple \\
				6 & \(+\) & \(-\) & \(+\) & $-$ \andrd \\
				7 & \(-\) & \(+\) & \(+\) & $-$ \andrd \\
				8 & \(+\) & \(+\) & \(+\) & $+$ \apple \\
				 \hline
			\end{tabulary}
		\column{0.1\textwidth}
		\column{0.4\textwidth}
		
			\vspace{1cm}
			\centerline{\includegraphics[width=\textwidth]{\imagedir/doe/examples/half-fraction-in-3-factors-Apple-Android.png}}
			
	\end{columns}
\end{frame}

\begin{frame}\frametitle{Planning for blocking: {\color{myOrange} add a new factor column to the standard order table}}
	
	\newcommand{\apple}{\scriptsize ~~\,Apple}
	\newcommand{\andrd}{\scriptsize Android}
	\newcommand{\white}{\color{white}}
	\begin{tabulary}{\linewidth}{ccccccc}\hline 
		\multirow{2}{*}{\textbf{\relax Run }} & \textbf{\relax A } & \textbf{\relax B } & \textbf{\relax C } & \textbf{\relax D = ABC	} & \textbf{\relax Outcome, $y$	}\\
		 & \scriptsize ``Promotion'' & \scriptsize ``Message'' & \scriptsize ``Price'' & \scriptsize ``Apple or Android'' \\
		\hline 
		1 & \(-\) & \(-\) & \(-\) & $-$ \andrd & $y_{(1)} \white + g = \widetilde{y}_{(1)}$\\
		2 & \(+\) & \(-\) & \(-\) & $+$ \apple & $y_{(2)} \white + h = \mathring{y}_{(2)}$\\
		3 & \(-\) & \(+\) & \(-\) & $+$ \apple & $y_{(3)} \white + h = \mathring{y}_{(3)}$\\
		4 & \(+\) & \(+\) & \(-\) & $-$ \andrd & $y_{(4)} \white + g = \widetilde{y}_{(4)}$\\
		5 & \(-\) & \(-\) & \(+\) & $+$ \apple & $y_{(5)} \white + h = \mathring{y}_{(5)}$\\
		6 & \(+\) & \(-\) & \(+\) & $-$ \andrd & $y_{(6)} \white + g = \widetilde{y}_{(6)}$\\
		7 & \(-\) & \(+\) & \(+\) & $-$ \andrd & $y_{(7)} \white + g = \widetilde{y}_{(7)}$\\
		8 & \(+\) & \(+\) & \(+\) & $+$ \apple & $y_{(8)} \white + h = \mathring{y}_{(8)}$\\
		 \hline
	\end{tabulary}
	
	\vspace{0.5cm}
	
	\begin{itemize}
		\item	Android users: ${\white\widetilde{y}_{(i)} =}\,\, y_{(i)} + g$
		\item	Apple users:\,\,\,\,\, ${\white \mathring{y}_{(i)} =}\,\, y_{(i)} + h$
	\end{itemize}
	
\end{frame}

\begin{frame}\frametitle{Planning for blocking: {\color{myOrange} add a new factor column to the standard order table}}
	
	\newcommand{\apple}{\scriptsize ~~\,Apple}
	\newcommand{\andrd}{\scriptsize Android}
	\newcommand{\white}{}
	\begin{tabulary}{\linewidth}{ccccccc}\hline 
		\multirow{2}{*}{\textbf{\relax Run }} & \textbf{\relax A } & \textbf{\relax B } & \textbf{\relax C } & \textbf{\relax D = ABC	} & \textbf{\relax Outcome, $y$	}\\
		 & \scriptsize ``Promotion'' & \scriptsize ``Message'' & \scriptsize ``Price'' & \scriptsize ``Apple or Android'' \\
		\hline 
		1 & \(-\) & \(-\) & \(-\) & $-$ \andrd & $y_{(1)} + g = \widetilde{y}_{(1)}$\\
		2 & \(+\) & \(-\) & \(-\) & $+$ \apple & $y_{(2)} + h = \mathring{y}_{(2)}$\\
		3 & \(-\) & \(+\) & \(-\) & $+$ \apple & $y_{(3)} + h = \mathring{y}_{(3)}$\\
		4 & \(+\) & \(+\) & \(-\) & $-$ \andrd & $y_{(4)} + g = \widetilde{y}_{(4)}$\\
		5 & \(-\) & \(-\) & \(+\) & $+$ \apple & $y_{(5)} + h = \mathring{y}_{(5)}$\\
		6 & \(+\) & \(-\) & \(+\) & $-$ \andrd & $y_{(6)} + g = \widetilde{y}_{(6)}$\\
		7 & \(-\) & \(+\) & \(+\) & $-$ \andrd & $y_{(7)} + g = \widetilde{y}_{(7)}$\\
		8 & \(+\) & \(+\) & \(+\) & $+$ \apple & $y_{(8)} + h = \mathring{y}_{(8)}$\\
		 \hline
	\end{tabulary}
	
	\vspace{0.5cm}

	
	\begin{itemize}
		\item	Android users: ${\widetilde{y}_{(i)} =}\,\, y_{(i)} + g$
		\item	Apple users:\,\,\,\,\, ${\mathring{y}_{(i)} =}\,\, y_{(i)} + h$
	\end{itemize} 
	
\end{frame}

\begin{frame}\frametitle{Why does blocking like this work so successfully? (math alert!)}
	
	The main effect of \textbf{A}
	
	\newcommand{\mo}{\color{myOrange}}
	
	\begin{align*}
	\mathbf{A} &= \dfrac{1}{2}\left[ \dfrac{\left( \mathring{y}_{(8)} - \widetilde{y}_{(7)}\right)
										+   \left( \widetilde{y}_{(4)} - \mathring{y}_{(3)}\right)	
										+   \left( \widetilde{y}_{(6)} - \mathring{y}_{(5)}\right)
										+   \left( \mathring{y}_{(2)} - \widetilde{y}_{(1)}\right)	} {4}\right]\\
		\intertext{\color{myOrange}Notice two $+\widetilde{y}$ values and two $-\widetilde{y}$; and also two $+\mathring{y}$ values and two $-\mathring{y}$}	
	\onslide+<2->{
	\mathbf{A} &=  \dfrac{1}{8}\bigg[   \left( y_{(8)} + h - y_{(7)} - g\right)
									+  \left( y_{(4)} + g - y_{(3)} - h\right)	\bigg.\\
							   &\bigg.\qquad\qquad\qquad\qquad\qquad+ \left( y_{(6)} + g - y_{(5)} - h\right)
									+  \left( y_{(2)} + h - y_{(1)} - g\right)	\bigg]\\
		}
	\onslide+<3->{
		\intertext{which simplifies to}
	\mathbf{A} &= \dfrac{1}{8} \Big[ -y_{(1)} + y_{(2)} - y_{(3)} + y_{(4)} - y_{(5)} + y_{(6)} - y_{(7)} + y_{(8)}	\,\,\cancelto{0}{-2g + 2g} \,\, \cancelto{0}{-2h+2h}\,\, \Big]\\
	}
	\onslide+<4->{
		\mathbf{A} &= \text{pure effect of \textbf{A},  without bias}
	}
	\end{align*}
\end{frame}

\begin{frame}\frametitle{Why does blocking like this work so successfully? (math alert!)}
	
	\vspace{1cm}
	In a similar way, you can show all effects are estimated without bias:
	\begin{itemize}
		\item	\textbf{A}
		\item	\textbf{B}
		\item	\textbf{C}
		\item	\textbf{AB}
		\item	\textbf{AC}
		\item	\textbf{BC}
		
		\vspace{1cm}
		\item	except, for the effect of \textbf{ABC}:
			\begin{align*}
			\mathbf{ABC} &= \dfrac{1}{8} \Big[  \underbrace{-y_{(1)} + y_{(2)} +  y_{(3)} - y_{(4)} + y_{(5)} - y_{(6)} - y_{(7)} + y_{(8)}}_{\mathclap{\text{pure effect of \textbf{ABC}}}} \underbrace{\,\,\,\,{\color{red}-4g} \,\, {\color{red}+ 4h_{\color{white}(i)}}}_{\mathclap{\text{\emph{with} bias}}}	 \Big]\\	
			\\
			%\mathbf{ABC} &= \underbrace{\text{, }}_{\mathclap{\text{confounding with the blocking effect}}}
			\end{align*}
	\end{itemize}
\end{frame}

\begin{frame}\frametitle{Why does blocking like this work so successfully? (math alert!)}
	
	\vspace{1cm}
	In a similar way, you can show all effects are estimated without bias:
	\begin{itemize}
		\item	\textbf{A}
		\item	\textbf{B}
		\item	\textbf{C}
		\item	\textbf{AB}
		\item	\textbf{AC}
		\item	\textbf{BC}
		
		\vspace{1cm}
		\item	except, for the effect of \textbf{ABC}:
			\begin{align*}
			\mathbf{ABC} &= \dfrac{1}{8} \Big[  \underbrace{-y_{(1)} + y_{(2)} +  y_{(3)} - y_{(4)} + y_{(5)} - y_{(6)} - y_{(7)} + y_{(8)}}_{\mathclap{\text{pure effect of \textbf{ABC}}}} \underbrace{\,\,\,\,{\color{red}-4g} \,\, {\color{red}+ 4h_{\color{white}(i)}}}_{\mathclap{\text{confounded with the blocking effect}}}	 \Big]\\	
			\\
			\end{align*}
	\end{itemize}
\end{frame}

\begin{frame}\frametitle{More than two blocks are easily possible}
	
	\begin{columns}[t]
		\column{0.25\textwidth}
			\textbf{CalApp}
			
			\includegraphics[width=\textwidth]{../4D/Supporting materials/4C-7-cell-phone-2830319467_1faaecc974_o-flickr.jpg}
			\\
			{\tiny{\href{https://secure.flickr.com/photos/williamhook/2830319467/}{Flickr: williamhook}}}
			
			\vspace{1cm}
			Apple, Android, Blackberry
		
		
			\column{0.25\textwidth}
			\onslide+<2->{
				\textbf{Baking}
				\includegraphics[width=\textwidth]{../4E/Supporting files/flour-and-flour-cropped.png}
		
				\vspace{1.1cm}
				4 different brands of flour
			}
			
		\column{0.25\textwidth}
			\onslide+<3->{
				\textbf{Shift work}
				%\vspace{0.1cm}
				\includegraphics[width=\textwidth]{../4E/Supporting files/flickr-4140348113_f0efc8235b_z-clock.jpg}
				\\
				{\tiny{\href{https://secure.flickr.com/photos/leehaywood/4140348113}{Flickr: leehaywood}}}
				
				
				
				\vspace{0.8cm}
				Day shift\\
				Afternoon shift\\
				Night shift
			}
		
		\column{0.25\textwidth}
			\onslide+<4->{
				\textbf{Gas mileage}
				
				%\vspace{-0.1cm}
				\includegraphics[width=\textwidth]{../4D/Supporting materials/4C-6-gas-mileage.png}
				
				\vspace{1.65cm}
				Gasoline brand 1\\
				Gasoline brand 2\\
				Gasoline brand 3\\
			}
			
	\end{columns}
	
	
\end{frame}

\begin{frame}\frametitle{}
	\centerline{\includegraphics[width=\textwidth]{../4E/Supporting files/blocking-tables.jpg}}
\end{frame}

\begin{frame}\frametitle{General rule for blocking in two blocks}
	\begin{columns}[T]
		\column{0.45\textwidth}
			\begin{enumerate}
				\item	Write out your existing standard order table
				\item	Add an extra ``factor'' to your standard order table
				\item	Generate a half-fraction, using the trade-off table, on this new factor
				\item	Signs with $-$ are in one block; and signs with $+$ are in the other block
			\end{enumerate}
		
		\column{0.48\textwidth}
			\newcommand{\apple}{\scriptsize ~~\,Apple}
			\newcommand{\andrd}{\scriptsize Android}
			\begin{tabulary}{\linewidth}{cccc|c}\hline 
				\multirow{1}{*}{\textbf{\relax Run }} & \textbf{\relax A } & \textbf{\relax B } & \textbf{\relax C } & \textbf{\relax D=ABC}  \\
				\hline 
				1 & \(-\) & \(-\) & \(-\) & $-$ \andrd \\
				2 & \(+\) & \(-\) & \(-\) & $+$ \apple \\
				3 & \(-\) & \(+\) & \(-\) & $+$ \apple \\
				4 & \(+\) & \(+\) & \(-\) & $-$ \andrd \\
				5 & \(-\) & \(-\) & \(+\) & $+$ \apple \\
				6 & \(+\) & \(-\) & \(+\) & $-$ \andrd \\
				7 & \(-\) & \(+\) & \(+\) & $-$ \andrd \\
				8 & \(+\) & \(+\) & \(+\) & $+$ \apple \\
				 \hline
			\end{tabulary}
	\end{columns}
	
	
\end{frame}



% /Users/kevindunn/Dropbox/Coursera/Media/All-course-slides/classes/CourseraMOOC-class-4F.tex

\begin{frame}\frametitle{There's more: we never stop learning\emph{!}}
	\centerline{
	\fbox{\parbox[b][6.2em][t]{.5\textwidth}{
		{\color{myGreen} There is ongoing research on how to efficiently run fewer experiments. \\}\\
		We give some pointers to this research at the end of this module.\\
		\\
		{\color{myOrange}In this section we only cover established techniques.}
	}}
	}
\end{frame}	

\begin{frame}\frametitle{Cell-culture example: long duration runs; and many factors are possible}
	\newcommand{\white}{\color{white}}
	\begin{columns}[c]
		\column{0.5\textwidth} 
			\begin{enumerate}
				\item	\textbf{T}: the temperature profile
				\item	\textbf{D}: dissolved oxygen
				\item	\textbf{A}: agitation rate
				\item	\textbf{P}: pH
				\item	\textbf{S}: substrate type (A or B)
				%\onslide+<2->{
				%\item	\textbf{W}: water type (distilled or tap water)
				%\item	\textbf{M}: mixer type (axial or radial)
				%}
			\end{enumerate}
		
		\column{0.5\textwidth}
			{\color{blue} \small Industrial scale: {\color{white}y}}   
			
			\vspace{0.2cm}
			
			\centerline{\includegraphics[height=.7\textwidth]{../4F/Supporting material/flickr-493740898_b98113ae44_o-cell-culture-mod.png}}
			\see{\href{https://secure.flickr.com/photos/londonmatt/493740898}{Flickr: londonmatt}}
	\end{columns}

	\vfill
	At 10 days per cell culture, it would take $\approx$ 1 year for all $2^5 = 32$ runs
	
\end{frame}

\begin{frame}\frametitle{Cell-culture example: long duration runs; and many factors are possible}
	\begin{columns}[c]
		\column{0.5\textwidth}
			\begin{enumerate}
				\item	\textbf{T}: the temperature profile
				\item	\textbf{D}: dissolved oxygen
				\item	\textbf{A}: agitation rate
				\item	\textbf{P}: pH
				\item	\textbf{S}: substrate type (A or B)
				%\onslide+<2->{
				%\item	\textbf{W}: water type (distilled or tap water)
				%\item	\textbf{M}: mixer type (axial or radial)
				%}
			\end{enumerate}
		
		\column{0.5\textwidth}
			{\color{blue} \small Laboratory scale:} 
			
			\vspace{0.2cm}
			
			\centerline{\includegraphics[height=.7\textwidth]{../4F/Supporting material/flickr-3644661574_79d0810177_b-cell-culture-lab}}
			\see{\href{https://secure.flickr.com/photos/kaibara/3644661574}{Flickr: kaibara}}
	\end{columns}

	\vfill
	At 10 days per cell culture, it would take $\approx$ 1 year for all $2^5 = 32$ runs
	
\end{frame}

\begin{frame}\frametitle{Cell-culture example: long duration runs; and many factors are possible}
	\begin{columns}[c]
		\column{0.5\textwidth}
			\begin{enumerate}
				\item	\textbf{T}: the temperature profile
				\item	\textbf{D}: dissolved oxygen
				\item	\textbf{A}: agitation rate
				\item	\textbf{P}: pH
				\item	\textbf{S}: substrate type (A or B)
				%\onslide+<2->{
				%\item	\textbf{W}: water type (distilled or tap water)
				%\item	\textbf{M}: mixer type (axial or radial)
				%}
			\end{enumerate}
		
		\column{0.5\textwidth}
			{\color{blue} \small Laboratory equipment to control the culture:} 
			
			\vspace{0.2cm}
			
			\centerline{\includegraphics[height=.7\textwidth]{../4F/Supporting material/flickr-3815183191_98c6296080_b-cell-culture-instrumentation}}
			\see{\href{https://secure.flickr.com/photos/mjanicki/3815183191}{Flickr: mjanicki}}
	\end{columns}

	\vfill
	3 months available: {\color{myOrange} that corresponds to 9 experiments}.
	
\end{frame}

\begin{frame}\frametitle{Cell-culture example: long duration runs; and many factors are possible}
	\begin{columns}[c]
		\column{0.5\textwidth}
			\begin{enumerate}
				\item	\textbf{T}: the temperature profile
				\item	\textbf{D}: dissolved oxygen
				\item	$\cancel{\text{\textbf{A}: agitation rate}}$
				\item	$\cancel{\text{\textbf{P}: pH}}$
				\item	\textbf{S}: substrate type (A or B)
				%\onslide+<2->{
				%\item	\textbf{W}: water type (distilled or tap water)
				%\item	\textbf{M}: mixer type (axial or radial)
				%}
			\end{enumerate}
		
		\column{0.5\textwidth}
			{\color{blue} \small Laboratory equipment to control the culture:} 
			
			\vspace{0.2cm}
			
			\centerline{\includegraphics[height=.7\textwidth]{../4F/Supporting material/flickr-3815183191_98c6296080_b-cell-culture-instrumentation}}
			\see{\href{https://secure.flickr.com/photos/mjanicki/3815183191}{Flickr: mjanicki}}
	\end{columns}

	\vfill
	{\color{red} Don't do this:} remove factors in order to get a full factorial.
	
\end{frame}

\begin{frame}\frametitle{Cell-culture example: long duration runs; and many factors are possible}
	\begin{columns}[c]
		\column{0.5\textwidth}
			\begin{enumerate}
				\item	\textbf{T}: the temperature profile
				\item	\textbf{D}: dissolved oxygen
				\item	\textbf{A}: agitation rate
				\item	\textbf{P}: pH
				\item	\textbf{S}: substrate type (A or B)
				\item	\textbf{W}: water type (distilled or tap water)
				\item	\textbf{M}: mixer type (axial or radial)
		
			\end{enumerate}
		
		\column{0.5\textwidth}
			{\color{blue} \small Different types of mixers (impellers):} 
			
			\vspace{0.2cm}
			
			\centerline{\includegraphics[height=.7\textwidth]{../4F/Supporting material/Mixing_-_flusso_assiale_e_radiale-wikipedia.jpg}}
			
			\see{\href{https://en.wikipedia.org/wiki/File:Mixing_-_flusso_assiale_e_radiale.jpg}{Wikipedia}}
	\end{columns}

	\vfill
	3 months available: {\color{myOrange} that corresponds to 9 experiments}.
	
\end{frame}

\begin{frame}\frametitle{}
	\begin{columns}[T]
		\column{0.9\textwidth}
			\includegraphics[height=\textheight]{\imagedir/doe/DOE-trade-off-table-MOOC-resolution.png}
		
		\column{0.15\textwidth}
			
			\onslide+<2->{
				\vspace{2cm}
				{\Huge
					$2^{5-2}$
					%$2^{k-p}$
				} 
			 
				\vspace{2cm}
				\onslide+<3->{ 
					\begin{align*}
						\textbf{D} &= \textbf{AB}\\
						\textbf{E}\, &= \textbf{AC} 
					\end{align*}
				}
			}
	\end{columns}
	
\end{frame}

\begin{frame}\frametitle{}
	\begin{columns}[T]
		\column{0.9\textwidth}
			\includegraphics[height=\textheight]{\imagedir/doe/DOE-trade-off-table-MOOC-resolution.png}
		
		\column{0.15\textwidth}
			
			
				\vspace{2cm}
				{\Huge
					$2^{k-p}$
				} 
			 
				\vspace{2cm}
				\onslide+<3->{ 
					\begin{align*}
						\textbf{D} &= \textbf{AB}\\
						\textbf{E}\, &= \textbf{AC} 
					\end{align*}
				}
	\end{columns}
	
\end{frame}

\begin{frame}\frametitle{}
	\begin{columns}[T]
		\column{0.9\textwidth}
			\includegraphics[height=\textheight]{\imagedir/doe/DOE-trade-off-table-MOOC-resolution-half-marked.png}
		
		\column{0.15\textwidth}
			
			
				\vspace{2cm}
				{\Huge
					$2^{k-p}$\\
					
					\vspace{1cm}
					
				} $p=1$
			 
				
	\end{columns}
	
\end{frame}

\begin{frame}\frametitle{}
	\begin{columns}[T]
		\column{0.9\textwidth}
			\includegraphics[height=\textheight]{\imagedir/doe/DOE-trade-off-table-MOOC-resolution-quarter-marked.png}
		
		\column{0.15\textwidth}
			
			
				\vspace{2cm}
				{\Huge
					$2^{k-p}$\\
					
					\vspace{1cm}
					
				} $p=2$
			 
				
	\end{columns}
	
\end{frame}

\begin{frame}\frametitle{Cell-culture example: creating the fractional factorial design}
	
	\vspace{0.5cm}
	\begin{tabulary}{\linewidth}{ccccccc}
		\textbf{\relax Experiment} & \textbf{\relax A$^\ast$ } & \textbf{\relax B$\,\mathring{}$} & \textbf{\relax C$\,\mathring{}$ } & \onslide+<2->{\textbf{\relax D$\,\mathring{}$ = AB}} & \onslide+<2->{\textbf{\relax E$^\ast$ = AC}}\\ \cline{1-6}
		\textbf{1} & \(-\) & \(-\) & \(-\) & \onslide+<2->{\(+\)} & \onslide+<2->{\(+\)} \\
		\textbf{2} & \(+\) & \(-\) & \(-\) & \onslide+<2->{\(-\)} & \onslide+<2->{\(-\)} \\
		\textbf{3} & \(-\) & \(+\) & \(-\) & \onslide+<2->{\(-\)} & \onslide+<2->{\(+\)} \\
		\textbf{4} & \(+\) & \(+\) & \(-\) & \onslide+<2->{\(+\)} & \onslide+<2->{\(-\)} \\
		\textbf{5} & \(-\) & \(-\) & \(+\) & \onslide+<2->{\(+\)} & \onslide+<2->{\(-\)} \\
		\textbf{6} & \(+\) & \(-\) & \(+\) & \onslide+<2->{\(-\)} & \onslide+<2->{\(+\)} \\
		\textbf{7} & \(-\) & \(+\) & \(+\) & \onslide+<2->{\(-\)} & \onslide+<2->{\(-\)} \\
		\textbf{8} & \(+\) & \(+\) & \(+\) & \onslide+<2->{\(+\)} & \onslide+<2->{\(+\)} \\
		\onslide+<4->{\textbf{9} &  $-1$	& 0 		& 0 & 0 & +1		&\color{myOrange}$\longleftarrow$ this row is a baseline $^\#$} 
	\end{tabulary}
	
	\vspace{0.5cm}
	\begin{flalign*}
		^\ast       & \text{categorical factor}\\
		\mathring{}\,\, & \text{continuous factor} \\
		\onslide+<4->{\color{myOrange}^\#  & \text{this entry does \textbf{not} use the generators; it should be run first to establish a baseline} &}
	\end{flalign*}	
\end{frame}

\begin{frame}\frametitle{$^\text{\color{purple}Video 4B}$ Aliasing: when we have more than one name for the same thing}
	
	\vspace{1cm}
	What is aliased in this experimental design (i.e. which columns are the same)?
		
		\vspace{0.5cm}
		\begin{itemize}
			\item	\textbf{A=BC}
			
			\onslide+<2->	{
			\vspace{1cm}
			\item	\textbf{B=AC}
			
			\vspace{1cm}
			\item	\textbf{C=AB} 
			
			\vspace{1cm}
			\item	\textbf{ABC = Intercept} (the intercept is indicated as $b_0$)
			}
		\end{itemize}

\end{frame}

\begin{frame}\frametitle{Calculating the aliases using an interesting technique}
	
	\begin{columns}[T]
		\column{0.25\textwidth}
			\includegraphics[width=\textwidth]{../4F/Supporting material/generators.png}
		
		\column{0.04\textwidth}
		\column{0.75\textwidth}
		\vspace{-.5cm}
		{\LARGE
		
			\begin{flalign*}			
				\textbf{D} &= \textbf{AB}&\\
				\onslide+<2->{
					\textbf{D\,D} &= \textbf{AB\,D}&
				}
				\onslide+<3->{
					\intertext{{\color{purple}rule:} \textbf{AA=I}; \textbf{BB=I}; ... \textbf{DD=I}, \emph{etc}}
				}
				\onslide+<5->{
					\textbf{I} &= \textbf{ABD}&\\
				}
			\end{flalign*}
		}

	\end{columns}	
	
	\uncover<4>{
		\vspace{-1.8cm}
		If we have $ \textbf{A} = \begin{pmatrix}-1\\+1\\-1\\+1\\-1\\+1\\-1\\+1\\ \end{pmatrix} \text{then we can say}\,\, \textbf{AA} =  \begin{pmatrix}-1\\+1\\-1\\+1\\-1\\+1\\-1\\+1\\ \end{pmatrix} \begin{pmatrix}-1\\+1\\-1\\+1\\-1\\+1\\-1\\+1\\ \end{pmatrix}  =  \begin{pmatrix}+1\\+1\\+1\\+1\\+1\\+1\\+1\\+1\\ \end{pmatrix} = \textbf{I}$
	}
	
	
	
\end{frame}

\begin{frame}\frametitle{Calculating the aliases using an interesting technique}
	
	\begin{columns}[T]
		\column{0.25\textwidth}
			\includegraphics[width=\textwidth]{../4F/Supporting material/generators.png}
		
		\column{0.04\textwidth}
		\column{0.75\textwidth}
		{\huge
			\begin{flalign*}			
				\textbf{E} &= \textbf{AC}&\\
				%\onslide+<2->{
					\textbf{E\,E} &= \textbf{AC\,E}&
				%}
				%\onslide+<3->{
					\intertext{{\color{purple}rule:} \textbf{AA=I}; \textbf{BB=I}; ... \textbf{DD=I}, \emph{etc}}
				%}
				\onslide+<2->{
					\textbf{I} &= \textbf{ACE}&\\
				}
			\end{flalign*}
		}

	\end{columns}	
	\uncover<2>{
		{\color{myOrange}Multiple the left and right by the symbol that's on the left.}
	}
	
	
\end{frame}

\begin{frame}\frametitle{Approach to calculate the aliasing pattern}
	
	\begin{columns}[T]
		\column{0.58\textwidth}
			\begin{enumerate}
				\item	Read the generators from the table 
				\onslide+<2->{
					\item	Rearrange the generators as  $\textbf{I = \ldots}$
				}
				\onslide+<4->{
				 	\item	Form the {\color{purple}\textbf{defining relationship}} taking all combinations of the words, so that $\textbf{I = \ldots}$
				}	
				\onslide+<5->{
				 	\item	Ensure the defining relationship has $2^p$ words
				}
				\onslide+<16->{
					\item	Use the defining relationship to compute the aliasing pattern
				}
			\end{enumerate}
			
		\column{0.5\textwidth}
			\begin{enumerate}
				\item	$\textbf{D = AB}$  and $\textbf{E = AC}$ 
				\onslide+<2->{
					\item	$\textbf{I = ABD}$ and $\textbf{I = ACE}$ 
				}
				
				\onslide+<6->{
 					\item	$\textbf{I =}$ \onslide+<7->{$\textbf{ABD}$} \onslide+<8->{$\textbf{= ACE}$} \onslide+<9->{$\textbf{= (ABD)(ACE)}$} 
					\\ \vspace{0.4cm}
				}
				\onslide+<6->{
								\item	$p=2$\onslide+<15->{, and we have 4 words.}
				}
				\onslide+<16->{
					\item	See the next slides.
					
				}
			\end{enumerate}			
			
	\end{columns}
	\vspace{0.5cm}
	\onslide+<3-5>{
		{\color{purple}\textbf{Word}}: a collection of sequential factor letters
	}
	\\
	\onslide+<10->{$\textbf{(ABD)(ACE)}$}
	\onslide+<11->{$\textbf{ = AABCDE}$}
	\onslide+<12->{$\textbf{ = (AA)BCDE}$}
	\onslide+<13->{$\textbf{ = (I)BCDE}$}
	\onslide+<14->{$\textbf{ = BCDE}$} 	
\end{frame}

\begin{frame}\frametitle{Approach to calculate the aliasing pattern}
	
	\begin{columns}[T]
		\column{0.58\textwidth}
			\begin{enumerate}
				\item	Read the generators from the table 
					\item	Rearrange the generators as  $\textbf{I = \ldots}$
				 	\item	Form the {\color{purple}\textbf{defining relationship}} taking all combinations of the words, so that $\textbf{I = \ldots}$
				 	\item	Ensure the defining relationship has $2^p$ words
				\onslide+<3->{
					\item	Use the defining relationship to compute the aliasing pattern
				}
			\end{enumerate}
			
		\column{0.5\textwidth}
			\begin{enumerate}
				\item	$\textbf{D = AB}$  and $\textbf{E = AC}$ 
					\item	$\textbf{I = ABD}$ and $\textbf{I = ACE}$ 
 					\item	$\textbf{I =}$ $\textbf{ABD}$ $\textbf{= ACE}$ $\textbf{= BCDE}$
					\\ \vspace{0.4cm}
				
					\item	$p=2$\onslide+<2->{, and we have 4 words.}
				
				\onslide+<3->{
					\item	We will in the next video.
					
				}
			\end{enumerate}			
			
	\end{columns}
	\vspace{0.5cm}
	\onslide+<4>{
		{\color{purple}\textbf{Word}}: a collection of sequential factor letters
	}
	\\
	\onslide+<1-2>{
		$\textbf{(ABD)(ACE)}$
		$\textbf{ = AABCDE}$
		$\textbf{ = (AA)BCDE}$
		$\textbf{ = (I)BCDE}$
		$\textbf{ = BCDE}$	
	}
\end{frame}

\begin{frame}\frametitle{Example to try yourself: find the defining relationship}
	
	\vspace{0.5cm}
	You'd like to investigate 6 factors, and have a budget for 15 to 20 experiments.
	
	\vspace{0.5cm}
	\begin{columns}[T]
		\column{0.7\textwidth}
			\begin{enumerate}
				\item	Read the generators from the table 
				\item	Rearrange the generators as  $\textbf{I = \ldots}$
			 	\item	Form the {\color{purple}defining relationship} taking all combinations of the words, so that $\textbf{I = \ldots}$
			 	\item	Ensure the defining relationship has $2^p$ words
				\item	In the next video: use this defining relationship to compute the aliasing pattern
			\end{enumerate}
			
		\column{0.3\textwidth}

	\end{columns}

	
\end{frame}

\begin{frame}\frametitle{Example to try yourself: {\color{myOrange}solution} for finding the defining relationship}
	
	\vspace{0.5cm}
	You'd like to investigate 6 factors, and have a budget for 15 to 20 experiments.
	
	\vspace{0.5cm}
	\begin{columns}[T]
		\column{0.85\textwidth}
			\begin{enumerate}
				\item	Read the generators from the table for $k=6$, and 16 experiments
					\begin{itemize}
						\item	\textbf{E = ABC}	and  \textbf{F = BCD}
					\end{itemize}
				\item	Rearrange the generators as  $\textbf{I = \ldots}$
					\begin{itemize}
						\item	\textbf{I = ABCE}	and  \textbf{I = BCDF}
					\end{itemize}
			 	\item	Form the {\color{purple}defining relationship} taking all combinations of the words, so that $\textbf{I = \ldots}$
					\begin{itemize}
						\item	\textbf{I = ABCE = BCDF = A(BB)(CC)DEF = ADEF}
					\end{itemize}
			 	\item	Ensure the defining relationship has $2^p$ words
					\begin{itemize}
						\item	It does; since $p=2$ in this case
					\end{itemize} 
				\item	We will use this defining relationship next.
			\end{enumerate}
			
		\column{0.3\textwidth}
			\centerline{\includegraphics[height=.7\textwidth]{../4F/Supporting material/ivq-final-solution.png}}

	\end{columns}

	
\end{frame}


% /Users/kevindunn/Dropbox/Coursera/Media/All-course-slides/classes/CourseraMOOC-class-4G.tex

\begin{frame}\frametitle{Cell-culture example: creating the fractional factorial design}
	
	\vspace{0.5cm}
	\begin{columns}[T]
		\column{0.45\textwidth}
			{\scriptsize
				\begin{tabulary}{\linewidth}{ccccccc}
					\textbf{\relax Experiment} & \textbf{\relax A } & \textbf{\relax B} & \textbf{\relax C } & \textbf{\relax D = AB} & \textbf{\relax E = AC}\\ \cline{1-6}
					\textbf{1} & \(-\) & \(-\) & \(-\) & \(+\) & \(+\) \\
					\textbf{2} & \(+\) & \(-\) & \(-\) & \(-\) & \(-\) \\
					\textbf{3} & \(-\) & \(+\) & \(-\) & \(-\) & \(+\) \\
					\textbf{4} & \(+\) & \(+\) & \(-\) & \(+\) & \(-\) \\
					\textbf{5} & \(-\) & \(-\) & \(+\) & \(+\) & \(-\) \\
					\textbf{6} & \(+\) & \(-\) & \(+\) & \(-\) & \(+\) \\
					\textbf{7} & \(-\) & \(+\) & \(+\) & \(-\) & \(-\) \\
					\textbf{8} & \(+\) & \(+\) & \(+\) & \(+\) & \(+\) \\
				\end{tabulary}
			}
		
		\column{0.48\textwidth}
			\vspace{-0.2cm}
			\centerline{\includegraphics[height=.5\textwidth]{../4F/Supporting material/flickr-493740898_b98113ae44_o-cell-culture-mod.png} \see{\href{https://secure.flickr.com/photos/londonmatt/493740898}{Flickr: londonmatt}}}
			
	\end{columns}
	
	\vspace{0.2cm}
	\hrule
	\begin{enumerate}
		\item	Read the generators from the trade off table 
			\begin{itemize}
				\item		$\textbf{D = AB}$  and $\textbf{E = AC}$ 
			\end{itemize}
		
		\item	Rearrange the generators as  $\textbf{I = \ldots}$
			\begin{itemize}
				\item	$\textbf{I = ABD}$ and $\textbf{I = ACE}$ 
			\end{itemize}
			
		\item	Form the {\color{purple}\textbf{defining relationship}} taking all combinations of the words: $\textbf{I = \ldots}$
			\begin{itemize}
				\item	$\textbf{I = ABD = ACE = BCDE}$
			\end{itemize}
			
		\item	Ensure the defining relationship has $2^p$ words
			\begin{itemize}
				\item	$p=2$, and we have 4 words.
			\end{itemize}
		\item	{\color{myOrange}\textbf{In this example:}} we will use the defining relationship to compute the aliasing pattern
		
	\end{enumerate}

		
	
\end{frame}

\begin{frame}\frametitle{The purpose of the defining relationship: to calculate all possible aliases}
	\newcommand{\mo}{\color{myOrange}}
	
	\begin{exampleblock}{}
		\color{myGreen}By example: what are the aliases (confounding) with factor \textbf{\mo B}?
	\end{exampleblock}
	\begin{align*}
		\textbf{I}\onslide+<2->{\mo\,\textbf{B}} &&=&& \textbf{ABD}\onslide+<2->{\mo\,\textbf{B}} &&=&& \textbf{ACE}\onslide+<2->{\mo\,\textbf{B}} &&=&& \textbf{BCDE}\onslide+<2->{\mo\,\textbf{B}}\\		
		\onslide+<3->{\textbf{\mo B} &&=&& \textbf{A(B{\mo B})D} &&=&& \textbf{A{\mo B}CE} &&=&& \textbf{(B{\mo B})CDE}\\}
		\onslide+<4->{\textbf{B} &&=&& \textbf{AID} &&=&& \textbf{ABCE} &&=&& \textbf{ICDE}\\}
		\onslide+<5->{\textbf{B} &&=&& \textbf{AD} &&=&& \textbf{ABCE} &&=&& \textbf{CDE}}
	\end{align*}
	
	\begin{itemize}
		\item	Take the defining relationship
		\onslide+<2->{
			\item	Multiply every word by a \textbf{\mo B} term
		}
		\onslide+<3->{
			\item	Rearrange the order
		}
		\onslide+<4->{
			\item	Simplify by using the rules: \textbf{AA=I}; \textbf{BB=I}; ... \textbf{DD=I}, \emph{etc}
		}
		\onslide+<5->{
			\item	Drop out the unnecessary identity terms
		}
	\end{itemize}
\end{frame}

\begin{frame}\frametitle{How do we use this aliasing information we obtain?}
	Now we know the aliases of \textbf{B}:
	\begin{align*}
		\textbf{B} &&=&& \textbf{AD} &&=&& \textbf{ABCE} &&=&& \textbf{CDE}
	\end{align*}
	
	\vspace{1cm}
	\begin{itemize}
		\item	These are the terms that B is going to be confounded with
		\onslide+<2->{\item	We cannot tell B apart from the AD interaction.}
	\end{itemize}
\end{frame}

\begin{frame}\frametitle{How do we use this aliasing information we obtain?}
	Now we know the aliases of \textbf{B}:
	\begin{align*}
		\textbf{B} &&=&& \textbf{AD} &&\color{lightgray}=&& {\color{lightgray}\textbf{ABCE}} &&\color{lightgray}=&& \color{lightgray} \textbf{CDE}
	\end{align*}
	
	\vspace{1cm}
	\begin{itemize}
		\item	These are the terms that B is going to be confounded with
		\onslide+<2->{\item	We cannot tell B apart from the AD interaction.}
	\end{itemize}
\end{frame}

\begin{frame}\frametitle{How do we use this aliasing information we obtain?}
	Try this yourself: what are the aliases of \textbf{C}?
	\pause
	\begin{align*}
		\textbf{C} &&=&& \color{lightgray}\textbf{ABCD} &&=&& \textbf{AE} &&=&&\color{lightgray} \textbf{BDE}
	\end{align*}
	
	
	\vspace{1cm}
	\begin{itemize}
		\item	These are the terms that C is going to be confounded with
		\onslide+<2->{\item	We cannot tell C apart from the AE interaction.}
	\end{itemize}
\end{frame}

\begin{frame}\frametitle{How do we use this aliasing information we obtain?}
	Try this yourself: what are the aliases of \textbf{A}?
	\pause
	\begin{align*}
		\textbf{A} &&=&& \textbf{BD} &&=&& \textbf{CE} &&=&&\color{lightgray} \textbf{ABCDE}
	\end{align*}
		
	\vspace{1cm}
	\begin{itemize}
		\item	These are the terms that A is going to be confounded with
		\onslide+<2->{\item	We cannot tell A apart from the BD interaction and the CE interaction.}
	\end{itemize}
\end{frame}

\begin{frame}\frametitle{How do we use this aliasing information we obtain?}
	
	\vspace{1cm}
	Write out all the aliases for the 5 main effects:
	
	\vspace{0.5cm}
	\begin{align*}
		\textbf{A} &&=&& \textbf{BD} &&=&& \textbf{CE} &&=&&\color{lightgray} \textbf{ABCDE}\\
		\textbf{B} &&=&& \textbf{AD} &&=&& {\color{lightgray}\textbf{ABCE}} &&=&& \color{lightgray} \textbf{CDE}\\
		\textbf{C} &&=&& \color{lightgray}\textbf{ABCD} &&=&& \textbf{AE} &&=&&\color{lightgray} \textbf{BDE}\\
		\textbf{D} &&=&& \textbf{AB} &&=&& \color{lightgray} \textbf{ACDE} &&=&& \color{lightgray}\textbf{BCE} \\
		\textbf{E} &&=&& \color{lightgray} \textbf{ABDE} &&=&& \textbf{AC} &&=&&\color{lightgray} \textbf{BCD}\\
	\end{align*}
	
\end{frame}

\begin{frame}\frametitle{How do we use this aliasing information we obtain?}
	
	\vspace{1cm}
	Aliases for the 5 main effects (dropping out 3rd order and higher interactions):
	
	\vspace{0.5cm}
	\begin{align*}
		\textbf{A} &&=&& \textbf{BD} &&=&& \textbf{CE} \\
		\textbf{B} &&=&& \textbf{AD} &&&& &&&& \\
		\textbf{C} &&=&& \textbf{AE} &&&& &&&& \\
		\textbf{D} &&=&& \textbf{AB} &&&& &&&& \\
		\textbf{E} &&=&& \textbf{AC} &&&& &&&& \\
		\color{white}  \textbf{E} &&\color{white}=&& \color{white} \textbf{ABCD}	 &&\color{white}=&& \color{white} \textbf{ABDE}	&&\color{white}=&& \color{white} \textbf{ABCDE}
	\end{align*}
		
	
\end{frame}

\begin{frame}\frametitle{Back to some more baking experiments}
	\begin{columns}[T]
		\column{0.45\textwidth}
		
			\vspace{1cm}
			{\small Recipe: \href{http://yint.org/honeycomb-cake}{http://yint.org/honeycomb-cake}}
		
			\vspace{1cm}
		
			\centerline{\includegraphics[height=.5\textwidth]{../4G/Supporting material/flickr-5268081618_97f2626f2a_b-baking-experiment.jpg}}
			
			 \see{\href{https://secure.flickr.com/photos/andrea_nguyen/5268081724}{Flickr: andrea\_nguyen}}
		
		\column{0.48\textwidth}
			Potential factors to consider are
			\begin{itemize}
				\item	stirring speed
				\item	type of coconut milk
				\item	baking soda
				\item	amount of wheat starch
				\item	baking temperature
			\end{itemize}
			
			\vspace{0.4cm}
			\hrule
			
			\vspace{0.1cm}
			Aliasing pattern for a $2^{5-2}_{\textrm{III}}$ design:
			\begin{tabulary}{\linewidth}{ccccccc}				
				\textbf{A} & = & \textbf{BD} & = & \textbf{CE}  \\
				\textbf{B} & = & \textbf{AD} & & \\
				\textbf{C} & = & \textbf{AE} & & \\
				\textbf{D} & = & \textbf{AB} & & \\
				\textbf{E} & = & \textbf{AC} & & 
			\end{tabulary}
			
	\end{columns}
	
	\vspace{1cm}

	
\end{frame}

\begin{frame}\frametitle{Back to some more baking experiments}
	\begin{columns}[T]
		\column{0.45\textwidth}
		
			\vspace{1cm}
			{\small Recipe: \href{http://yint.org/honeycomb-cake}{http://yint.org/honeycomb-cake}}
		
			\vspace{1cm}
		
			\centerline{\includegraphics[height=.5\textwidth]{../4G/Supporting material/flickr-5268081618_97f2626f2a_b-baking-experiment.jpg}}
			
			 \see{\href{https://secure.flickr.com/photos/andrea_nguyen/5268081724}{Flickr: andrea\_nguyen}}
		
		\column{0.48\textwidth}
			{\color{myOrange}So my factor assignment so far is:}
			\begin{itemize}
				\item	stirring speed
				\item	type of coconut milk
				\item	baking soda
				\item	amount of wheat starch
				\item	\textbf{A} = baking temperature
			\end{itemize}
			
			\vspace{0.4cm}
			\hrule
			
			\vspace{0.1cm}
			Aliasing pattern for a $2^{5-2}_{\textrm{III}}$ design:
			\begin{tabulary}{\linewidth}{ccccccc}				
				\textbf{A} & = & \textbf{BD} & = & \textbf{CE}  \\
				\textbf{B} & = & \textbf{AD} & & \\
				\textbf{C} & = & \textbf{AE} & & \\
				\textbf{D} & = & \textbf{AB} & & \\
				\textbf{E} & = & \textbf{AC} & & 
			\end{tabulary}
			
	\end{columns}
	
	\vspace{1cm}

	
\end{frame}

\begin{frame}\frametitle{Back to some more baking experiments}
	\begin{columns}[T]
		\column{0.45\textwidth}
		
			\vspace{1cm}
			{\small Recipe: \href{http://yint.org/honeycomb-cake}{http://yint.org/honeycomb-cake}}
		
			\vspace{1cm}
		
			\centerline{\includegraphics[height=.5\textwidth]{../4G/Supporting material/flickr-5268081618_97f2626f2a_b-baking-experiment.jpg}}
			
			 \see{\href{https://secure.flickr.com/photos/andrea_nguyen/5268081724}{Flickr: andrea\_nguyen}}
		
		\column{0.48\textwidth}
			{\color{myOrange}So my factor assignment so far is:}
			\begin{itemize}
				\item	\textbf{B} = stirring speed {\tiny (to avoid an AB interaction)}
				\item	type of coconut milk
				\item	baking soda
				\item	amount of wheat starch
				\item	\textbf{A} = baking temperature
			\end{itemize}
			
			\vspace{0.4cm}
			\hrule
			
			\vspace{0.1cm}
			Aliasing pattern for a $2^{5-2}_{\textrm{III}}$ design:
			\begin{tabulary}{\linewidth}{ccccccc}				
				\textbf{A} & = & \textbf{BD} & = & \textbf{CE}  \\
				\textbf{B} & = & \textbf{AD} &  & \\
				\textbf{C} & = & \textbf{AE} &  & \\
				\textbf{D} & = & \textbf{AB} &  & \\
				\textbf{E} & = & \textbf{AC} &  & 
			\end{tabulary}
			
	\end{columns}
	
	\vspace{1cm}

	
\end{frame}

\begin{frame}\frametitle{Back to some more baking experiments}
	\begin{columns}[T]
		\column{0.45\textwidth}
		
			\vspace{1cm}
			{\small Recipe: \href{http://yint.org/honeycomb-cake}{http://yint.org/honeycomb-cake}}
		
			\vspace{1cm}
		
			\centerline{\includegraphics[height=.5\textwidth]{../4G/Supporting material/flickr-5268081618_97f2626f2a_b-baking-experiment.jpg}}
			
			 \see{\href{https://secure.flickr.com/photos/andrea_nguyen/5268081724}{Flickr: andrea\_nguyen}}
		
		\column{0.48\textwidth}
			{\color{myOrange}So my factor assignment so far is:}
			\begin{itemize}
				\item	\textbf{B} = stirring speed {\tiny (to avoid an AB interaction)}
				\item	type of coconut milk
				\item	\textbf{D} = baking soda {\tiny (we expect this to be sensitive)}
				\item	amount of wheat starch
				\item	\textbf{A} = baking temperature
			\end{itemize}
			
			\vspace{0.4cm}
			\hrule
			
			\vspace{0.1cm}
			Aliasing pattern for a $2^{5-2}_{\textrm{III}}$ design:
			\begin{tabulary}{\linewidth}{ccccccc}				
				\textbf{A} & = & \textbf{BD} & = & \textbf{CE}  \\
				\textbf{B} & = & \textbf{AD} &  \\
				\textbf{C} & = & \textbf{AE} &  \\
				\textbf{D} & = & \multicolumn{5}{l}{$\cancelto{_0}{\textbf{AB}}$ {\tiny (to get an unbiased estimate)}}   \\
				\textbf{E} & = & \textbf{AC} & & 
			\end{tabulary}
			
	\end{columns}
	
	\vspace{1cm}
		
\end{frame}

\begin{frame}\frametitle{How do we use this aliasing information we obtain?}
	
	\vspace{1cm}
	The aliases for the 5 main effects in this a $2^{5-2}_{\textrm{III}}$ design:
	
	\vspace{0.5cm}
	\begin{align*}
		\textbf{A} &&=&& \textbf{BD} &&=&& \textbf{CE} &&=&&\color{lightgray} \textbf{ABCDE}\\
		\textbf{B} &&=&& \textbf{AD} &&=&& \color{lightgray} \textbf{CDE} &&=&& {\color{lightgray}\textbf{ABCE}}\\
		\textbf{C} &&=&& \textbf{AE} &&=&& \color{lightgray} \textbf{BDE} &&=&& \color{lightgray}\textbf{ABCD}\\
		\textbf{D} &&=&& \textbf{AB} &&=&& \color{lightgray}\textbf{BCE}  &&=&& \color{lightgray} \textbf{ACDE}\\
		\textbf{E} &&=&& \textbf{AC} &&=&& \color{lightgray} \textbf{BCD} &&=&& \color{lightgray} \textbf{ABDE}\\
	\end{align*}
	
\end{frame}

\begin{frame}\frametitle{Is is possible to get a more favourable fractional factorial? }
	
	\begin{exampleblock}{}
		In other words, can we get better confounding?
	\end{exampleblock}
	
	
	
	\begin{itemize}
		\item	In the 5-factor example:\\
			\qquad\qquad \emph{main effects are confounded with two factor interactions}
		
	\end{itemize}
	\vspace{1cm}
	
			Is it possible to achieve a design where main effects are confounded with 3-factor interactions?
			
	\vspace{1cm}
	
		\pause
		These are called resolution \textrm{IV} designs.
\end{frame}

\begin{frame}\frametitle{Example to try yourself: find the defining relationship}
	
	\vspace{0.5cm}
	You'd like to investigate 6 factors, and have a budget for 15 to 20 experiments.
	
	\vspace{0.5cm}
	\begin{columns}[T]
		\column{0.7\textwidth}
			\begin{enumerate}
				\item	Read the generators from the table 
				\item	Rearrange the generators as  $\textbf{I = \ldots}$
			 	\item	Form the {\color{purple}defining relationship} taking all combinations of the words, so that $\textbf{I = \ldots}$
			 	\item	Ensure the defining relationship has $2^p$ words
				\item	Use this defining relationship to compute the aliasing pattern
			\end{enumerate}
			
		\column{0.3\textwidth}

	\end{columns}

	
\end{frame}

\begin{frame}\frametitle{Example for practice: {\color{myOrange}solution} for finding the defining relationship}
	
	\vspace{0.5cm}
	You'd like to investigate 6 factors, and have a budget for 15 to 20 experiments.
	
	\vspace{0.5cm}
	\begin{columns}[T]
		\column{0.85\textwidth}
			\begin{enumerate}
				\item	Read the generators from the table for $k=6$, and 16 experiments
					\begin{itemize}
						\item	\textbf{E = ABC}	and  \textbf{F = BCD}
					\end{itemize}
				\item	Rearrange the generators as  $\textbf{I = \ldots}$
					\begin{itemize}
						\item	\textbf{I = ABCE}	and  \textbf{I = BCDF}
					\end{itemize}
			 	\item	Form the {\color{purple}defining relationship} taking all combinations of the words, so that $\textbf{I = \ldots}$
					\begin{itemize}
						\item	\textbf{I = ABCE = BCDF = A(BB)(CC)DEF = ADEF}
					\end{itemize}
			 	\item	Ensure the defining relationship has $2^p$ words
					\begin{itemize}
						\item	It does; since $p=2$ in this case
					\end{itemize} 
				\item	We will use this defining relationship now
			\end{enumerate}
			
		\column{0.3\textwidth}
			\centerline{\includegraphics[height=.7\textwidth]{../4F/Supporting material/ivq-final-solution.png}}

	\end{columns}

	
\end{frame}

\begin{frame}\frametitle{The 6-factor example in 16 experiments: a resolution \textrm{IV} design}

	
	\begin{exampleblock}{}
		\color{myGreen}By example: what are the aliases (confounding) with factor \textbf{\color{myOrange} A}?
	\end{exampleblock}
	
	
	\begin{align*}
		\intertext{The defining relationship is}
		\textbf{I} &&=&& \textbf{ABCE} &&=&& \textbf{BCDF} &&=&& \textbf{ADEF}\\
		\onslide+<2->{
			\intertext{Now multiply all words by \textbf{\color{myOrange} A}:}
			\textbf{\color{myOrange} A} &&=&& \textbf{BCE} &&=&& \textbf{ABCDF} &&=&& \textbf{DEF}
		}
	\end{align*}
	
	\onslide+<3->{
		\vspace{0.5cm}
		There are only 3-factor and higher level interactions\emph{!}
	}
	
\end{frame}

\begin{frame}\frametitle{The 6-factor example in 16 experiments: a resolution \textrm{IV} design}
	\begin{exampleblock}{}
		\color{myGreen}What are the aliases (confounding) with a two-factor interaction term: \textbf{\color{myOrange} CD}?
	\end{exampleblock}
	
	\begin{align*}
		\intertext{The defining relationship is}
		\textbf{I} &&=&& \textbf{ABCE} &&=&& \textbf{BCDF} &&=&& \textbf{ADEF}\\
		\onslide+<2->{
			\intertext{Now multiply all words by \textbf{\color{myOrange} CD}:}
			\textbf{\color{myOrange} CD} &&=&& \textbf{ABDE} &&=&& \textbf{BF} &&=&& \textbf{ACEF}
		}
	\end{align*}
	
	\onslide+<3->{
		\vspace{0.5cm}
		Two factor interactions are aliased with other two factor interactions.
	}

\end{frame}

\begin{frame}\frametitle{}
	%\vspace{-10pt}
	\centerline{\includegraphics[width=1\textwidth]{\imagedir/doe/DOE-trade-off-table.png}}
\end{frame}

\begin{frame}\frametitle{What the design's resolution tells us}

	\textbf{Resolution \textrm{III} designs}
	\begin{itemize}
		\item	Excellent for an initial screening
			\begin{itemize}
				\item	e.g. developing a new product 
				\item	e.g. troubleshooting a process, such as moving production from one location to another, but struggling to get a similar product on the two different machines
			\end{itemize}
	\end{itemize}

	\pause
	\vspace{0.5cm}
	\textbf{Resolution \textrm{IV} designs}
	\begin{itemize}
		\item	Used for learning about, and understanding, a system (characterization)
	\end{itemize}
	
	\pause
	\vspace{0.5cm}
	\textbf{Resolution \textrm{V} designs and higher, and full factorial designs}
	\begin{itemize}
		\item	Used for optimizing a process, understanding complex effects
		\item	To develop high-accuracy models
		\item	You will need to justify the budget for this carefully: expensive\emph{!}
	\end{itemize}
\end{frame}

\begin{frame}\frametitle{}
	\vspace{0.5cm}
	\centerline{\includegraphics[height=1.1\textheight]{../4G/Supporting material/course-textbook-fractional-factorial-example.png}}
		
\end{frame}

\begin{frame}\frametitle{}
	\vspace{0.5cm}
	\centerline{\includegraphics[height=1.1\textheight]{../4G/Supporting material/course-textbook-fractional-factorial-example-continued.png}}
		
\end{frame}

\begin{frame}\frametitle{General approach for figuring out the aliasing \emph{before starting} the experiments}
	
	\vspace{0.5cm}
	Define the number of factors to investigate; and determine what your budget is.
	
	\vspace{0.5cm}
	\begin{columns}[T]
		\column{0.9\textwidth}
			\begin{enumerate}
				\item	Read the generators from the trade off table 
				\item	Rearrange the generators as  $\textbf{I = \ldots}$
			 	\item	Form the {\color{purple}defining relationship} taking all combinations of the words, so that $\textbf{I = \ldots}$
			 	\item	Ensure the defining relationship has $2^p$ words
				\item	Compute the aliasing pattern
				\item	{\color{myGreen}Is the aliasing problematic?}
				
					\hbox{\hspace{3.5em}If \textbf{yes}: reassign factor letters; or pick another design (start over)}
					
					\hbox{\hspace{3.5em}If \textbf{no}: you are ready to start your experiments}
			\end{enumerate}
			
		\column{0.1\textwidth}
			\onslide+<2->{
				\centerline{\includegraphics[width=2\textwidth]{../4G/Supporting material/flickr-9568156463_1809c97b21_o-checklist}}
			
				\see{\href{https://secure.flickr.com/photos/ajc1/9568156463}{Flickr: ajc1}}
			}
	\end{columns}

	
\end{frame}

\begin{frame}\frametitle{}
	%\vspace{-10pt}
	\centerline{\includegraphics[width=1\textwidth]{\imagedir/doe/DOE-trade-off-table.png}}
\end{frame}

\begin{frame}\frametitle{\includegraphics[width=0.3\textwidth]{\imagedir/doe/examples/advice-logo.png}\,\, pick a design that meets the objective}
	
	\begin{columns}[T]
		\column{0.7\textwidth}
		
			\begin{itemize}
				\item	If you are just starting out, avoid eliminating factors to simply get a full factorial.
				\item	Use the experimental evidence to eliminate factors.
			\end{itemize}
			
			
			\vspace{1cm}
			\onslide+<2->{
				\begin{itemize}
					\item	Remember: these are experimental building blocks. The experiments you run first can be extended on later.
					\vspace{0.5cm}
					\onslide+<3->{
						\item	In the next example, we show how factors are eliminated, {\color{myOrange}\emph{based on evidence}}.
					}
				\end{itemize}
			}
	
			\vspace{1cm}
			
		\column{0.3\textwidth}
			\onslide+<2->{
				\centerline{\includegraphics[width=\textwidth]{../4G/Supporting material/flickr-6479064129_25ce3bb07f_o-building-block.png}}
		
				\see{\href{https://secure.flickr.com/photos/rahego/6479064129}{Flickr: rahego}}
			}
	\end{columns}
\end{frame}


% /Users/kevindunn/Dropbox/Coursera/Media/All-course-slides/classes/CourseraMOOC-class-4H.tex

\begin{frame}\frametitle{\includegraphics[width=0.3\textwidth]{\imagedir/doe/examples/advice-logo.png}\,\, pick a design that meets the objective}
	
	\begin{columns}[T]
		\column{0.7\textwidth}
		
			\begin{itemize}
				\item	If you are just starting out, avoid eliminating factors to simply get a full factorial.
				\item	Use the experimental evidence to eliminate factors.
			\end{itemize}
			
			
			\vspace{1cm}
			\onslide+<2->{
				\begin{itemize}
					\item	Remember: these are experimental building blocks. The experiments you run first can be extended on later.
					\vspace{0.5cm}
					\onslide+<3->{
						\item	In the next example, we show how factors are eliminated, {\color{myOrange}\emph{based on evidence}}.
					}
				\end{itemize}
			}
	
			\vspace{1cm}
			
		\column{0.3\textwidth}
			\onslide+<2->{
				\centerline{\includegraphics[width=\textwidth]{../4G/Supporting material/flickr-6479064129_25ce3bb07f_o-building-block.png}}
		
				\see{\href{https://secure.flickr.com/photos/rahego/6479064129}{Flickr: rahego}}
			}
	\end{columns}
\end{frame}

\begin{frame}\frametitle{An example to demonstrate a saturated fractional factorial analyis}
	We have 7 factors: \textbf{A}, \textbf{B}, \textbf{C}, \textbf{D}, \textbf{E}, \textbf{F}, and \textbf{G}.
	
	\vspace{1cm}
	The fewest number of experiments is: 8 runs
	\vfill
\end{frame}

\begin{frame}\frametitle{Creating the standard order table for the fractional factorial}
	\vspace{0.4cm}
	\begin{tabulary}{\linewidth}{c|ccccccc}
		\textbf{\relax Experiment} & \textbf{\relax A } & \textbf{\relax B} & \textbf{\relax C } & \onslide+<3->{\textbf{\relax D = AB}} & \onslide+<3->{\textbf{\relax E = AC}} & \onslide+<3->{\textbf{\relax F = BC}} & \onslide+<3->{\textbf{\relax G = ABC}} \\ \cline{1-8}
		\textbf{1} & \(-\) & \(-\) & \(-\) & \onslide+<3->{\(+\)} & \onslide+<3->{\(+\)} & \onslide+<3->{\(+\)} & \onslide+<3->{\(-\)} \\
		\textbf{2} & \(+\) & \(-\) & \(-\) & \onslide+<3->{\(-\)} & \onslide+<3->{\(-\)} & \onslide+<3->{\(+\)} & \onslide+<3->{\(+\)} \\
		\textbf{3} & \(-\) & \(+\) & \(-\) & \onslide+<3->{\(-\)} & \onslide+<3->{\(+\)} & \onslide+<3->{\(-\)} & \onslide+<3->{\(+\)} \\
		\textbf{4} & \(+\) & \(+\) & \(-\) & \onslide+<3->{\(+\)} & \onslide+<3->{\(-\)} & \onslide+<3->{\(-\)} & \onslide+<3->{\(-\)} \\
		\textbf{5} & \(-\) & \(-\) & \(+\) & \onslide+<3->{\(+\)} & \onslide+<3->{\(-\)} & \onslide+<3->{\(-\)} & \onslide+<3->{\(+\)} \\
		\textbf{6} & \(+\) & \(-\) & \(+\) & \onslide+<3->{\(-\)} & \onslide+<3->{\(+\)} & \onslide+<3->{\(-\)} & \onslide+<3->{\(-\)} \\
		\textbf{7} & \(-\) & \(+\) & \(+\) & \onslide+<3->{\(-\)} & \onslide+<3->{\(-\)} & \onslide+<3->{\(+\)} & \onslide+<3->{\(-\)} \\
		\textbf{8} & \(+\) & \(+\) & \(+\) & \onslide+<3->{\(+\)} & \onslide+<3->{\(+\)} & \onslide+<3->{\(+\)} & \onslide+<3->{\(+\)} \\  
	\end{tabulary}
	
	\vspace{0.4cm}
	
	
	\begin{columns}[T]
		
		\column{0.45\textwidth}
			\centerline{\includegraphics[width=.5\textwidth]{../4G/Supporting material/trade-off-design-example-saturated.png}}
		
		\column{0.48\textwidth}
			\onslide+<2->{
				{\huge $2^{k-p}$\\
				{\normalsize with} $p=4$}
			}
	\end{columns}	
\end{frame}

\begin{frame}\frametitle{Creating the standard order table for the fractional factorial}
	\vspace{0.4cm}
	\begin{tabulary}{\linewidth}{c|ccccccc|c}
		\textbf{\relax Experiment} & \textbf{\relax A } & \textbf{\relax B} & \textbf{\relax C } & \onslide+<1->{\textbf{\relax D = AB}} & \onslide+<1->{\textbf{\relax E = AC}} & \onslide+<1->{\textbf{\relax F = BC}} & \onslide+<1->{\textbf{\relax G = ABC}} & \onslide+<1->{$y$}\\ \cline{1-9}
		\textbf{1} & \(-\) & \(-\) & \(-\) & \onslide+<1->{\(+\)} & \onslide+<1->{\(+\)} & \onslide+<1->{\(+\)} & \onslide+<1->{\(-\)} \\
		\textbf{2} & \(+\) & \(-\) & \(-\) & \onslide+<1->{\(-\)} & \onslide+<1->{\(-\)} & \onslide+<1->{\(+\)} & \onslide+<1->{\(+\)} \\
		\textbf{3} & \(-\) & \(+\) & \(-\) & \onslide+<1->{\(-\)} & \onslide+<1->{\(+\)} & \onslide+<1->{\(-\)} & \onslide+<1->{\(+\)} \\
		\textbf{4} & \(+\) & \(+\) & \(-\) & \onslide+<1->{\(+\)} & \onslide+<1->{\(-\)} & \onslide+<1->{\(-\)} & \onslide+<1->{\(-\)} \\
		\textbf{5} & \(-\) & \(-\) & \(+\) & \onslide+<1->{\(+\)} & \onslide+<1->{\(-\)} & \onslide+<1->{\(-\)} & \onslide+<1->{\(+\)} \\
		\textbf{6} & \(+\) & \(-\) & \(+\) & \onslide+<1->{\(-\)} & \onslide+<1->{\(+\)} & \onslide+<1->{\(-\)} & \onslide+<1->{\(-\)} \\
		\textbf{7} & \(-\) & \(+\) & \(+\) & \onslide+<1->{\(-\)} & \onslide+<1->{\(-\)} & \onslide+<1->{\(+\)} & \onslide+<1->{\(-\)} \\
		\textbf{8} & \(+\) & \(+\) & \(+\) & \onslide+<1->{\(+\)} & \onslide+<1->{\(+\)} & \onslide+<1->{\(+\)} & \onslide+<1->{\(+\)} \\ 
	\end{tabulary}
	
	\vspace{0.4cm}
	
	
	\begin{columns}[T]
		
		\column{0.45\textwidth}
			\centerline{\includegraphics[width=.5\textwidth]{../4G/Supporting material/trade-off-design-example-saturated.png}}
		
		\column{0.48\textwidth}
			\onslide+<1->{
				{\huge $2^{k-p}$\\
				{\normalsize with} $p=4$}
			}
	\end{columns}
\end{frame}

\begin{frame}\frametitle{\includegraphics[width=.07\textwidth]{../4G/Supporting material/flickr-9568156463_1809c97b21_o-checklist.jpg} Let's follow the recommended approach shown earlier in the video}
	
	\begin{enumerate}
		\item	Read the generators from the trade off table 
			\begin{itemize}
				\item		$\textbf{D = AB}$  and $\textbf{E = AC}$ and $\textbf{F = BC}$  and $\textbf{G = ABC}$ 
			\end{itemize}
\onslide+<2->{
		\item	Rearrange the generators as  $\textbf{I = \ldots}$
			\begin{itemize}
				\item	$\textbf{I = ABD}$ and $\textbf{I = ACE}$ and $\textbf{I = BCF}$ and $\textbf{I = ABCG}$
			\end{itemize}
}
\onslide+<3->{
		\item	Form the {\color{purple}\textbf{defining relationship}} taking all combinations of the words: $\textbf{I = \ldots}$
}
\onslide+<4->{
	   	 \fbox{\parbox[b][4.5em][t]{0.92\textwidth}{
			\textbf{I = ABD = ACE = BCF = ABCG = $\ldots$} {\small \color{myOrange} \hfill $\longleftarrow$ that's 5 so far} \\
\onslide+<5->{
			\,\,\textbf{BDCE}}\onslide+<6->{\textbf{ = ACDF}}\onslide+<7->{\textbf{ = CDG = ABEF = BEG = AFG = $\ldots$} {\small \color{myOrange} \hfill $\longleftarrow$ that's 11} \\
}
\onslide+<8->{
			\textbf{DEF}} \onslide+<9->{\textbf{= ADEG = CEFG = BDFG = $\ldots$}{\small \color{myOrange} \hfill $\longleftarrow$ we are up to 15} \\
}
\onslide+<10->{
			\textbf{ABCDEFG}{\small \color{myOrange} \hfill $\longleftarrow$ we have all 16 here} \\
}
		 	
	   	 } }
}
\onslide+<5-5>{\textbf{(ABD)(ACE) = BCDE}}\,\onslide+<6-6>{\textbf{(ABD)(BCF) = ACDF}}\, \onslide+<8-8>{\textbf{(ABD)(ACE)(BCF) = DEF}}
\onslide+<3->{
		\item	Ensure the defining relationship has $2^p$ words
			\begin{itemize}
				\item	$p=4$, so we have 16 words.
			\end{itemize}
}
\onslide+<11->{
		\item	Use the defining relationship to compute the aliasing pattern
		\item	Ensure the aliasing is acceptable
}		
	\end{enumerate}
\end{frame}

\begin{frame}\frametitle{Check the alias patterns by using the defining relationship}
	
	
	\begin{align*}
		\textbf{A} &= \textbf{BD = CE = FG \color{lightgray} = BCG = CDF = BEF = DEG =  ABCF = ABEG = $\ldots$} \\		
		&\qquad\qquad\qquad\color{lightgray} \textbf{ = ACDG = ADEF = ABDCE = ABDFG = ACEFG = BCDEFG}\\
\onslide+<2->{
		\textbf{B} &= \textbf{AD = CF = EG} \color{lightgray} \,+\, \text{other higher order interactions} \\
		\textbf{C} &= \textbf{AE = BF = DG}\\
		\textbf{D} &= \textbf{AB = CG = EF}\\
		\textbf{E} &= \textbf{AC = BG = DF}\\
		\textbf{F} &= \textbf{BC = AG = DE}\\
		\textbf{G} &= \textbf{CD = BE = AF}
}
	\end{align*}
\end{frame}

\begin{frame}\frametitle{Finally! You get to do the experiments and record the outcome value}
	\vspace{0.4cm}
	\begin{tabulary}{\linewidth}{c|ccccccc|c}\hline
		\textbf{\relax Experiment} & \textbf{\relax A } & \textbf{\relax B} & \textbf{\relax C } & \textbf{\relax D = AB} & \textbf{\relax E = AC} & \textbf{\relax F = BC} & \textbf{\relax G = ABC} & $y$\\ \hline
		\textbf{1} & \(-\) & \(-\) & \(-\) & \(+\) & \(+\) & \(+\) & \(-\) & 320\\
		\textbf{2} & \(+\) & \(-\) & \(-\) & \(-\) & \(-\) & \(+\) & \(+\) & 276\\
		\textbf{3} & \(-\) & \(+\) & \(-\) & \(-\) & \(+\) & \(-\) & \(+\) & 306\\
		\textbf{4} & \(+\) & \(+\) & \(-\) & \(+\) & \(-\) & \(-\) & \(-\) & 290\\
		\textbf{5} & \(-\) & \(-\) & \(+\) & \(+\) & \(-\) & \(-\) & \(+\) & 272\\
		\textbf{6} & \(+\) & \(-\) & \(+\) & \(-\) & \(+\) & \(-\) & \(-\) & 274\\
		\textbf{7} & \(-\) & \(+\) & \(+\) & \(-\) & \(-\) & \(+\) & \(-\) & 290\\
		\textbf{8} & \(+\) & \(+\) & \(+\) & \(+\) & \(+\) & \(+\) & \(+\) & 255\\ \hline
	\end{tabulary}
\end{frame}

\begin{frame}\frametitle{The reduced model: only has four main effect factors}
	\renewcommand{\lg}{\color{lightgray}}
	
	\vspace{0.4cm}
	\begin{tabulary}{\linewidth}{c|ccccccc|c}\hline
		\textbf{\relax Experiment} & \textbf{\relax A } & \textbf{\relax C} & \textbf{\relax E } & \textbf{\relax G} & \lg \textbf{\relax B} & \lg \textbf{\relax D} & \lg \textbf{\relax F} & $y$\\ \hline
		\textbf{1} & \(-\) & \(-\) & \(+\) & \(-\) & \lg \(-\) & \lg  \(+\) & \lg  \(+\) & 320\\
		\textbf{2} & \(+\) & \(-\) & \(-\) & \(+\) & \lg \(-\) & \lg  \(-\) & \lg  \(+\) & 276\\
		\textbf{3} & \(-\) & \(-\) & \(+\) & \(+\) & \lg \(+\) & \lg  \(-\) & \lg  \(-\) & 306\\
		\textbf{4} & \(+\) & \(-\) & \(-\) & \(-\) & \lg \(+\) & \lg  \(+\) & \lg  \(-\) & 290\\
		\textbf{5} & \(-\) & \(+\) & \(-\) & \(+\) & \lg \(-\) & \lg  \(+\) & \lg  \(-\) & 272\\
		\textbf{6} & \(+\) & \(+\) & \(+\) & \(-\) & \lg \(-\) & \lg  \(-\) & \lg  \(-\) & 274\\
		\textbf{7} & \(-\) & \(+\) & \(-\) & \(-\) & \lg \(+\) & \lg  \(-\) & \lg  \(+\) & 290\\
		\textbf{8} & \(+\) & \(+\) & \(+\) & \(+\) & \lg \(+\) & \lg  \(+\) & \lg  \(+\) & 255\\ \hline
	\end{tabulary}
	
	\vspace{1cm}
	Those that are more math oriented: please verify that each column is uncorrelated with the others. So rebuilding the model implies the factor estimates are the same.
\end{frame}

\begin{frame}\frametitle{Other fractional factorial designs: Plackett-Burman designs}
 	\begin{columns}[T]
 		
 		\column{0.55\textwidth}
			
			Plackett-Burman designs exists in \\
			multiples of 4:
			
				\begin{itemize}
					\item	4, 8, 12, 16, 20, 24, 28, 32, \\
							36, 40, $\ldots$
					
					\item	Main effects are confounded \\
							with two-factor interactions,\\
							but in a complicated way.
							
					\item	Such designs are most usefully \\
							created by software, unlike the\\
							fractional factorial designs \\
							shown in this section.
							
					\item	e.g. a Placket-Burman design\\
							in 20 runs, can screen for\\
							19 factors\emph{!}
							
				\end{itemize}
 		\column{0.5\textwidth}
			\hbox{\hspace{-5.5em}
				\includegraphics[height=.9\textheight]{\imagedir/doe/DOE-trade-off-table-MOOC-resolution-plain-cropped.png}
			}
					
 	\end{columns}
	 
	 
	
\end{frame}

\begin{frame}\frametitle{Definitive Screening Designs: a type of optimal design}
	\begin{columns}[T]
		\column{0.6\textwidth}
			\textbf{Optimal designs}
			
			\begin{itemize}
				\item	There are several desirable mathematical criteria that can be optimal.
				\item	We won't go into the details, but factorial designs often meet these optimal criteria.
				
				\item	Interested in the details? Search for:
					\begin{itemize}
						\item	\texttt{D-optimal designs}\\
							it is the most common optimal design
					\end{itemize}
			\end{itemize}
			
			\onslide+<2->{
			\textbf{Definitive screening designs}
				\begin{itemize}
					\item	Factors can be at 3 levels (not 2!)
					\item	Small number of runs
					\item	Main effects and 2-factor interactions \textbf{are not} aliased - a great advantage.
				\end{itemize}
			}
		\column{0.02\textwidth}
			\rule[3mm]{0.01cm}{90mm}
		\column{0.48\textwidth}
			
			\centerline{\includegraphics[width=.8\textwidth]{../4G/Supporting material/D-optimal-paper.png}}
			\see{\href{http://www.jstor.org/discover/10.2307/1267635}{http://www.jstor.org/discover/10.2307/1267635}}
		
			\onslide+<2->{
				\vspace{1cm}
				\centerline{\includegraphics[width=.8\textwidth]{../4H/Supporting material/definitive-screening-design-paper.png}}
				\see{\href{http://yint.org/dsdesign}{http://yint.org/dsdesign}}
			}
			
	\end{columns}	
\end{frame}

\begin{frame}\frametitle{\includegraphics[width=0.3\textwidth]{\imagedir/doe/examples/advice-logo.png}\,\, Practice, fail, start over, and persist}

	\begin{itemize}
		\item	There are case studies in the course textbook
		\item	There are other textbooks, listed on the course website
		\item	Create your own datasets
			\begin{itemize}
				\item	biscuits
				\item	coffee
				\item	growing plants, or
				\item	many of the experiments suggested in the course forums
				
			\end{itemize}
	\end{itemize}
\end{frame}





% /Users/kevindunn/Dropbox/Coursera/Media/All-course-slides/classes/CourseraMOOC-class-5A.tex

\begin{frame}\frametitle{Achievable objectives when improving a process -- based on data}
	\pause
	\begin{columns}[t]
		\column{0.65\textwidth}
			\begin{enumerate}
				\item	learn more about, and increase your understanding  \pause
				\item	troubleshoot a problem  \pause
				\item	make predictions  \pause
				\item	try to optimize the process  \pause
				\item	monitor the system to ensure performance gains are retained  \pause
			\end{enumerate}
	
		\column{0.35\textwidth}
			\onslide+<5->{
				\includegraphics[width=\textwidth]{\imagedir/examples/reactor-design-example/response-surface.png}
			}
			\onslide+<6->{
				\includegraphics[width=\textwidth]{\imagedir/monitoring/Kappa-phaseII-testing.png}
			}
			
	\end{columns}
	
	\vspace{-1cm}
	\onslide+<7->{
		Think back to your prior projects; \\what were your objectives?
	}

	\vspace{1cm}
	\onslide+<8->{
		\hfill \includegraphics[width=0.3\textwidth]{\imagedir/doe/examples/advice-logo.png}
		\,\,{\color{blue}Always have a clear objective in mind}
	}
\end{frame}

\begin{frame}\frametitle{1. Learning more and increasing our understanding}
	\begin{columns}[T]
		\column{0.45\textwidth}
			\includegraphics[width=\textwidth]{\imagedir/doe/examples/solar-panel-mendelu-cz-website.png}
			
			
			\see{\href{http://yint.org/solar-panel-study}{http://yint.org/solar-panel-study}}
			
		\column{0.48\textwidth}
			\includegraphics[width=1.2\textwidth]{../5A/Supporting files/5A-1-important-factors.pdf}
		
			\vspace{.5cm}
			\color{myOrange}\tiny We studied this example earlier.
			
	\end{columns}
\end{frame}

\begin{frame}\frametitle{2. Troubleshooting a difficult problem in your company}
	\begin{columns}[T]
		\column{0.6\textwidth}
			\includegraphics[width=\textwidth]{\imagedir/examples/competitor-product/competitor-product.png}
			
			
		\column{0.45\textwidth}
			Why is {\color{purple} my company's product (purple)} so different from my {\color{red} competitor's}? \\
			I'm losing customers because of this!
			
			\vspace{0.5cm}
			{\color{blue}\emph{Potential causes:}}
			\begin{itemize}
				\item	raw material has become more brittle
				\item	seasonal variability affects our process
				\item	we are operating at slower rates
				\item	was it the move to higher pressures last month
				\item	the new staff that we hired?
			\end{itemize}
			
	\end{columns}
\end{frame}

\begin{frame}\frametitle{3. Making predictions from your data}
	\includegraphics[width=\textwidth]{../5A/Supporting files/2C-screenshot.png}
\end{frame}

\begin{frame}\frametitle{Back to making popcorn}
	\begin{columns}[T]
		\column{0.7\textwidth}
			\includegraphics[width=\textwidth]{../5A/Supporting files/flickr-ed_welker-4137738962_064e1f4604_o-popcorn.jpg}

		\column{0.30\textwidth}
			\see{\href{https://secure.flickr.com/photos/ed_welker/4137738962/}{Flickr: ed\_welker}}
			
			\vspace{5cm}
			%{\small Read the forum posts: \href{http://yint.org/popcorn-post}{http://yint.org/popcorn-post}}
	\end{columns}
	
\end{frame}

\begin{frame}\frametitle{Back to making popcorn: our factors, and our objective function}
	\begin{columns}[T]
		\column{0.60\textwidth}
			\vspace{6cm}
			{\tiny Space for animated calculations here}

		\column{0.4\textwidth}
			\includegraphics[width=\textwidth]{../5A/Supporting files/flickr-skaty222-404186384_9e48c08087_o-burned.jpg}
			
			\see{\href{https://secure.flickr.com/photos/skaty222/404186384}{Flickr: skaty222}}
			
	\end{columns}
	
\end{frame}

\begin{frame}\frametitle{What ``Response Surface Methods'' are all about}
	
	\begin{columns}[c]
		\column{0.55\textwidth}
			\begin{exampleblock}{}
				Sequential groups of experiments\\
				to create empirical models\\
				to reach an optimum\\
				efficiently\\
				using only factors that affect your outcome
			\end{exampleblock}
	
		\column{0.45\textwidth}
			\includegraphics[width=\textwidth]{\imagedir/examples/reactor-design-example/response-surface.png}
			
			\includegraphics[width=\textwidth]{../5A/Supporting files/StatsMOOCLogo2c.jpg}
	\end{columns}
	
\end{frame}



% /Users/kevindunn/Dropbox/Coursera/Media/All-course-slides/classes/CourseraMOOC-class-5B.tex


%$\hat{y} = b_0 + b_\text{A}x_\text{A} + {\color{blue}b_\text{AA}x^2_\text{AA}}$
%$\hat{y} = 91.8 + 14.9 x_\text{A}  -2 x^2_\text{AA}$
%$\color{red}\hat{y} = 91.8 + 14.9 (+1)  -2 (+1)^2 \approx 105 $
% \color[rgb]{0.538744,0.117572,0.032380}

\begin{frame}\frametitle{The {\color{red} critical concepts} covered in this video using the {\color{myOrange}popcorn case study}}
	\begin{enumerate}
		\item	Real-world units and coded units
		\item	\textbf{Linear vs nonlinear systems}
		\item	Prediction models are wrong, but still useful
		\item	\textbf{Noise and error} and the need for replicated experiments
		\item	How to systematically reach an optimum
		\item	Justifying the choice of every experiment
	\end{enumerate}
\end{frame}

\begin{frame}\frametitle{The connection between our model's coded units, and real-world units}
	\begin{columns}[b]
		
		
		\column{0.65\textwidth}
			\textbf{The general formula for continuous variables}
			\begin{flalign*}
				\onslide+<2->{
					\text{coded value} &= \dfrac{(\text{real value}) - (\text{center value})}{\tfrac{1}{2}\left(\text{range}\right)}\\ \\
				}
				\onslide+<3->{
					\text{center value} &= \dfrac{(\text{low value}) + (\text{high value})}{2}\\ \\
				}
				\onslide+<4->{
					\text{range} &= (\text{high value}) - (\text{low value}) &\\
				}
			\end{flalign*}
			
		\column{0.02\textwidth}
			
			\rule[3mm]{0.01cm}{55mm}%
			
		
		\column{0.40\textwidth}
			\textbf{Example:} cooking time, factor \textbf{A}
			\begin{flalign*}
				\onslide+<5->{	
					x_\text{A} &= \dfrac{\,\text{A} - 135}{\tfrac{1}{2}\left(30\right)\,} = \dfrac{\,\text{A} - 135}{15}\\ \\
				}
				\onslide+<3->{
					\text{center}_\text{A} &= \dfrac{120 + 150}{2} = 135\\ \\
				}
				\onslide+<4->{
					\text{range}_\text{A} &= 150 - 120 = 30 & \\
				}
			\end{flalign*}		
			
	\end{columns}
	\vspace{-.2cm}
	\centerline{\includegraphics[width=.4\textwidth]{../5B/Supporting materials/popcorn-coded-real.png}}
\end{frame}

\begin{frame}\frametitle{Let's try it out}
	
	\begin{flalign*}
		\text{coded value} &= \dfrac{(\text{real value}) - (\text{center value})}{\tfrac{1}{2}\left(\text{range}\right)} &
		\qquad x_\text{A} &= \dfrac{\,\text{A} - 135}{\tfrac{1}{2}\left(30\right)\,} = \dfrac{\,\text{A} - 135}{15}\\ 
		\\
		\onslide+<2->{
			&\text{If cooking time is 135 seconds, what is $x_\text{A}$?} &
		}
		\onslide+<3->{
			x_\text{A} &= \dfrac{135 - 135}{\tfrac{1}{2}\left(30\right)\,} = \dfrac{0}{15} = 0
		}\\ \\
		\onslide+<4->{
			&\text{If cooking time is 150 seconds, what is $x_\text{A}$?} &
		}
		\onslide+<5->{
			x_\text{A} &= \dfrac{150 - 135}{\tfrac{1}{2}\left(30\right)\,} = \dfrac{15}{15} = +1
		}
	\end{flalign*}
			
	
\end{frame}

\begin{frame}\frametitle{The connection between real-world units and  coded units (in reverse\emph{!})}
	\begin{columns}[b]
				
		\column{0.65\textwidth}
			\begin{flalign*}
				\intertext{\textbf{Going forwards}:}
				\text{coded value} &= \dfrac{(\text{real value}) - (\text{center value})}{\tfrac{1}{2}\left(\text{range}\right)}\\ \\
				\intertext{\textbf{Going backwards}:}
				\text{real value} &= (\text{coded value}) \times \tfrac{1}{2}\left(\text{range}\right) + (\text{center value}) \\ \\ \\
			\end{flalign*}
			
		\column{0.02\textwidth}
			\rule[3mm]{0.01cm}{55mm}
		
		\column{0.40\textwidth}
			
			\begin{flalign*}
				\intertext{\textbf{Example:} cooking time, factor \textbf{A}}
				x_\text{A} &= \dfrac{\,\text{A} - 135}{\tfrac{1}{2}\left(30\right)\,} = \dfrac{\,\text{A} - 135}{15}\\ \\
				\intertext{\color{myOrange} What is $x_\text{A} = +2$?:} 
				\onslide+<2->{
				\text{real value} &= (+2) \times \tfrac{1}{2}\left(30\right) + (135)\\
					\onslide+<3->{
					 			  &= (+2) \times 15 + (135)}\\
					\onslide+<4->{
								  &= 30 + 135 = 165\\}
				}
			\end{flalign*}		
			
	\end{columns}
	
	\vspace{2cm}
	\vfill
\end{frame}

\begin{frame}\frametitle{}
	\centerline{\includegraphics[width=.98\textwidth]{../5B/Supporting materials/popcorn-experiments-01.png}}
\end{frame}
\begin{frame}\frametitle{}
	\centerline{\includegraphics[width=.98\textwidth]{../5B/Supporting materials/popcorn-experiments-02.png}}
\end{frame}
\begin{frame}\frametitle{}
	\centerline{\includegraphics[width=.98\textwidth]{../5B/Supporting materials/popcorn-experiments-03.png}}
\end{frame}
\begin{frame}\frametitle{}
	\centerline{\includegraphics[width=.98\textwidth]{../5B/Supporting materials/popcorn-experiments-04.png}}
\end{frame}
\begin{frame}\frametitle{}
	\centerline{\includegraphics[width=.98\textwidth]{../5B/Supporting materials/popcorn-experiments-05.png}}
\end{frame}
\begin{frame}\frametitle{}
	\centerline{\includegraphics[width=.98\textwidth]{../5B/Supporting materials/popcorn-experiments-06.png}}
\end{frame}
\begin{frame}\frametitle{The famous quote by George Box}
	\begin{quote}
		``... all models are wrong, but some are useful.''
	\end{quote}
	
	\begin{quote}
		``...the practical question is how wrong do they have to be [before they are] not useful?''
	\end{quote}
	
	\vspace{1cm}
	G. E. P. Box and  N. R. Draper (1987), ``\emph{Empirical Model Building and Response Surfaces}'', John Wiley \& Sons, New York, NY.
\end{frame}
\begin{frame}\frametitle{Answering George Box's question for the popcorn example}
	
	What is ``\emph{not useful}\,'' in this case?
	
	\vspace{1cm}
	In an earlier video we said: ``always have an objective in mind''.
	
	\vspace{1cm}
	\begin{exampleblock}{The answer to the question}
		Our model is not useful when the predictions are not accurate; we need accurate predictions to optimize.
	\end{exampleblock}
\end{frame}
\begin{frame}\frametitle{}
	\centerline{\includegraphics[width=.98\textwidth]{../5B/Supporting materials/popcorn-experiments-07.png}}
\end{frame}
\begin{frame}\frametitle{}
	\centerline{\includegraphics[width=.98\textwidth]{../5B/Supporting materials/popcorn-experiments-08.png}}
\end{frame}
\begin{frame}\frametitle{}
	\centerline{\includegraphics[width=.98\textwidth]{../5B/Supporting materials/popcorn-experiments-09.png}}
\end{frame}
\begin{frame}\frametitle{}
	\centerline{\includegraphics[width=.98\textwidth]{../5B/Supporting materials/popcorn-experiments-10.png}}
\end{frame}
\begin{frame}\frametitle{}
	\centerline{\includegraphics[width=.98\textwidth]{../5B/Supporting materials/popcorn-experiments-11.png}}
\end{frame}
\begin{frame}\frametitle{}
	\centerline{\includegraphics[width=.98\textwidth]{../5B/Supporting materials/popcorn-experiments-12.png}}
\end{frame}
\begin{frame}\frametitle{}
	\centerline{\includegraphics[width=.98\textwidth]{../5B/Supporting materials/popcorn-experiments-13.png}}
\end{frame}
\begin{frame}\frametitle{}
	\centerline{\includegraphics[width=.98\textwidth]{../5B/Supporting materials/popcorn-experiments-14.png}}
\end{frame}
\begin{frame}\frametitle{}
	\centerline{\includegraphics[width=.98\textwidth]{../5B/Supporting materials/popcorn-experiments-15.png}}
\end{frame}
\begin{frame}\frametitle{}
	\centerline{\includegraphics[width=.98\textwidth]{../5B/Supporting materials/popcorn-experiments-16.png}}
\end{frame}
\begin{frame}\frametitle{}
	\centerline{\includegraphics[width=.98\textwidth]{../5B/Supporting materials/popcorn-experiments-17.png}}
\end{frame}
\begin{frame}\frametitle{}
	\centerline{\includegraphics[width=.98\textwidth]{../5B/Supporting materials/popcorn-experiments-18.png}}
\end{frame}
\begin{frame}\frametitle{}
	\centerline{\includegraphics[width=.98\textwidth]{../5B/Supporting materials/popcorn-experiments-19.png}}
\end{frame}
\begin{frame}\frametitle{}
	\centerline{\includegraphics[width=.98\textwidth]{../5B/Supporting materials/popcorn-experiments-20.png}}
\end{frame}
\begin{frame}\frametitle{}
	\centerline{\includegraphics[width=.98\textwidth]{../5B/Supporting materials/popcorn-experiments-21.png}}
\end{frame}
\begin{frame}\frametitle{}
	\centerline{\includegraphics[width=.98\textwidth]{../5B/Supporting materials/popcorn-experiments-22.png}}
\end{frame}
\begin{frame}\frametitle{}
	\centerline{\includegraphics[width=.98\textwidth]{../5B/Supporting materials/popcorn-experiments-23.png}}
\end{frame}
\begin{frame}\frametitle{}
	\centerline{\includegraphics[width=.98\textwidth]{../5B/Supporting materials/popcorn-experiments-24.png}}
\end{frame}
\begin{frame}\frametitle{}
	\centerline{\includegraphics[width=.98\textwidth]{../5B/Supporting materials/popcorn-experiments-25.png}}
\end{frame}
\begin{frame}\frametitle{}
	\centerline{\includegraphics[width=.98\textwidth]{../5B/Supporting materials/popcorn-experiments-26.png}}
\end{frame}
\begin{frame}\frametitle{}
	\centerline{\includegraphics[width=.98\textwidth]{../5B/Supporting materials/popcorn-experiments-27.png}}
\end{frame}
\begin{frame}\frametitle{}
	\centerline{\includegraphics[width=.98\textwidth]{../5B/Supporting materials/popcorn-experiments-28.png}}
\end{frame}
\begin{frame}\frametitle{}
	\centerline{\includegraphics[width=.98\textwidth]{../5B/Supporting materials/popcorn-experiments-29.png}}
\end{frame}
\begin{frame}\frametitle{}
	\centerline{\includegraphics[width=.98\textwidth]{../5B/Supporting materials/popcorn-experiments-30.png}}
\end{frame}
\begin{frame}\frametitle{}
	\centerline{\includegraphics[width=.98\textwidth]{../5B/Supporting materials/popcorn-experiments-31.png}}
\end{frame}
\begin{frame}\frametitle{}
	\centerline{\includegraphics[width=.98\textwidth]{../5B/Supporting materials/popcorn-experiments-32.png}}
\end{frame}
\begin{frame}\frametitle{}
	\centerline{\includegraphics[width=.98\textwidth]{../5B/Supporting materials/popcorn-experiments-33.png}}
\end{frame}
\begin{frame}\frametitle{}
	\centerline{\includegraphics[width=.98\textwidth]{../5B/Supporting materials/popcorn-experiments-34.png}}
\end{frame}
\begin{frame}\frametitle{}
	\centerline{\includegraphics[width=.98\textwidth]{../5B/Supporting materials/popcorn-experiments-35.png}}
\end{frame}
\begin{frame}\frametitle{}
	\centerline{\includegraphics[width=.98\textwidth]{../5B/Supporting materials/popcorn-experiments-36.png}}
\end{frame}
\begin{frame}\frametitle{Recap of each experiment used so far}
	\begin{itemize}
		\item	\textbf{1, 2, 3, 4}
		
			\hspace{1ex} the used to rule out factor B (oil type)\\
			\hspace{1ex} used  to provide an initial model
			
		\pause
		
		\item	\textbf{5}
		
			\hspace{1ex} center point: confirmed our model was linear\\
			\hspace{1ex} used to check prediction quality
		
		\pause
			
		\item	\textbf{6}
		
			\hspace{1ex} exploratory first step outside the initial factorial\\
			\hspace{1ex} to test the model's prediction quality\\
			\hspace{1ex} it suggested we rebuild the model (add quadratic term)
		
		\pause
		
		\item	\textbf{7}
		
			\hspace{1ex} seems like we are near an optimum

	\end{itemize}
	\onslide+<5->{
		\hspace{1cm} \includegraphics[width=0.3\textwidth]{\imagedir/doe/examples/advice-logo.png}
				\,\,{\color{blue}are you able to justify the need for each experiment?}
	}
\end{frame}
\begin{frame}\frametitle{The {\color{red} critical concepts} covered in this video using the {\color{myOrange}popcorn case study}}
	\begin{enumerate}
		\item	Real-world units and coded units
		\item	Linear vs nonlinear systems
		\item	Prediction models are wrong, but still useful
		\item	Noise and error and the need for replicated experiments
		\item	How to systematically reach an optimum
		\item	Justifying the choice of every experiment
	\end{enumerate}
\end{frame}


% /Users/kevindunn/Dropbox/Coursera/Media/All-course-slides/classes/CourseraMOOC-class-5C.tex

\begin{frame}\frametitle{}
	\centerline{\includegraphics[height=.8\textheight]{../5C/Supporting material/flickr-quinnanya-2186957732_12f67e557b_o-grocery-shelf.jpg}}
	\see{\href{https://secure.flickr.com/photos/quinnanya/2186957732}{Flickr: quinnanya}}
\end{frame}
\begin{frame}\frametitle{}
	\centerline{\includegraphics[height=\textheight]{../5C/Supporting material/COST-contours-shopping-01a.png}}
\end{frame}
\begin{frame}\frametitle{}
	\centerline{\includegraphics[height=\textheight]{../5C/Supporting material/COST-contours-shopping-01b.png}}
\end{frame}
\begin{frame}\frametitle{}
	\centerline{\includegraphics[height=\textheight]{../5C/Supporting material/COST-contours-shopping-02a.png}}
\end{frame}
\begin{frame}\frametitle{}
	\centerline{\includegraphics[height=\textheight]{../5C/Supporting material/COST-contours-shopping-02b.png}}
\end{frame}
\begin{frame}\frametitle{}
	\centerline{\includegraphics[height=\textheight]{../5C/Supporting material/COST-contours-shopping-03.png}}
\end{frame}
\begin{frame}\frametitle{}
	\centerline{\includegraphics[height=\textheight]{../5C/Supporting material/COST-contours-shopping-04.png}}
\end{frame}
\begin{frame}\frametitle{}
	\centerline{\includegraphics[height=\textheight]{../5C/Supporting material/COST-contours-shopping-05.png}}
\end{frame}
\begin{frame}\frametitle{}
	\centerline{\includegraphics[height=\textheight]{../5C/Supporting material/COST-contours-shopping-06a.png}}
\end{frame}
\begin{frame}\frametitle{}
	\centerline{\includegraphics[height=\textheight]{../5C/Supporting material/COST-contours-shopping-06b.png}}
\end{frame}
\begin{frame}\frametitle{}
	\centerline{\includegraphics[height=\textheight]{../5C/Supporting material/COST-contours-shopping-07.png}}
\end{frame}
\begin{frame}\frametitle{}
	\centerline{\includegraphics[height=\textheight]{../5C/Supporting material/COST-contours-shopping-08.png}}
\end{frame}
\begin{frame}\frametitle{}
	\centerline{\includegraphics[height=\textheight]{../5C/Supporting material/COST-contours-shopping-09.png}}
\end{frame}
\begin{frame}\frametitle{}
	\centerline{\includegraphics[height=\textheight]{../5C/Supporting material/COST-contours-shopping-10.png}}
\end{frame}
\begin{frame}\frametitle{}
	\centerline{\includegraphics[height=\textheight]{../5C/Supporting material/COST-contours-shopping-11.png}}
\end{frame}
\begin{frame}\frametitle{}
	\centerline{\includegraphics[height=\textheight]{../5C/Supporting material/COST-contours-shopping-12.png}}
\end{frame}

\begin{frame}\frametitle{The COST approach: {\color{myGreen}``Change One Single Thing'' at a time}}
	\begin{columns}[b]
		\column{0.6\textwidth}
			\begin{exampleblock}{}
				\begin{itemize}
					\item	leads you into a false belief that you have reached the optimum \pause
					\item	COST can possibly work in 2-dimensions, and maybe with 3 variables
					\item	but the chance you hit the optimum gets lower \pause
					\item	interactions and other unusual surface shapes makes COST inefficient \pause
					
					\item	OFAT = one factor at a time \\(another name for COST)
				\end{itemize}
			\end{exampleblock}
		
		\column{0.40\textwidth}
			\centerline{\includegraphics[width=1.1\textwidth]{../5C/Supporting material/COST-contours-shopping-12.png}}
	\end{columns}
	
	\pause
	\vspace{1cm}
	{\color{myOrange}COST works in a lab: to prove a conclusive cause-effect relationship}
	\\
	\pause
	\vspace{0.5cm}
	{\color{blue} Recall, RSM {\small (response surface methods)} is used \textbf{after} screening for known causal factors}

\end{frame}


% /Users/kevindunn/Dropbox/Coursera/Media/All-course-slides/classes/CourseraMOOC-class-5D.tex

\begin{frame}\frametitle{Interpreting what contour plots are}
	\begin{columns}[b]
		\column{0.5\textwidth}
			\centerline{\includegraphics[width=1.1\textwidth]{../5D/Supporting material/wikipedia-20121125213159!FujiSunriseKawaguchiko2025WP.jpg}}
			
			\see{Wikipedia: \href{https://commons.wikimedia.org/wiki/File:FujiSunriseKawaguchiko2025WP.jpg}{Wikipedia}}
		
		\column{0.5\textwidth}
			\centerline{\includegraphics[width=1.102\textwidth]{../5D/Supporting material/Google-Map-Screenshot-wide.png}}

			\see{\href{https://www.google.com/maps/preview?f=q&hl=en&geocode=&q=Mt.+Fuji&ie=UTF8&t=p&ll=35.366656,138.733292&spn=0.099668,0.207367&z=13&iwloc=addr;}{Google Maps link}}
	\end{columns}
	
\end{frame}

\begin{frame}\frametitle{Why we need response surface methods}
	\centerline{\includegraphics[width=.7\textwidth]{../5D/Supporting material/flickr-slgc-5812112251_f0b0e253c5_o-basement-construction.jpg}}
	\see{\href{https://secure.flickr.com/photos/slgc/5812112251}{Flickr: slgc}}	
\end{frame}

\begin{frame}\frametitle{Some questions on how we will use response surface methods }
	\begin{columns}[c]
		\column{0.5\textwidth}
			\centerline{\includegraphics[width=1.102\textwidth]{../5C/Supporting material/COST-contours-shopping-11.png}}

		\column{0.5\textwidth}
			\begin{itemize}
				\item	Observing the surface is costly
						
						\hspace{1ex} {\scriptsize each touch on the surface = experiment }\\
						\hspace{1ex} {\scriptsize we want to do as few experiments as possible }
				\pause
				\item	How do we get to the top quickly and efficiently?
				
				\pause
				\item	How will we know we are actually at the top?
			\end{itemize}
	\end{columns}
	
\end{frame}

\begin{frame}\frametitle{What does the response surface consist of?}
	
	The response surface is a plot of the outcome variable:
	\begin{itemize}
		\item	total sales (to maximize)
		\item	total number of unburned popcorn (to maximize)
		\item	height of plants (to maximize)
	\end{itemize}
	\vspace{1cm}
	If you are stuck, use ``profit''
	\begin{exampleblock}{}
		\centerline{\color{myGreen}Profit = (total income) $-$ (total expenses)}
	\end{exampleblock}
	
\end{frame}

\begin{frame}\frametitle{What do we do if we want to \emph{\textbf{minimize}}?}
	
	We still maximize, but just turn our response surface upside-down
	
	\vspace{.5cm}
	\begin{exampleblock}{}
		\centerline{\color{myGreen}maximization = $-$ (minimization)}
	\end{exampleblock}
	\centerline{\includegraphics[width=.8\textwidth]{../5C/Supporting material/waste-water-screenshot-example.png}}
	
\end{frame}

\begin{frame}\frametitle{Some questions to leave you with for the next video }
	\begin{columns}[c]
		\column{0.5\textwidth}
			\centerline{\includegraphics[width=1.102\textwidth]{../5C/Supporting material/COST-contours-shopping-11.png}}

		\column{0.5\textwidth}
			\begin{itemize}
				\item	Which direction should we climb up that mountain? 
				\item	How do we get to the top quickly and efficiently? \pause
				\item	What size of steps should we take? \pause
				\item	What if that surface is nonlinear? \pause
				\item	When do we stop? ``How do we confirm we are at the top?''
			\end{itemize}
	\end{columns}
	 
	\pause
	\vspace{1cm}
	{\color{myOrange}\emph{Hint}: some of the answers were given in the prior video (popcorn optimization)}
\end{frame}


% /Users/kevindunn/Dropbox/Coursera/Media/All-course-slides/classes/CourseraMOOC-class-5E.tex

\begin{frame}\frametitle{Case study: manufacturing of a mass produced product}
	\begin{columns}[b]
		\column{0.6\textwidth}
				\centerline{\includegraphics[width=\textwidth]{../5E/Supporting materials/flickr-jurvetson-6858583426_39dbf3910f_o-production-line-tesla.jpg}}
				\see{\href{https://secure.flickr.com/photos/jurvetson/6858583426/}{Flickr: jurvetson}}
		\column{0.40\textwidth}
			Tesla Motors assembly line
	\end{columns}
\end{frame}

\begin{frame}\frametitle{Case study: manufacturing of a mass produced product}
	\begin{columns}[b]
		\column{0.6\textwidth}
 				\centerline{\includegraphics[width=\textwidth]{../5E/Supporting materials/flickr-archangel-6966237299_c8995710b1_o-petroleum.jpg}}

 				\see{\href{https://secure.flickr.com/photos/archangel12/6966237299/}{Flickr: archangel12}}
		\column{0.40\textwidth}

	\end{columns}
\end{frame}
	
\begin{frame}\frametitle{Case study: manufacturing of a mass produced product}
	\begin{columns}[b]
		\column{0.6\textwidth}
				\centerline{\includegraphics[width=\textwidth]{../5E/Supporting materials/flickr-bensutherland-5587949321_971f94e306_o-production.jpg}}
				\see{\href{https://secure.flickr.com/photos/bensutherland/5587949321/}{Flickr: bensutherland}}
		\column{0.40\textwidth}

	\end{columns}
\end{frame}

\begin{frame}\frametitle{Case study: manufacturing of a mass produced product}
	\begin{columns}[c]
		\column{0.4\textwidth}
				\centerline{\includegraphics[height=.7\textheight]{../5E/Supporting materials/flickr-bitchbuzz-6510472979_bf2db00108_o-beer-bottles.jpg}}
				\see{Flickr: 6510472979}
		\column{0.60\textwidth}
			Two factors are available to vary:
			\begin{itemize}
				

				\item	\textbf{T}hroughput: number of parts per hour
				\item	\textbf{P}rice: selling price per part produced
			\end{itemize}
			
			\vspace{1cm}
			\pause
			The outcome variable $y$ = profit [\$ per hour]
			
			\begin{itemize}
				\item	profit = (all income) $-$ (all expenses) \pause
				\item	both factors affect the profit \pause
				\item	profit is easily calculated \pause
			\end{itemize}
	\end{columns}
\end{frame}

\begin{frame}\frametitle{}
	\centerline{\includegraphics[height=\textheight]{../5E/Supporting materials/RSM-01A.png}}
\end{frame}
\begin{frame}\frametitle{}
	\centerline{\includegraphics[height=\textheight]{../5E/Supporting materials/RSM-01B.png}}
\end{frame}
\begin{frame}\frametitle{}
	\centerline{\includegraphics[height=\textheight]{../5E/Supporting materials/RSM-02.png}}
\end{frame}
\begin{frame}\frametitle{}
	\centerline{\includegraphics[height=\textheight]{../5E/Supporting materials/RSM-03.png}}
\end{frame}
\begin{frame}\frametitle{}
	\centerline{\includegraphics[height=\textheight]{../5E/Supporting materials/RSM-04.png}}
\end{frame}
\begin{frame}\frametitle{}
	\centerline{\includegraphics[height=\textheight]{../5E/Supporting materials/RSM-04-A.png}}
\end{frame}
\begin{frame}\frametitle{}
	\centerline{\includegraphics[height=\textheight]{../5E/Supporting materials/RSM-04-B.png}}
\end{frame}
\begin{frame}\frametitle{}
	\centerline{\includegraphics[height=\textheight]{../5E/Supporting materials/RSM-05.png}}
\end{frame}
\begin{frame}\frametitle{}
	\centerline{\includegraphics[height=\textheight]{../5E/Supporting materials/RSM-06.png}}
\end{frame}
\begin{frame}\frametitle{}
	\centerline{\includegraphics[height=\textheight]{../5E/Supporting materials/RSM-07.png}}
\end{frame}
\begin{frame}\frametitle{}
	\centerline{\includegraphics[height=\textheight]{../5E/Supporting materials/RSM-08.png}}
\end{frame}

\begin{frame}\frametitle{List of all experiments}
	\begin{tabulary}{\linewidth}{c|cc|cc|c|c|cc}
		\textbf{\relax Experiment} & \textbf{\relax P } & \textbf{\relax T} & \textbf{\relax $x_\text{P}$} & \textbf{\relax $x_\text{T}$} & \textbf{\relax Prediction = $\hat{y}$} & \textbf{\relax Actual = $y$} & \textbf{\relax Model } \\ \hline
		Current point & \$0.75 & 325 & 0 & 0 & \onslide+<3->{\$ 390}& \$ 407 & ~1   
		\onslide+<2->{ \\
		
			1 & ~~0.50 & 320 & $-1$ & $-1$ & \onslide+<3->{~~197} & ~~193  & 1\\
			2 & ~~1.00 & 320 & $+1$ & $-1$ & \onslide+<3->{~~472} & ~~468  & 1\\
			3 & ~~0.50 & 330 & $-1$ & $+1$ & \onslide+<3->{~~314} & ~~310  & 1\\
			4 & ~~1.00 & 330 & $+1$ & $+1$ & \onslide+<3->{~~575} & ~~571  &~1}
		\onslide+<4->{ \\
			5 & ~~1.36 & 330 & 2.44 & 1.0  & 764 & \onslide+<4->{\$ 669} &~~1
		}
		\onslide+<5->{ \\\hline
			5 & 1.36 & 330 & 2.44 & 1.0  & 764 & \onslide+<4->{\$ 669} &~1
		}
	\end{tabulary}
\end{frame}
	
\begin{frame}\frametitle{\includegraphics[width=0.3\textwidth]{\imagedir/doe/examples/advice-logo.png}\,\, for selecting factorial ranges on a response surface}
	\begin{columns}[t]
		\column{0.6\textwidth}
			\begin{enumerate}
				\item	You want to notice a difference between the low and high levels
					\begin{itemize}
						\item	Too close: and you just pick up noise
						\item	Too far: and you are misled by nonlinearities
					\end{itemize}
				\onslide+<2->{
					\item	Don't go to the extremes (a very common mistake)
				}
				\onslide+<3->{
					\item	No idea? Start with $\approx 25\%$ of the extreme range
				}
		
			\end{enumerate}
		
		\column{0.50\textwidth}
			
	\end{columns}
	
	
	
\end{frame}

\begin{frame}\frametitle{The calculations from real-world units to coded units}
	
	\[ \color{myOrange}\text{coded value} = \dfrac{(\text{real value}) - (\text{center point})}{\tfrac{1}{2}\left(\text{range}\right)} \]
	
	\vspace{-16pt}
	
	\begin{columns}[t]
		\column{0.5\textwidth}
		
		\onslide+<3->{
			\centerline{\textbf{Price}}
			
			\vspace{-6pt}
			\begin{flalign*} 
				\text{center}_\text{P} &= \$0.75\\
				\text{range}_\text{P} &= \$0.50 \\
				x_\text{P} &= \dfrac{\text{P} - \text{center}_\text{P}} {\tfrac{1}{2} \text{range}_\text{P}} &\\
			\end{flalign*}
				
			\vspace{.11cm}
			\emph{Example}: coded value for $\text{P} = \$1.00$?
			\[x_\text{P} = \dfrac{1.00 - 0.75} {\tfrac{1}{2} (0.50)} =  \dfrac{0.25} {0.25} = +1\]
		}
		
		\column{0.50\textwidth}
		
		\onslide+<2->{
		
			\centerline{\textbf{Throughput}}
			
			\vspace{-6pt}
			\begin{flalign*} 
				\text{center}_\text{T} &= 325\\
				\text{range}_\text{T} &= 10 \\
				x_\text{T} &= \dfrac{\text{T} - \text{center}_\text{T}} {\tfrac{1}{2} \text{range}_\text{T}} &\\
			\end{flalign*}
			
			\vspace{.11cm}
			\emph{Example}: coded value for $\text{T} = 320$?
			\[x_\text{T} = \dfrac{320 - 325} {\tfrac{1}{2} (10)} =  \dfrac{-5} {5} = -1\]
		}
	\end{columns}
	
\end{frame}

\begin{frame}\frametitle{Using the prediction model}
	\[y = 390 + 134 x_\text{P} + 55 x_\text{T} - 3.5x_\text{P}x_\text{T} \]
	
	\vspace{1cm}
	Check the model's ``goodness of fit'' at the center point:
	\begin{itemize}
		\item	At the center: $x_\text{P}=0$ and $x_\text{T}=0$ \pause
		\item	Predicted $ \hat{y} = 390 + 0 + 0 + 0 = \$390$
		\item	Actual $y = \$407$ \pause
		\item	That's a difference of $\$17$
	\end{itemize}
	
	 \pause
	\vspace{1cm}
	Recall the concept of noise from the prior videos?\\
	Perform replicate experiments to estimate noise.
\end{frame}

\begin{frame}\frametitle{The steepest {\color{myOrange}path of ascent} using the local model of the system}

	\begin{exampleblock}{}
		\begin{align*} 
			y &=& b_0 &&+&& b_\text{P} x_\text{P} &&+&& b_\text{T} x_\text{T} &&+&& b_\text{PT}\,\,x_\text{P}x_\text{T} \\
			y &=& 390 &&+&& 134 x_\text{P}        &&+&& 55 x_\text{T}         &&+&& (-3.5)\,x_\text{P}x_\text{T} 
		\end{align*}		
	\end{exampleblock}
	
	\begin{columns}[c]
		\column{0.6\textwidth}
			\begin{itemize}
				\item	$b_\text{P} =134$ interpretation:
				\begin{itemize}
					\item	each $\Delta x_\text{P} = 1$ increase in $x_\text{P}$ (coded value) improves $y$ by $\$134$
				\end{itemize}
			\end{itemize}
			\pause
			\begin{itemize}
				\item	$b_\text{T} = 55$ interpretation:
				\begin{itemize}
					\item	each $\Delta x_\text{T} = 1$ increase in $x_\text{T}$ (coded value)  improves $y$ by $\$55$
				\end{itemize}
			\end{itemize}
		\column{0.450\textwidth}
			\centerline{\includegraphics[width=\textwidth]{../5E/Supporting materials/first-factorial-analysis-delta-P.png}}
	\end{columns}
\end{frame}

\begin{frame}\frametitle{The steepest {\color{myOrange}path of ascent} using the local model of the system}

	\begin{exampleblock}{}
		\begin{align*} 
			y &=& b_0 &&+&& b_\text{P} x_\text{P} &&+&& b_\text{T} x_\text{T} &&+&& b_\text{PT}\,\,x_\text{P}x_\text{T} \\
			y &=& 390 &&+&& 134 x_\text{P}        &&+&& 55 x_\text{T}         &&+&& (-3.5)\,x_\text{P}x_\text{T} 
		\end{align*}		
	\end{exampleblock}
	
	\begin{columns}[c]
		\column{0.6\textwidth}
			\begin{itemize}
				\item	$b_\text{P} =134$ interpretation:
				\begin{itemize}
					\item	each $\Delta x_\text{P} = 1$ increase in $x_\text{P}$ (coded value) improves $y$ by $\$134$
				\end{itemize}
			\end{itemize}
			\pause
			\begin{itemize}
				\item	$b_\text{T} = 55$ interpretation:
				\begin{itemize}
					\item	each $\Delta x_\text{T} = 1$ increase in $x_\text{T}$ (coded value)  improves $y$ by $\$55$
				\end{itemize}
			\end{itemize}
		\column{0.450\textwidth}
			\centerline{\includegraphics[width=\textwidth]{../5E/Supporting materials/first-factorial-analysis-delta-T.png}}
	\end{columns}
\end{frame}

\begin{frame}\frametitle{A convenient link between coded unit \textbf{\emph{changes}} and real-world \textbf{\emph{changes}}}

	\begin{exampleblock}{}

		\begin{align*} 
			\color{myOrange}\text{coded value} &= \dfrac{(\text{real value}) - (\text{center point})}{\tfrac{1}{2}\text{range}} \\
			\color{myOrange}\Delta (\text{coded value}) &= \color{myOrange}	\dfrac{\Delta (\text{real-world value})}{\tfrac{1}{2}\text{range}}
		\end{align*}		
	\end{exampleblock}
	\pause
	\begin{columns}[t]
		\column{0.5\textwidth}
			\centerline{\textbf{Example for throughput}}
			\begin{flalign*} 
				\Delta x_\text{T} = \dfrac{\Delta \text{T}}{\tfrac{1}{2}\text{range}_\text{T}}
			\end{flalign*}
			
			\vspace{.11cm}
			What does the coded value of $\Delta x_\text{T} =1$ represent in the real-world?
		
		\column{0.50\textwidth}
			\begin{flalign*} 
				\Delta x_\text{T} &= \dfrac{\Delta \text{T}}{\tfrac{1}{2}\text{range}_\text{T}} \\
				+1 &= \dfrac{\Delta \text{T}}{\tfrac{1}{2}\left(10\right)} \\
				\Delta \text{T} &= 5\,\text{parts per hour} \\
			\end{flalign*}
	\end{columns}
	\centerline{So $\Delta x_\text{T} = +1$ is equivalent to $\Delta \text{T} = +5$} 
\end{frame}

\begin{frame}\frametitle{A convenient link between coded unit \textbf{\emph{changes}} and real-world \textbf{\emph{changes}}}

	\begin{exampleblock}{}

		\begin{align*} 
			\color{myOrange}\text{coded value} &= \dfrac{(\text{real value}) - (\text{center point})}{\tfrac{1}{2}\text{range}} \\
			\color{myOrange}\Delta (\text{coded value}) &= \color{myOrange}	\dfrac{\Delta (\text{real-world value})}{\tfrac{1}{2}\text{range}}
		\end{align*}		
	\end{exampleblock}
	
	\begin{columns}[t]
		\column{0.5\textwidth}
			\centerline{\textbf{Example for price}}
			\begin{flalign*} 
				\Delta x_\text{P} = \dfrac{\Delta \text{P}}{\tfrac{1}{2}\text{range}_\text{P}}
			\end{flalign*}
			
			\vspace{.11cm}
			What does the coded value of $\Delta x_\text{P} =1$ represent in the real-world?
		
		\column{0.50\textwidth}
			\begin{flalign*} 
				\Delta x_\text{P} &= \dfrac{\Delta \text{P}}{\tfrac{1}{2}\text{range}_\text{P}} \\
				+1 &= \dfrac{\Delta \text{P}}{\tfrac{1}{2}\left(0.50\right)} \\
				\Delta \text{P} &= \$0.25  \\
			\end{flalign*}
	\end{columns}
	\centerline{So $\Delta x_\text{P} = +1$ is equivalent to $\Delta \text{P} = +\$0.25 $} 
\end{frame}

\begin{frame}\frametitle{The steepest {\color{myOrange}path of ascent} using the local model of the system}

	\begin{exampleblock}{}
		\begin{align*} 
			y &=& b_0 &&+&& b_\text{P} x_\text{P} &&+&& b_\text{T} x_\text{T} &&+&& b_\text{PT}\,\,x_\text{P}x_\text{T} \\
			y &=& 390 &&+&& 134 x_\text{P}        &&+&& 55 x_\text{T}         &&+&& (-3.5)\,x_\text{P}x_\text{T} 
		\end{align*}		
	\end{exampleblock}
	
	\begin{columns}[c]
		\column{0.6\textwidth}
			\centerline{\includegraphics[width=.9\textwidth]{../5E/Supporting materials/first-factorial-analysis-climbing-up.png}}
		
			
		\column{0.50\textwidth}
			\vspace{.1cm}\onslide+<2->{\color{myGreen}Take a step of $b_\text{T}=55$ in throughput\\}
			
			\onslide+<3->{
				\color{myGreen}for every $b_\text{P}=134$ steps in price
			}
			
			\vspace{.3cm}
			\onslide+<4->{
				\color{myOrange}But, our actual step is $\Delta x_\text{T}$, so ratio it:
			}
			
			\vspace{.3cm}
			\onslide+<5->{
				\color{myGreen}Take a step of $\dfrac{b_\text{T}}{\Delta x_\text{T}}$ in throughput\\  
			}
			\onslide+<6->{
				\color{myGreen}for every $\dfrac{b_\text{P}}{\Delta x_\text{P}}$ steps in price
			}
			\vspace{0.3cm}
		
	   	 
			\onslide+<7->{
				\fbox{\parbox[b][3em][t]{0.85\textwidth}{
					\color{blue}
					\vspace{0.1cm}
					$\dfrac{b_\text{T}}{\Delta x_\text{T}} = \dfrac{b_\text{P}}{\Delta x_\text{P}} \quad \Longrightarrow \quad
					\dfrac{\Delta x_\text{P}}{\Delta x_\text{T}} = \dfrac{b_\text{P}}{b_\text{T}}
					$
					\vspace{0.1cm}
				} }
			}
		   	 
	\end{columns}
\end{frame}

\begin{frame}\frametitle{Systematic approach to take a step towards the optimum}
	\begin{columns}[T]
		\column{0.2\textwidth}
		
			\vspace{0.1cm}
			{\tiny 
				\begin{enumerate}
					\item	Pick change in coded units in one factor.
				\end{enumerate}
			 \par}
			\onslide+<2->{
				{\tiny 
					\begin{enumerate}\setcounter{enumi}{1}
						\item	Find the ratios for the other factor(s).
					\end{enumerate}
				
				\par}
			}
			
			\vspace{0.9cm}
			\onslide+<3->{
				{\tiny 
					\begin{enumerate}\setcounter{enumi}{2}
						\item	Calculate step size in coded units.
					\end{enumerate}
				
				\par}
			}
			
			\vspace{0.4cm}
			\onslide+<4->{
				{\tiny 
					\begin{enumerate}\setcounter{enumi}{3}
						\item	Convert these to real-world \emph{changes}.
					\end{enumerate}
				
				\par}
			}
			
			\vspace{1cm}
			\onslide+<5->{
				{\tiny 
					\begin{enumerate}\setcounter{enumi}{4}
						\item	Finally, take a step from the baseline! Get the real-world location
						of the next experiment.
					\end{enumerate}
				
				\par}
			}
				
		\column{0.01\textwidth}
			\rule[3mm]{0.01cm}{85mm}%
			
			
		\column{0.4\textwidth}
			\centerline{\textbf{Price}}
			
			\onslide+<2->{
				\vspace{0.cm}
				\begin{align*}
					\Delta x_\text{P} &= \dfrac{b_\text{P}}{b_\text{T}}\cdot \Delta x_\text{T}\\ 
					\Delta x_\text{P} &= \dfrac{134}{55}\cdot 1\\
					\onslide+<3->{\Delta x_\text{P} &= 2.44}
				\end{align*}
			}
			%$ \text{(real-world change)}= (\text{coded change})\cdot \tfrac{1}{2}\text{range}$
			
			\vspace{-0.45cm}
			\onslide+<4->{
				\vspace{-0.6cm}
				\begin{align*} 
					\Delta \text{P} &= \Delta x_\text{P} \cdot   \tfrac{1}{2}(0.50) \\
					\Delta \text{P} &= \$0.61
				\end{align*}
			}
			
			\vspace{-0.9cm}
			\onslide+<5->{
				\begin{align*} 
					\text{P}^{(5)} &= \text{P}^{(0)} + \Delta \text{P} \\
					\text{P}^{(5)} &= \$0.75 + 0.61 \\
					\text{P}^{(5)} &= \$1.36
				\end{align*}
			}
			
			\color{myOrange} 
			
		
		\column{0.01\textwidth}
			\rule[3mm]{0.01cm}{85mm}%
			
		\column{0.4\textwidth}
			\centerline{\textbf{Throughput}}
			
			$\Delta x_\text{T} = 1$ (this was chosen)
			
					
			\vspace{2.15cm}
			\onslide+<3->{
				$\Delta x_\text{T} = 1$
			}
			
			\vspace{-0.25cm}
			\onslide+<4->{
				\begin{align*} 
					\Delta \text{T} &= \Delta x_\text{T} \cdot   \tfrac{1}{2}(10) \\
					\Delta \text{T} &= 5~\text{parts per hour}
				\end{align*}
			}
			
			\vspace{-0.8cm}
			\onslide+<5->{
				\begin{align*} 
					\text{T}^{(5)} &= \text{T}^{(0)} + \Delta \text{T} \\
					\text{T}^{(5)} &= 325 + 5 \\
					\text{T}^{(5)} &= 330 ~\text{parts per hour}
				\end{align*}
			}
	\end{columns}
\end{frame}

\begin{frame}\frametitle{Systematic approach to take a step towards the optimum}
	\begin{columns}[T]
		\column{0.2\textwidth}

			\vspace{1cm}
			\onslide+<1->{
				{\tiny 
					\begin{enumerate}\setcounter{enumi}{4}
						\item	Get the real-world location
						of the next experiment.
					\end{enumerate}
				
				\par}
			}
			
			\onslide+<2->{
				{\tiny 
					\begin{enumerate}\setcounter{enumi}{5}
						\item	Convert these back to coded-units.
					\end{enumerate}
				
				\par}
			}
			
				
		\column{0.01\textwidth}
			\rule[3mm]{0.01cm}{25mm}%
			
			
		\column{0.4\textwidth}
			\centerline{\textbf{Price}}
			
	
			\onslide+<1->{
				\begin{align*} 
					\text{P}^{(5)} &= \$1.36
				\end{align*}
			}
			
			\vspace{-1.1cm}
			\onslide+<2->{
				\begin{align*} 
					x_\text{P}^{(5)} &= 2.44
				\end{align*}
			}
			
			
			
		
		\column{0.01\textwidth}
			\rule[3mm]{0.01cm}{30mm}%
			
		\column{0.4\textwidth}
			\centerline{\textbf{Throughput}}
			
			\onslide+<1->{
				\begin{align*} 
					\text{T}^{(5)} &= 330 ~\text{parts per hour}
				\end{align*}
			}
			
			\vspace{-1.1cm}
			\onslide+<2->{	
				\begin{align*} 
					x_\text{T}^{(5)} &= 1.0
				\end{align*}
			}
	\end{columns}

	\vspace{-1.1cm}
	\begin{columns}[T]
		\column{0.2\textwidth}

			\vspace{1cm}
			\onslide+<3->{
				{\tiny 
					\begin{enumerate}\setcounter{enumi}{6}
						\item	Predict the next experiment's outcome.
					\end{enumerate}
				
				\par}
			}
			
			\vspace{2cm}
			\onslide+<4->{
				{\tiny 
					\begin{enumerate}\setcounter{enumi}{7}
						\item	Now run the next experiment, and record the values
					\end{enumerate}
				
				\par}
			}
			
		\column{0.01\textwidth}
			\rule[3mm]{0.01cm}{85mm}%
			
		\column{0.851\textwidth}
			
			\vspace{1cm}
			
			
			
			\onslide+<3->{	
				\hrule
				
				\begin{align*}
					\hat{y} &= 390 + 134 x_\text{P} + 55 x_\text{T} - 3.5x_\text{P}x_\text{T} \\
					\hat{y}^{(5)} &= 390 + 134 (2.44) + 55 (1.0) - 3.5(2.44)(1.0)\\
					\hat{y}^{(5)} &= 390 + 327 + 55 - 8.50\\
					\hat{y}^{(5)} &= 764~\text{profit per hour}
				\end{align*}
			}
			
			\vspace{-1.15cm}
			\onslide+<4->{	
				\begin{align*}
					y^{(5)} &= \$669~\text{profit per hour}
				\end{align*}
			}
	\end{columns}
\end{frame}

\begin{frame}\frametitle{A convenient link between coded unit \textbf{\emph{changes}} and real-world \textbf{\emph{changes}}}

	\begin{exampleblock}{}

		\begin{align*} 
			\color{myOrange}\text{coded value} &= \dfrac{(\text{real value}) - (\text{center point})}{\tfrac{1}{2}\text{range}} \\
			\color{myOrange}\Delta (\text{coded change}) &= \color{myOrange}	\dfrac{\Delta (\text{real-world change})}{\tfrac{1}{2}\text{range}}
		\end{align*}		
	\end{exampleblock}
	\pause
	\begin{columns}[t]
		\column{0.5\textwidth}
			\centerline{\textbf{Example for throughput}}
			\begin{flalign*} 
				\Delta x_\text{T} = \dfrac{\Delta \text{T}}{\tfrac{1}{2}\text{range}_\text{T}}
			\end{flalign*}
			
			\vspace{.11cm}
			What does the coded value of $\Delta x_\text{T} =1$ represent in the real-world?
		
		\column{0.50\textwidth}
			\begin{flalign*} 
				\Delta x_\text{T} &= \dfrac{\Delta \text{T}}{\tfrac{1}{2}\text{range}_\text{T}} \\
				+1 &= \dfrac{\Delta \text{T}}{\tfrac{1}{2}\left(10\right)} \\
				\Delta \text{T} &= 5\,\text{parts per hour} \\
			\end{flalign*}
	\end{columns}
	\centerline{So $\Delta x_\text{T} = +1$ is equivalent to $\Delta \text{T} = +5$} 
\end{frame}

\begin{frame}\frametitle{Judging the predictive ability from the model}
	An approximate way to judge the model's prediction ability:
	\begin{itemize}
		\item	$\hat{y} = \$764$
		\item	$y = \$669$
		\item	The prediction error is $\$95$.
		\item	Note that the two main effects are: $b_\text{P} =134$ and  $b_\text{T} = 55$ 
		\item	So this error is comparable to these; 
			\begin{itemize}
				\item	it's smaller than the effect of a $\Delta x_\text{P}=1$
				\item	it's larger than the effect of a $\Delta x_\text{T}=1$
				\item	so that error is  substantial
			
			\end{itemize}
	\end{itemize}
\end{frame}

\begin{frame}\frametitle{}
	\centerline{\includegraphics[height=\textheight]{../5E/Supporting materials/RSM-08.png}}
\end{frame}
\begin{frame}\frametitle{}
	\centerline{\includegraphics[height=\textheight]{../5E/Supporting materials/RSM-09.png}}
\end{frame}
\begin{frame}\frametitle{}
	\centerline{\includegraphics[height=\textheight]{../5E/Supporting materials/RSM-10.png}}
\end{frame}
\begin{frame}\frametitle{}
	\centerline{\includegraphics[height=\textheight]{../5E/Supporting materials/RSM-11.png}}
\end{frame}
\begin{frame}\frametitle{}
	\centerline{\includegraphics[height=\textheight]{../5E/Supporting materials/RSM-12.png}}
\end{frame}
\begin{frame}\frametitle{}
	\centerline{\includegraphics[height=\textheight]{../5E/Supporting materials/RSM-13.png}}
\end{frame}
\begin{frame}\frametitle{}
	\centerline{\includegraphics[height=\textheight]{../5E/Supporting materials/RSM-14.png}}
\end{frame}

\begin{frame}\frametitle{List of all experiments}
	\begin{tabulary}{\linewidth}{c|cc|cc|c|c|cc}
		\textbf{\relax Experiment} & \textbf{\relax P } & \textbf{\relax T} & \textbf{\relax $x_\text{P}$} & \textbf{\relax $x_\text{T}$} & \textbf{\relax Prediction = $\hat{y}$} & \textbf{\relax Actual = $y$} & \textbf{\relax Model } \\ \hline
	%	Current point & \$0.75 & 325 & 0 & 0 & \onslide+<1->{\$ 390}& \$ 407 & ~1   
	%	\onslide+<1->{ \\
	%	
	%		1 & ~~0.50 & 320 & $-1$ & $-1$ & \onslide+<1->{~~197} & ~~193  & 1\\
	%		2 & ~~1.00 & 320 & $+1$ & $-1$ & \onslide+<1->{~~472} & ~~468  & 1\\
	%		3 & ~~0.50 & 330 & $-1$ & $+1$ & \onslide+<1->{~~314} & ~~310  & 1\\
	%		4 & ~~1.00 & 330 & $+1$ & $+1$ & \onslide+<1->{~~575} & ~~571  &~1}
	%	\onslide+<1->{ \\
	%		5 & ~~1.36 & 330 & 2.44 & 1.0  & \onslide+<1->{~~764} & \onslide+<1->{\$ 669} &~~1
	%	}
		\onslide+<2->{~4 & ~~1.00 & 330 & $-1$ & $-1$ &  & \onslide+<3->{\$ 571 } &~2 \\
			~5 & ~~1.36 & 330 & $+1$ & $-1$ &  & \onslide+<3->{~~669 } &~2 \\
		}
		\onslide+<2->{6 & ~~1.00 & 338 & $-1$ & $+1$ &  & \onslide+<3->{~~620 } &~2 \\
			~7 & ~~1.36 & 338 & $+1$ & $+1$ &  & \onslide+<3->{~~710 } &~2 \\ 
			~8 & ~~1.18 & 334 & $0$  & $0$  &  & \onslide+<3->{~~657 } &~~2
		}
	\end{tabulary}
\end{frame}

\begin{frame}\frametitle{Try taking the next step up the mountain on your own}
	
	\begin{columns}[T]
		\column{0.6\textwidth}
		
			\begin{itemize}
				\item	Visualize the results first
				\item	Build a model using computer software
				\item	Sketch a contour plot by hand, or \\with software
			\end{itemize}
		\column{0.4\textwidth}
			\centerline{\includegraphics[width=1.5\textwidth, trim=100 70 400 200,clip,]{../5E/Supporting materials/RSM-15.png}}
	\end{columns}
	
	\pause
	\vspace{-2.5cm}
	{\color{myOrange}Now use the 8 step approach we showed earlier:}
	\vspace{0.2cm}
	\begin{enumerate}

			\item	Pick change in coded units in one factor. 
				\begin{itemize}
					\item	Use $\Delta x_\text{P}= 1.5$
				\end{itemize}
			\onslide+<2->{
				\item	Find the ratios for the other factor(s).
			}
			\onslide+<3->{
				\item	Calculate step size in coded units.
				\item	Convert these to real-world \emph{changes}.
			}
			\onslide+<4->{
				\item	Get the real-world location	of the next experiment.
				\item	Convert these back to coded-units.
				\item	Predict the next experiment's outcome.
			}
			\onslide+<5->{
				\item	Run the next experiment, and record the outcome value.  \href{http://yint.org/run-expt}{http://yint.org/run-expt}
			}
	\end{enumerate}
	
\end{frame}

\begin{frame}\frametitle{}
	\centerline{\includegraphics[height=\textheight]{../5E/Supporting materials/RSM-15.png}}
\end{frame}




% /Users/kevindunn/Dropbox/Coursera/Media/All-course-slides/classes/CourseraMOOC-class-5F.tex

\begin{frame}\frametitle{Outline of topics in this video}
	\begin{enumerate}
		\item	We keep climbing up the mountain ... \pause
		\item	but, what happens if we encounter a constraint? \pause
		\item	and, what happens when mistakes happen?\pause
	\end{enumerate}
	
	\begin{align*}
		\color{myOrange}\text{coded value} &= \dfrac{(\text{real value}) - (\text{center point})}{\tfrac{1}{2}\text{range}} \\
		x_\text{T} &= \dfrac{337 - 339}{\tfrac{1}{2}(6)} = -\dfrac{2}{3}
	\end{align*}
\end{frame}

\begin{frame}\frametitle{}
	\centerline{\includegraphics[height=\textheight]{../5E/Supporting materials/RSM-14.png}}
\end{frame}

\begin{frame}\frametitle{}
	\centerline{\includegraphics[height=\textheight]{../5E/Supporting materials/RSM-15.png}}
\end{frame}

\begin{frame}\frametitle{}
	\centerline{\includegraphics[height=\textheight]{../5E/Supporting materials/RSM-16.png}}
\end{frame}

\begin{frame}\frametitle{Taking the next step for experiment 9: {\color{myOrange}fill in the blanks}}
	\begin{columns}[T]
		\column{0.2\textwidth}
		
			\vspace{0.1cm}
			{\tiny 
				\begin{enumerate}
					\item	Pick change in coded units in one factor.
				\end{enumerate}
			 \par}
			 
			\onslide+<1->{
				{\tiny 
					\begin{enumerate}\setcounter{enumi}{1}
						\item	Find the ratios for the other factor(s).
					\end{enumerate}
				
				\par}
			}
			
			\vspace{0.0cm}
			\onslide+<1->{
				{\tiny 
					\begin{enumerate}\setcounter{enumi}{2}
						\item	Calculate step size in coded units.
					\end{enumerate}
				
				\par}
			}
			
			\onslide+<1->{
				{\tiny 
					\begin{enumerate}\setcounter{enumi}{3}
						\item	Convert these to real-world \emph{changes}.
					\end{enumerate}
				
				\par}
			}
			
			\onslide+<1->{
				{\tiny 
					\begin{enumerate}\setcounter{enumi}{4}
						\item	Get the real-world location
						of the next experiment.
					\end{enumerate}
				
				\par}
			}
			
			
			\vspace{-0.2cm}
			\onslide+<1->{
				{\tiny 
					\begin{enumerate}\setcounter{enumi}{5}
						\item	Convert these back to coded-units.
					\end{enumerate}
				
				\par}
			}
				
		\column{0.01\textwidth}
			\rule[3mm]{0.01cm}{58mm}%
			
			
		\column{0.4\textwidth}
			\centerline{\textbf{Price}}
			
			$\Delta x_\text{P} = 1.5$ (this was chosen)
			
			\vspace{1.3cm}
			$\Delta x_\text{P} = 1.5$ 
		
			\vspace{-0.45cm}
			\onslide+<1->{
				\begin{align*} 
					\Delta \text{P} &= \rule{2cm}{.5pt}~~~~~~~~~~~~~~~~~~~ \\
				\end{align*}
			}
			
			\vspace{-1.8cm}
			\onslide+<1->{
				\begin{align*} 
					\text{P}^{(9)} &= \rule{2cm}{.5pt} ~~~~~~~~~~~~~~~~~~~\\
				\end{align*}
			}
			
			\vspace{-1.9cm}
			\onslide+<1->{	
				\begin{align*} 
					x_\text{P}^{(9)} &= \rule{2cm}{.5pt}~~~~~~~~~~~~~~~~~~~
				\end{align*}
			}
			
		
		\column{0.01\textwidth}
			\rule[3mm]{0.01cm}{58mm}%
			
		\column{0.4\textwidth}
			\centerline{\textbf{Throughput}}
			
			\vspace{0.85cm}
			\onslide+<1->{
				\vspace{0.cm}
				\begin{align*}
					\Delta x_\text{T} &= \rule{2cm}{.5pt}
				\end{align*}
			}
			
			\vspace{-0.65cm}
			\onslide+<1->{
				\vspace{-0.6cm}
				\begin{align*} 
					\Delta \text{T} &= \rule{2cm}{.5pt}~~\text{parts per hour}
				\end{align*}
			}
			
			\vspace{-1.2cm}
			\onslide+<1->{
				\begin{align*} 
					\text{T}^{(9)} &= \rule{2cm}{.5pt} ~\text{parts per hour}
				\end{align*}
			}
			
			\vspace{-1.3cm}
			\onslide+<1->{	
				\begin{align*} 
					x_\text{T}^{(9)} &= \rule{2cm}{.5pt}~~~~~~~~~~~~~~~~~~~
				\end{align*}
			}
	\end{columns}
	
	\vspace{-0.3cm}
	\begin{columns}[T]
		\column{0.2\textwidth}

			\vspace{0cm}
			\onslide+<1->{
				{\tiny 
					\begin{enumerate}\setcounter{enumi}{6}
						\item	Predict the next experiment's outcome.
					\end{enumerate}
				
				\par}
			}
			
			\vspace{0cm}
			\onslide+<1->{
				{\tiny 
					\begin{enumerate}\setcounter{enumi}{7}
						\item	Now run the next experiment, and record the values
					\end{enumerate}
				
				\par}
			}
			
		\column{0.01\textwidth}
			\rule[3mm]{0.01cm}{85mm}%
			
		\column{0.853\textwidth}
			
			\onslide+<1->{	
				\hrule
				
				\begin{align*}
					~~~\hat{y}  = \rule{2cm}{.5pt} ~\text{profit per hour} 
				\end{align*}
			}
			
			\vspace{-1.15cm}
			\onslide+<1->{	
				\begin{align*}
					y^{(9)} &=  \rule{2cm}{.5pt} ~\text{profit per hour}
				\end{align*}
			}
	\end{columns}
	
\end{frame}

\begin{frame}\frametitle{Taking the next step for experiment 9}
	\begin{columns}[T]
		\column{0.2\textwidth}
		
			\vspace{0.1cm}
			{\tiny 
				\begin{enumerate}
					\item	Pick change in coded units in one factor.
				\end{enumerate}
			 \par}
			\onslide+<2->{
				{\tiny 
					\begin{enumerate}\setcounter{enumi}{1}
						\item	Find the ratios for the other factor(s).
					\end{enumerate}
				
				\par}
			}
			
			\vspace{0.9cm}
			\onslide+<3->{
				{\tiny 
					\begin{enumerate}\setcounter{enumi}{2}
						\item	Calculate step size in coded units.
					\end{enumerate}
				
				\par}
			}
			
			\vspace{0.4cm}
			\onslide+<4->{
				{\tiny 
					\begin{enumerate}\setcounter{enumi}{3}
						\item	Convert these to real-world \emph{changes}.
					\end{enumerate}
				
				\par}
			}
			
			\vspace{1cm}
			\onslide+<5->{
				{\tiny 
					\begin{enumerate}\setcounter{enumi}{4}
						\item	Finally, take a step from the baseline! Get the real-world location
						of the next experiment.
					\end{enumerate}
				
				\par}
			}
				
		\column{0.01\textwidth}
			\rule[3mm]{0.01cm}{85mm}%
			
			
		\column{0.4\textwidth}
			\centerline{\textbf{Price}}
			
			$\Delta x_\text{P} = 1.5$ (this was chosen)
			
			\vspace{2.15cm}
			\onslide+<3->{
				$\Delta x_\text{P} = 1.5$
			}
			
			\vspace{-0.25cm}
			\onslide+<4->{
				\begin{align*} 
					\Delta \text{P} &= \Delta x_\text{P} \cdot   \tfrac{1}{2}(0.36) \\
					\Delta \text{P} &= \$0.27
				\end{align*}
			}
			
			\vspace{-1cm}
			\uncover<4>{
				\only<beamer>{
					{\color{myOrange}
					\begin{align*} 
						\text{range}_\text{P} &= \$0.18\\
						\Delta \text{P} &= \Delta x_\text{P} \cdot   \tfrac{1}{2}(\text{range}_\text{P})
					\end{align*}}
				}
				\only<handout>{\vspace{2.8cm}}
			}
			
			\vspace{-2.8cm}
			\onslide+<5->{
				\begin{align*} 
					\text{P}^{(9)} &= \text{P}^{(8)} + \Delta \text{P} \\
					\text{P}^{(9)} &= \$1.18 + 0.27 \\
					\text{P}^{(9)} &= \$1.45
				\end{align*}
			}
		
		\column{0.01\textwidth}
			\rule[3mm]{0.01cm}{85mm}%
			
		\column{0.4\textwidth}
			\centerline{\textbf{Throughput}}
			
			\onslide+<2->{
				\vspace{0.cm}
				\begin{align*}
					\Delta x_\text{T} &= \dfrac{b_\text{T}}{b_\text{P}}\cdot \Delta x_\text{P}\\ 
					\Delta x_\text{T} &= \dfrac{22.5}{47}\cdot 1.5\\
					\onslide+<3->{\Delta x_\text{T} &= 0.718}
				\end{align*}
			}
			%$ \text{(real-world change)}= (\text{coded change})\cdot \tfrac{1}{2}\text{range}$
			
			\vspace{-0.45cm}
			\onslide+<4->{
				\vspace{-0.6cm}
				\begin{align*} 
					\Delta \text{T} &= \Delta x_\text{T} \cdot   \tfrac{1}{2}(8) \\
					\Delta \text{T} &= 2.87 \approx 3 ~\text{parts per hour}
				\end{align*}
			}
			
			\vspace{-1.1cm}
			\uncover<4>{
				\only<beamer>{
					{\color{myOrange}
					\begin{align*} 
						\text{range}_\text{T} &= 8\\
						\Delta \text{T} &= \Delta x_\text{T} \cdot   \tfrac{1}{2}(\text{range}_\text{T})
					\end{align*}}
				}
				\only<handout>{\vspace{2.8cm}}
			}
			
			\vspace{-2.8cm}
			
			\onslide+<5->{
				\begin{align*} 
					\text{T}^{(9)} &= \text{T}^{(8)} + \Delta \text{P} \\
					\text{T}^{(9)} &= 334 + 3 \\
					\text{T}^{(9)} &= 337 ~\text{parts per hour}
				\end{align*}
			}
			
			
			
	\end{columns}
\end{frame}

\begin{frame}\frametitle{Taking the next step for experiment 9}
	\begin{columns}[T]
		\column{0.2\textwidth}

			\vspace{1cm}
			\onslide+<1->{
				{\tiny 
					\begin{enumerate}\setcounter{enumi}{4}
						\item	Get the real-world location
						of the next experiment.
					\end{enumerate}
				
				\par}
			}
			
			\onslide+<2->{
				{\tiny 
					\begin{enumerate}\setcounter{enumi}{5}
						\item	Convert these back to coded-units.
					\end{enumerate}
				
				\par}
			}
			
				
		\column{0.01\textwidth}
			\rule[3mm]{0.01cm}{25mm}%
			
			
		\column{0.4\textwidth}
			\centerline{\textbf{Price}}
			
	
			\onslide+<1->{
				\begin{align*} 
					\text{P}^{(9)} &= \$1.45
				\end{align*}
			}
			
			\vspace{-1.1cm}
			\onslide+<2->{
				\begin{align*} 
					x_\text{P}^{(9)} &= 1.5
				\end{align*}
			}
		
		\column{0.01\textwidth}
			\rule[3mm]{0.01cm}{30mm}%
			
		\column{0.4\textwidth}
			\centerline{\textbf{Throughput}}
			
			\onslide+<1->{
				\begin{align*} 
					\text{T}^{(9)} &= 337 ~\text{parts per hour}
				\end{align*}
			}
			
			\vspace{-1.1cm}
			\onslide+<2->{	
				\begin{align*} 
					x_\text{T}^{(9)} &= \dfrac{337 - 334}{\tfrac{1}{2}(8)} = 0.75~(\text{not $0.718$})
				\end{align*}
			}
	\end{columns}

	\vspace{-1.1cm}
	\begin{columns}[T]
		\column{0.2\textwidth}

			\vspace{1cm}
			\onslide+<3->{
				{\tiny 
					\begin{enumerate}\setcounter{enumi}{6}
						\item	Predict the next experiment's outcome.
					\end{enumerate}
				
				\par}
			}
			
			\vspace{2cm}
			\onslide+<4->{
				{\tiny 
					\begin{enumerate}\setcounter{enumi}{7}
						\item	Now run the next experiment, and record the values
					\end{enumerate}
				
				\par}
			}
			
		\column{0.01\textwidth}
			\rule[3mm]{0.01cm}{85mm}%
			
		\column{0.8703\textwidth}
			
			\vspace{1cm}
			
			
			
			\onslide+<3->{	
				\hrule
				
				\begin{align*}
					\hat{y}       &&=&& 645 &&+&& 47 x_\text{P} &&+&& 22.5 x_\text{T} &&-&& 2x_\text{P}x_\text{T}&& \\
					\hat{y}^{(5)} &&=&& 645 &&+&& 47 (1.5)      &&+&& 22.5 (0.75)     &&-&& 2(1.5)(0.75)&&\\
					\hat{y}^{(5)} &&\approx&& 645 &&+&& 71            &&+&& 17              &&-&& 2&&\\
					\hat{y}^{(5)} &&\approx&& 731~\text{profit per hour} \span\omit\span\omit\span\omit\span\omit\span\omit
				\end{align*}
			}
			
			\vspace{-1.15cm}
			\onslide+<4->{	
				\begin{align*}
					y^{(5)} &= \$717~\text{profit per hour}
				\end{align*}
			}
	\end{columns}
\end{frame}

\begin{frame}\frametitle{Judging the prediction error}
	\begin{enumerate}
		\item	We should have replicated experiments to judge the noise level
		
		\centerline{
			\includegraphics[width=.7\textwidth, trim=50 120 0 250,clip,]{../5F/Supporting material/popcorn-experiments-11-noise.jpg}}
		{\small We will see more on replicated experiments in the next video}
		
		\vspace{1cm}
		\pause
		
		\item	Contrast prediction error with the model's coefficients:
		\begin{itemize}
			\item	$\hat{y}^{(9)}  = 645 + {\color{myOrange}47} x_\text{P}+ {\color{myOrange}22.5} x_\text{T} - 2x_\text{P}x_\text{T} = \$731$
			\item	$y_\text{actual}^{(9)} = \$717$ 
			\item	error = $\hat{y} - y \approx \$ 13 $ (ignoring the decimals)
		\end{itemize}
		
	\end{enumerate}
\end{frame}

\begin{frame}\frametitle{}
	\centerline{\includegraphics[height=\textheight]{../5E/Supporting materials/RSM-17.png}}
\end{frame}
\begin{frame}\frametitle{}
	\centerline{\includegraphics[height=\textheight]{../5E/Supporting materials/RSM-18.png}}
\end{frame}
\begin{frame}\frametitle{}
	\centerline{\includegraphics[height=\textheight]{../5E/Supporting materials/RSM-19.png}}
\end{frame}
\begin{frame}\frametitle{}
	\centerline{\includegraphics[height=\textheight]{../5E/Supporting materials/RSM-20.png}}
\end{frame}

\begin{frame}\frametitle{Taking the next step for experiment 10: {\color{myOrange}fill in the blanks}}
	\begin{columns}[T]
		\column{0.195\textwidth}
		
			\vspace{0.1cm}
			{\tiny 
				\begin{enumerate}
					\item	Pick change in coded units in one factor.
				\end{enumerate}
			 \par}
			 
			\onslide+<1->{
				{\tiny 
					\begin{enumerate}\setcounter{enumi}{1}
						\item	Find the ratios for \\the other factor(s).
					\end{enumerate}
				
				\par}
			}
			
			\vspace{0.0cm}
			\onslide+<1->{
				{\tiny 
					\begin{enumerate}\setcounter{enumi}{2}
						\item	Calculate step size in coded units.
					\end{enumerate}
				
				\par}
			}
			
			\onslide+<1->{
				{\tiny 
					\begin{enumerate}\setcounter{enumi}{3}
						\item	Convert these to real-world \emph{changes}.
					\end{enumerate}
				
				\par}
			}
			
			\onslide+<1->{
				{\tiny 
					\begin{enumerate}\setcounter{enumi}{4}
						\item	Get the real-world location
						of the next experiment.
					\end{enumerate}
				
				\par}
			}
			
			
			\vspace{-0.2cm}
			\onslide+<1->{
				{\tiny 
					\begin{enumerate}\setcounter{enumi}{5}
						\item	Convert these back\\ to coded-units.
					\end{enumerate}
				
				\par}
			}
				
		\column{0.01\textwidth}
			\rule[3mm]{0.01cm}{58mm}%
			
			
		\column{0.4\textwidth}
			\centerline{\textbf{Price}}
			
			$\Delta x_\text{P} = 2.5$ (this was chosen)
			
			\vspace{1.3cm}
			$\Delta x_\text{P} = 2.5$ 
		
			\vspace{-0.45cm}
			\onslide+<1->{
				\begin{align*} 
					\Delta \text{P} &= \rule{2cm}{.5pt} \\
				\end{align*}
			}
			
			\vspace{-1.8cm}
			\onslide+<1->{
				\begin{align*} 
					\text{P}^{(10)} &= \rule{2cm}{.5pt} \\
				\end{align*}
			}
			
			\vspace{-1.9cm}
			\onslide+<1->{	
				\begin{align*} 
					x_\text{P}^{(10)} &= \rule{2cm}{.5pt}
				\end{align*}
			}
			
		
		\column{0.01\textwidth}
			\rule[3mm]{0.01cm}{58mm}%
			
		\column{0.4\textwidth}
			\centerline{\textbf{Throughput}}
			
			\vspace{0.85cm}
			\onslide+<1->{
				\vspace{0.cm}
				\begin{align*}
					\Delta x_\text{T} &= \rule{2cm}{.5pt}
				\end{align*}
			}
			
			\vspace{-0.65cm}
			\onslide+<1->{
				\vspace{-0.6cm}
				\begin{align*} 
					\Delta \text{T} &= \rule{2cm}{.5pt}~~\text{parts per hour}
				\end{align*}
			}
			
			\vspace{-1.2cm}
			\onslide+<1->{
				\begin{align*} 
					\text{T}^{(10)} &= \rule{2cm}{.5pt} ~\text{parts per hour}
				\end{align*}
			}
			
			\vspace{-1.3cm}
			\onslide+<1->{	
				\begin{align*} 
					x_\text{T}^{(10)} &= \rule{2cm}{.5pt}
				\end{align*}
			}
	\end{columns}
	
	\vspace{-0.3cm}
	\begin{columns}[T]
		\column{0.2\textwidth}

			\vspace{0cm}
			\onslide+<1->{
				{\tiny 
					\begin{enumerate}\setcounter{enumi}{6}
						\item	Predict the next experiment's outcome.
					\end{enumerate}
				
				\par}
			}
			
			\vspace{0cm}
			\onslide+<1->{
				{\tiny 
					\begin{enumerate}\setcounter{enumi}{7}
						\item	Now run the next experiment, and record the values
					\end{enumerate}
				
				\par}
			}
			
		\column{0.01\textwidth}
			\rule[3mm]{0.01cm}{85mm}%
			
		\column{0.864\textwidth}
			
			\onslide+<1->{	
				\hrule
				
				\begin{align*}
					~~~\hat{y}^{(10)}  = \rule{2cm}{.5pt} ~\text{profit per hour} 
				\end{align*}
			}
			
			\vspace{-1.15cm}
			\onslide+<1->{	
				\begin{align*}
					y^{(10)} &=  \rule{2cm}{.5pt} ~\text{profit per hour}
				\end{align*}
			}
	\end{columns}
	
\end{frame}

\begin{frame}\frametitle{Taking the next step for experiment 10: {\color{myOrange}solution}}
	\begin{columns}[T]
		\column{0.2066\textwidth}
		
			\vspace{0.1cm}
			{\tiny 
				\begin{enumerate}
					\item	Pick change in coded units in one factor.
				\end{enumerate}
			 \par}
			 
			\onslide+<1->{
				{\tiny 
					\begin{enumerate}\setcounter{enumi}{1}
						\item	Find the ratios for\\ the other factor(s).
					\end{enumerate}
				
				\par}
			}
			
			\vspace{0.0cm}
			\onslide+<1->{
				{\tiny 
					\begin{enumerate}\setcounter{enumi}{2}
						\item	Calculate step size in coded units.
					\end{enumerate}
				
				\par}
			}
			
			\onslide+<1->{
				{\tiny 
					\begin{enumerate}\setcounter{enumi}{3}
						\item	Convert these to real-world \emph{changes}.
					\end{enumerate}
				
				\par}
			}
			
			\onslide+<1->{
				{\tiny 
					\begin{enumerate}\setcounter{enumi}{4}
						\item	Get the real-world location
						of the next experiment.
					\end{enumerate}
				
				\par}
			}
			
			
			\vspace{-0.2cm}
			\onslide+<1->{
				{\tiny 
					\begin{enumerate}\setcounter{enumi}{5}
						\item	Convert these back\\ to coded-units.
					\end{enumerate}
				
				\par}
			}
				
		\column{0.01\textwidth}
			\rule[3mm]{0.01cm}{58mm}%
			
			
		\column{0.4\textwidth}
			\centerline{\textbf{Price}}
			
			$\Delta x_\text{P} = 2.5$ (this was chosen)
			
			\vspace{1.3cm}
			$\Delta x_\text{P} = 2.5$ 
		
			\vspace{-0.45cm}
			\onslide+<4->{
				\begin{align*} 
					\Delta \text{P} &= {\color{blue} \$0.45} \\
				\end{align*}
			}
			
			\vspace{-1.8cm}
			\onslide+<6->{
				\begin{align*} 
					\text{P}^{(10)} &= {\color{blue} \$1.63}\\
				\end{align*}
			}
			
			\vspace{-1.9cm}
			\onslide+<8->{	
				\begin{align*} 
					x_\text{P}^{(10)} &= {\color{blue} 2.5} 
				\end{align*}
			}
			
		
		\column{0.01\textwidth}
			\rule[3mm]{0.01cm}{58mm}%
			
		\column{0.4\textwidth}
			\centerline{\textbf{Throughput}}
			
			\vspace{0.85cm}
			\onslide+<2->{
				\vspace{0.cm}
				\begin{align*}
					\Delta x_\text{T} &= {\color{blue} 1.2 = \dfrac{b_\text{T}}{b_\text{P}} \times \Delta x_\text{P} = \dfrac{22.5}{47} \times 2.5}
				\end{align*}
			}
			
			\vspace{-0.65cm}
			\onslide+<3->{
				\vspace{-0.6cm}
				\begin{align*} 
					\Delta \text{T} &= {\color{blue} 4.8 \approx 5~~\text{parts per hour}}
				\end{align*}
			}
			
			\vspace{-1.2cm}
			\onslide+<5->{
				\begin{align*} 
					\text{T}^{(10)} &= {\color{blue} 334 + 5 = 339 ~\text{parts per hour}}
				\end{align*}
			}
			
			\vspace{-1.5cm}
			\onslide+<7->{	
				\begin{align*} 
					x_\text{T}^{(10)} &={\color{blue}\dfrac{339-334}{\tfrac{1}{2}\cdot (8)} = 1.25}} 
				\end{align*}
	\end{columns}
	
	\vspace{-0.3cm}
	\begin{columns}[T]
		\column{0.2\textwidth}

			\vspace{0cm}
			\onslide+<1->{
				{\tiny 
					\begin{enumerate}\setcounter{enumi}{6}
						\item	Predict the next experiment's outcome.
					\end{enumerate}
				
				\par}
			}
			
			\vspace{0cm}
			\onslide+<1->{
				{\tiny 
					\begin{enumerate}\setcounter{enumi}{7}
						\item	Now run the next experiment, and record the values
					\end{enumerate}
				
				\par}
			}
			
		\column{0.01\textwidth}
			\rule[3mm]{0.01cm}{85mm}%
			
		\column{0.864\textwidth}
			
			\onslide+<9->{	
				\hrule
				\vspace{-0.5cm}
				\begin{align*}
					\hat{y}       &&=&& 645 &&+&& 47 x_\text{P} &&+&& 22.5 x_\text{T} &&-&& 2x_\text{P}x_\text{T}&& \\
					\hat{y}^{(10)} && \approx&& {\color{blue}\$ 785~\text{profit per hour}} \span\omit\span\omit\span\omit\span\omit\span\omit
				\end{align*}
			}
			
			\vspace{-1.3cm}
			\onslide+<10->{	
				\begin{align*}
					y^{(10)} &=  \color{blue} \$ 732 ~\text{profit per hour}
				\end{align*}
			}
	\end{columns}
	
\end{frame}

\begin{frame}\frametitle{}
	\centerline{\includegraphics[height=\textheight]{../5E/Supporting materials/RSM-20.png}}
\end{frame}
\begin{frame}\frametitle{}
	\centerline{\includegraphics[height=\textheight]{../5E/Supporting materials/RSM-21.png}}
\end{frame}
\begin{frame}\frametitle{}
	\centerline{\includegraphics[height=\textheight]{../5E/Supporting materials/RSM-22.png}}
\end{frame}
\begin{frame}\frametitle{}
	\centerline{\includegraphics[height=\textheight]{../5E/Supporting materials/RSM-23.png}}
\end{frame}
\begin{frame}\frametitle{}
	\centerline{\includegraphics[height=\textheight]{../5E/Supporting materials/RSM-24.png}}
\end{frame}
\begin{frame}\frametitle{}
	\centerline{\includegraphics[height=\textheight]{../5E/Supporting materials/RSM-25.png}}
\end{frame}

\begin{frame}\frametitle{Step size is a function of the factorial range}
	Let's work backwards:\vspace{1cm}
	\begin{itemize}
		\item	You choose the step size, e.g. $\Delta x_\text{P} = 2.5$
		\vspace{0.5cm}
		\item	Real-world step size $= \Delta \text{P} = \Delta x_\text{P} \cdot \tfrac{1}{2}\left(\text{range}_\text{P} \right) $
		\vspace{0.5cm}
		\item	where the ``$\text{range}_\text{P}$'' is picked by the experimenter 
	\end{itemize}
\end{frame}

\begin{frame}\frametitle{}
	\centerline{\includegraphics[height=\textheight]{../5E/Supporting materials/RSM-26.png}}
\end{frame}

\begin{frame}\frametitle{List of all experiments}
	\begin{tabulary}{\linewidth}{c|cc|cc|c|c|cc}
		\textbf{\relax Experiment} & \textbf{\relax P } & \textbf{\relax T} & \textbf{\relax $x_\text{P}$} & \textbf{\relax $x_\text{T}$} & \textbf{\relax Prediction = $\hat{y}$} & \textbf{\relax Actual = $y$} & \textbf{\relax Model } \\ \hline
	%	Current point & \$0.75 & 325 & 0 & 0 & \onslide+<1->{\$ 390}& \$ 407 & ~1   
	%	\onslide+<1->{ \\
	%	
	%		1 & ~~0.50 & 320 & $-1$ & $-1$ & \onslide+<1->{~~197} & ~~193  & 1\\
	%		2 & ~~1.00 & 320 & $+1$ & $-1$ & \onslide+<1->{~~472} & ~~468  & 1\\
	%		3 & ~~0.50 & 330 & $-1$ & $+1$ & \onslide+<1->{~~314} & ~~310  & 1\\
	%		4 & ~~1.00 & 330 & $+1$ & $+1$ & \onslide+<1->{~~575} & ~~571  &~1}
	%	\onslide+<1->{ \\
	%		5 & ~~1.36 & 330 & 2.44 & 1.0  & \onslide+<1->{~~764} & \onslide+<1->{\$ 669} &~~1
	%	}
		9 & ~~1.45 & 337 & $1.5$  & $0.75$ &  & \$ 717 &~2 \\ \hline
		\onslide+<1->{10 & ~~1.63 & 339 & $~0$ & $~0$ &  & \$ 732 &~3 \\
		   11 & ~~1.45 & 336 & $-1$ & $-1$ &  & ~~715  &~3 \\
		   12 & ~~1.81 & 336 & $+1$ & $-1$ &  & ~~713  &~3 \\
		   13 & ~~1.45 & 342 & $-1$ & $+1$ &  & ~~733  &~3 \\ 
		   14 & ~~1.81 & 342 & $+1$ & $+1$ &  & ~~725 &~~3
		}
	\end{tabulary}
\end{frame}

\begin{frame}\frametitle{}
	\centerline{\includegraphics[height=\textheight]{\imagedir/doe/two-factors-with-mistake-01.png}}
\end{frame}
\begin{frame}\frametitle{}
	\centerline{\includegraphics[height=\textheight]{\imagedir/doe/two-factors-with-mistake-02.png}}
\end{frame}
\begin{frame}\frametitle{}
	\centerline{\includegraphics[height=\textheight]{\imagedir/doe/two-factors-with-mistake-03.png}}
\end{frame}
\begin{frame}\frametitle{}
	\centerline{\includegraphics[height=\textheight]{\imagedir/doe/two-factors-with-mistake-04.png}}
\end{frame}
\begin{frame}\frametitle{}
	\centerline{\includegraphics[height=\textheight]{\imagedir/doe/two-factors-with-mistake-06.png}}
\end{frame}
\begin{frame}\frametitle{}
	\centerline{\includegraphics[height=\textheight]{\imagedir/doe/two-factors-with-mistake-07.png}}
\end{frame}
\begin{frame}\frametitle{}
	\centerline{\includegraphics[height=\textheight]{\imagedir/doe/two-factors-with-mistake-08.png}}
\end{frame}
\begin{frame}\frametitle{}
	\centerline{\includegraphics[height=\textheight]{\imagedir/doe/two-factors-with-mistake-09.png}}
\end{frame}
\begin{frame}\frametitle{}
	\centerline{\includegraphics[height=\textheight]{\imagedir/doe/two-factors-with-mistake-10.png}}
\end{frame}
\begin{frame}\frametitle{}
	\centerline{\includegraphics[height=\textheight]{\imagedir/doe/two-factors-with-mistake-11.png}}
\end{frame}
\begin{frame}\frametitle{}
	\centerline{\includegraphics[height=\textheight]{\imagedir/doe/two-factors-with-mistake-12.png}}
\end{frame}
\begin{frame}\frametitle{}
	\centerline{\includegraphics[height=\textheight]{\imagedir/doe/two-factors-with-mistake-13.png}}
\end{frame}
\begin{frame}\frametitle{}
	\centerline{\includegraphics[height=\textheight]{\imagedir/doe/two-factors-with-mistake-14.png}}
\end{frame}
\begin{frame}\frametitle{}
	\centerline{\includegraphics[height=\textheight]{\imagedir/doe/two-factors-with-mistake-15.png}}
\end{frame}
\begin{frame}\frametitle{}
	\centerline{\includegraphics[height=\textheight]{\imagedir/doe/two-factors-with-mistake-16.png}}
\end{frame}

\begin{frame}\frametitle{} %Contrasting the models from a \textbf{botched} \emph{vs} \textbf{regular} design
	
	\centerline{\includegraphics[width=.3\textwidth]{../5F/Supporting material/botched-design.png}}
	
	\vspace{-3cm}
	\begin{columns}[T]
		\column{0.48\textwidth}
		
		\vspace{0.1cm}
		\textbf{Botched design}
		
		\vspace{0.25cm}
		
		
		\includegraphics[width=.5\textwidth]{../5F/Supporting material/botched-design-code.png}
		
		\vspace{0.3cm}
		{\small Data points used: 10, \textbf{9}, 12, 13, 14}
		
		\vspace{-.65cm}
		\[\hat{y} = {\color{purple}723.4}  {\color{black}-2.243 x_\text{P}} +  {\color{blue}7.542 x_\text{T}}  {\color{red}-1.542 x_\text{P}x_\text{T}} \]
		
		\vspace{-0.1cm}
		\centerline{\includegraphics[width=.7\textwidth]{../5F/Supporting material/botched-contour.pdf}}

		\column{0.01\textwidth}
			\vspace{3.5cm}
			\rule[3mm]{0.01cm}{58mm}%
			
		\column{0.48\textwidth}
		
			\vspace{-0.3cm}
			\begin{flushright}
			\textbf{Regular design}
			
			\vspace{0.25cm}
			
			
			\includegraphics[width=.5\textwidth]{../5F/Supporting material/regular-design-code.png}
			
			\vspace{0.3cm}
			{\small Data points used: 10, \textbf{11}, 12, 13, 14}
			\end{flushright}
			
			\vspace{-1cm}
			\[\hat{y} = {\color{purple}723.6}  {\color{black}-2.500 x_\text{P} }+  {\color{blue}7.500 x_\text{T} } {\color{red}-1.500 x_\text{P}x_\text{T}} \]
			
			\vspace{-0.1cm}
			\centerline{\includegraphics[width=.7\textwidth]{../5F/Supporting material/regular-contour.pdf}}
	\end{columns}

\end{frame}

\begin{frame}\frametitle{}
	\centerline{\includegraphics[width=\textwidth]{\imagedir/doe/two-factors-with-constraint-MOOC-01.png}}
\end{frame}
\begin{frame}\frametitle{}
	\centerline{\includegraphics[width=\textwidth]{\imagedir/doe/two-factors-with-constraint-MOOC-02.png}}
\end{frame}
\begin{frame}\frametitle{}
	\centerline{\includegraphics[width=\textwidth]{\imagedir/doe/two-factors-with-constraint-MOOC-03.png}}
\end{frame}
\begin{frame}\frametitle{}
	\centerline{\includegraphics[width=\textwidth]{\imagedir/doe/two-factors-with-constraint-MOOC-04.png}}
\end{frame}
\begin{frame}\frametitle{}
	\centerline{\includegraphics[width=\textwidth]{\imagedir/doe/two-factors-with-constraint-MOOC-05.png}}
\end{frame}
\begin{frame}\frametitle{}
	\centerline{\includegraphics[width=\textwidth]{\imagedir/doe/two-factors-with-constraint-MOOC-06.png}}
\end{frame}

% /Users/kevindunn/Dropbox/Coursera/Media/All-course-slides/classes/CourseraMOOC-class-5G.tex

\begin{frame}\frametitle{Before we get started: a quick look back at OFAT (COST)}
	
	OFAT is not recommended for a variety of reasons:
	\begin{columns}[T]
		\column{0.75\textwidth}
			\vspace{0.5cm}
			\begin{itemize}
				\item	With OFAT you are never really sure you are at the peak.
					\begin{itemize}
						\item	you keep iterating through all the factors
						\item	no conclusive indication that you converged
					\end{itemize}
				
		\onslide+<2->{
				\vspace{0.5cm}
				\item	OFAT is order-dependent (it is a lottery\emph{!})
		}
		\onslide+<3->{
				\vspace{0.5cm}
		
				\item	OFAT does not scale well 
					\begin{itemize}
						\item	For 3 or more factors: we often end up using more runs
						\item	For 2 factors: I find OFAT and RSM (response surface methods) use about the same number of
								runs
					\end{itemize}
		}
		\onslide+<4->{
				\vspace{0.5cm}
		
				\item	We don't learn about interactions; only about  main effects 
		}
			\end{itemize}
		\column{0.3\textwidth}
		
			\centerline{\includegraphics[width=\textwidth]{../5G/Supporting materials/COST-contours-shopping-extend-20.png}}
			\onslide+<2->{
				\centerline{\includegraphics[width=\textwidth]{../5G/Supporting materials/COST-contours-shopping-reorder-09.png}}
			}

	\end{columns}
	
	

\end{frame}
\begin{frame}\frametitle{}
	\centerline{\includegraphics[height=\textheight]{../5E/Supporting materials/RSM-26.png}}
\end{frame}
\begin{frame}\frametitle{}
	\centerline{\includegraphics[height=\textheight]{../5G/Supporting materials/COST-contours-shopping-extend-01.png}}
\end{frame}
\begin{frame}\frametitle{}
	\centerline{\includegraphics[height=\textheight]{../5G/Supporting materials/COST-contours-shopping-extend-02.png}}
\end{frame}
\begin{frame}\frametitle{}
	\centerline{\includegraphics[height=\textheight]{../5G/Supporting materials/COST-contours-shopping-extend-03.png}}
\end{frame}
\begin{frame}\frametitle{}
	\centerline{\includegraphics[height=\textheight]{../5G/Supporting materials/COST-contours-shopping-extend-04.png}}
\end{frame}
\begin{frame}\frametitle{}
	\centerline{\includegraphics[height=\textheight]{../5G/Supporting materials/COST-contours-shopping-extend-05.png}}
\end{frame} 
\begin{frame}\frametitle{}
	\centerline{\includegraphics[height=\textheight]{../5G/Supporting materials/COST-contours-shopping-extend-06.png}}
\end{frame}
\begin{frame}\frametitle{}
	\centerline{\includegraphics[height=\textheight]{../5G/Supporting materials/COST-contours-shopping-extend-07.png}}
\end{frame}
\begin{frame}\frametitle{}
	\centerline{\includegraphics[height=\textheight]{../5G/Supporting materials/COST-contours-shopping-extend-08.png}}
\end{frame}
\begin{frame}\frametitle{}
	\centerline{\includegraphics[height=\textheight]{../5G/Supporting materials/COST-contours-shopping-extend-09.png}}
\end{frame}
\begin{frame}\frametitle{}
	\centerline{\includegraphics[height=\textheight]{../5G/Supporting materials/COST-contours-shopping-extend-13.png}}
\end{frame}
\begin{frame}\frametitle{}
	\centerline{\includegraphics[height=\textheight]{../5G/Supporting materials/COST-contours-shopping-extend-14.png}}
\end{frame}
\begin{frame}\frametitle{}
	\centerline{\includegraphics[height=\textheight]{../5G/Supporting materials/COST-contours-shopping-extend-15.png}}
\end{frame}
\begin{frame}\frametitle{}
	\centerline{\includegraphics[height=\textheight]{../5G/Supporting materials/COST-contours-shopping-extend-16.png}}
\end{frame}
\begin{frame}\frametitle{}
	\centerline{\includegraphics[height=\textheight]{../5G/Supporting materials/COST-contours-shopping-extend-17.png}}
\end{frame}
\begin{frame}\frametitle{}
	\centerline{\includegraphics[height=\textheight]{../5G/Supporting materials/COST-contours-shopping-extend-18.png}}
\end{frame}
\begin{frame}\frametitle{}
	\centerline{\includegraphics[height=\textheight]{../5G/Supporting materials/COST-contours-shopping-extend-19.png}}
\end{frame}
\begin{frame}\frametitle{}
	\centerline{\includegraphics[height=\textheight]{../5G/Supporting materials/COST-contours-shopping-extend-20.png}}
\end{frame}
\begin{frame}\frametitle{}
	\centerline{\includegraphics[height=\textheight]{../5G/Supporting materials/COST-contours-shopping-reorder-01.png}}
\end{frame}
\begin{frame}\frametitle{}
	\centerline{\includegraphics[height=\textheight]{../5G/Supporting materials/COST-contours-shopping-reorder-02.png}}
\end{frame}
\begin{frame}\frametitle{}
	\centerline{\includegraphics[height=\textheight]{../5G/Supporting materials/COST-contours-shopping-reorder-03.png}}
\end{frame}
\begin{frame}\frametitle{}
	\centerline{\includegraphics[height=\textheight]{../5G/Supporting materials/COST-contours-shopping-reorder-04.png}}
\end{frame}
\begin{frame}\frametitle{}
	\centerline{\includegraphics[height=\textheight]{../5G/Supporting materials/COST-contours-shopping-reorder-05.png}}
\end{frame}
\begin{frame}\frametitle{}
	\centerline{\includegraphics[height=\textheight]{../5G/Supporting materials/COST-contours-shopping-reorder-06.png}}
\end{frame}
\begin{frame}\frametitle{}
	\centerline{\includegraphics[height=\textheight]{../5G/Supporting materials/COST-contours-shopping-reorder-07.png}}
\end{frame}
\begin{frame}\frametitle{}
	\centerline{\includegraphics[height=\textheight]{../5G/Supporting materials/COST-contours-shopping-reorder-08.png}}
\end{frame}
\begin{frame}\frametitle{}
	\centerline{\includegraphics[height=\textheight]{../5G/Supporting materials/COST-contours-shopping-reorder-09.png}}
\end{frame}
\begin{frame}\frametitle{}
	\centerline{\includegraphics[height=\textheight]{../5E/Supporting materials/RSM-26.png}}
\end{frame}
\begin{frame}\frametitle{}
	\centerline{\includegraphics[height=\textheight]{../5E/Supporting materials/RSM-27.png}}
\end{frame}
\begin{frame}\frametitle{}
	\centerline{\includegraphics[height=\textheight]{../5E/Supporting materials/RSM-28.png}}
\end{frame}
\begin{frame}\frametitle{}
	\centerline{\includegraphics[height=\textheight]{../5E/Supporting materials/RSM-29.png}}
\end{frame}
\begin{frame}\frametitle{}
	\centerline{\includegraphics[height=\textheight]{../5E/Supporting materials/RSM-30.png}}
\end{frame}
\begin{frame}\frametitle{}
	\centerline{\includegraphics[height=\textheight]{../5E/Supporting materials/RSM-31.png}}
\end{frame}
\begin{frame}\frametitle{}
	\centerline{\includegraphics[height=\textheight]{../5E/Supporting materials/RSM-32.png}}
\end{frame}
\begin{frame}\frametitle{}
	\centerline{\includegraphics[height=\textheight]{../5E/Supporting materials/RSM-33.png}}
\end{frame}
\begin{frame}\frametitle{}
	\centerline{\includegraphics[height=\textheight]{../5E/Supporting materials/RSM-34.png}}
\end{frame}
\begin{frame}\frametitle{}
	\centerline{\includegraphics[height=\textheight]{../5G/Supporting materials/no-interaction.png}}
\end{frame}
\begin{frame}\frametitle{}
	\centerline{\includegraphics[height=\textheight]{../5G/Supporting materials/with-interaction.png}}
\end{frame}

\begin{frame}\frametitle{The 3 models we have built so far}
	Factorial 1:  
	\[\color{blue}\hat{y} = 389.8  +134  x_\text{P} +  55 x_\text{T} -3.5 x_\text{P}x_\text{T} \]
	Factorial 2: 
	\[\color{myGreen} \hat{y} =  645.4  +47 x_\text{P} +  22.500 x_\text{T} -2.0 x_\text{P}x_\text{T} \]
	Factorial 3:
	\[\hat{y} = 723.6  -2.5 x_\text{P} +  7.500 x_\text{T} -1.5 x_\text{P}x_\text{T} \]
	
\end{frame}

\begin{frame}\frametitle{Taking the next step for experiment 15: {\color{myOrange}solution}}
	\begin{columns}[T]
		\column{0.2066\textwidth}
		
			\vspace{0.1cm}
			{\tiny 
				\begin{enumerate}
					\item	Pick change in coded units in one factor.
				\end{enumerate}
			 \par}
			 
			\onslide+<1->{
				{\tiny 
					\begin{enumerate}\setcounter{enumi}{1}
						\item	Find the ratios for\\ the other factor(s).
					\end{enumerate}
				
				\par}
			}
			
			\vspace{0.0cm}
			\onslide+<1->{
				{\tiny 
					\begin{enumerate}\setcounter{enumi}{2}
						\item	Calculate step size in coded units.
					\end{enumerate}
				
				\par}
			}
			
			\onslide+<1->{
				{\tiny 
					\begin{enumerate}\setcounter{enumi}{3}
						\item	Convert these to real-world \emph{changes}.
					\end{enumerate}
				
				\par}
			}
			
			\onslide+<1->{
				{\tiny 
					\begin{enumerate}\setcounter{enumi}{4}
						\item	Get the real-world location
						of the next experiment.
					\end{enumerate}
				
				\par}
			}
			
			
			\vspace{-0.2cm}
			\onslide+<1->{
				{\tiny 
					\begin{enumerate}\setcounter{enumi}{5}
						\item	Convert these back\\ to coded-units.
					\end{enumerate}
				
				\par}
			}
				
		\column{0.01\textwidth}
			\rule[3mm]{0.01cm}{60mm}%
			
			
		\column{0.4\textwidth}
			\centerline{\textbf{Price}}
			
			\vspace{-0.5cm}
			\onslide+<1->{
				\vspace{0.cm}
				\begin{align*}
					\Delta x_\text{P} &= {\color{blue} = \dfrac{b_\text{P}}{b_\text{T}} \times \Delta x_\text{T} }
				\end{align*}
			}
			
			\vspace{-1.15cm}
			\onslide+<1->{
				\vspace{0.cm}
				\begin{align*}
					\Delta x_\text{P} &= {\color{blue}  \dfrac{-2.5}{7.5} \times 2 = -\dfrac{2}{3}}
				\end{align*}
			}
			
			\vspace{-0.65cm}
			\onslide+<1->{
				\vspace{-0.6cm}
				\begin{align*} 
					\Delta \text{P} &= {\color{blue} -\dfrac{2}{3}\cdot\frac{1}{2}(0.36) = -\$0.12}
				\end{align*}
			}
			
			\vspace{-1.2cm}
			\onslide+<1->{
				\begin{align*} 
					\text{P}^{(15)} &= {\color{blue} \text{P}^{(10)} + \Delta \text{P} = \$1.63 - 0.12 = \$1.51}
				\end{align*}
			}
			
			\vspace{-1.5cm}
			\onslide+<1->{	
				\begin{align*} 
					x_\text{P}^{(15)} &={\color{blue}\dfrac{1.51-1.63}{\tfrac{1}{2}\cdot (0.36)} = -\dfrac{2}{3}}
				\end{align*}
			}
		
		\column{0.01\textwidth}
			\rule[3mm]{0.01cm}{56.5mm}%
			
		\column{0.4\textwidth}
			\centerline{\textbf{Throughput}}
			
			$\Delta x_\text{T} = 2$ (this was chosen)
			
			\vspace{1.3cm}
			$\Delta x_\text{T} = 2$ 
		
			\vspace{-0.85cm}
			\onslide+<1->{
				\begin{align*} 
					\Delta \text{T} &= {\color{blue} 6} \\
				\end{align*}
			}
			
			\vspace{-2cm}
			\onslide+<1->{
				\begin{align*} 
					 \text{T}^{(15)} &=  {\color{blue} \text{T}^{(10)} + \Delta \text{T}}\\
					 \text{T}^{(15)} &= 339 + 6 = 345\\
				\end{align*}
			}
			
			\vspace{-1.9cm}
			\onslide+<1->{	
				\begin{align*} 
					x_\text{T}^{(15)} &= {\color{blue} 2} 
				\end{align*}
			}
	\end{columns}
	
	\vspace{-0.5cm}
	\begin{columns}[T]
		\column{0.2\textwidth}

			\vspace{-0.0cm}
			\onslide+<1->{
				{\tiny 
					\begin{enumerate}\setcounter{enumi}{6}
						\item	Predict the next experiment's outcome.
					\end{enumerate}
				
				\par}
			}
			
			\vspace{0cm}
			\onslide+<2->{
				{\tiny 
					\begin{enumerate}\setcounter{enumi}{7}
						\item	Now run the next experiment, and record the values
					\end{enumerate}
				
				\par}
			}
			
		\column{0.01\textwidth}
			\rule[3mm]{0.01cm}{85mm}%
			
		\column{0.912\textwidth}
			
			\onslide+<1->{	
				\hrule
				\vspace{-0.2cm}
				\begin{align*}
					\hat{y}       &&=&& 723.6 &&-&& 2.5  x_\text{P} &&+&& 7.5 x_\text{T} &&-&& 1.5x_\text{P}x_\text{T}&& \\
					\hat{y}^{(15)} && \approx&& {\color{blue}\$ 742~\text{profit per hour}} \span\omit\span\omit\span\omit\span\omit\span\omit
				\end{align*}
			}
			
			\vspace{-1.6cm}
			\onslide+<2->{	
				\begin{align*}
					y^{(15)} &=  \color{blue} \$ 735 ~\text{profit per hour}
				\end{align*}
			}
	\end{columns}
	
\end{frame}

\begin{frame}\frametitle{Dealing with curvature when we suspect we are approaching an optimum}
	{\color{myOrange}We detect curvature in several ways (in practice, you will detect one or more of these)}
	
	\vspace{.7cm}
	\begin{enumerate}
		\item	interaction terms are comparable to main effects
	\end{enumerate}
	
	
	\begin{columns}[T]
		\column{0.33\textwidth}
			\centerline{\includegraphics[width=\textwidth]{../5G/Supporting materials/no-interaction.pdf}}
			\vspace{-0.5cm}
			\[ \hat{y} = -2.5 x_\text{P} +7.5 x_\text{T} + \cancelto{0}{0 \cdot x_\text{P}x_\text{T}} \]
		\column{0.33\textwidth}
			\centerline{\includegraphics[width=\textwidth]{../5G/Supporting materials/with-interaction.pdf}}
			\[\hat{y} = -2.5 x_\text{P} +  7.5 x_\text{T} \color{red}-1.5 x_\text{P}x_\text{T}\]
		\column{0.33\textwidth}
			\centerline{\includegraphics[width=\textwidth]{../5G/Supporting materials/system-with-a-saddle.pdf}}
			\vspace{-0.5cm}
			\[\hat{y} = -2.5 x_\text{P} +  5.35 x_\text{T} \color{red}- 7.5 x_\text{P}x_\text{T}\]
	\end{columns}
	
\end{frame}

\begin{frame}\frametitle{Dealing with curvature when we suspect we are approaching an optimum}
	{\color{myOrange}We detect curvature in several ways (in practice, you will detect one or more of these)}
	
	\vspace{.7cm}
	\begin{enumerate}\setcounter{enumi}{1}
		\item	differences in the {\color{myGreen} spread}$^\ast$ are becoming smaller and smaller
	\end{enumerate}
	
	\vspace{.3cm}
	$^\ast${\color{myGreen}\scriptsize can be crudely quantified as (highest outcome) $-$ (lowest outcome)}
	\vspace{.5cm}
	\begin{columns}[T]
		\column{0.33\textwidth}
			Factorial 1
			
			\vspace{.5cm}
			\centerline{\includegraphics[width=\textwidth]{../5G/Supporting materials/factorial-1.png}}
			
			spread = \$ 378
		\column{0.33\textwidth}
		
			\onslide+<2->{
				Factorial 2 
			
				\vspace{.5cm}
				\centerline{\includegraphics[width=\textwidth]{../5G/Supporting materials/factorial-2.png}}
			
				spread = \$ 139
			}
		\column{0.33\textwidth}
			\onslide+<3->{
				Factorial 3
				
				\vspace{.5cm}
				\centerline{\includegraphics[width=\textwidth]{../5G/Supporting materials/factorial-3.png}}
			
				\vspace{0.4cm}
				spread = \$ 20
			}
	\end{columns}
	
\end{frame}

\begin{frame}\frametitle{{\color{red} Aside}: If we are judging differences, we need an estimate of ``noise''}
	\begin{columns}[T]
		\column{0.5\textwidth}
			\centerline{\includegraphics[width=\textwidth]{../5E/Supporting materials/RSM-35.png}}
		\column{0.5\textwidth}
			\centerline{\includegraphics[width=\textwidth]{../5E/Supporting materials/RSM-36.png}}
	\end{columns}	
\end{frame}

\begin{frame}\frametitle{Dealing with curvature when we suspect we are approaching an optimum}
	{\color{myOrange}We detect curvature in several ways (in practice, you will detect one or more of these)}
	
	\vspace{.7cm}
	\begin{enumerate}\setcounter{enumi}{2}
		\item	we notice prediction errors with our model that indicate the surface is changing
	\end{enumerate}
	
	
	\vspace{.5cm}
	\begin{columns}[T]
		\column{0.33\textwidth}
			
			\onslide+<2->{
			
				\centerline{\includegraphics[width=\textwidth]{../5G/Supporting materials/flickr-rmkoske-2558420300_a9a3005dc7_o-burned-modified.jpg}}
				\see{\href{https://secure.flickr.com/photos/67146024@N00/2558420300/}{Flickr: rmkoske}}
			}
			
		\column{0.33\textwidth}
		
			\onslide+<2->{
			
				\vspace{.5cm}
				
			
			}
		\column{0.33\textwidth}
			\onslide+<3->{
				
				\centerline{\includegraphics[width=\textwidth]{../5G/Supporting materials/flickr-93653851-8514973597_3e189570e2_o-cliff.jpg}}
				\vspace{-1.2cm}
				\see{\href{https://secure.flickr.com/photos/93653851@N04/8514973597/}{Flickr: Digital Temi}}
			}
	\end{columns}
	
\end{frame}

\begin{frame}\frametitle{}
	\centerline{\includegraphics[height=\textheight]{../5E/Supporting materials/RSM-37.png}}
\end{frame}

\begin{frame}\frametitle{Dealing with curvature when we suspect we are approaching an optimum}
	{\color{myOrange}We detect curvature in several ways (in practice, you will detect one or more of these)}
	
	\vspace{.7cm}
	\begin{enumerate}\setcounter{enumi}{3}
		\item	we detect ``lack of fit'' in our empirical model
	\end{enumerate}
	
	\begin{columns}[T]
		\column{0.2\textwidth}

		\column{0.912\textwidth}
			
			\centerline{\includegraphics[height=0.7\textheight]{\imagedir/doe/lack-of-fit-illustration.png}}
	\end{columns}
	
\end{frame}

\begin{frame}\frametitle{Dealing with curvature when we suspect we are approaching an optimum}
	{\color{myOrange}We detect curvature in several ways (in practice, you will detect one or more of these)}
	
	\vspace{.7cm}
	\begin{enumerate}\setcounter{enumi}{3}
		\item	we detect ``lack of fit'' in our empirical model
	\end{enumerate}
	
	\begin{columns}[T]
		\column{0.35\textwidth}
			\onslide+<1->{
				\centerline{\includegraphics[width=\textwidth]{../5G/Supporting materials/factorial-1.png}}
				
				\scriptsize
				$\hat{y} = 390  +134  x_\text{P} +  55 x_\text{T} -3.5 x_\text{P}x_\text{T}$
				\normalsize
				
				\vspace{0.2cm}
				
					$\left.\begin{array}{c}
						y^{(0)} = \$ 407\\
						\hat{y}^{(0)} = \$ 390
					\end{array}\onslide+<2->{  \right\} }$
					 \onslide+<2->{
					 	{\scriptsize difference = \$17}
					} 
			}
			
		\column{0.01\textwidth}
			\rule[3mm]{0.01cm}{60mm}%
			
		\column{0.35\textwidth}
			\onslide+<3->{
				\centerline{\includegraphics[width=\textwidth]{../5G/Supporting materials/factorial-2.png}}
				
				\scriptsize
				$\hat{y} =  645  +47 x_\text{P} +  22.5 x_\text{T} -2.0 x_\text{P}x_\text{T}$
				\normalsize
				
				\vspace{0.2cm}
				
				
					$\left.\begin{array}{c}
						y^{(8)} = \$ 657\\
						\hat{y}^{(8)} = \$ 645
					\end{array}\right\}$ {\scriptsize difference = \$12}
			}
			
		\column{0.01\textwidth}
			\rule[3mm]{0.01cm}{60mm}%
			
		\column{0.37\textwidth}
			\onslide+<4->{
				\centerline{\includegraphics[width=\textwidth]{../5G/Supporting materials/factorial-3-with-multiple-centers.png}}
				
				\vspace{0.4cm}
				\scriptsize
				$\hat{y} = 724  -2.5 x_\text{P} +  7.5 x_\text{T} -1.5 x_\text{P}x_\text{T} $
				\normalsize
				
				\vspace{0.2cm}
				
				$\left.\begin{array}{c}
					y^{(\text{mid})} = \$ 734\\
					\hat{y}^{(\text{mid})} = \$ 724
				\end{array}\right\}$ {\scriptsize difference = \$10}
			}
	\end{columns}
\end{frame}

\begin{frame}\frametitle{The one-dimensional equivalent of what we are seeing in two dimensions}
	
	{\color{myOrange}We detect curvature in several ways (in practice, you will detect one or more of these)}
	
	\vspace{.7cm}
	\begin{enumerate}\setcounter{enumi}{3}
		\item	we detect ``lack of fit'' in our empirical model
	\end{enumerate}
	
	\begin{columns}[T]
		\column{0.68\textwidth}
			\centerline{\includegraphics[width=5cm]{\imagedir/doe/lack-of-fit-illustration-linear-region.png}}
		\column{0.01\textwidth}
			\rule[3mm]{0.01cm}{60mm}%
		\column{0.32\textwidth}
			\centerline{\includegraphics[width=5cm]{\imagedir/doe/lack-of-fit-illustration.png}}
	\end{columns}	
\end{frame}

\begin{frame}\frametitle{Dealing with curvature when we suspect we are approaching an optimum}
	{\color{myOrange}We detect curvature in several ways (in practice, you will detect one or more of these)}
	
	\vspace{.7cm}
	\begin{enumerate}
		\item	interaction terms are comparable to main effects
		\item	differences in the spread are becoming smaller and smaller
		\item	we notice prediction errors with our model that indicate the surface is changing
		\item	we detect ``lack of fit''
		\item	confidence intervals show some model coefficients are not significant
	\end{enumerate}
	
\end{frame}

\begin{frame}\frametitle{Improving the model's prediction ability by adding quadratic terms}
	Current situation, displaying lack of fit:
	
	\centerline{\includegraphics[height=.8\textheight]{\imagedir/doe/lack-of-fit-illustration-nonlinear.png}}
\end{frame}

\begin{frame}\frametitle{Improving the model's prediction ability by adding quadratic terms}
	Add specially placed points:
	
	\centerline{\includegraphics[height=.8\textheight]{\imagedir/doe/lack-of-fit-illustration-nonlinear-extra-points.png}}
\end{frame}

\begin{frame}\frametitle{Improving the model's prediction ability by adding quadratic terms}
	And fit a quadratic model now:
	
	\centerline{\includegraphics[height=.8\textheight]{\imagedir/doe/lack-of-fit-illustration-nonlinear-extra-points-quadratic.png}}
\end{frame}

\begin{frame}\frametitle{Adding experiments so we can fit quadratic terms: where do we put them?}
	\begin{columns}[T]
		\column{0.5\textwidth}
			\centerline{\includegraphics[width=\textwidth]{\imagedir/doe/central-composite-design-MOOC-FCD-points-no-persective.png}}
		\column{0.5\textwidth}			
			\centerline{\includegraphics[width=\textwidth]{\imagedir/doe/central-composite-design-MOOC-CCD-points-no-persective.png}}
			\vspace{-0.5cm}
			\[\alpha = 1.41\]
	\end{columns}
\end{frame}

\begin{frame}\frametitle{Adding experiments so we can fit quadratic terms: where do we put them?}
	\begin{columns}[T]
		\column{0.5\textwidth}
			\centerline{\includegraphics[width=\textwidth]{\imagedir/doe/central-composite-design-MOOC-FCD-points.png}}
		\column{0.5\textwidth}			
			\centerline{\includegraphics[width=\textwidth]{\imagedir/doe/central-composite-design-MOOC-CCD-points.png}}
			
			\vspace{-0.5cm}
			\[\alpha = 1.41\]
	\end{columns}
\end{frame}

\begin{frame}\frametitle{Adding experiments so we can fit quadratic terms: where do we put them?}
	\begin{columns}[T]
		\column{0.5\textwidth}
		
			\vspace{1cm}
			
			\begin{itemize}
				\item	Run the factorial points first
				\onslide+<2->{
					\item	Then run the star (axial) points after$^\ast$
				}
				\onslide+<2->{
					\Large
					\[\alpha  = \left(2^k\right)^{0.25}\]
					
			
					\begin{itemize}
						\onslide+<2->{\item	$k = 2$; then $\alpha = \sqrt{2} \approx 1.41$ }
						\onslide+<2->{\item	$k = 3$; then $\alpha = 2^{\tfrac{3}{4}} \approx 1.68$}
					\end{itemize}
					\normalsize
				}
				\onslide+<2->{
					\item	Run several center points randomly, during the above two stages
				}
			\end{itemize}
			
			\onslide+<2->{
				\vspace{0cm}
				\tiny
				$^\ast$ Astute viewers will wonder about confounding disturbances, since these runs are not randomized.
			
				You can show that these {\color[rgb]{0,0.5,1}\textbf{two}} {\color[rgb]{0.5,0, 0.5}\textbf{groups}} of experiments are blocked.
			}
			
			
		\column{0.5\textwidth}			
			\centerline{\includegraphics[width=\textwidth]{\imagedir/doe/central-composite-design-MOOC-anim-01.png}}
			
			\vspace{-0.5cm}
			\[\alpha = 1.41\]
	\end{columns}
\end{frame}

\begin{frame}\frametitle{Adding experiments so we can fit quadratic terms: where do we put them?}
	\begin{columns}[T]
		\column{0.5\textwidth}
		
			\vspace{1cm}
			
			\begin{itemize}
				\item	Run the factorial points first
				\onslide+<1->{
					\item	Then run the star (axial) points after$^\ast$
				}
				\onslide+<1->{
					\Large
					\[\alpha  = \left(2^k\right)^{0.25}\]
					
			
					\begin{itemize}
						\onslide+<2->{\item	$k = 2$; then $\alpha = \sqrt{2} \approx 1.41$ }
						\onslide+<3->{\item	$k = 3$; then $\alpha = 2^{\tfrac{3}{4}} \approx 1.68$}
					\end{itemize}
					\normalsize
				}
				\onslide+<4->{
					\item	Run several center points randomly, during the above two stages
				}
			\end{itemize}
			
			\onslide+<4->{
				\vspace{0cm}
				\tiny
				$^\ast$ Astute viewers will wonder about confounding disturbances, since these runs are not randomized.
			
				You can show that these {\color[rgb]{0,0.5,1}\textbf{two}} {\color[rgb]{0.5,0, 0.5}\textbf{groups}} of experiments are blocked.
			}
			
			
		\column{0.5\textwidth}			
			\centerline{\includegraphics[width=\textwidth]{\imagedir/doe/central-composite-design-MOOC-anim-02.png}}
			
			\vspace{-0.5cm}
			\[\alpha = 1.41\]
	\end{columns}
\end{frame}

\begin{frame}\frametitle{Adding experiments so we can fit quadratic terms: where do we put them?}
	\begin{columns}[T]
		\column{0.5\textwidth}
		
			\vspace{1cm}
			
			\begin{itemize}
				\item	Run the factorial points first
				\onslide+<1->{
					\item	Then run the star (axial) points after$^\ast$
				}
				\onslide+<1->{
					\Large
					\[\alpha  = \left(2^k\right)^{0.25}\]
					
			
					\begin{itemize}
						\onslide+<1->{\item	$k = 2$; then $\alpha = \sqrt{2} \approx 1.41$ }
						\onslide+<1->{\item	$k = 3$; then $\alpha = 2^{\tfrac{3}{4}} \approx 1.68$}
					\end{itemize}
					\normalsize
				}
				\onslide+<2->{
					\item	Run several center points randomly, during the above two stages
				}
			\end{itemize}
			
			\onslide+<2->{
				\vspace{0cm}
				\tiny
				$^\ast$ Astute viewers will wonder about confounding disturbances, since these runs are not randomized.
			
				You can show that these {\color[rgb]{0,0.5,1}\textbf{two}} {\color[rgb]{0.5,0, 0.5}\textbf{groups}} of experiments are blocked.
			}
			
			
		\column{0.5\textwidth}			
			\centerline{\includegraphics[width=\textwidth]{\imagedir/doe/central-composite-design-MOOC-3-factors.png}}
			
			\vspace{-0.5cm}
			\[\alpha = 1.68\]
	\end{columns}
\end{frame}

\begin{frame}\frametitle{Adding experiments so we can fit quadratic terms: where do we put them?}
	\begin{columns}[T]
		\column{0.5\textwidth}
		
			\vspace{1cm}
			
			\begin{itemize}
				\item	Run the factorial points first
				\onslide+<1->{
					\item	Then run the star (axial) points after$^\ast$
				}
				\onslide+<1->{
					\Large
					\[\alpha  = \left(2^k\right)^{0.25}\]
					
			
					\begin{itemize}
						\onslide+<1->{\item	$k = 2$; then $\alpha = \sqrt{2} \approx 1.41$ }
						\onslide+<1->{\item	$k = 3$; then $\alpha = 2^{\tfrac{3}{4}} \approx 1.68$}
					\end{itemize}
					\normalsize
				}
				\onslide+<1->{
					\item	Run several center points randomly, during the above two stages
				}
			\end{itemize}
			
			\onslide+<2->{
				\vspace{0cm}
				\tiny
				$^\ast$ Astute viewers will wonder about confounding disturbances, since these runs are not randomized.
			
				You can show that these {\color[rgb]{0,0.5,1}\textbf{two}} {\color[rgb]{0.5,0, 0.5}\textbf{groups}} of experiments are blocked.
			}
			
			
		\column{0.5\textwidth}			
			\centerline{\includegraphics[width=\textwidth]{\imagedir/doe/central-composite-design-MOOC-anim-03.png}}
			
			\vspace{-0.5cm}
			\[\alpha = 1.41\]
	\end{columns}
\end{frame}

\begin{frame}\frametitle{Adding experiments so we can fit quadratic terms: where do we put them?}
	\begin{columns}[T]
		\column{0.5\textwidth}
		
			\vspace{1cm}
			
			\begin{itemize}
				\item	Run the factorial points first
				\onslide+<1->{
					\item	Then run the star (axial) points after$^\ast$
				}
				\onslide+<1->{
					\Large
					\[\alpha  = \left(2^k\right)^{0.25}\]
					
			
					\begin{itemize}
						\onslide+<1->{\item	$k = 2$; then $\alpha = \sqrt{2} \approx 1.41$ }
						\onslide+<1->{\item	$k = 3$; then $\alpha = 2^{\tfrac{3}{4}} \approx 1.68$}
					\end{itemize}
					\normalsize
				}
				\onslide+<1->{
					\item	Run several center points randomly, during the above two stages
				}
			\end{itemize}
			
			\onslide+<1->{
				\vspace{0cm}
				\tiny
				$^\ast$ Astute viewers will wonder about confounding disturbances, since these runs are not randomized.
			
				You can show that these {\color[rgb]{0,0.5,1}\textbf{two}} {\color[rgb]{0.5,0, 0.5}\textbf{groups}} of experiments are blocked.
			}
			
			
		\column{0.5\textwidth}			
			\centerline{\includegraphics[width=\textwidth]{\imagedir/doe/central-composite-design-MOOC-anim-04.png}}
			
			\vspace{-0.5cm}
			\[\alpha = 1.41\]
	\end{columns}
\end{frame}

\begin{frame}\frametitle{Adding experiments so we can fit quadratic terms: where do we put them?}
	\begin{columns}[T]
		\column{0.5\textwidth}
		
			\vspace{1cm}
			
			\begin{itemize}
				\item	Run the factorial points first
				\onslide+<1->{
					\item	Then run the star (axial) points after$^\ast$
				}
				\onslide+<1->{
					\Large
					\[\alpha  = \left(2^k\right)^{0.25}\]
					
			
					\begin{itemize}
						\onslide+<1->{\item	$k = 2$; then $\alpha = \sqrt{2} \approx 1.41$ }
						\onslide+<1->{\item	$k = 3$; then $\alpha = 2^{\tfrac{3}{4}} \approx 1.68$}
					\end{itemize}
					\normalsize
				}
				\onslide+<1->{
					\item	Run several center points randomly, during the above two stages
				}
			\end{itemize}
			
			\onslide+<1->{
				\vspace{0cm}
				\tiny
				$^\ast$ Astute viewers will wonder about confounding disturbances, since these runs are not randomized.
			
				You can show that these {\color[rgb]{0,0.5,1}\textbf{two}} {\color[rgb]{0.5,0, 0.5}\textbf{groups}} of experiments are blocked.
			}
			
			
		\column{0.5\textwidth}			
			\centerline{\includegraphics[width=\textwidth]{\imagedir/doe/central-composite-design-MOOC-anim-05.png}}
			
			\vspace{-0.5cm}
			\[\alpha = 1.41\]
	\end{columns}
\end{frame}

\begin{frame}\frametitle{}
	\centerline{\includegraphics[height=\textheight]{../5E/Supporting materials/RSM-38.png}}
\end{frame}
\begin{frame}\frametitle{}
	\centerline{\includegraphics[height=\textheight]{../5E/Supporting materials/RSM-39.png}}
\end{frame}
\begin{frame}\frametitle{}
	\centerline{\includegraphics[height=\textheight]{../5E/Supporting materials/RSM-40.png}}
\end{frame}
\begin{frame}\frametitle{}
	\centerline{\includegraphics[height=\textheight]{../5E/Supporting materials/RSM-41.png}}
\end{frame}
\begin{frame}\frametitle{}
	\centerline{\includegraphics[height=\textheight]{../5E/Supporting materials/RSM-42.png}}
\end{frame}
\begin{frame}\frametitle{List of all experiments}
	\begin{tabulary}{\linewidth}{c|cc|cc|c|c|cc}
		\textbf{\relax Experiment} & \textbf{\relax P } & \textbf{\relax T} & \textbf{\relax $x_\text{P}$} & \textbf{\relax $x_\text{T}$} & \textbf{\relax $\hat{y}$} & \textbf{\relax Actual = $y$} & \textbf{\relax Type } \\ \hline
			10 & ~~1.63 & 339 & $~0$ & $~0$ &  & \$ 732 &~Center \\
			11 & ~~1.45 & 336 & $-1$ & $-1$ &  & ~~715  &~Factorial \\
			12 & ~~1.81 & 336 & $+1$ & $-1$ &  & ~~713  &~Factorial \\
			13 & ~~1.45 & 342 & $-1$ & $+1$ &  & ~~733  &~Factorial \\ 
			14 & ~~1.81 & 342 & $+1$ & $+1$ &  & ~~725  &~Factorial \\
			15 & ~~1.51 & 345 & $-\tfrac{2}{3}$ & $+2$ & \$ 742 & ~~735  & Steepest path \\ \hline
			16 & ~~1.63 & 339 & $~0$ & $~0$ &  & \$ 733 &~Center \\
			17 & ~~1.63 & 339 & $~0$ & $~0$ &  & ~~737 &~Center \\ \hline
			\onslide+<2->{18 & ~~1.63 & 343 & $~0$ & $+1.41$ &  & \onslide+<6->{\$ 738} &~Star} \\ 
			\onslide+<3->{19 & ~~1.38 & 339 & $-1.41$ & $0$ &  & \onslide+<6->{~~717} &~Star \\ 
				20 & ~~1.63 & 335 & $0$ & $-1.41$ &  & \onslide+<6->{~~721} &~Star \\ 
				21 & ~~1.88 & 339 & $+1.41$ & $0$ &  & \onslide+<6->{~~710} &~Star}
			\onslide+<4->{\\ \hline
				22 & ~~1.63 & 339 & $~0$ & $~0$ &  & \onslide+<6->{\$ 735} &~Center} 
			\onslide+<7->{\\ \hline
				23 & ~~ &  &  &  &  &  &~Optimum?} 
			\end{tabulary}
\end{frame}
\begin{frame}\frametitle{Results of the 12 runs: (factorial + star + center). Are we nearly there?}
	\[\hat{y} = 734.23 -2.5x_\text{P}    +    6.97  x_\text{T}    -10.6  x^2_\text{P}     -2.5  x^2_\text{T}     -1.5x_\text{P}x_\text{T}\]
	\begin{columns}[T]
		\column{0.6\textwidth}
			\begin{enumerate}
				\onslide+<2->{
					\item	We get good predictions at the center,
							indicating a ``small lack of fit''
							\begin{itemize}
								\item	$\hat{y}^{(\text{center})} = \$ 734$
								\item	$y^{(\text{center})}_\text{actual} = \$ 734.25$
							\end{itemize}
				}
				\onslide+<3->{
					\vspace{0.5cm}
					\item	We have good predictions of other points in the region
						\centerline{\includegraphics[width=.8\textwidth]{../5G/Supporting materials/software-prediction.png}}
						\vspace{-0.3cm}
						\begin{itemize}
							\item	$\hat{y}^{(15)} = \$ 737$
							\item	$y^{(15)}_\text{actual} = \$ 735$
						\end{itemize}
					}
			\end{enumerate}
			
		\column{0.01\textwidth}
			\rule[3mm]{0.01cm}{60mm}%
		
		\column{0.4\textwidth}	
			Output from the R software:

			\centerline{\includegraphics[width=\textwidth]{../5G/Supporting materials/R-software-output.png}}

	\end{columns}
	
	
\end{frame}
\begin{frame}\frametitle{}
	\centerline{\includegraphics[height=\textheight]{../5E/Supporting materials/RSM-43.png}}
\end{frame}
\begin{frame}\frametitle{}
	\centerline{\includegraphics[height=\textheight]{../5E/Supporting materials/RSM-44.png}}
\end{frame}
\begin{frame}\frametitle{}
	\centerline{\includegraphics[height=\textheight]{../5E/Supporting materials/RSM-45.png}}
\end{frame}
\begin{frame}\frametitle{}
	\centerline{\includegraphics[height=\textheight]{../5E/Supporting materials/RSM-46.png}}
\end{frame}
\begin{frame}\frametitle{Finding the optimum analytically (with differentiation)}
	\[\hat{y} =	 734  -2.5x_\text{P}    +    6.97  x_\text{T}    -10.6  x^2_\text{P}     -2.5  x^2_\text{T}     -1.5x_\text{P}x_\text{T}	 \]

	\onslide+<2->{
		\begin{align*}
			\dfrac{\partial \hat{y}}{\partial x_\text{P}} &&=&& -2.5 &&-&& 21.2x_\text{P} &&-&& 1.5x_\text{T} &&=&& 0 \\
			\dfrac{\partial \hat{y}}{\partial x_\text{T}} &&=&& 6.97 &&-&& 1.5x_\text{P} &&-&& 5.0x_\text{T} &&=&& 0 \\
		\end{align*}
	}
	\vspace{-0.5cm}
	\onslide+<3->{
		\[
		\begin{pmatrix}-21.2 && -1.5 \\ \\ -1.5 && -5.0\end{pmatrix}\begin{pmatrix}x_\text{P} \\ \\ x_\text{T}\end{pmatrix} = \begin{pmatrix}2.5 \\ \\ -6.97\end{pmatrix}
		\]
	}
	\vspace{0.4cm}
	\onslide+<4->{
		\begin{align*}
			x^{(23)}_\text{P} &= -0.22 &&& \onslide+<5->{\text{P}^{(23)}& = -0.22 \cdot \tfrac{1}{2}(0.36) +1.63 = {\color{red}\$1.59}}\\
			x^{(23)}_\text{T} &= 1.46 &&&  \onslide+<5->{\text{T}^{(23)} &= 1.46\cdot \tfrac{1}{2}(6) +339 \approx {\color{red} 343~\text{parts per hour}}} \\
		\end{align*}
	}
	
\end{frame}
\begin{frame}\frametitle{}
	\centerline{\includegraphics[height=\textheight]{../5E/Supporting materials/RSM-47.png}}
\end{frame}
\begin{frame}\frametitle{}
	\centerline{\includegraphics[height=\textheight]{../5E/Supporting materials/RSM-48.png}}
\end{frame}
\begin{frame}\frametitle{}
	\centerline{\includegraphics[height=\textheight]{../5E/Supporting materials/RSM-49.png}}
\end{frame}
\begin{frame}\frametitle{}
	\centerline{\includegraphics[height=\textheight]{../5E/Supporting materials/RSM-50.png}}
\end{frame}
\begin{frame}\frametitle{}
	\centerline{\includegraphics[height=\textheight]{../5E/Supporting materials/RSM-51.png}}
\end{frame}
\begin{frame}\frametitle{}
	\centerline{\includegraphics[height=\textheight]{../5E/Supporting materials/RSM-52.png}}
\end{frame}
\begin{frame}\frametitle{}
	\centerline{\includegraphics[height=\textheight]{../5E/Supporting materials/RSM-53.png}}
\end{frame}
\begin{frame}\frametitle{}
	\centerline{\includegraphics[height=\textheight]{../5E/Supporting materials/RSM-54.png}}
\end{frame}
\begin{frame}\frametitle{}
	\centerline{\includegraphics[height=\textheight]{../5E/Supporting materials/RSM-55.png}}
\end{frame}
\begin{frame}\frametitle{}
	\centerline{\includegraphics[height=\textheight]{../5E/Supporting materials/RSM-56.png}}
\end{frame}
\begin{frame}\frametitle{}
	\centerline{\includegraphics[height=\textheight]{../5E/Supporting materials/RSM-57.png}}
\end{frame}
\begin{frame}\frametitle{}
	\centerline{\includegraphics[height=\textheight]{../5E/Supporting materials/RSM-58.png}}
\end{frame}
\begin{frame}\frametitle{}
	\centerline{\includegraphics[height=\textheight]{../5E/Supporting materials/RSM-59.png}}
\end{frame}
\begin{frame}\frametitle{}
	\centerline{\includegraphics[height=\textheight]{../5E/Supporting materials/RSM-60.png}}
\end{frame}


% /Users/kevindunn/Dropbox/Coursera/Media/All-course-slides/classes/CourseraMOOC-class-6.tex

\begin{frame}\frametitle{The course outline over the past five modules}
	\begin{columns}[T]
		\column{1.0\textwidth}
			\centerline{\includegraphics[width=\textwidth]{../6/Supporting material/Module-1.png}}
		%\column{0.3\textwidth}
			%\centerline{\includegraphics[width=\textwidth]{../6/Supporting material/weekly-plan-for-Coursera-MOOC.png}}
	\end{columns}
\end{frame}

\begin{frame}\frametitle{The course outline over the past five modules}
	\begin{columns}[T]
		\column{1.0\textwidth}
			\centerline{\includegraphics[width=\textwidth]{../6/Supporting material/Module-2.png}}
		%\column{0.3\textwidth}
		%	\centerline{\includegraphics[width=\textwidth]{../6/Supporting material/weekly-plan-for-Coursera-MOOC.png}}
	\end{columns}
\end{frame}

\begin{frame}\frametitle{The course outline over the past five modules}
	\begin{columns}[T]
		\column{1.0\textwidth}
			\centerline{\includegraphics[width=\textwidth]{../6/Supporting material/Module-3.png}}
		%\column{0.3\textwidth}
		%	\centerline{\includegraphics[width=\textwidth]{../6/Supporting material/weekly-plan-for-Coursera-MOOC.png}}
	\end{columns}
\end{frame}

\begin{frame}\frametitle{The course outline over the past five modules}
	\begin{columns}[T]
		\column{1.0\textwidth}
			\centerline{\includegraphics[width=\textwidth]{../6/Supporting material/Module-4.png}}
		%\column{0.3\textwidth}
		%	\centerline{\includegraphics[width=\textwidth]{../6/Supporting material/weekly-plan-for-Coursera-MOOC.png}}
	\end{columns}
\end{frame}

\begin{frame}\frametitle{The course outline over the past five modules}
	\begin{columns}[T]
		\column{1.0\textwidth}
			\centerline{\includegraphics[width=\textwidth]{../6/Supporting material/Module-5.png}}
		%\column{0.3\textwidth}
		%	\centerline{\includegraphics[width=\textwidth]{../6/Supporting material/weekly-plan-for-Coursera-MOOC.png}}
	\end{columns}
\end{frame}

\begin{frame}\frametitle{}
			\centerline{\includegraphics[height=\textheight]{../5E/Supporting materials/RSM-51.png}}
\end{frame}

\begin{frame}\frametitle{Dealing with multiple criteria in experiments}
	\begin{columns}[c]
		\column{0.75\textwidth}
			\begin{itemize}
				\item	The weighted sum = $\varphi$
				\begin{align*}
					\varphi &&=&& {\color{purple}w_1} {\color[rgb]{0,0.56,0.93}(\text{colour})} &&+&& {\color{purple}w_2} {\color[rgb]{0.93,0,0}(\text{breakability})} \\
				\onslide+<2->{	
				\text{For example}\qquad	\varphi &&=&& {\color{purple}0.3} {\color[rgb]{0,0.56,0.93}(\text{colour})} &&+&& {\color{purple}0.7} {\color[rgb]{0.93,0,0}(\text{breakability})}
				}
				\end{align*}
			\end{itemize}
			
			\centerline{\includegraphics[width=.7\textwidth]{\imagedir/doe/examples/snackfood-tradeoffs-01.png}}

		\column{0.3\textwidth}
			Optimizing the properties of a snack food product that is fried in oil
			\centerline{\includegraphics[width=\textwidth]{../6/Supporting material/snack-food.jpg}}
	\end{columns}
\end{frame}

\begin{frame}\frametitle{Dealing with multiple criteria in experiments}
	\begin{columns}[c]
		\column{0.75\textwidth}
			\begin{itemize}
				\item	The weighted sum = $\varphi$
				\begin{align*}
					\varphi &&=&& {\color{purple}w_1} {\color[rgb]{0,0.56,0.93}(\text{colour})} &&+&& {\color{purple}w_2} {\color[rgb]{0.93,0,0}(\text{breakability})} \\
				\onslide+<1->{	
				\text{For example}\qquad	\varphi &&=&& {\color{purple}0.3} {\color[rgb]{0,0.56,0.93}(\text{colour})} &&+&& {\color{purple}0.7} {\color[rgb]{0.93,0,0}(\text{breakability})}
				}
				\end{align*}
			\end{itemize}
			
			\centerline{\includegraphics[width=.7\textwidth]{\imagedir/doe/examples/snackfood-tradeoffs-02.png}}

		\column{0.3\textwidth}
			Optimizing the properties of a snack food product that is fried in oil
			\centerline{\includegraphics[width=\textwidth]{../6/Supporting material/snack-food.jpg}}
	\end{columns}
\end{frame}

\begin{frame}\frametitle{Dealing with multiple criteria in experiments}
	\begin{columns}[c]
		\column{0.75\textwidth}
			\begin{itemize}
				\item	The weighted sum = $\varphi$
				\begin{align*}
					\varphi &&=&& {\color{purple}w_1} {\color[rgb]{0,0.56,0.93}(\text{colour})} &&+&& {\color{purple}w_2} {\color[rgb]{0.93,0,0}(\text{breakability})} \\
				\onslide+<1->{	
				\text{For example}\qquad	\varphi &&=&& {\color{purple}0.3} {\color[rgb]{0,0.56,0.93}(\text{colour})} &&+&& {\color{purple}0.7} {\color[rgb]{0.93,0,0}(\text{breakability})}
				}
				\end{align*}
			\end{itemize}
			
			\centerline{\includegraphics[width=.7\textwidth]{\imagedir/doe/examples/snackfood-tradeoffs-03.png}}

		\column{0.3\textwidth}
			Optimizing the properties of a snack food product that is fried in oil
			\centerline{\includegraphics[width=\textwidth]{../6/Supporting material/snack-food.jpg}}
	\end{columns}
\end{frame}

\begin{frame}\frametitle{}
	\centerline{\includegraphics[width=\textwidth]{\imagedir/doe/DOE-trade-off-table.png}}
\end{frame}


\begin{frame}\frametitle{The book by Goos and Jones: \emph{Optimal Design of Experiments}}
	\begin{itemize}
		\item	A case-study based approach \pause
		\item	RSM with categorical factors \pause
		\item	Screening designs \pause
		\item	Mixture designs \pause
		\item	Blocking and covariates \pause
		\item	Split-plot designs (an important, practical topic) 
	\end{itemize}
\end{frame}

\begin{frame}\frametitle{}
	\centerline{\includegraphics[width=.96\textwidth]{../5B/Supporting materials/popcorn-experiments-32.png}}
\end{frame}

\begin{frame}\frametitle{Some additional advice on response surface methods}
	
	\begin{columns}[T]
		\column{0.95\textwidth}
			\begin{itemize}
				\item	RSM with multiple factors, e.g. \textbf{A}, \textbf{B}, \textbf{C}, and \textbf{D}. Pick a
				value for $\Delta x_\text{A}$, then:
				\begin{itemize}
					\item	$\Delta x_\text{B} = \dfrac{b_\text{B}}{b_\text{A}} \Delta x_\text{A}$
					\item	$\Delta x_\text{C} = \dfrac{b_\text{C}}{b_\text{A}} \Delta x_\text{A}$
					\item	$\Delta x_\text{D} = \dfrac{b_\text{D}}{b_\text{A}} \Delta x_\text{A}$
				\end{itemize}
				
				\item	Now with 4 factors, it does not make sense to run $2^4=16$ experiments in every factorial. Use fractional factorials.
				\item	Unless, you are doing the experiments on a small scale, and they are cheap(er) than full-scale experiments.
				\item	But watch the aliases when you approach the optimum. You will likely need to run a full CCD near the optimum to estimate curvature correctly.
						\\
				\item	Check out the {\color{myOrange}``rsm''} package in R
						
			\end{itemize}
		\column{0.05\textwidth}
			
	\end{columns}
% categorical -> continuous
\end{frame}

\begin{frame}\frametitle{Resources for Designed Experiments}
	The list is available here: 
	\href{http://yint.org/resources}{http://yint.org/resources}
	
	\vspace{24pt}
	Contact me to keep the list up-to-date: see the link on that webpage.
\end{frame}

