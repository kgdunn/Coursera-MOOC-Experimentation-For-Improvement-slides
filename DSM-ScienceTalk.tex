\documentclass[handout,11pt,aspectratio=169,mathserif]{beamer}

\input{preamble.tex}

\newdateformat{mydate}{\THEYEAR} % \twodigit{\THEDAY}-\twodigit{\THEMONTH}-\THEYEAR
\mydate
\title[]{\LARGE Experimentation for Improvement}
\subtitle[]{\vspace{0.5cm} \includegraphics[width=0.5\textwidth]{\imagedir/teaching/logos/Coursera-MOOC-logo-2014.png} \vspace{-1.5cm} }\author[]{}
\institute[]{}
\date[]{\copyright~ Kevin Dunn, \today \\ \vspace{0cm}
{{\footnotesize {\tt } \href{http://learnche.org/}{http://learnche.org/}\\} \vspace{0.5cm}}

{\LARGE \color{blue}{Design and Analysis of Experiments - An Overview}}

}

\begin{document}
	
\begin{frame} \titlepage \end{frame}

\begin{frame}\frametitle{Copyright, sharing, and attribution notice}

	{\footnotesize This work is licensed under the Creative Commons Attribution-ShareAlike 4.0 Unported License. To view a copy of this license, 
	please visit \href{http://creativecommons.org/licenses/by-sa/3.0/}{http://creativecommons.org/licenses/by-sa/4.0/}}
	
	\vspace{0.0cm}
	\begin{flushright}
		\includegraphics[width=0.2\textwidth]{\imagedir/common/creative-commons-by-sa.png}
	\end{flushright}	
	\vspace{-0.4cm}
	\begin{exampleblock}{}
		{\small This license allows you: }
		\begin{itemize}
			\item	{\color{myGreen}{\textbf{to share}}} - to copy, distribute and transmit the work, including print it
			\item	{\color{myOrange}{\textbf{to adapt}}} - but you must distribute the new result under the same or similar license to this one
			\item	{\color{myRed}{\textbf{commercialize}}} - you \underline{\emph{are allowed}} to use this work for commercial purposes 
			\item	{\color{blue}{\textbf{attribution}}} - but you must attribute the work as follows:
			\begin{itemize}
				\item	``Portions of this work are the copyright of Kevin Dunn'', \emph{or}
				\item	``This work is the copyright of Kevin Dunn'' \\{\tiny (when used without modification)}
			\end{itemize} 
		\end{itemize}
	\end{exampleblock}
\end{frame}


\begin{frame}\frametitle{Some important terminology we will use all the time}
	
	\begin{columns}[T]
		\column{0.75\textwidth}
			\textbf{{\color{purple} Outcome}}
				\begin{itemize}
					\item	What we measure after the experiment is finished  \pause
					\item	It is the aspect you are interested in improving.\pause
			 
				\end{itemize}
		\column{0.3\textwidth}
		
			\centerline{\includegraphics[width=\textwidth]{\imagedir/doe/measure-4904403417_93baa750a6-flickr.jpg}}

	\end{columns}
	
			
	\vspace{24pt}
	\textbf{{\color{purple} Factors}}
		\begin{itemize}
			\item	Things which you actively change to influence the outcome.
			\item	We typically change 2, 3, 4, or many more factors. \pause
			\item	Don't fixate on changing 1 factor at a time.
		\end{itemize}
\end{frame}

\begin{frame}\frametitle{}
	
	\begin{columns}[c]
		\column{0.5\textwidth}
			\centerline{\includegraphics[width=0.8\textwidth]{../1B/Supporting-material/plant.png}}
		
		\column{0.5\textwidth}
			Various outcomes are possible in your experiment:
			
			\begin{itemize}
				\item	height of the plant
				\item	average length of leaves
				\item	the number of flowers
			\end{itemize}
			
			\vspace{12pt}
			These are examples of numeric measurements (quantitative).

	\end{columns}
\end{frame}

\begin{frame}\frametitle{}
	
	\begin{columns}[c]
		\column{0.5\textwidth}
			\centerline{\includegraphics[width=0.8\textwidth]{../1B/Supporting-material/plant.png}}
		
		\column{0.5\textwidth}
			%Different outcomes are possible:
			
			\begin{itemize}
				\item	colour of the flower
			\end{itemize}
			
			\vspace{12pt}
			This is a qualitative measurement.
			
			(We use qualitative outcomes infrequently)

	\end{columns}
\end{frame}

\begin{frame}\frametitle{}
	
	\textbf{{\color{purple} Outcome}} \onslide+<5->{ = \textbf{{\color{purple} Response}}}
		\begin{itemize}
			\item	What we measure after the experiment is finished  
			\item	It is the aspect you are interested in improving.
		\end{itemize}
	
	\vspace{12pt}
	\pause
	
	\textbf{{\color{purple} Objective}}
	
	
		\qquad combine the {\color{purple} outcome} with ``a desire to \emph{adjust} the outcome''
	\vspace{24pt}
	\pause
	
	{\textbf{{Various examples of ``objectives''}}}
	
		\begin{itemize}
			\item	maximize $(\uparrow)$ the profit
			\item	maximize $(\uparrow)$ the height of the plant
			\item	minimize $(\downarrow)$ pollution
			\item	minimize $(\downarrow)$ energy used to produce a product
		\end{itemize} 	
		
	\pause
		\vspace{12pt}
		But sometimes the objective is ``the same as before'' $(=)$
\end{frame}

\begin{frame}\frametitle{}
	
	\textbf{{\color{purple} Factors}} \onslide+<6->{ = \textbf{{\color{purple} Variables}}}
	\pause
		
		\begin{columns}[T]
			\column{0.33\textwidth}
				\onslide+<2->{\centerline{\includegraphics[width=\textwidth]{../1B/Supporting-material/water.png}}}
				
				
			\column{0.33\textwidth}
				\onslide+<3->{\centerline{\includegraphics[width=\textwidth]{../1B/Supporting-material/fertilizer.png}}}
				
				
			\column{0.33\textwidth}
				\onslide+<4->{\centerline{\includegraphics[width=\textwidth]{../1B/Supporting-material/soil.png}}}
			
		\end{columns}
		
		
	\vspace{24pt}
	
	\onslide+<5->{
		\textbf{{\color{purple} Types of factors}}
	
	
		\qquad \emph{numeric} factors (quantitative) can be measured and adjusted to different levels
		
			\qquad \qquad 
		\vspace{12pt}
	
		\qquad \emph{categorical} factors (qualitative) are simply selected for their characteristic
	}
	
	
\end{frame}



\end{document}

