%\documentclass[11pt,aspectratio=169,mathserif]{beamer}
\documentclass[handout,11pt,aspectratio=169,mathserif]{beamer}

\usepackage[latin1]{inputenc}  % allows direct input of Unicode characters
% %\usepackage[utf8x]{inputenc}   % allows direct input of Unicode characters
\inputencoding{utf8} %<--- try this next time
%\usepackage{textcomp}
\usepackage{amsmath,amssymb,amsfonts,euscript,mathrsfs,wasysym,textcomp}
\usepackage{array}
\usepackage{transparent} 		% to get alpha channel in "\color"
\usepackage{multirow}
\usepackage{multimedia}
\usepackage{fancybox}
\usepackage{mathtools}
\usepackage{psfrag}
\usepackage{listings}
\usepackage{hyperref}
\usepackage[normalem]{ulem}  	% For strikeout text: \sout{text goes here}
\usepackage{datetime} 			% For better date/time display
\usepackage{comment}  			% To block out parts of the notes
\usepackage[multidot]{grffile} 	% Handles file names with spaces and dots in them
\usepackage{cancel}				% To cancel terms in an equation
%\usepackage{rotating}			% Rotated text
\usepackage{etoolbox}			% For selective compiling: figure sources might be on different machines
\usepackage{tabulary}			% Better package for tables?

\providetoggle{internal}
\settoggle{internal}{false}      % set "true" when the PDF is for internal use; use "false" when publishing it publically, on-line

\providetoggle{hrymak}          % usage: \iftoggle{hrymak}{}{}
\settoggle{hrymak}{false}

\providetoggle{instructor}
\providetoggle{student}

\settoggle{instructor}{true}    % usage: \iftoggle{instructor}{}{}
\settoggle{student}{false}      % usage: \iftoggle{student}{}{}

\providetoggle{unixmac}
\settoggle{unixmac}{true}
\iftoggle{unixmac}
{
	\newcommand{\imagedir}{/Users/kevindunn/Dropbox/figures-master}
}{
	\newcommand{\imagedir}{C:/Figures}
}

%\newcommand{\angstrom}{\mbox{\normalfont\AA}}
% \tiny\scriptsize\footnotesize\small\normalsize\large\Large\LARGE\huge\Huge

\usetheme{default} 
\setbeamertemplate{navigation symbols}{}          % suppress all navigation symbols
\setbeamertemplate{blocks}[rounded][shadow=true]  % use rounded blocks (boxes), with shadows

% http://www.latex-community.org/forum/viewtopic.php?f=4&t=2251
\newenvironment<>{varblock}[2][\textwidth]{%
	\begin{center}
		\begin{minipage}{#1}
			\setlength{\textwidth}{#1}
			\begin{actionenv}#3%
				\def\insertblocktitle{#2}%
				\par%
				\usebeamertemplate{block begin}}
				{\par%
				\usebeamertemplate{block end}%
			\end{actionenv}
		\end{minipage}
	\end{center}
}
\definecolor{todoGreen}{rgb}{0.0, 0.9, 0.0}
\definecolor{myGreen}{rgb}{0.,0.4,0.}
\definecolor{myOrange}{rgb}{1.,0.5,0.}
\definecolor{myBlue}{rgb}{0.0,0.1,0.9}
\definecolor{myRed}{rgb}{1.0,0.0,0.0}
\definecolor{myLightGrey}{rgb}{0.8,0.8,0.8}
\definecolor{Brown}{cmyk}{0,0.81,1,0.60}
\definecolor{OliveGreen}{cmyk}{0.64,0,0.95,0.40}
\definecolor{CadetBlue}{cmyk}{0.62,0.57,0.23,0}
\definecolor{purple}{rgb}{0.70,0.22,0.92}

%https://tex.stackexchange.com/questions/31323/cross-out-with-arrow-as-in-goes-to-zero
\renewcommand{\CancelColor}{\color{red}} %change cancel color to red
\makeatletter
\let\my@cancelto\cancelto %copy over the original cancelto command
\newcommand<>{\cancelto}[2]{\alt#3{\my@cancelto{#1}{#2}}{\mathrlap{#2}\phantom{\my@cancelto{#1}{#2}}}}
% redefine the cancelto command, using \phantom to assure that the
% result doesn't wiggle up and down with and without the arrow
\makeatother



% Show page numbers and dates on slides
% KGD: removed footer completely: 01 Oct 2012: Presentation Zen: no redundant information
%\setbeamertemplate{footline}{\begin{beamercolorbox}[right]{section in head/foot}{\color{black}{\tiny \insertframenumber}} ~~~ \vskip5pt \end{beamercolorbox}}

\setbeamercovered{transparent=1} 

\hypersetup{colorlinks        = true,    
 			linkcolor         = blue,    
%}
% 			linkbordercolor   = {1 0 0},    
% 			urlcolor          = cyan,    
% 			bookmarks         = {true,},    
% 			bookmarksopen     = {true,},    
% 			bookmarksnumbered = {false,},    
 			pdftitle          = {Kevin Dunn},
 			pdfsubject        = {Kevin Dunn},
 			pdfauthor         = {Kevin Dunn},
 			pdfproducer       = {LaTeX, wiki2beamer, beamer, BeamerPDF},    
 			pdfkeywords       = http://learnche.mcmaster.ca/,
 		}
\usepackage{pgfpages}

\makeatletter
\def\hlinewd#1{%
\noalign{\ifnum0=`}\fi\hrule \@height #1 %
\futurelet\reserved@a\@xhline}
\makeatother

% Source code listings
\lstdefinestyle{python}{
    captionpos=t,%
    basicstyle=\footnotesize\ttfamily,%
    numberstyle=\tiny,%
    numbers=none,%
    stepnumber=1,%
    frame=single,%
    showspaces=false,%
    showstringspaces=false,%
    showtabs=false,%
    keywordstyle=\color{blue},%
    identifierstyle=,%
    commentstyle=\color{gray},%
    stringstyle=\color{blue}%
}
% Override the Python settings with these
\lstset{language=python,
    basicstyle=\normalsize,
    frame=tb,
    framesep=5pt,
    keywordstyle=\ttfamily\color{OliveGreen},
    identifierstyle=\ttfamily\color{CadetBlue}\bfseries, 
    commentstyle=\color{Brown},
    stringstyle=\ttfamily\color{blue},
    xleftmargin=2pt,xrightmargin=2pt,
    showstringspaces=false
}

\lstset{language=R,
    basicstyle=\small,
    frame=tb,
    framesep=5pt,
    keywordstyle=\ttfamily\color{OliveGreen},
    identifierstyle=\ttfamily\color{CadetBlue}\bfseries, 
    commentstyle=\color{Brown},
    stringstyle=\ttfamily\color{blue},
    xleftmargin=2pt,xrightmargin=2pt,
    showstringspaces=false
}

% Some definitions
\newcommand{\todo}[1]{{\center{\color{todoGreen} #1}}}
\newcommand{\q}{{\textbf{Q}}}
\newcommand{\adv}{{\small {\color{Brown} (advanced)}}}
\newcommand{\extra}{{\small {\color{Brown} (extra)}}}
\newcommand{\lit}[1]{\href{http://literature.connectmv.com/item/#1}{http://literature.connectmv.com/item/#1}}
\newcommand{\lititem}[2]{\href{http://literature.connectmv.com/item/#1}{#2 {\tiny (http://literature.connectmv.com/item/#1)}}}
\newcommand{\liturl}[2]{\href{#1}{#2 {\tiny ~(#1)}}}
\newcommand{\micron}{$\mu $m}
\newcommand{\see}[1]{{\tiny [{\color{myBlue}{#1}}]}}
\newcommand{\seefull}[1]{{\color{myBlue}{#1}}}
\newcommand{\degC}{$^\circ\text{C}$}
\newcommand{\myhrule}{\vspace{4pt}\hrule\vspace{4pt}}

\newdateformat{mydate}{\THEYEAR} % \twodigit{\THEDAY}-\twodigit{\THEMONTH}-\THEYEAR
\mydate
\title[]{\LARGE Experimentation for Improvement}
\subtitle[]{\vspace{0.5cm} \includegraphics[width=0.5\textwidth]{\imagedir/teaching/logos/Coursera-MOOC-logo-2014.png} \vspace{-1.5cm} }\author[]{}
\institute[]{}
\date[]{\copyright~ Kevin Dunn, \today \\ \vspace{0cm}
{{\footnotesize {\tt } \href{http://learnche.org/}{http://learnche.org/}\\} \vspace{0.5cm}}

{\LARGE \color{blue}{Design and Analysis of Experiments - An Overview}}

}

\begin{document}

\begin{frame} \titlepage \end{frame}

\begin{frame}\frametitle{Copyright, sharing, and attribution notice}

	{\footnotesize This work is licensed under the Creative Commons Attribution-ShareAlike 4.0 Unported License. To view a copy of this license,
	please visit \href{http://creativecommons.org/licenses/by-sa/3.0/}{http://creativecommons.org/licenses/by-sa/4.0/}}

	\vspace{0.0cm}
	\begin{flushright}
		\includegraphics[width=0.2\textwidth]{\imagedir/common/creative-commons-by-sa.png}
	\end{flushright}
	\vspace{-0.4cm}
	\begin{exampleblock}{}
		{\small This license allows you: }
		\begin{itemize}
			\item	{\color{myGreen}{\textbf{to share}}} - to copy, distribute and transmit the work, including print it
			\item	{\color{myOrange}{\textbf{to adapt}}} - but you must distribute the new result under the same or similar license to this one
			\item	{\color{myRed}{\textbf{commercialize}}} - you \underline{\emph{are allowed}} to use this work for commercial purposes
			\item	{\color{blue}{\textbf{attribution}}} - but you must attribute the work as follows:
			\begin{itemize}
				\item	``Portions of this work are the copyright of Kevin Dunn'', \emph{or}
				\item	``This work is the copyright of Kevin Dunn'' \\{\tiny (when used without modification)}
			\end{itemize}
		\end{itemize}
	\end{exampleblock}
\end{frame}


\begin{frame}\frametitle{Achievable objectives when improving a process -- based on data}
	
	\begin{columns}[t]
		\column{0.6\textwidth}
		\Large
			\begin{enumerate}
				\item	learn more about, and increase your understanding  \pause
				\item	troubleshoot a problem  \pause
				\item	make predictions  \pause
				\item	try to optimize the process  \pause
				\item	monitor the system to ensure performance gains are retained  \pause
			\end{enumerate}
	
		\column{0.4\textwidth}
			\onslide+<5->{
				\includegraphics[width=\textwidth]{\imagedir/examples/reactor-design-example/response-surface.png}
			}
			\onslide+<6->{
				\includegraphics[width=\textwidth]{\imagedir/monitoring/Kappa-phaseII-testing.png}
			}
			
	\end{columns}
	
	\vspace{-1cm}

	\vspace{1cm}
	\onslide+<8->{
		\hfill \includegraphics[width=0.3\textwidth]{\imagedir/doe/examples/advice-logo.png}
		\,\,{\color{blue}Always have a clear objective in mind}
	}
\end{frame}



\begin{frame}\frametitle{The 5 goals for any data analysis work}

	\textbf{{\color{purple} 1. Understand}}
	\begin{itemize}
		\item	our process or system better
	\end{itemize}

	\vspace{12pt}
	\textbf{{\color{purple} 2. Troubleshoot}}
	\begin{itemize}
		\item	a problem
		\item	develop or improve a method
	\end{itemize}

	\vspace{12pt}
	\textbf{{\color{purple} 3. Predict}}
	\begin{itemize}
		\item	what our system will do under new conditions
	\end{itemize}

	\vspace{12pt}
	\textbf{{\color{purple} 4. Optimize}}
	\begin{itemize}
		\item	make the system perform better, safer, cheaper, faster
	\end{itemize}

	\vspace{12pt}
	\textbf{{\color{purple} 5. Monitor}}
		\begin{itemize}
			\item	ensure we keep the gains we have made
		\end{itemize}

\end{frame}

\begin{frame}\frametitle{Some important terminology we will use all the time}

	\begin{columns}[T]
		\column{0.75\textwidth}
			\textbf{{\color{purple} Outcome}}
				\begin{itemize}
					\item	What we measure after the experiment is finished  \pause
					\item	It is the aspect you are interested in improving  \pause
					\item	also called the \emph{Response}

				\end{itemize}
		\column{0.3\textwidth}

			\centerline{\includegraphics[width=\textwidth]{\imagedir/doe/measure-4904403417_93baa750a6-flickr.jpg}}
	\end{columns}

	\vspace{24pt}
	\textbf{{\color{purple} Factors}}
		\begin{itemize}
			\item	Things which you actively change to influence the outcome.
			\item	We typically change 2, 3, 4, or many more factors. \pause
			\item	Don't fixate on changing 1 factor at a time. {\color{myOrange}  $\longleftarrow$	\emph{Common misconception!}}
		\end{itemize}
\end{frame}

\begin{frame}\frametitle{}

	\begin{columns}[c]
		\column{0.5\textwidth}
			\centerline{\includegraphics[width=0.8\textwidth]{../1B/Supporting-material/plant.jpg}}

		\column{0.5\textwidth}
			Various outcomes are possible in your experiment:

			\begin{itemize}
				\item	height of the plant
				\item	average length of leaves
				\item	the number of flowers
			\end{itemize}

			\vspace{12pt}
			These are examples of numeric measurements (quantitative).

	\end{columns}
\end{frame}


\begin{frame}\frametitle{}

	\textbf{{\color{purple} Outcome}} = \textbf{{\color{purple} Response}}
		\begin{itemize}
			\item	What we measure after the experiment is finished
			\item	It is the aspect you are interested in improving.
		\end{itemize}

	\vspace{12pt}
	\pause

	\textbf{{\color{purple} Objective}}


		\qquad combine the {\color{purple} outcome} with ``a desire to \emph{adjust} the outcome''
	\vspace{24pt}
	\pause

	{\textbf{{Various examples of ``objectives''}}}

		\begin{itemize}
			\item	maximize $(\uparrow)$ the profit
			\item	maximize $(\uparrow)$ the height of the plant
			\item	minimize $(\downarrow)$ pollution
			\item	minimize $(\downarrow)$ energy used to produce a product
		\end{itemize}

	\pause
		\vspace{12pt}
		But sometimes the objective is ``the same as before'' $(=)$
\end{frame}

\begin{frame}\frametitle{}

	\textbf{{\color{purple} Factors}}  = \textbf{{\color{purple} Variables}}
	

		\begin{columns}[T]
			\column{0.33\textwidth}
				\centerline{\includegraphics[width=\textwidth]{../1B/Supporting-material/water.png}}


			\column{0.33\textwidth}
				\centerline{\includegraphics[width=\textwidth]{../1B/Supporting-material/fertilizer.png}}


			\column{0.33\textwidth}
				\centerline{\includegraphics[width=\textwidth]{../1B/Supporting-material/soil.png}}

		\end{columns}


	\vspace{24pt}

	
	\textbf{{\color{purple} Types of factors}}

	\vspace{12pt}
	\qquad \emph{numeric} factors (quantitative) can be measured and adjusted to different levels

	\qquad \qquad
	\vspace{12pt}

	\qquad \emph{categorical} factors (qualitative) are simply selected for their characteristic



\end{frame}


\begin{frame}\frametitle{A systematic approach } 
	\Large
	 \begin{enumerate}
	 	\item	What's my outcome? \pause
	 	\item	What's my objective? \pause
	 	\item	Which factors? \pause
		\item	At what levels?\pause
	 	\item	Plan the experiment \pause
	 	\item	Implement the experiment\pause
	 	\item	Analyze the results\pause
		\item	Repeat (almost always required) [25\% rule]
	 \end{enumerate}
	 
 	 \vspace{12pt}
	 \small
	 {\color{purple}  \href{http://yint.org/template}{Template for you to use}}
\end{frame}


\begin{frame}\frametitle{}
	%\vspace{-10pt}
	\centerline{\includegraphics[width=1.2\textwidth]{\imagedir/doe/DOE-trade-off-table.png}}
\end{frame}


\begin{itemize}
	
	\item   Collapsability sequence
	\item 	Interaction: what is?
	\item 	Tradeoff: 4 factors and 5 expts. But with 8 expts you can docs lot more
	\item 	Randomization is important
	\item 	Interaction plots
	\item 	Contour plot concept
	\item 	Disturbance and covariates
	\item 	Slide 283 for fraction factorial
	\item 	Rsm example walk through, incl slide 521. All models useful
	\item 	Cost approach incl slide 491
	\item 	When things go wrong you have fallback if you had those extra experiments
	
	\item	Rsm slide 529on interaction
	\item 	Rsm slide 543
	
	


\end{itemize}




\end{document}
