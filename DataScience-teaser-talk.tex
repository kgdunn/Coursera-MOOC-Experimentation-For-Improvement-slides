%\documentclass[11pt,aspectratio=169,mathserif]{beamer}
\documentclass[handout,11pt,aspectratio=169,mathserif]{beamer}

\usepackage[latin1]{inputenc}  % allows direct input of Unicode characters
% %\usepackage[utf8x]{inputenc}   % allows direct input of Unicode characters
\inputencoding{utf8} %<--- try this next time
%\usepackage{textcomp}
\usepackage{amsmath,amssymb,amsfonts,euscript,mathrsfs,wasysym,textcomp}
\usepackage{array}
\usepackage{transparent} 		% to get alpha channel in "\color"
\usepackage{multirow}
\usepackage{multimedia}
\usepackage{fancybox}
\usepackage{mathtools}
\usepackage{psfrag}
\usepackage{listings}
\usepackage{hyperref}
\usepackage[normalem]{ulem}  	% For strikeout text: \sout{text goes here}
\usepackage{datetime} 			% For better date/time display
\usepackage{comment}  			% To block out parts of the notes
\usepackage[multidot]{grffile} 	% Handles file names with spaces and dots in them
\usepackage{cancel}				% To cancel terms in an equation
%\usepackage{rotating}			% Rotated text
\usepackage{etoolbox}			% For selective compiling: figure sources might be on different machines
\usepackage{tabulary}			% Better package for tables?

\providetoggle{internal}
\settoggle{internal}{false}      % set "true" when the PDF is for internal use; use "false" when publishing it publically, on-line

\providetoggle{hrymak}          % usage: \iftoggle{hrymak}{}{}
\settoggle{hrymak}{false}

\providetoggle{instructor}
\providetoggle{student}

\settoggle{instructor}{true}    % usage: \iftoggle{instructor}{}{}
\settoggle{student}{false}      % usage: \iftoggle{student}{}{}

\providetoggle{unixmac}
\settoggle{unixmac}{true}
\iftoggle{unixmac}
{
	\newcommand{\imagedir}{/Users/kevindunn/Dropbox/figures-master}
}{
	\newcommand{\imagedir}{C:/Figures}
}

%\newcommand{\angstrom}{\mbox{\normalfont\AA}}
% \tiny\scriptsize\footnotesize\small\normalsize\large\Large\LARGE\huge\Huge

\usetheme{default} 
\setbeamertemplate{navigation symbols}{}          % suppress all navigation symbols
\setbeamertemplate{blocks}[rounded][shadow=true]  % use rounded blocks (boxes), with shadows

% http://www.latex-community.org/forum/viewtopic.php?f=4&t=2251
\newenvironment<>{varblock}[2][\textwidth]{%
	\begin{center}
		\begin{minipage}{#1}
			\setlength{\textwidth}{#1}
			\begin{actionenv}#3%
				\def\insertblocktitle{#2}%
				\par%
				\usebeamertemplate{block begin}}
				{\par%
				\usebeamertemplate{block end}%
			\end{actionenv}
		\end{minipage}
	\end{center}
}
\definecolor{todoGreen}{rgb}{0.0, 0.9, 0.0}
\definecolor{myGreen}{rgb}{0.,0.4,0.}
\definecolor{myOrange}{rgb}{1.,0.5,0.}
\definecolor{myBlue}{rgb}{0.0,0.1,0.9}
\definecolor{myRed}{rgb}{1.0,0.0,0.0}
\definecolor{myLightGrey}{rgb}{0.8,0.8,0.8}
\definecolor{Brown}{cmyk}{0,0.81,1,0.60}
\definecolor{OliveGreen}{cmyk}{0.64,0,0.95,0.40}
\definecolor{CadetBlue}{cmyk}{0.62,0.57,0.23,0}
\definecolor{purple}{rgb}{0.70,0.22,0.92}

%https://tex.stackexchange.com/questions/31323/cross-out-with-arrow-as-in-goes-to-zero
\renewcommand{\CancelColor}{\color{red}} %change cancel color to red
\makeatletter
\let\my@cancelto\cancelto %copy over the original cancelto command
\newcommand<>{\cancelto}[2]{\alt#3{\my@cancelto{#1}{#2}}{\mathrlap{#2}\phantom{\my@cancelto{#1}{#2}}}}
% redefine the cancelto command, using \phantom to assure that the
% result doesn't wiggle up and down with and without the arrow
\makeatother



% Show page numbers and dates on slides
% KGD: removed footer completely: 01 Oct 2012: Presentation Zen: no redundant information
%\setbeamertemplate{footline}{\begin{beamercolorbox}[right]{section in head/foot}{\color{black}{\tiny \insertframenumber}} ~~~ \vskip5pt \end{beamercolorbox}}

\setbeamercovered{transparent=1} 

\hypersetup{colorlinks        = true,    
 			linkcolor         = blue,    
%}
% 			linkbordercolor   = {1 0 0},    
% 			urlcolor          = cyan,    
% 			bookmarks         = {true,},    
% 			bookmarksopen     = {true,},    
% 			bookmarksnumbered = {false,},    
 			pdftitle          = {Kevin Dunn},
 			pdfsubject        = {Kevin Dunn},
 			pdfauthor         = {Kevin Dunn},
 			pdfproducer       = {LaTeX, wiki2beamer, beamer, BeamerPDF},    
 			pdfkeywords       = http://learnche.mcmaster.ca/,
 		}
\usepackage{pgfpages}

\makeatletter
\def\hlinewd#1{%
\noalign{\ifnum0=`}\fi\hrule \@height #1 %
\futurelet\reserved@a\@xhline}
\makeatother

% Source code listings
\lstdefinestyle{python}{
    captionpos=t,%
    basicstyle=\footnotesize\ttfamily,%
    numberstyle=\tiny,%
    numbers=none,%
    stepnumber=1,%
    frame=single,%
    showspaces=false,%
    showstringspaces=false,%
    showtabs=false,%
    keywordstyle=\color{blue},%
    identifierstyle=,%
    commentstyle=\color{gray},%
    stringstyle=\color{blue}%
}
% Override the Python settings with these
\lstset{language=python,
    basicstyle=\normalsize,
    frame=tb,
    framesep=5pt,
    keywordstyle=\ttfamily\color{OliveGreen},
    identifierstyle=\ttfamily\color{CadetBlue}\bfseries, 
    commentstyle=\color{Brown},
    stringstyle=\ttfamily\color{blue},
    xleftmargin=2pt,xrightmargin=2pt,
    showstringspaces=false
}

\lstset{language=R,
    basicstyle=\small,
    frame=tb,
    framesep=5pt,
    keywordstyle=\ttfamily\color{OliveGreen},
    identifierstyle=\ttfamily\color{CadetBlue}\bfseries, 
    commentstyle=\color{Brown},
    stringstyle=\ttfamily\color{blue},
    xleftmargin=2pt,xrightmargin=2pt,
    showstringspaces=false
}

% Some definitions
\newcommand{\todo}[1]{{\center{\color{todoGreen} #1}}}
\newcommand{\q}{{\textbf{Q}}}
\newcommand{\adv}{{\small {\color{Brown} (advanced)}}}
\newcommand{\extra}{{\small {\color{Brown} (extra)}}}
\newcommand{\lit}[1]{\href{http://literature.connectmv.com/item/#1}{http://literature.connectmv.com/item/#1}}
\newcommand{\lititem}[2]{\href{http://literature.connectmv.com/item/#1}{#2 {\tiny (http://literature.connectmv.com/item/#1)}}}
\newcommand{\liturl}[2]{\href{#1}{#2 {\tiny ~(#1)}}}
\newcommand{\micron}{$\mu $m}
\newcommand{\see}[1]{{\tiny [{\color{myBlue}{#1}}]}}
\newcommand{\seefull}[1]{{\color{myBlue}{#1}}}
\newcommand{\degC}{$^\circ\text{C}$}
\newcommand{\myhrule}{\vspace{4pt}\hrule\vspace{4pt}}

\newdateformat{mydate}{\THEYEAR} % \twodigit{\THEDAY}-\twodigit{\THEMONTH}-\THEYEAR
\mydate
\title[]{\LARGE Experimentation for Improvement}
\subtitle[]{\vspace{0.5cm} \includegraphics[width=0.5\textwidth]{\imagedir/teaching/logos/Coursera-MOOC-logo-2014.png} \vspace{-1.5cm} }\author[]{}
\institute[]{}
\date[]{Kevin Dunn, \today \\ \vspace{0cm}
{{\footnotesize {\tt } \href{https://learnche.org/}{https://learnche.org/}\\} \vspace{0.5cm}}
{\LARGE \color{blue}{Design and Analysis of Experiments - An Overview}}
}

\begin{document}

\begin{frame} \titlepage \end{frame}
\begin{frame}\frametitle{Copyright, sharing, and attribution notice}

	{\footnotesize This work is licensed under the Creative Commons Attribution-ShareAlike 4.0 Unported License. To view a copy of this license,
	please visit \href{https://creativecommons.org/licenses/by-sa/4.0/}{https://creativecommons.org/licenses/by-sa/4.0/}}

	\vspace{0.0cm}
	\begin{flushright}
		\includegraphics[width=0.2\textwidth]{\imagedir/common/creative-commons-by-sa.png}
	\end{flushright}
	\vspace{-0.4cm}
	\begin{exampleblock}{}
		{\small This license allows you: }
		\begin{itemize}
			\item	{\color{myGreen}{\textbf{to share}}} - to copy, distribute and transmit the work, including print it
			\item	{\color{myOrange}{\textbf{to adapt}}} - but you must distribute the new result under the same or similar license to this one
			\item	{\color{myRed}{\textbf{commercialize}}} - you \underline{\emph{are allowed}} to use this work for commercial purposes
			\item	{\color{blue}{\textbf{attribution}}} - but you must attribute the work as follows:
			\begin{itemize}
				\item	``Portions of this work are the copyright of Kevin Dunn'', \emph{or}
				\item	``This work is the copyright of Kevin Dunn'' \\{\tiny (when used without modification)}
			\end{itemize}
		\end{itemize}
	\end{exampleblock}
\end{frame}

\begin{frame}\frametitle{Some important terminology}

	\begin{columns}[T]
		\column{0.75\textwidth}
			\textbf{{\color{purple} Outcome}}
				\begin{itemize}
					\item	What we measure after the experiment is finished  \pause
					\item	It is the aspect you are interested in improving  
					\item	also called the \emph{Response}

				\end{itemize}
			
			\pause
			\vspace{12pt}	
			\textbf{{\color{purple} Objective}}
			
				\vspace{1pt}	
				\qquad combine the {\color{purple} outcome} with ``a desire to \emph{adjust} the outcome''
				
				
			\pause
			\vspace{12pt}
			
			{\textbf{{Various examples of ``{\color{purple} objectives}''}}}

				\begin{itemize}
					\item	maximize $(\uparrow)$ the profit
					\item	maximize $(\uparrow)$ the height of the plant
					\item	minimize $(\downarrow)$ pollution
					\item	minimize $(\downarrow)$ energy used to produce a product
				\end{itemize}
			
			
		\column{0.3\textwidth}
			\centerline{\includegraphics[width=0.8\textwidth]{\courseradir/1B/Supporting-material/plant.jpg}}
			%\centerline{\includegraphics[width=\textwidth]{\imagedir/doe/measure-4904403417_93baa750a6-flickr.jpg}}
	\end{columns}


\end{frame}
\begin{frame}\frametitle{Some important terminology}
	\vspace{12pt}
	\textbf{{\color{purple} Factors = Variables}}
		\begin{itemize}
			\item	Things which you actively change to influence the outcome.
			\item	We typically change 2, 3, 4, or many more factors in a set of experiments. \pause
			\item	Don't fixate on changing 1 factor at a time. {\color{myOrange}  $\longleftarrow$	\emph{Common misconception!}}
		\end{itemize}
	
	\begin{columns}[T]
		\column{0.33\textwidth}
			\centerline{\includegraphics[width=\textwidth]{\courseradir/1B/Supporting-material/water.png}}
			
			\centerline{Numeric}


		\column{0.33\textwidth}
			\centerline{\includegraphics[width=\textwidth]{\courseradir/1B/Supporting-material/fertilizer.png}}
			
			\centerline{Categorical}


		\column{0.33\textwidth}	
			\centerline{\includegraphics[width=\textwidth]{\courseradir/1B/Supporting-material/soil.png}}
	\end{columns}
	
	\begin{itemize}
		\item	\emph{numeric} factors (quantitative) can be measured and adjusted to different levels
		\item	\emph{categorical} factors (qualitative) are simply selected for their characteristic
	\end{itemize}	
\end{frame}
\begin{frame}\frametitle{A systematic approach} 
	\begin{columns}[T]
		\column{0.6\textwidth}
			\Large
			\begin{enumerate}
			\item	What's my outcome? 
			\item	What's my objective? \pause
			\item	Which factors? 
			\item	At what levels?\pause
			\item	Plan the experiment {\small (\href{https://yint.org/template}{template})}\pause
			\item	Implement the experiments \pause
			\item	Analyze the results\pause
			\item	Repeat the process \newline (almost always required) [25\% rule]
			\end{enumerate}
			
		\column{0.5\textwidth}
			\onslide+<3->{
				\includegraphics[width=.8\textwidth]{\imagedir/doe/screenshot-of-DOE-template.png}
			}
	\end{columns}
\end{frame}

% Sequence: from baseline to interactions
\begin{frame}\frametitle{Basic concept of Designed Experiments}
	\centerline{\includegraphics[height=.8\textheight]{\imagedir/doe/intplot-ofat-vs-full-factorial-base.png}}
\end{frame}
\begin{frame}\frametitle{}
	\centerline{\includegraphics[height=.8\textheight]{\imagedir/doe/ofat-vs-full-factorial-base-colours.png}}
\end{frame}
\begin{frame}\frametitle{}
	\centerline{\includegraphics[height=.8\textheight]{\imagedir/doe/ofat-vs-full-factorial-base-colours-results.png}}
\end{frame}
\begin{frame}\frametitle{}
	\centerline{\includegraphics[height=.8\textheight]{\imagedir/doe/intplotA-ofat-vs-full-factorial-base-colours-results.png}}
\end{frame}
\begin{frame}\frametitle{}
	\centerline{\includegraphics[height=.8\textheight]{\imagedir/doe/intplotB-ofat-vs-full-factorial-base-colours-results.png}}
\end{frame}
\begin{frame}\frametitle{}
	\centerline{\includegraphics[height=.8\textheight]{\imagedir/doe/ofat-vs-full-factorial-2-factors-full-factorial-colours.png}}
\end{frame}
\begin{frame}\frametitle{}
	\centerline{\includegraphics[height=.8\textheight]{\imagedir/doe/intplotA-ofat-vs-full-factorial-base-colours-full-results.png}}
\end{frame}
\begin{frame}\frametitle{}
	\centerline{\includegraphics[height=.8\textheight]{\imagedir/doe/intplotB-ofat-vs-full-factorial-base-colours-full-results.png}}
\end{frame}
%\begin{frame}\frametitle{}
%	\centerline{\includegraphics[height=.8\textheight]{\imagedir/doe/intplotA-ofat-vs-full-factorial-base-colours-full-results.png}}
%\end{frame}

% 3-factor full system
\begin{frame}\frametitle{}
	\centerline{\includegraphics[height=.8\textheight]{\imagedir/doe/ofat-vs-full-factorial-3rd-factor-colours.png}}
\end{frame}
\begin{frame}\frametitle{}
	\centerline{\includegraphics[height=.8\textheight]{\imagedir/doe/ofat-vs-full-factorial-3-factors-full-factorial-colours.png}}
\end{frame}
\begin{frame}\frametitle{}
	
	\Large{Total number of experiments = } $\scalebox{6}{${\color{myBlue} 2}^{\color{brickred} k}$} $
	
	\Huge
	
	\vspace{24pt}
	{\color{brickred}{$k$} = number of factors}
	
	{\color{myBlue}{$2$} = number of levels for all factors}
\end{frame}


\begin{frame}\frametitle{}
	\centerline{\includegraphics[height=.8\textheight]{\imagedir/doe/intplotA-ofat-vs-full-factorial-base-colours-interactions-results.png}}
\end{frame}

% Interactions
\begin{frame}\frametitle{Understanding interactions: water and soap example}
	\begin{center}
		\includegraphics[width=\textwidth]{\imagedir/doe/examples/hand-washing-flickr-7008312299_2fcf07309c_k.jpg}
	\end{center}
	\vspace{-4cm}
	\see{\href{https://secure.flickr.com/photos/usdagov/7008312299/}{Flickr}}
\end{frame}
\begin{frame}\frametitle{Understanding interactions: water and soap example}
	
	\large
	\begin{itemize}
		\item	\textbf{Using soap} works better with warm water (instead of cold water)

			\makebox[14cm][c]{We say: ``{\color{OliveGreen}\emph{the effect of warm water enhances the effect of soap}}''}
		
		\vspace{24pt}
		\item	\textbf{Warm water} works better with soap (instead of no soap)
			\makebox[14cm][c]{We say: ``{\color{OliveGreen}\emph{the effect of soap is enhanced by using warm water}}''}
			
	\end{itemize}

	\vspace{24pt}
	This interaction works in our favour in this example.
\end{frame}
\begin{frame}\frametitle{The advantages of the factorial designs}
	\begin{itemize}
		\item	redundancy [e.g. an experiment fails, measurement outlier]
		
		\item	information about interactions
		
		\item	clear signal which factors truly have an effect 
		
		\item	quantify the effect of each factor
		
		\item 	each factor's effect is quantified independent of the other
	\end{itemize}
\end{frame}



% DOE tradeoff table
\begin{frame}\frametitle{}
	%\vspace{-10pt}
	\centerline{\includegraphics[width=1.2\textwidth]{\imagedir/doe/DOE-trade-off-table-main-focus.png}}
\end{frame}

% Example of using the tradeoff table
\begin{frame}\frametitle{Cell-culture example: long duration runs; and many factors are possible}
	\newcommand{\white}{\color{white}}
	\begin{columns}[c]
		\column{0.5\textwidth} 
			\begin{enumerate}
				\item	\textbf{T}: the temperature profile
				\item	\textbf{D}: dissolved oxygen
				\item	\textbf{A}: agitation rate
				\item	\textbf{P}: pH
				\item	\textbf{S}: substrate type (A or B)
				%\onslide+<2->{
				%\item	\textbf{W}: water type (distilled or tap water)
				%\item	\textbf{M}: mixer type (axial or radial)
				%}
			\end{enumerate}
		
		\column{0.5\textwidth}
			%{\color{blue} \small Industrial scale: {\color{white}y}}   
			
			\vspace{0.2cm}
			
			\centerline{\includegraphics[height=.7\textwidth]{\courseradir/4F/Supporting material/flickr-493740898_b98113ae44_o-cell-culture-mod.png}}
			\see{\href{https://secure.flickr.com/photos/londonmatt/493740898}{Flickr: londonmatt}}
	\end{columns}

	\vfill
	At 10 days per cell culture, it would take $\approx$ 1 year for all  ${\color{myBlue} 2}^{\color{brickred} 5} = 32$.
	
	Budget = 3 months, {\color{myOrange} that corresponds to 9 experiments}.
		
\end{frame}
\begin{frame}\frametitle{Cell-culture example: long duration runs; and many factors are possible}
	\begin{columns}[c]
		\column{0.5\textwidth}
			\begin{enumerate}
				\item	\textbf{T}: the temperature profile
				\item	\textbf{D}: dissolved oxygen
				\item	$\cancel{\text{\textbf{A}: agitation rate}}$
				\item	$\cancel{\text{\textbf{P}: pH}}$
				\item	\textbf{S}: substrate type (A or B)
				%\onslide+<2->{
				%\item	\textbf{W}: water type (distilled or tap water)
				%\item	\textbf{M}: mixer type (axial or radial)
				%}
			\end{enumerate}
		
		\column{0.5\textwidth}
			%{\color{blue} \small Laboratory equipment to control the culture:} 
			
			\vspace{0.2cm}
			
			\centerline{\includegraphics[height=.7\textwidth]{\courseradir/4F/Supporting material/flickr-3815183191_98c6296080_b-cell-culture-instrumentation.png}}
			\see{\href{https://secure.flickr.com/photos/mjanicki/3815183191}{Flickr: mjanicki}}
	\end{columns}

	\vfill
	{\color{red} Don't do this:} remove factors in order to get a full factorial.
	
\end{frame}

% DOE tradeoff table
\begin{frame}\frametitle{}
	%\vspace{-10pt}
	\centerline{\includegraphics[width=1.2\textwidth]{\imagedir/doe/DOE-trade-off-table-main-focus.png}}
\end{frame}

\begin{frame}\frametitle{Cell-culture example: long duration runs; and many factors are possible}
	\begin{columns}[c]
		\column{0.5\textwidth}
			\begin{enumerate}
				\item	\textbf{T}: the temperature profile
				\item	\textbf{D}: dissolved oxygen
				\item	\textbf{A}: agitation rate
				\item	\textbf{P}: pH
				\item	\textbf{S}: substrate type (A or B)
				\item	\textbf{W}: water type (distilled or tap water)
				\item	\textbf{M}: mixer type (axial or radial)
		
			\end{enumerate}
		
		\column{0.5\textwidth}
			
			
			\vspace{0.2cm}
			
			\centerline{\includegraphics[height=.7\textwidth]{\courseradir/4F/Supporting material/Mixing_-_flusso_assiale_e_radiale-wikipedia.jpg}}
			
			{\color{blue} \small Different types of mixers (impellers)} \see{\href{https://en.wikipedia.org/wiki/File:Mixing_-_flusso_assiale_e_radiale.jpg}{Wikipedia}}
	\end{columns}
\end{frame}

% Grocery shelft OFAT example
\begin{frame}\frametitle{}
	\centerline{\includegraphics[height=.8\textheight]{\courseradir/5C/Supporting material/flickr-quinnanya-2186957732_12f67e557b_o-grocery-shelf.jpg}}
	\see{\href{https://secure.flickr.com/photos/quinnanya/2186957732}{Flickr: quinnanya}}
\end{frame}
\begin{frame}\frametitle{}
	\centerline{\includegraphics[height=\textheight]{\courseradir/5C/Supporting material/COST-contours-shopping-01a.png}}
\end{frame}
\begin{frame}\frametitle{}
	\centerline{\includegraphics[height=\textheight]{\courseradir/5C/Supporting material/COST-contours-shopping-01b.png}}
\end{frame}
\begin{frame}\frametitle{}
	\centerline{\includegraphics[height=\textheight]{\courseradir/5C/Supporting material/COST-contours-shopping-02a.png}}
\end{frame}
\begin{frame}\frametitle{}
	\centerline{\includegraphics[height=\textheight]{\courseradir/5C/Supporting material/COST-contours-shopping-02b.png}}
\end{frame}
\begin{frame}\frametitle{}
	\centerline{\includegraphics[height=\textheight]{\courseradir/5C/Supporting material/COST-contours-shopping-03.png}}
\end{frame}
\begin{frame}\frametitle{}
	\centerline{\includegraphics[height=\textheight]{\courseradir/5C/Supporting material/COST-contours-shopping-04.png}}
\end{frame}
\begin{frame}\frametitle{}
	\centerline{\includegraphics[height=\textheight]{\courseradir/5C/Supporting material/COST-contours-shopping-05.png}}
\end{frame}
\begin{frame}\frametitle{}
	\centerline{\includegraphics[height=\textheight]{\courseradir/5C/Supporting material/COST-contours-shopping-06a.png}}
\end{frame}
\begin{frame}\frametitle{}
	\centerline{\includegraphics[height=\textheight]{\courseradir/5C/Supporting material/COST-contours-shopping-06b.png}}
\end{frame}
\begin{frame}\frametitle{}
	\centerline{\includegraphics[height=\textheight]{\courseradir/5C/Supporting material/COST-contours-shopping-07.png}}
\end{frame}
\begin{frame}\frametitle{}
	\centerline{\includegraphics[height=\textheight]{\courseradir/5C/Supporting material/COST-contours-shopping-08.png}}
\end{frame}
\begin{frame}\frametitle{}
	\centerline{\includegraphics[height=\textheight]{\courseradir/5C/Supporting material/COST-contours-shopping-09.png}}
\end{frame}
\begin{frame}\frametitle{}
	\centerline{\includegraphics[height=\textheight]{\courseradir/5C/Supporting material/COST-contours-shopping-10.png}}
\end{frame}
\begin{frame}\frametitle{}
	\centerline{\includegraphics[height=\textheight]{\courseradir/5C/Supporting material/COST-contours-shopping-11.png}}
\end{frame}
\begin{frame}\frametitle{Interpreting what contour plots are}
	\begin{columns}[b]
		\column{0.5\textwidth}
			\centerline{\includegraphics[width=1.1\textwidth]{\courseradir/5D/Supporting material/wikipedia-20121125213159!FujiSunriseKawaguchiko2025WP.jpg}}
			
			\see{Wikipedia: \href{https://commons.wikimedia.org/wiki/File:FujiSunriseKawaguchiko2025WP.jpg}{Wikipedia}}
		
		\column{0.5\textwidth}
			\centerline{\includegraphics[width=1.102\textwidth]{\courseradir/5D/Supporting material/Google-Map-Screenshot-wide.png}}

			\see{\href{https://www.google.com/maps/preview?f=q&hl=en&geocode=&q=Mt.+Fuji&ie=UTF8&t=p&ll=35.366656,138.733292&spn=0.099668,0.207367&z=13&iwloc=addr;}{Google Maps link}}
	\end{columns}
\end{frame}
\begin{frame}\frametitle{The COST approach: {\color{myGreen}``Change One Single Thing'' at a time}}
	\begin{columns}[T]
		\column{0.6\textwidth}
			\begin{exampleblock}{}
				\begin{itemize}
					\item	\onslide+<1->{leads you into a false belief that you have reached the optimum }
					\item	\onslide+<2->{is order-dependent (it is a lottery\emph{!})}
					\item	\onslide+<3->{{\color{myOrange}COST works in a lab: to prove a conclusive cause-effect relationship}}
					\item	\onslide+<4->{interactions and other unusual surface shapes makes COST inefficient }
					\item	\onslide+<5->{does not scale to more factors: 3, 4 or more dimensions}
					
				\end{itemize}
			\end{exampleblock}
		
		\column{0.40\textwidth}
			
				\centerline{\includegraphics<1>[width=1.1\textwidth]{\courseradir/5C/Supporting material/COST-contours-shopping-12.png}}
				\centerline{\includegraphics<2->[width=.9\textwidth]{\courseradir/5G/Supporting materials/COST-contours-shopping-extend-20.png}}
				\centerline{\includegraphics<2->[width=.9\textwidth]{\courseradir/5G/Supporting materials/COST-contours-shopping-reorder-09.png}}
	\end{columns}
	
	%\pause
	%\vspace{0.5cm}
	%{\color{blue} Recall, RSM {\small (response surface methods)} is used \textbf{after} screening for known causal factors}

\end{frame}

% RSM sequence
\begin{frame}\frametitle{Case study: manufacturing of a mass produced product}
	\begin{columns}[c]
		\column{0.4\textwidth}
				\centerline{\includegraphics[height=.7\textheight]{\courseradir/5E/Supporting materials/flickr-bitchbuzz-6510472979_bf2db00108_o-beer-bottles.jpg}}
				\see{Flickr: 6510472979}
		\column{0.60\textwidth}
			Two factors are available to vary:
			\begin{itemize}
				\item	\textbf{T}hroughput: number of parts per hour
				\item	\textbf{P}rice: selling price per part produced
			\end{itemize}
			
			\vspace{1cm}
			\pause
			The outcome variable $y$ = profit [\$ per hour]
			
			\begin{itemize}
				\item	profit = (all income) $-$ (all expenses) \pause
				\item	both factors affect the profit
				\item	profit is easily calculated 
			\end{itemize}
	\end{columns}
\end{frame}
\begin{frame}\frametitle{}
	\centerline{\includegraphics[height=\textheight]{\courseradir/5E/Supporting materials/RSM-02.png}}
\end{frame}
\begin{frame}\frametitle{}
	\centerline{\includegraphics[height=\textheight]{\courseradir/5E/Supporting materials/RSM-05.png}}
\end{frame}
\begin{frame}\frametitle{}
	\centerline{\includegraphics[height=\textheight]{\courseradir/5E/Supporting materials/RSM-08.png}}
\end{frame}
\begin{frame}\frametitle{}
	\centerline{\includegraphics[height=\textheight]{\courseradir/5E/Supporting materials/RSM-07.png}}
\end{frame}
\begin{frame}\frametitle{}
	\centerline{\includegraphics[height=\textheight]{\courseradir/5E/Supporting materials/RSM-09.png}}
\end{frame}
\begin{frame}\frametitle{}
	\centerline{\includegraphics[height=\textheight]{\courseradir/5E/Supporting materials/RSM-15.png}}
\end{frame}
\begin{frame}\frametitle{}
	\centerline{\includegraphics[height=\textheight]{\courseradir/5E/Supporting materials/RSM-22.png}}
\end{frame}
\begin{frame}\frametitle{}
	\centerline{\includegraphics[height=\textheight]{\courseradir/5E/Supporting materials/RSM-31.png}}
\end{frame}
\begin{frame}\frametitle{}
	\centerline{\includegraphics[height=\textheight]{\courseradir/5E/Supporting materials/RSM-35.png}}
\end{frame}
\begin{frame}\frametitle{Improving the model's prediction ability by adding quadratic terms}
	Add specially placed points:
	
	\centerline{\includegraphics[height=.8\textheight]{\imagedir/doe/lack-of-fit-illustration-nonlinear-extra-points.png}}
\end{frame}
\begin{frame}\frametitle{Improving the model's prediction ability by adding quadratic terms}
	And fit a quadratic model now:
	
	\centerline{\includegraphics[height=.8\textheight]{\imagedir/doe/lack-of-fit-illustration-nonlinear-extra-points-quadratic.png}}
\end{frame}
\begin{frame}\frametitle{}
	\centerline{\includegraphics[height=\textheight]{\courseradir/5E/Supporting materials/RSM-43.png}}
\end{frame}
\begin{frame}\frametitle{}
	\centerline{\includegraphics[height=\textheight]{\courseradir/5E/Supporting materials/RSM-54.png}}
\end{frame}

\begin{frame}\frametitle{The 5 goals for any data analysis work}

	\begin{columns}[t]
		\column{0.6\textwidth}
		
			\vspace{-12pt}
			\textbf{{\color{purple} 1. Understand}}
			\begin{itemize}
				\item	our process or system better\pause
			\end{itemize}

			%\vspace{4pt}
			\textbf{{\color{purple} 2. Troubleshoot}}
			\begin{itemize}
				\item	a problem
				\item	develop or improve a method\pause
			\end{itemize}

			%\vspace{4pt}
			\textbf{{\color{purple} 3. Predict}}
			\begin{itemize}
				\item	what our system will do under new conditions\pause
			\end{itemize}

			%\vspace{4pt}
			\textbf{{\color{purple} 4. Optimize}}
			\begin{itemize}
				\item	make the system perform better, safer, cheaper, faster\pause
			\end{itemize}

			%\vspace{pt}
			\textbf{{\color{purple} 5. Monitor}}
			\begin{itemize}
				\item	ensure we keep the gains we have made\pause
			\end{itemize}
			
		\column{0.4\textwidth}
			\onslide+<3->{
				\includegraphics[width=\textwidth]{\imagedir/examples/reactor-design-example/response-surface.png}
			}
			\onslide+<5->{
				\includegraphics[width=\textwidth]{\imagedir/monitoring/Kappa-phaseII-testing.png}
			}
			
	\end{columns}
	
	\onslide+<6->{
		\hfill \includegraphics[width=0.3\textwidth]{\imagedir/doe/examples/advice-logo.png} {\color{blue}Always have a clear objective in mind}
	}

\end{frame}

\begin{frame}\frametitle{Resources}
	
	\begin{exampleblock}{}
		\centerline{\LARGE  \href{https://yint.org/resources}{yint.org/resources}}
	\end{exampleblock}
	
	\begin{itemize}
		\item	Free online course: \href{http://yint.org/expt}{yint.org/expt}
		\item	Free online textbook: \href{https://learnche.org/pid/design-analysis-experiments/index}{https://learnche.org/pid} (chapter 5)
		\item	12 week internal training
		\item	Python code for DoE analysis: \href{https://github.com/kgdunn/process\_improve}{https://github.com/kgdunn/process\_improve}
	\end{itemize}
\end{frame}

\iffalse
\begin{itemize}
	\item   Collapsability sequence
	\item 	Interaction: what is?
	\item 	Tradeoff: 4 factors and 5 expts. But with 8 expts you can docs lot more
	  - tradeoff: uncertainty from materials, people, or equipment.
	  \item   blocking: in the tradeoff table
	\item 	Randomization is important
	\item 	Interaction plots
	\item 	Contour plot concept
	\item 	Disturbance and covariates	
	\item 	Slide 283 for fraction factorial
	\item 	Rsm example walk through, incl slide 521. All models useful
	\item 	Cost approach incl slide 491
	\item 	When things go wrong you have fallback if you had those extra experiments
\end{itemize}
\fi

\end{document}
